\begin{to_def}
    Пусть $\E |\xi|^k < + \infty$. Число $\nu_k = \E \xi^k$ называется \textit{моментом порядка $k$}, или \textit{$k$-м моментом}  случайной величины $\xi$, число $\E |\xi|^k$ называется \textit{абсолютным $k$-м моментом}. Число
    $
        \E \left[
            \xi - \E (\xi)
        \right]^k
    $
    называется \textit{центральным $k$-м моментом}, $\E |\xi - \E \xi|^2$ -- \textit{абсолютным центральным $k$-м моментом}.
\end{to_def}

\begin{to_def}
    Число $\D (\xi) = \E (\xi - \E \xi)^2$ (центральный момент второго порядка)  называется \textit{дисперсией} случайной величины $\xi$.  Другими словами, это <<среднее значение квадрата отклонения случайной величины $\xi$ от своего среднего>>.
\end{to_def}

\begin{to_def}
    Число $\sigma = \sqrt{ D \xi}$ называют \textit{среднеквадратичным отклонением} случайной величины $\xi$.
\end{to_def}

\begin{to_thr}[неравенство Йенсена]
    Пусть вещественнозначная функция $g$ выпукла. Тогда для любой случайной величины $\xi$ с конечным первым моментом верно неравенство
    \begin{equation*}
        \E g(\xi) \geq g (\E \xi),
    \end{equation*}
    где для вогнутых функций знак неравенства меняется на противоположный. 
\end{to_thr}

\begin{to_lem}
    Если $\E |\xi|^t < \infty$, то для любого $0 < s < t$ верно, что
    \begin{equation*}
        \sqrt[s]{\E |\xi|^s} \leq \sqrt[t]{\E |\xi|^t}.
    \end{equation*}
\end{to_lem}

\noindent
Также из неравенства Йенсена вытекает ряд удобных неравенств:
\begin{equation*}
    \E e^{\xi} \geq e^{\E \xi},
    \hspace{5 mm}
    \red{\ldots}
\end{equation*}