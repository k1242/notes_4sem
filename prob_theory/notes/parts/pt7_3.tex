\textbf{Многомерное нормальное распределение}. Пусть $\Sigma > 0$ -- положительно определенная симметричная матрица. Говорят, что вектор $(\xi_1, \ldots, \xi_n)$ имеет многомерное нормально распределение $\textnormal{N}_{\vv{a}, \Sigma}$ с вектором средних $\vc{a}$ и матрицей ковариации $\Sigma$, если плотность совместного распределения $f_{\xi_1, \ldots, \xi_n} (x_1, \ldots, x_n)$ равна
\begin{equation*}
    f_{\xi} (\vc{x}) = \frac{1}{\sqrt{\det \Sigma} (\sqrt{2 \pi})^n}
    \exp\left(
        - \frac{1}{2} (\vc{x}-\vc{a})\T \cdot \Sigma^{-1} \cdot (\vc{x} - \vc{a})
    \right)
\end{equation*}
В частном случае, когда $\Sigma$ -- диагональная матрица с элементами $\sigma_1^2, \ldots, \sigma_n^2$ на диагонали, совместная плотность превращается в произведение плотностей нормальных величин. Вообще это равенство означает независимость величин $\xi_1, \ldots, \xi_n$.