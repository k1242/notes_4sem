\begin{to_def}
    События $A$ и $B$ называются \textit{независимыми}, если $\P(A \cap B) = \P(A) \P(B)$.
\end{to_def}
\noindent
Из этого определения вытекают следующие леммы.
\begin{to_lem}
    Пусть $\P(B) > 0$. Тогда события $A$ и $B$ независимы тогда и только, когда $\P(A|B) = \P(A)$. 
\end{to_lem}
\begin{to_lem}
    Пусть $A$ и $B$ несовместны. Тогда независимыми они будут только в том случае, если $\P(A) = 0$ или $\P(B) = 0$.
\end{to_lem}

Другими словами несовместные события не могут быть независимыми. Зависимость между ними -- просто причинно-следственная: если $A\cap B = \varnothing$, то $A \subseteq \bar{B}$, т.е. при выполнении $A$ события $B$  \textit{не происходит}.


\begin{to_lem}
    % свойство 6
    Если события $A$ и $B$ независимы, то независимы и события $A$ и $\bar{B}$, $\bar{A}$ и $B$, $\bar{A}$ и $\bar{B}$.
\end{to_lem}


\begin{to_def}
    События $A_1, \ldots, A_n$ называются \textit{независимыми в совокупности}, если для любого $1 \leq k \leq n$ и любого набора различных меж собой индекс $1 \leq i_1 < \ldots < i_k \leq n$ имеет место равенство
    \begin{equation*}
        \P(A_{i_1} \cap \ldots \cap A_{i_k}) = \P (A_{i_1}) \cdot \ldots \cdot \P(A_{i_k}).
    \end{equation*}
\end{to_def}