\subsection{Протокол BB84}
Пусть есть вертикальная и диагональная поляризация, а также 4 квантовых состояния


шифр Вернама
Протокол Диффи — Хеллмана
Алгоритм RSA

Классическая кирптография: 
    + изученность, стандартизированность
    - не выдерживает создание квантового компьютера

Квантовая криптография:
    + не ставит перехватчик перед вычислительными задачами
    - мало изучены, возможны атаки

Постквантовая криптография:
    + выдерживает существование квантового компьютера
    - недостаточная изученность, авось и классический может взломать


\subsection{Теория Информации}    
пусть $h(p)$ -- информационное содержание события вероятности $p$.
Верно следующее утверждение:
\begin{equation*}
    h(p_1) > h(p_2) \ \ \Leftarrow \ \ p_1 < p_2.
\end{equation*}
Также вполне логично предположить, что $h(1) = 0$, а также что $h(p_1 p_2) = h(p_1)+h(p_2)$. 

Это приводит к функции вида
\begin{equation*}
    h(x) = - \log x = \log_2 \frac{1}{x}
\end{equation*}

\textit{Распределением вероятностей} будем считать некоторый набор $\{p_i\}$ такой, что $\sum p_i = 1$.
Информация, выдаваемая источником может быть найдена, как \textit{матожидание} 
\begin{equation*}
    H(P) = - \sum_i p_i \log p_i,
\end{equation*}
иначе функция называется энтропией Шеннона, -- мера того, насколько неизвестно что выдаст источник. 

Также энтропия Шеннона -- среднее количество вопросов, которые необходимо задать. Ещё это среднее количество битов, которое необходимо, чтобы закодировать выход источника. 

Неравенство Крафта позволяет сформулировать условие к префиксному коду. 

Также можно сформулировать, что разность между практической и теоретической длиной слова $\geq 0$, что соответсвует неравенству Гиббса. 


\begin{itemize}
    \item Коды Хаффмана
    \item Maassen-Uffink entropic
\end{itemize}

\subsection{Измерения в базисе}

Возвращаемся к состояниям 
\begin{align*}
    \cqs{0}{+} &= \qs{0} = \begin{pmatrix}
        1 \\ 0
    \end{pmatrix} \\
    \cqs{1}{+} &= \qs{1} = \begin{pmatrix}
        0 \\ 1
    \end{pmatrix} \\
    \cqs{0}{\times} &= \frac{\qs{0}+\qs{1}}{\sqrt{2}} = \frac{1}{\sqrt{2}} \begin{pmatrix}
        1  \\ 1
    \end{pmatrix} \\
    \cqs{1}{\times} &= \frac{\qs{0}-\qs{1}}{\sqrt{2}} = \frac{1}{\sqrt{2}} \begin{pmatrix}
        1 \\ -1
    \end{pmatrix} \\
\end{align*}
Пусть есть некоторый ортонормированный базис $\{\qs{e_i}\}$ и  состояние $\qs{\xi}$. Вероятность исхода $i$ при измерении $\qs{\xi}$ в базисе $\{\qs{e_i}\}$ равна
\begin{equation*}
    \textnormal{Pr}\, (i) = | \langle e_i | \xi \rangle|^2.
\end{equation*}

