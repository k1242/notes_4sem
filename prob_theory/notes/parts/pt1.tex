Для начала подружимся с комбинаторикой, взяв некоторую её проекцию на теорвер

\begin{to_thr}[]
    Пусть множества $A = \{a_1, \ldots, a_k\}$ состоит из $k$ элементов, а множество $B = \{b_1, \ldots, b_m\}$ -- из $m$ элементов. Тогда можно образовать равно $k\cdot m$ пар $(a_i, b_j)$.
\end{to_thr}

\begin{to_thr}[]
    Общее количество различных наборов при выборе $k$ элементов из $n$ \textbf{без} возвращения и \textbf{с} учётом порядка равняется
    \begin{equation*}
        A_n^k = n \cdot (n-1) \cdot \ldots \cdot (n-k+1) = \frac{n!}{(n-k)!},
    \end{equation*}
    где $A_n^k$ называется \textit{числом размещений} из $n$ элементов по $k$ элементов. 
\end{to_thr}

\begin{to_thr}[]
    Общее количество различных наборов при выборе $k$ элементов из $n$ \textbf{без} возвращения и \textbf{без} учета порядка равняется
    \begin{equation*}
        C_n^k = \frac{A_n^k}{k!} = \frac{n!}{k! (n-k)!},
    \end{equation*}
    где число $C_n^k$ называется \textit{числом сочетаний} из $n$ элементов по $k$ элементов. 
\end{to_thr}

\green{
\begin{to_thr}[]
% вот тут не понял почему, но ладно
    Общее количество различных наборов при выборе $k$ элементов из $n$ с возвращением и без учёта порядка равняется
    \begin{equation*}
        C_{n+k-1}^k = C_{n+k-1}^{n-1}.
    \end{equation*}
\end{to_thr}
}