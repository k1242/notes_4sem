\begin{to_def}
    \textit{Пространством элементарных исходов} называют множество $\Omega$, содержащее все возможные взаимоисключающие результаты данного случайного эксперимента. Элементы множества $\Omega$ называются \textit{элементарными исходами} и обозначаются $\omega$.
\end{to_def}

\begin{to_def}
    \textit{Событиями} называются подмножества $\Omega$. Говорят, что \textit{произошло событие} $A$, если эксперимент завершился одним из элементарных исходов, входящих в множество $A$. 
\end{to_def}

Вообще в силу таких определений события и множества оказываются очень похожими, так что определены операции \textit{объединения}, \textit{пересечения}, \textit{дополнения}, а также взятия \textit{противоположеного} $\bar{A} = \Omega \backslash A$. Также можно выделить достоверное событие $\Omega$ и невозможное $\varnothing$. 

События $A$ и $B$ называются \textit{несовместными}, если они не могут произойти одновременно: $A \cap B = \varnothing$. События $A_1, \ldots, A_n$ называются \textit{попарно несовместными}, если несовместны любые два из них: $A_i \cap A_j = \varnothing, \ \forall i \neq j$. Говорят, что событие $A$ \textit{влечет} событие $B$ ($A \subseteq B$), если $A \Rightarrow B$. 

