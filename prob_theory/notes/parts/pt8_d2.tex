\begin{to_def}
    \textit{Производящей функцией моментов} случайной величины $\xi$ называют математическое ожидание случайной величины $e^{s \xi}$, где $s$ -- действительный параметр:
    \begin{equation}
        \psi_\xi (s) = \E (e^{s \xi}).
    \end{equation}
\end{to_def}

\begin{to_thr}[]
    Если случайная величина $\xi$ имеет начальный момент порядка $n$, то производящая функция $\psi_\xi (s)$ $n$ раз дифференцируема по $s$, и для всех $k \leq n$ выполняется соотношение
    \begin{equation}
        \nu_k = \psi^{(k)}_\xi (0).
    \end{equation}
\end{to_thr}

Действительно, разлагая функции моментов в ряд Маклорена, можно получить её разложение в ряд с начальными моментами
\begin{equation*}
    \psi_\xi (s) = 1 + \nu_1 s + \frac{\nu_2}{2!} s^2 + \ldots
\end{equation*}