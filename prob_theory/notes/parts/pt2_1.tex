\begin{to_def}
    Множество $\mathcal A$, элементами которого являются некоторые подмножества $\Omega$ называют \textit{алгеброй}, если оно удовлетворяет следующим условиям:
    \begin{itemize}
        \item[А1)] $\Omega \in \mathcal A$ 
        (алгебра содержит достоверные события);
        \item[А2)] если $A \in \mathcal A$, то $\bar{A} \in \mathcal A$ 
        (вместе с любым множеством алгебра содержит противоположное к нему);
        \item[А3)] если $A \in \mathcal A$ и $B \in \mathcal A$, то $A \cup B \in \mathcal A$ 
        (вместе с любыми двумя множествами алгебра содержит их объединение).
    \end{itemize}
\end{to_def}


Вообще из А1 и А2 следует, что $\varnothing = \bar{\Omega} \in \mathcal A$. Пункт А3 экстраполируется на любой конечный набор. Кстати, объединение можно заменить (в силу закона де Моргана) на пересечение:
\begin{equation*}
    xy \in \mathcal A
    \hspace{0.5cm} \Leftrightarrow \hspace{0.5cm}
    \overline{xy} \in \mathcal A
    \hspace{0.5cm} \Leftrightarrow \hspace{0.5cm}
    \overline{x} + \overline{y} \in \mathcal A.
\end{equation*}

\begin{to_thr}[закон де Моргана]
    Для множеств $x$, $y$ верно, что
    \begin{equation*}
        \overline{x + y} = \overline{x} \cdot \overline{y},
        \hspace{1 cm}
        \overline{xy} = \overline{x} + \overline{y},
    \end{equation*}
    где $xy = x \cap y$, $x + y = x \cup y$. 
\end{to_thr}

В случае счётного пространства элементарных исходов А3 алгебры оказывается недостаточно, так приходим к $\sigma$-алгебре:
\begin{to_def}
    Множество $\mathcal F$, элементами которого являются некоторые подмножества $\Omega$ называется $\sigma$-алгеброй, если выполнены следующий условия:
    \begin{itemize}
        \item[S1)] $\Omega \in \mathcal F$ 
        (алгебра содержит достоверные события);
        \item[S2)] если $A \in \mathcal F$, то $\bar{A} \in \mathcal F$ 
        (вместе с любым множеством алгебра содержит противоположное к нему);
        \item[S3)] если $\{A_i\} \in \mathcal F$, то $\cup_i A_i \in \mathcal F$  
        (вместе с любым \textit{счетным} набором событий $\sigma$-алгебра содержит их объединение).
    \end{itemize}
\end{to_def}

\begin{to_def}
    Минимальной $\sigma$-алгеброй, содержащей набор множеств $\mathcal U$, называется пересечение всех $\sigma$-алгебр, содержащих $\mathcal U$. 
\end{to_def}


\begin{to_def}
    Минимальная $\sigma$-алгебра, содержащая множество $\mathcal U$ всех интервалов на вещественной прямой называется \textit{борелевской} \textit{сигма}-\textit{алгеброй} в $\mathbb{R}$ и обозначается $\mathfrak B (\mathbb{R})$.
\end{to_def}

Итак, оказался определен специальный класс $\mathcal F$ подмножеств $\Omega$, названный $\sigma$-алгеброй событий. Применение счетного числа любых операция к множествам из $\mathcal F$ снова дает множество из $\mathcal F$. \textit{Событиями} будем называть только множества $A \in \mathcal F$.