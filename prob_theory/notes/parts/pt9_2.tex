\begin{to_def}
    \textit{Коэффициентом корреляции} $\rho (\xi, \eta)$ случайных величин $\xi$ и $\eta$, дисперсии которых существуют и отличны от нуля, называется число
    \begin{equation}
        \rho (\xi, \eta) = 
        \frac{\cov (\xi, \eta)}{\sqrt{\D \xi} \sqrt{\D \eta}}.
    \end{equation}
\end{to_def}

Можно наполнить это достаточно глубоким смыслом. На самом деле это <<косинус угла>> между двумя элементами $\xi - \E \xi$ и $\eta - \E \eta$ гильбертова пространства, образованного случайными величинами с нулевым матожиданием и конечным вторым моментом. Пространство набжено скалярным произведением $\cov (\xi, \eta)$ и <<нормой>>, равной корню из дисперсии, или $\sqrt{ \cov (\xi, \xi)}$.

\begin{to_thr}[]
    Коэффициент корреляции обладает свойствами:
    \begin{itemize}
        \item[1)] если $\xi$ и $\eta$ независимы, то $\rho(\xi, \eta) = 0$;
        \item[2)] всегда $|\rho(\xi, \eta)| \leq 1$;
        \item[3)] $|\rho(\xi, \eta)|=1$ тогда и только тогда, когда $\xi$ и $\eta$ почти наверное линейно связаны. 
    \end{itemize}
\end{to_thr}

\begin{to_def}
    \textit{Стандартизацией} случайной величины называется преобразование
    \begin{equation}
        \hat{\xi} = \frac{\xi - \E(\xi)}{\sqrt{\D (\xi)}}.
    \end{equation}
\end{to_def}

В терминах стандартизации чуть проще записывается коэффициент корреляции:
\begin{equation*}
    \rho (\xi, \eta) = \E \left(
        \hat{\xi} \cdot \hat{\eta}
    \right).
\end{equation*}

\begin{to_def}
    Говорят, что $\xi$ и $\eta$ \textit{отрицательно коррелированы}, если $\rho(\xi, \eta) < 0$; \textit{положительно коррелированы}, если $\rho (\xi, \eta) > 0$; \textit{некоррелированы}, если $\rho(\xi, \eta) = 0$. 
\end{to_def}

\begin{to_lem}
    Для любых случайных величин $\xi$ и $\eta$ с конечной и ненулевой дсперсией при любых постоянных $a \neq 0$ и $b$ имеет место равенство
    \begin{equation}
        \rho (\alpha \xi + b, \eta) = \sign(a) \cdot \rho (\xi, \eta).
    \end{equation}
\end{to_lem}


\noindent
\red{Разобрать пример 67 и далее.}