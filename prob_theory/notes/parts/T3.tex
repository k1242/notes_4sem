\subsection*{Т3}

Введем для каждого места величину $\xi_i$, равную $1$ в случае нечётного числа и $0$ иначе. Число, наверное, подразумевает, что на первом месте не может стоять ноль, но на всякий случай пока обозначим вероятность быть первой цифре нечетной за $\gamma$, остальных местах равновероятны значения 0 и 1.

Вероятность существования хотя бы одной нечётной цифры найдём через вероятность их отсутсвия:
\begin{equation*}
    \P(\exists a_i \in \textnormal{Odd}) = 1 - \left(\frac{1}{2}\right)^2 \cdot (1-\gamma).
\end{equation*}
Матожидание же величины $\xi = \sum_{i=1}^8 \xi_i$ легко найти, в силу независимости $\xi$:
\begin{equation*}
    \E(\xi) = \sum_{i=1}^8 \E(\xi_i) = 7 \E(\xi_i) + \gamma.
\end{equation*}

Если на первом месте может стоять $0$, то $\gamma = 0.5$ и, соответственно,
\begin{equation*}
    \P(\exists a_i \in \textnormal{Odd}) = \frac{255}{256} \approx 1 - 3.9 \cdot 10^{-3},
    \hspace{5 mm}
    \E(\xi) = 4.
\end{equation*}
Если же $0$ стоять на первом месте не может, то  $\gamma = 5/9$ и, соответственно,
\begin{equation*}
    \P(\exists a_i \in \textnormal{Odd}) = 1 - \frac{1}{128} \frac{5}{9} \approx 1 - 4.3 \cdot 10^{-3},
    \hspace{10 mm}
    \E(\xi) =  \frac{73}{18} \approx 4.06.
\end{equation*}