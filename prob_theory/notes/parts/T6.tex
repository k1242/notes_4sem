\subsection*{Т6}

Для поиска коэффициента корреляции сначала найдём дисперсию и матожидание количества людей, выходящих на 8 этаже, ($\xi$) и количества людей, выходящих на 8 этаже или выше ($\eta$).

Представим величину $\xi$ как сумму четырёх других $\xi = \sum_{i=1}^4 \xi_i$, где $\xi$ -- вероятность выйти на 8 этаже для каждого из четырех людей:
\begin{center}
    \begin{tabular}{c|c|c|}
        $\xi$ & $1$ & $0$  \\
        \hline
        $\P$ & $1/12$ & $11/12$ \\
    \end{tabular}
\end{center}
В силу независимости $\xi_i$ верно, что
\begin{equation*}
    \E(\xi) = \E\left(
        \sum_{i=1}^{4} \xi_i
    \right) = \sum_{i=1}^{4} \E(\xi_i) = \frac{4}{12},
    \hspace{5 mm}
    \D(\xi) = \D\left(
        \sum_{i=1}^{4} \xi_i
    \right) = \sum_{i=1}^{4} \D(\xi_i) = 4
    \left(\frac{1}{12} - \left(\frac{1}{12}\right)^2\right) = \frac{11}{36}.
\end{equation*}

Аналогично найдём характеристики $\eta$, представив через сумму независимых величин $\eta = \sum_{i=1}^{4} \eta_i$, где $\eta_i$ -- вероятность для каждого человека выйти на этаже, выше восьмого
\begin{center}
    \begin{tabular}{c|c|c|}
        $\eta$ & $1$ & $0$  \\
        \hline
        $\P$ & $5/12$ & $7/12$ \\
    \end{tabular}
\end{center}
Тогда, аналогично, в силу незаивимости $\eta_i$, находим
\begin{equation*}
    \E (\eta) = \sum_{i=1}^{4} \E(\eta_i) = 4 \cdot \frac{5}{12} = \frac{5}{3},
    \hspace{5 mm}
    \D (\eta) = \sum_{i=1}^{4} \D (\eta_i) = 4 \cdot \left(
        \frac{5}{12} - \frac{25}{144}
    \right) = \frac{35}{36}.
\end{equation*}

Теперь найдём матожидание $\E(\xi \eta)$, построив таблицу $\P(\xi, \eta)$, где $\xi$ и $\eta$ принимает значения от $0$ до $4$. Заметим, что таблица будет верхнетреугольной: если на $8$ этаже вышло $n$ людей, то $\eta \geq n$. Сформировать вероятность $\P(\xi = \xi_0, \eta = \eta_0)$ можно, выбирая $\xi_0$ людей из $4$ -- оставшихся на $8$ этаже, выбирая $\eta-\xi$ людей из $4-\xi$ -- оказавшихся на 9 этаже и выше, где вероятность оказаться ниже 8 -- $(7/12)$,  и вероятность быть на $9$ и выше -- $(5/12)$, итого находим
\begin{equation*}
    \P(\xi = \xi_0, \eta = \eta_0) = \begin{pmatrix}
        4 \\ \xi_0
    \end{pmatrix}
    \left(
        \frac{1}{12}
    \right)^{\xi_0} 
    \begin{pmatrix}
        4 - \xi_0 \\ \eta_0 - \xi_0
    \end{pmatrix}
    \left(
        \frac{7}{12}
    \right)^{4-\eta_0}
    \left(
        \frac{5}{12}
    \right)^{\eta_0-\xi_0},
\end{equation*}
а искомое матожидание тогда будет равно
\begin{equation*}
    \E (\xi \eta) =  \sum_{\xi_0 = 0}^{4} \sum_{\eta_0 = 0}^{4}
    \xi_0 \eta_0 \P(\xi=_{\xi_0}, \eta={\eta_0}) = \frac{3}{4},
\end{equation*}
что нетрудно получить прямым вычислением. Конечно, судя по простоте ответа, его можно было получить и более простым путём, но зато мы уверены в результатах. 

Наконец, корреляция $\xi$ и $\eta$, равна
\begin{equation*}
    \rho(\xi, \eta) = \frac{\cov(\xi, \eta)}{\sqrt{\D \xi} \sqrt{\D \eta}} = 
    \frac{\E(\xi \eta) - \E(\xi) \E(\eta)}{\sqrt{\D \xi} \sqrt{\D \eta}} = \sqrt{\frac{7}{55}} \approx 0.36.
\end{equation*}



