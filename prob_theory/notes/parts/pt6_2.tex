\textbf{Линейные и монотонные преобразования}.
Если с дискретными распределениями всё понятно, то с абсолютно непрерывными чуть интереснее,  о них дальше и поговорим. Пусть случайная величина $\xi$ имеет функцию распределения $F_\xi (x)$ и плотность распределения $f_\xi (x)$. Построим с помощью борелевской функции $g \colon \mathbb{R} \to \mathbb{R}$ случайную величину $\eta = g(\xi)$,и найдём плотность распределения (если она существует). 

\begin{to_thr}[]
    Пусть $\xi$ имеет функцию распределения $F_\xi (x)$ и плотность распределения $f_\xi (x)$, и постоянная $a$ отлична от нуля. Тогда случайная величина $\eta = a \xi + b$ имеет плотность распределения
    \begin{equation*}
        f_\eta (x) = \frac{1}{|a|} f_\xi \left(
            \frac{x-b}{a}
        \right).
    \end{equation*}
\end{to_thr}


\textbf{Квантильное преобразование}. Полезно уметь строить случайные величины с заданным распределением по равномерно распределенной случайной величине.

\begin{to_thr}[]
    Пусть функция распределения $F(x) = F_\xi (x)$ непрерывна. Тогда случайная величина $\eta = F(\xi)$ имеет равномерное на отрезке $[0, 1]$ распределение. 
\end{to_thr}

\begin{to_thr}[$\mathfrak{alarm}$]
    Пусть $\eta \in U_{0, 1}$, а $F$ -- произвольная функция распределения. Тогда случайная величина $\xi = F^{-1} (\eta)$ (\textbf{<<квантильное преобразование}>>  над $\eta$) имеет функцию распределения $F$. 
\end{to_thr}

\noindent
Как следствие,
для $\eta \in \textnormal{U}_{0, 1}$
,верны следующие утверждения:
\begin{equation*}
    - \frac{1}{\alpha} \ln (1 - \eta) \in \textnormal{E}_\alpha, 
    \hspace{5 mm}  
    a + \sigma \tg (\pi \eta - \pi / 2) \in \textnormal{C}_{\alpha, \sigma},
    \hspace{5 mm}
    \Phi_{0, 1}^{-1} (\eta) \in \textnormal{N}_{0, 1}.
\end{equation*}