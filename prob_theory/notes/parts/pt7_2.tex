Далее рассмотрим два типичных случая, когда совместное распределение либо дискретно, либо непрерывно.
\texttt{Сингулярное распределение не является редкостью: стоит выбрать отрезок на плоскости.} 

\begin{to_def}
    Случайные величины $\xi_1, \, \xi_2$ имеют \textit{дискретное} совместное распределение, если существует конечный или счётный набор пар числе $\{a_i, b_j\}$ такой, что
    \begin{equation*}
        \sum_{i=1}^{\infty} \sum_{j=1}^{\infty} \P (\xi_1 = a_i, \ \xi_2 = b_j) = 1.
    \end{equation*}
    Таблицу, на пересечении $i$-й строки и $j$-го столбца которых стоит $\P(\xi_1 = a_i, \ \xi_2 = b_j)$, называют таблицей совместного распределения случайных величин $\xi_1$ и $\xi_2$. 
\end{to_def}

\begin{to_def}
    Случайные величины $\xi_1$, $\xi_2$ имеют \textit{абсолютно непрерывное} совместное распредеение, если существует неотрицательная функция $f_{\xi_1, \xi_2} (x, y)$ такая, что для любого множества $B \in \mathfrak B (\mathbb{R}^2)$ имеет место равенство
    \begin{equation*}
         \P \left(
            (\xi_1, \xi_2) \in B
         \right) = \iint_B f_{\xi_1, \xi_2} (x, y) \d x \d y.
     \end{equation*} 
     Функция $f_{\xi_1, \xi_2} (x, y)$ называется \textit{плотностью совместного распределения} случайных величин $\xi_1$ и $\xi_2$.
\end{to_def}

Если случайные величины $\xi_1$ и $\xi_2$ имеют абсолютно непрерывное совместное распределение, то для любых $x_1$, $x_2$ имеет место равенство
\begin{equation*}
    F_{\xi_1, \xi_2} (x_1, x_2) = \P(\xi_1 < x_1, \ \xi_2 < x_2) = \int_{-\infty}^{x_1} \left(
        \int_{-\infty}^{x_2} f_{\xi_1, \xi_2} (x, y) \d y
    \right) \d x.
\end{equation*}
По функции совместного распределения его плотность находится как смешанная частная производная:
\begin{equation*}
    f_{\xi_1, \xi_2} (x, y) = \frac{\partial^2 }{\partial x \partial y} F_{\xi_1, \xi_2} (x, y)
\end{equation*}
для почти всех $(x, y)$. 

\begin{to_thr}[]
    Если случайные величины $\xi_1$ и $\xi_2$ имеютабсолютно непреывное совместное распределение с плотностью $f(x, y)$, то $\xi_1$ и $\xi_2$ в отдельности также имеют абсолютно непрерывное распределение с плотностями:
    \begin{equation*}
        f_{\xi_1} (x) = \int_{-\infty}^{+\infty} f(x, y) \d y;
        \hspace{10 mm}
        f_{\xi_2} (y) = \int_{-\infty}^{+\infty} f(x, y) \d x.
    \end{equation*}
\end{to_thr}