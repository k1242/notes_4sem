\begin{to_thr}[формула Байеса]
    Пусть $\{H_i\}$ -- полная группа событий, и $A$ -- некоторое событие, $\P (A) > 0$. Тогда условная вероятность того, что имело место событие $H_k$, если в результате эксперимента наблюдалось событие $A$, может быть вычислена по формуле
    \begin{equation}
        \P (H_k | A) = \frac{
        \P (H_k) \cdot \P (A | H_k)
        }{
        \sum\limits_{i=1}^{\infty} \P(H_i)  \cdot \P (A | H_i)
        }.
    \end{equation}
\end{to_thr}

\begin{to_def}
    Вероятности $\P (H_i)$, вычисленные заранее, до проведения эксперимента, называют\footnote{
        \textit{a'priori}  -- << до опыта >>.
    }  \textit{априорными} вероятностями. Условные вероятности $\P(H_i | A)$ называют\footnote{
        \textit{a'priori}  -- << после опыта >>.
    }  \textit{апостериорными} вероятностями.
\end{to_def}

Формула Байеса позволяет переоценить заранее известные вероятности после того, как получено знание о результате эксперимента. 
\texttt{Эта формула находит многочисленные применения в экономике, статистике, соци- логии и т.п}

