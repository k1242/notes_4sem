Далее всегда предполагается, что матожидание существует. 

(E1) Для $\forall$ борелевской $g \colon  \mathbb{R} \mapsto \mathbb{R}$, для дискретного и непрерывного распределения, при существующем $\E$:
\begin{equation*}
    \E g(\xi) = \sum_k g(a_k) \P(\xi = a_k),
    \hspace{10 mm}
    \E g(\xi) = \int_{-\infty}^{+\infty} g(x) f_\xi (x) \d x.
\end{equation*}

(E3) Матожидание линейно по константам: $\E (c \xi) = c \E (\xi)$. 

(E4) Матожидание суммы \textit{любых} случайных величин равно сумме их матожиданий: $\E (\xi + \eta) = \E (\xi) + \E (\eta)$.

(E7) Если $\xi$ и $\eta$ независимы и их матожидания существуют, то $\E (\xi \eta) = \E(\xi) \cdot \E(\eta)$.











