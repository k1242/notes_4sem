\begin{to_def}
    Конечный или счётный набор попарно несовместных событий $\{H_i\}$ таких, что $\P(H_i)>0$ $\forall i$ и 
    $\cup_i H_i = \Omega$, называется \textit{полной группой событий} или разбиением пространства $\Omega$. 
    Также события, образующие полную группу событий, часто называют \textit{гипотезами}.
\end{to_def}

При подходящем выборе гипотез для любого события $A$ могут быть сравнительно просто вычислены $\P(A | H_i)$ и, собственно, $\P(H_i)$. Как посчитать вероятность события $A$?

\begin{to_thr}[формула полной вероятности]
    Пусть дана полная группа событий $\{H_i\}$. Тогда вероятность любого события $A$ может быть вычислена по формуле
    \begin{equation*}
        \P(A) = \sum_{i=1}^{\infty} \P(H_i) \cdot \P(A | H_i).
    \end{equation*}
\end{to_thr}