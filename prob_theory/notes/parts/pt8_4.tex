Во всех свойствах ниже предполагается существование вторых моментов случайных величин.

(D1) 
Дисперсия может быть вычислена по формуле $\D \xi = \E \xi^2 - (\E \xi)^2$.
\begin{equation*}
    \D \xi = \E (\xi - E \xi)^2 = \bigg/
        a = \E \xi
    \bigg/ = \E (\xi^2) - 2 a \E \xi + a^2 = \E \xi^2 - (\E \xi)^2.
\end{equation*}

(D2)
Считая $c$ константной: $\D (c \xi) = c^2 \D \xi$. 

(D3)
Дисперсия нетрицательна: $\D \xi \geq 0$, более того обращается в ноль, только при $\xi = \const$ почти наверное. 

(D4)
$\D (\xi + c) = \D \xi$.

(D5)
Если $\xi$ и $\eta$ независимы, то $\D (\xi + \eta) = \D \xi + \D \eta$. Вообще верна формула
\begin{equation}
    D(\xi + \eta) = \D \xi + \D \eta + 2 (\E (\xi \eta) - \E \xi \E \eta).
\end{equation}

(D6)
Минимум среднеквадратичного отклонения $\xi$ от точек числовой прямой есть $\D \xi$:
\begin{equation*}
    \D \xi = \E (\xi - \E \xi)^2 = \min_a \E (\xi - a)^2.
\end{equation*}