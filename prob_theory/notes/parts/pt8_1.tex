Далее будет использовать термин \textit{математического ожидания}, и также можно встретить наименования: \textit{среднее значение}, \textit{первый момент}. 

\begin{to_def}
    \textit{Математическим ожиданием} 
    $\E(\xi)$ случайной величины $\xi$ с дискретным распределением называется \textit{число}
    \begin{equation*}
        \E (\xi) = \sum_k a_k p_k = \sum_k a_k \P (\xi = a_k),
    \end{equation*}
    если данный ряд абсолютно сходится, т.е. если $\sum_i |a_i| \, p_i < +\infty$. В противном случае говорят, что математическое ожидание \textit{не существует}. 
\end{to_def}


\begin{to_def}
    \textit{Математическим ожиданием} $\E (\xi)$ случайное величины $\xi$ c абсолютно непрерывным распределением с плотностью распределения $f_\xi (x)$ называется \textit{число}
    \begin{equation*}
        \E (\xi) = \int_{-\infty}^{+\infty} 
        x f_\xi (x) \d x,
    \end{equation*}
    если этот интеграл абсолютно сходится, т.е. если $\int |x| f_\xi (x) \d x < + \infty$. 
\end{to_def}

