\textbf{Вырожденное распределение}. 
Для удобства вводят \textit{вырожденное распределение}, когда возможен единственный результат при $\P (\xi = c)=1$, тогда функция распрееления имеет вид
\begin{equation*}
    F_\xi (x) = \P (\xi < x) = \P(c < x) = \left\{\begin{aligned}
        &0, &x \leq x, \\
        &1, &x > c.
    \end{aligned}\right.
\end{equation*}
В таком случае принято писать, что $\xi \in \textnormal{I}_c$.


\textbf{Распределение Бернулли}.
Говорят про \textit{распределение Бернулли} с параметром $p$ ($\xi \in \textnormal{B}_p$), если $\xi$ принимает значения $1$ и $0$ с вероятностью $p$ и $1-p$ соответственно. Случайная величина $\xi$ с таким распределением равна \textit{числу упехов} в одном испытании схемы Бернулли с вероятностью успеха $p$. Функция распредления случайной величины $\xi$ тогда равна
\begin{equation*}
    F_\xi (x) = P(\xi < x) = \left\{\begin{aligned}
        &0, &x \leq 0, \\
        &1-p, &0 < x \leq 1, \\
        &1, &x > 1.
    \end{aligned}\right.
\end{equation*}


\textbf{Биномиальое распределение}. 
Говорят, что случайная величина $\xi$ имеет биномиальное распределение с параметрами $n \in \mathbb{N}$ и $p \on (0, 1)$, и пишут $\xi \in \textnormal{B}_{n,p}$, если $\xi$ принимает значения $k = 0, \ldots, n$ с вероятностями $\P (\xi = k) = C_n^k p^k (1-p)^{n-k}$. Случайная величиная с таким распределением имеет смысл \textit{числа успехов} в $n$ исыпытаниях схемы Бернулли с вероятностью успеха $p$. 

\textbf{Геометрическое распределение}.
Говорят, что случайная величина $\tau$ имеет геометрическое распределение с параметром $p \in (0, 1)$, и пишут $\tau \in \textnormal{G}_p$, если $\tau$ принимает значения $k =1, 2, 3, \ldots$ с вероятностями $\P(\tau = k) = p (1-p)^{k-1}$. 
Случайная величина с таким распределением имеет смысл номера первого успешного испытания в схеме Бернулли с вероятностью успеха p.



\textbf{Распрееление Пуассона}. Говорят, что случайная величина $\xi$ имеет распределение Пуассона с параметром $\lambda > 0$, и пишут $\xi \in \Pi_\lambda$, если $\xi$ принимает значения $k = 0, 1, \ldots$ с вероятностью $\P(\xi = k) = \frac{\lambda^k}{k!} e^{-\lambda}$. Иначе распределение Пуассона называют \textit{распределением числа редких событий}. 


\textbf{Гипергеметрическое распределение}. Говорят, что случайная величина $\xi$ имеет гипергеометрическое распределение с параметрами $N$, $n \leq N$ и $K \leq N$, если $\xi$ принимает целые значения $k$ такие, что $0 \leq k \leq K$, $0 \leq n- k \leq N _ K$, с вероятностями $\P(\xi = k) = C_K^k C_{N_K}^{n-k} / C_N^n$. 
Случайная величина с таким распределением имеет смысл числа белых шаров среди $n$ шаров, выбранных наудачу и без возвращения из урны, содержащей $K$ белых и $N-K$ не белых.


