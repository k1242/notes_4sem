\begin{to_lem}
    Для любого $x \in \mathbb{R}$ справедливо соотношение:
    \begin{equation*}
        \Phi_{a, \sigma^2} (x) = \Phi_{0, 1} \left(
            \frac{x-a}{\sigma}
        \right).
    \end{equation*}
\end{to_lem}


Аналогичное утверждение для случайных величичн: если $\xi \in \textnormal{N}_{a, \sigma^2}$, то $\eta = \frac{\xi-a}{\sigma} \in \textnormal{N}_{0, 1}$. Более того, если $\xi \in \textnormal{N}_{a, \sigma^2}$, то
\begin{equation*}
    \P (x_1 < \xi < x_2) = \Phi_{a, \sigma^2} (x_2) - \Phi_{a, \sigma^2} (x_2) = \Phi_{0, 1} \left(
        \frac{x_2 - a}{\sigma}
    \right) - \Phi_{0, 1} \left(
        \frac{x_1 - a}{\sigma}
    \right).
\end{equation*}
В общем вычисления любых вероятностей для нормального распределения сводятся к вычислению $\Phi_{0, 1} (x)$, которое обладает следующими свойствами:
\begin{itemize}
    \item $\Phi_{0, 1} (0) = 0.5, \hspace{5 mm} \Phi_{0, 1} (-x) = 1 - \Phi_{0, 1} (x)$.
    \item Если $\xi \in \textnormal{N}_{0, 1}$, то для любого $x > 0$, верно что
    $
        \P (|\xi| < x) = 1 - 2 \Phi_{0, 1} (-x) = 2 \Phi_{0, 1} (x) - 1.
    $
\end{itemize}

