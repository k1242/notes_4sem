Пусть $\xi_1$ и $\xi_2$ -- случайные величины с плотностью совместного распределения $f_{\xi_1, \xi_2} (x_1, x_2)$, и задана борелевская функция $g \colon  \mathbb{R}^2 \mapsto \mathbb{R}$. Требуется найти функцию (и плотность, если повезет) распределения случайной величины $\eta = g(\xi_1, \xi_2)$. 

\begin{to_thr}[]
    Пусть $x \in \mathbb{R}$, и область $D_x \subseteq \mathbb{R}^2$ состоит из точек $(u, v)$ таких, что $g(u, v) < x$. Тогда случайная величина $\eta = g(\xi_1, \xi_2)$ имеет функцию распределения
    \begin{equation*}
        F_\eta (x) = \P (g(\xi_1, \xi_2) < x) = \P \left((\xi_1, \xi_2) \in D_x\right) = \iint_{D_x} f_{\xi_1, \xi_2} (u, v) \d u \d v.
    \end{equation*}
\end{to_thr}

Если $\xi_1$ и $\xi_2$ независимы, то распределение $g(\xi_1, \xi_2)$ полностью определяется частными распределениями величин $\xi_1$ и $\xi_2$.

\begin{to_thr}[формула свёртки]
    Если случайные величины $\xi_1$ и $\xi_2$ независимы и имеют абсолютно непрерывные распределения с плотностями $f_{\xi_1} (u)$ и $f_{\xi_2} (v)$, то плотность распределения суммы $\xi_1 + \xi_2$ существует и равна <<свёртке>> плотностей $f_{\xi_1}$ и $f_{\xi_2}$:
    \begin{equation}
        f_{\xi_1 + \xi_2} (t) = \int_{-\infty}^{+\infty} f_{\xi_1} (u) f_{\xi_2} (t-u) \d u =
        \int_{-\infty}^{+\infty}  f_{\xi_2} (u) f_{\xi_1} (t-u) \d u.
    \end{equation}
\end{to_thr}