\begin{to_def}
    \textit{Схемой Бернулли} называется последовательность независимых в совокупности испытаний, в каждом из 
    которых возможны лишь два исхода -- <<успех>> и <<неудача>>, при этом успех \cmark в одном испытании происходит с вероятностью $p \in (0, 1)$, а неудача \xmark -- с вероятностью $q = 1 - p$.
\end{to_def}

В испытаниях схемы Бернулли  независимость в совокупности испытаний означает, что при любом $n$ независимы в совокупности события успехов в каждом событие. 

Эти события принадлежат одному и тому же пространству элементарных исходов, полученному декартовым произведением бесконечного числа двухэлементных множеств $\{\cmark, \xmark\}$:
\begin{equation*}
    \Omega = \{(a_1, a_2, \ldots, a_n) \mid a_i \in \{\cmark, \xmark\}, n \in \mathbb{Z}_+\}.
\end{equation*}
Далее количество успехов для $n$ испытаний схемы Бернулли будем называть $\nu_n$. Заметим, что $\nu_n \in \mathbb{Z}_+ \cap [0, n]$. 

\begin{to_thr}[формула Бенулли]
    При любом $k = 0, 1, \ldots, n$ имеет место равенство:
    \begin{equation*}
        \P (\nu_n = k) = C_n^k p^k q^{n-k}.
    \end{equation*}
\end{to_thr}

\begin{to_def}[$\mathfrak D$]
    Набор чисел $\{C-n^k p^k q^{n-k}, \ k = 0, 1, \ldots, n\}$ называется \textit{биномиальным} распределением.
\end{to_def}

