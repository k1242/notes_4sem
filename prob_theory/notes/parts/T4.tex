\subsection*{Т4}


Найдём производящую функция для биномиального распределения, вида
\begin{equation*}
    \P(\xi = k) = C_n^k p^k q^{n-k},
\end{equation*}
что соответствует количеству успехов в схеме Бернулли, где вероятность успеха $p$.

Коэффициенты в биноме Ньютона выглядят очень похоже на $\P$, так что заметим, что производящая функция вида
\begin{equation*}
    P(s) = (q + ps)^n,
\end{equation*}
нам подходит. 
Найдём матожидание и дисперсию, как
\begin{equation*}
    \E(\xi) = P'_s (s= 1),
    \hspace{10 mm}
    \D(\xi) = P''(1) + P'(1) - \E^2(\xi).
\end{equation*}
Производные $P(s)$:
\begin{equation*}
    P'(s) = n p (q + ps)^{n-1}, \hspace{5 mm}  
    P'(1) = n p,
    \hspace{10 mm}
    P''(s) = n (n-1) p^2 (q+ps)^{n-1},
    \hspace{5 mm}
    P''(1) = n (n-1) p^2,
\end{equation*}
тогда искомые величины:
\begin{equation*}
    \E(\xi) = np,
    \hspace{10 mm}
    \D(\xi) = np(1-p) = npq.
\end{equation*}