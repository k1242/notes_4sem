\subsection*{Т7}

Так как доска небольшая, то, имея калькулятор, ничего в принципе не мешает просто посчитать количество доступных путей:
\begin{table}[h]
\centering
% \begin{tabular}{lllllllllllllll}
\begin{tabular}{rrrrrrrrrrrrrrr}
% \midrule
\toprule
  1 &   1 &   1 &   1 &   1 &    1 &    1 &    1 &     1 &     1 &     1 &     1 &      1 &      1 &      1 \\
  1 &   2 &   3 &   4 &   5 &    6 &    7 &    8 &     9 &    10 &    11 &    12 &     13 &     14 &     15 \\
  1 &   3 &   6 &  10 &  15 &   21 &   28 &   36 &    45 &    55 &    66 &    78 &     91 &    105 &    120 \\
  1 &   4 &  10 &  20 &  35 &   56 &   84 &  120 &   165 &   220 &   286 &   364 &    455 &    560 &    680 \\
  1 &   5 &  15 &  35 &  70 &  126 &  210 &  330 &   495 &   715 &  1001 &  1365 &   1820 &   2380 &   3060 \\
  1 &   6 &  21 &  56 & 126 &  252 &  462 &  792 &  1287 &  2002 &  3003 &  4368 &   6188 &   8568 &  11628 \\
  1 &   7 &  28 &  84 & 210 &  462 &  924 & 1716 &  3003 &  5005 &  8008 & 12376 &  18564 &  27132 &  38760 \\
  1 &   8 &  36 & 120 & 330 &  792 & 1716 & 3432 &  6435 & 11440 & 19448 & 31824 &  50388 &  77520 & 116280 \\
  1 &   9 &  45 & 165 & 495 & 1287 & 3003 & 6435 & 12870 & 24310 & 43758 & 75582 & 125970 & 203490 & 319770 \\
\bottomrule
\end{tabular}
\caption{Заполненная количеством доступных путей доска к задаче №Т7}
\end{table}

\noindent
Расчёт происходит из предположения о том, что у количество путей $N[i, j]$ равно
\begin{equation*}
    N[i, j] = N[i-1, j] + N[i, j-1],
\end{equation*}
а левый столбей и верхняя строка <<заполнены>> единицами -- существует единственный способ добраться до этой клетки.

Если нас интересует движение такое, что последние три клетки были сделаны по короткой стороне доски, то вероятность такого маршрута:
\begin{equation*}
    \P_0 = \frac{11628}{319770} = \frac{2}{55} \approx 3.64 \cdot 10^{-2}.
\end{equation*}


Вообще можно заметить в числах биномиальные коэффициенты -- действительно, достигая\footnote{
    Формулы удобнее выглядят, когда $i$ и $j$ нумеруются с $0$. 
}  $[i, j]$ клетки, мы делаем $i$ шагов вправо и $j$ вниз, то есть необходимо в $i+j$ элементах выбрать $i$ элементов (или $j$), тогда искомая вероятность
\begin{equation*}
    P_0  = \begin{pmatrix}
        14 + 8 \\ 8
    \end{pmatrix}
    \bigg/ 
    \begin{pmatrix}
        14 + 5 \\ 5
    \end{pmatrix} = \frac{2}{55},
\end{equation*}
что сходится с прямым вычислением.