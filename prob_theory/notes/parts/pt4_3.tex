Теперь рассмотрим схему независимых испытаний независимых испытаний уже не с двумя, а с болбшим количество возможных результатов в каждом испытании. 

Пусть возможны $m$ исходов, $i$-й исход в одном испытании случается с вероятностью $p_i$, где $\sum_i p_i = 1$. Через $\P(n_1, \ldots, n_m)$ обозначим вероятность того, что в $n$ независимых испытаниях первый исход случится $n_1$ раз, \ldots, $m$-исход -- $n_m$ раз.

\green{
\begin{to_thr}
    Для любого $n$ и любых неотрицательных целых чисел $\{n_i\}$, сумма которых равна $n$, верна формула
    \begin{equation*}
        \P(n_1, \ldots, n_m) = \frac{n!}{n_1! \ldots n_m!} p_1^{n_1} \cdot \ldots \cdot p_m^{n_m}.
    \end{equation*}
\end{to_thr}
}

\begin{to_def}[$\mathfrak D$]
    Набор чисел
    \begin{equation*}
         \left\{
         \frac{n!}{n_1! \ldots n_m!} p_1^{n_1} \cdot \ldots \cdot p_m^{n_m}
         \ \bigg| \ n = 1, 2,  \ldots\right\}
    \end{equation*} 
    называется \textit{мультиномиальным (полиномиaльным)} распределением.
\end{to_def}

