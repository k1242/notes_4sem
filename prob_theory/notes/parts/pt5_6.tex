\textbf{Общие свойства функций распределения}. Функцией распределения случайной величины $\xi$ мы назвали функцию $F_\xi (x) = \P (\xi < x)$. 

\begin{to_thr}
    Любая функция распределения обладает свойствами
    \begin{itemize}
        \item[F1)] она не убвает;
        \item[F2)] в прелелах $x \to - \infty$, и $x \to + \infty$ равна $0$ и $1$ соответственно;
        \item[F3)] она в любой точке непрерывна слева.
    \end{itemize}
\end{to_thr}


\begin{to_thr}[]
    Если функция $F \colon \mathbb{R} \mapsto [0, 1]$ удовлетворяет свойствам (F1)-(F3), то $F$ есть функция распределения некоторой случайной величины $\xi$, т.е. найдётся вероятностное пространство $\langle \Omega, \mathcal F, \P \rangle$ и случайная величина $\xi$ на нём такая, что $F(x) \equiv F_\xi (x)$. 
\end{to_thr}

\begin{to_lem}
     любой точке $x_0$ разница $F_\xi (x_0 + 0) - F_{\xi} (x_0)$ равна $\P (\xi = x_0)$:
     \begin{equation*}
          F_\xi (x_0 + 0) = F_\xi (x_0) + \P (\xi = x_0) = P (\xi \leq x_0).
      \end{equation*} 
\end{to_lem}

\begin{to_lem}
    Для любой случайной величины $\xi$
    \begin{equation*}
        \P (a \leq \xi < b) = F_\xi (b) - F_\xi (a).
    \end{equation*}
\end{to_lem}




\textbf{Функция распределения дискретного распределения}. Как мы поним, функция распределения может быть найдена по талице распределения, как сумма $F_\xi (x) = \P( \xi < x) = \sum_k \P(\xi = a_k)$, где $a_k < x$.

\begin{to_lem}
    Случайная величина $\xi$ имеет дискретное распределение тогда и только тогда, когда функция распределения $F_\xi (x)$ имеет в точках $a_i$ скачки с величиной $p_i = \P(\xi = a_i) = F_\xi (a_i +0) - F_\xi (a_i)$, и растёт только за счёт скачков.
\end{to_lem}


\textbf{Свойства абсолютно непрерывного распределения}. Пусть слу-
чайная величина $\xi$ имеет абсолюлютно непрерывное распределение с плотностью $f_\xi (t)$. Тогда функция распределения может быть найдена, как интеград. 

\begin{to_lem}
    Если случайная величина $\xi$ имеет абсолютно непрерывное распределение, то её функция распределения всюду непрерывна. Более того её функция распределенеия дифференцируема почти всюду: $f_\xi (x) = F'_\xi (x) = d_x F_\xi (x)$. 
\end{to_lem}


\red{
\textbf{Функция распределения сингулярного распределения}.
}

\red{
\textbf{Функция распределения смешанного распределения}.
}