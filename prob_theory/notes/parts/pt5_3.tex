Хотелось бы найти некоторый универсальный способ  для описания распределения.

\begin{to_def}
    \textit{Функцией распределения} случайной величины $\xi$ называется функция $F_\xi \colon  \mathbb{R} \mapsto [0, 1]$, при каждом $x \in \mathbb{R}$ равная вероятности случайной величине $\xi$ принимать значения, меньшие $x$:
    \begin{equation*}
        F_\xi (x) = \P (\xi < x) = \P \{\omega \mid \xi (\omega) < x\}.
    \end{equation*}
\end{to_def}

Далее перечислены основные дискретные и абсолютно непрерывные распределения и найдены их функции распределения.