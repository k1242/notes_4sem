Пусть задано вероятностное пространство $\langle \Omega, \mathcal F, \P \rangle$.

\begin{to_def}
    Функция $\xi \colon  \Omega \mapsto \mathbb{R}$ называется \textit{случайное величиной}, если для любого борелевского множества $B \in \mathfrak B (\mathbb{R})$ множество $\xi^{-1} (B)$ является событием, т.е принадлежит $\sigma$-алгебре $\mathcal F$. 
\end{to_def}

Множество $\xi^{-1} (B) = \{ \omega \mid \xi(\omega) \in B\}$, состоящее из элементарных исходов $\omega$, называется \textit{полным прообразом множества} $B$. Можно немного другим способом сформулировать требования к величине:

\begin{to_def}
    Функция $\xi \colon  \Omega \mapsto \mathbb{R}$ называется случайной величиной, если для любых веществнных $a < b$ множество $$\{\omega \colon  \xi(\omega) \in (a, b)\} \in \mathcal F$$
    принадлежит $\sigma$-алгебре.
\end{to_def}