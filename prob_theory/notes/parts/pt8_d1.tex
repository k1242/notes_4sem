Дискретные величины, рассмотренные раннее, принимают только целые значения $X = 0, 1, \ldots$. Нахождение числовых характеристик упрощается, если рассматреть \textit{производящие функции}.

\begin{to_def}
    \textit{Производящей функцией} дискретной целочисленной случайной величины $\xi$ с законом распределения
    $\P (\xi = k) = p_k$ , где $k = 0, 1, \ldots$ называется функция, заданная степенным рядом
    \begin{equation}
        \E(s^\xi) = P (s) = p_0 + p_1 s + p_2 s^2 + \ldots,
    \end{equation}
    который сходится по крайней мере для $|s| \leq 1$. 
\end{to_def}

\begin{to_thr}[]
    Производящая функция суммы независимых случайных величин $\xi$ и $\eta$ равна произведению производящих функций слагаемых
    \begin{equation}
        P_{\xi + \eta} (s) = P_\xi (s) \cdot P_\eta (s).
    \end{equation}
\end{to_thr}

Так например для биномального распределения производящая функция примет вид
\begin{equation*}
    P (s) = (q + ps)^n.
\end{equation*}
А для геометрического закона распределения
\begin{equation*}
    \P (s) = p s + p q s^2 + p q^2 s^3 + \ldots = \frac{ps}{1-qs}.
\end{equation*}
В случае же Пуассона
\begin{equation*}
    P (s) = \sum_{k=0}^{\infty} \frac{\lambda^k}{k!}e^{-\lambda} s^k = e^{-\lambda} \sum_{k=0}^\infty 
    \frac{(\lambda s)^k}{k!} = e^{-\lambda} e^{\lambda s} = e^{\lambda (s-1)}.
\end{equation*}

\begin{to_thr}
    Сумма независимых случайных величин, распределенных по закону Пуассона, распределена по тому же закону.
\end{to_thr}

\begin{to_thr}[]
    Для дискретной случайной величины $\xi$ с производящей функцией $\P (s)$ выполняются следующие требования:
    \begin{equation}
        \E (\xi) = P'_s (1),
        \hspace{10 mm}
        \D (\xi) = P''_{s,s} (1) + P'_s (1) - [P'_{s} (1)]^2.
    \end{equation}
\end{to_thr}


