(Ф1). Характеристическая функция всегда существует: $|\varphi_\xi (t) | = |\E e^{i t \xi} | \leq 1$.

(Ф2). По харакетристической функции однозначно восстанавливается распределение. Например, если модуль характеристической функции интегрируем на всей прямой, то 
\begin{equation*}
    f_\xi (x) = \frac{1}{2\pi} \int_{-\infty}^{+\infty} e^{-itx} \varphi_x (t) \d t.
\end{equation*}

(Ф3). Характерестическая функция случайной величины $a + b \xi$ связана с характеристической функцией случайной величины $\xi$ равенством
\begin{equation*}
    \varphi_{a + b \xi} (t) = \E e^{i t (a + b \xi)} = e^{i t a} \E \left(
        i (tb) \xi
    \right) = e^{i t a} \varphi_\xi (tb).
\end{equation*}

(Ф4). Характеристическая функция суммы независимых случайных величин равна произведению характеричтических функций слагаемых: если случайные величины $\xi$ и $\eta$ независимы, то
\begin{equation*}
    \varphi_{\xi + \eta} (t) = \E e^{i t (\xi + \eta)} = \E (e^{i t \xi}) \E (e^{i t \eta}) = \varphi_\xi (t) \varphi_\eta (t).
\end{equation*}
\texttt{Собственно, это очень простой и приятный инструмент для доказательства \textit{устойчивости} распределений}. \red{Чем надо было бы и воспользоваться}.

(Ф5). Пусть существует момент порядка $k \in \mathbb{N}$ случайной величины $\xi$. Тогда характеристическая функция $\varphi_\xi (t)$ непрерывно дифференцируема $k$ раз и её $k$-я производная в \textit{нуле} связана с моментом порядка $k$ равенством
\begin{equation*}
    \varphi_\xi^{(k)} (0) = \left(
        \frac{d^k }{d t^k} \E e^{i t \xi}
    \right) \bigg|_{t=0} = \left(
        \E i^k \xi^k e^{i t \xi}
    \right) \bigg|_{t=0} = i^k \E(\xi^k).
\end{equation*} 

\begin{to_lem}
    Для случайной величины $\xi$ со стандартным нормальным распределением момент чёного порядка $2k$ равен
    \begin{equation*}
        \E (\xi^{2k}) = (2 k - 1) !! = (2k -1) \cdot (2k-3) \cdot \ldots \cdot 3 \cdot 1.
    \end{equation*}
    Все моменты нечётных порядков существуют и равны нулю.
\end{to_lem}

\texttt{Как только появились производные высших порядков, самое время разложить функцию в ряд Тейлора:} 

(Ф6). Пусть существует момент порядка $k \in \mathbb{N}$ случайной величина $\xi$, тогда характеричтическая функция $\varphi_\xi (t)$ в окрестности точки $t = 0$ разлагается в ряд Тейлора
\begin{equation*}
    \varphi_\xi (t) = \varphi_\xi (0) + \sum_{j=1}^{k} \frac{t^j}{j!} \varphi_\xi^{(j)} (0) + o(|t^k|) 
    = 
    1 + \sum_{j=1}^{k} \frac{i^j t^j}{j!} \E (\xi^j) + o(|t^k|).
\end{equation*}

\begin{to_thr}[теорема о непрерывно соответствии]
    Случайные величины $\xi_n$ слабо сходятся к случайной величине $\xi$ тогда и только тогда, когда для любого $t$ характеристические функции $\varphi_{\xi-b} (t)$ сходятся к  характеристической функции $\varphi_\xi (t)$. 
\end{to_thr}