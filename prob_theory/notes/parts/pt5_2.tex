\begin{to_def}
    \textit{Распределением} случайной величины $\xi$ называется \blue{вероятностная мера} $\mu(B) = \P(\xi \in B)$ на множестве борелевских подмножеств $\mathbb{R}$.
\end{to_def}

Можно представить себе распределение случайной величины $\xi$ как соответствие между множествами $B \in \mathfrak B (\mathbb{R})$ и вероятностями $\P(\xi \in B)$.

\begin{to_def}
    Если две функции $\xi$ и $\eta$  отличаются на множестве меры нуль, при этом имеют одинаковое распределение, то говорят, что $\xi$ и $\eta$ совпадают \textit{почти наверное}: $\P (\xi = \eta) =1$.
\end{to_def}

\begin{to_def}
    Случайная велчина $\xi$ имеет \textit{дискретное} распределение, если существует конечный, или счётный набор чисел $\{a_i\}$ такой, что
    \begin{equation*}
        \P(\xi = a_i) > 0 \ \ \ \forall i, \hspace{10 mm}
        \sum_{i=1}^{\infty} \P (\xi = \alpha_i) = 1.
    \end{equation*}
    Значения эти называют \textit{атомами}: $\xi$ имеет атом в точке $x$, если $\P(\xi = x) > 0$.
\end{to_def}

Если случайная величина $\xi$ имеет дискретное распределение, то для любого $B \subseteq \mathbb{R}$
\begin{equation*}
    \P(\xi \in B) = \sum_{a_i \in B} \P (\xi = a_i).
\end{equation*}
Вообще дискретные распредления удобно задавать вероятностной таблицей
\begin{center}
    \begin{tabular}{c|c|c|c|c}
        $\xi$ & $a_1$ & $a_2$ & $a_3$ & $\ldots$ \\
        \hline
        $\P$ & $p_1$ & $p_2$ & $p_3$ & $\ldots$ \\
    \end{tabular}
\end{center}

\begin{to_def}
    Случайная величина $\xi$ имеет \textit{абсолютно непрерывно} распределение, если существует неотрицательная функция $f_\xi (x)$ такая, что для любого борелевского множества $B$ имеет место равенство:
    \begin{equation*}
        \P (\xi \in B) = \int_B f_\xi (x) \d x.
    \end{equation*}
    Функцию $f_\xi (x)$ называют \textit{плотностью распределения} величины $\xi$.
\end{to_def}

\begin{to_thr}[]
    Плотность распределения обладает свойствами:
    \begin{equation*}
        \textnormal{(f1)} \hspace{5 mm} f_\xi (x) \geq 0 \ \ \forall x, 
        \hspace{10 mm}
        \textnormal{(f2)} \hspace{5 mm} \int_{-\infty}^{+\infty} f_\xi (t) \d t = 1
        .
    \end{equation*}
\end{to_thr}

\begin{to_thr}
    Если функция $f$ обладает свойствами \textnormal{(f1)} и \textnormal{(f2)}, то существует вероятностное пространство и случаяная величина $\xi$ на нём, для которой $f$ является плотностью распределения.
\end{to_thr}

Ещё бывает сингулярное распределение\footnote{
    На континуальном множестве меры нуль.
}, смешанные варианты, и всё (\textit{Лебег} \textit{approved}).