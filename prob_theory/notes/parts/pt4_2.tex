Далее, для схемы Бернулли, введем величину $\tau \in \mathbb{Z}_+ \cap [1, +\infty)$ равную номеру перого успешного испытания. 

\begin{to_thr}[]
    Вероятность того, что первый успех произойдёт в испытании с номером $k \in \mathbb{N} \cap [1, +\infty)$, равна
    \begin{equation*}
        \P(\tau = k) = p q^{k-1}.
    \end{equation*}
\end{to_thr}

\begin{to_def}[$\mathfrak D$]
    Набор чисел $\{p q^{k-1} \mid k = 1, 2, \ldots\}$ называется \textit{геометрическим} распределением вероятностей.
\end{to_def}

\green{
\begin{to_thr}[<<Нестарение>> геометрического распределения]
    Пусть $\P(\tau = k) = p q^{k-1}$ $\forall k \in \mathbb{N}$. Тогда для любых неотрицательных целых $n$ и $k$ имеет место равенство:
    \begin{equation*}
        \P(\tau > n + k \mid \tau > n) = \P(\tau > k).
    \end{equation*}
    Другими название -- свойство отсутствия последствия.
\end{to_thr}
}