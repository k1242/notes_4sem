Далее кратко познакомимся с другими показателями из статистики. 

\begin{to_def}
    \textit{Медианой} распределения случайной величины $\xi$ называется любое из чисел $\mu$ таких, что
    \begin{equation*}
        \P (\xi \leq \mu) \geq \frac{1}{2}, \hspace{5 mm}
        \P (\xi \geq \mu) \geq \frac{1}{2}.
    \end{equation*}
    Обобщая, приходим к понятию \textit{квантили} уровня $\delta \in (0, 1)$, так назывется решение уравнения $\P (x_\delta) = \delta$, где $x_\delta$ отрезает площадь $\delta$ слева от себя и $1-\delta$ справа. 
\end{to_def}

Вообще ещё есть такой зоопарк, что квантили уровней кратных $0.01$ в прикладной статистике называют \textit{процентилями}, кратных $0.1$ -- \textit{децилями}, кратных $0.25$  -- \textit{квартилями}.

\begin{to_def}
    \textit{Модой} абсолютно непрерывного распределения называют любую точку локального максимума плотности распределения. Для дискретных распределений модой считают любое значение $a_i$, вероятность которого больше соседних.
\end{to_def}

Для описания \textit{унимодеальных} распределений используют следующие величины:

\begin{to_def}
    \textit{Коэффициентом асимметрии} распределения с конечным третьим моментом называют число
    \begin{equation*}
        \beta_1 = \E \left(
            \frac{x-a}{\sigma}
        \right)^3,
    \end{equation*}
    где $a = \E \xi$,  а $\sigma = \sqrt{D \xi}$. 
\end{to_def}

Для симметричных распределений коэффициент асимметрии равен нулю, если $\beta_1 > 0$, то график плотности имеет более крутой наклон слева, и более пологий справа. 

\begin{to_def}
    \textit{Коэффициентом эксцесса} распределения с конечным четвертым моментом называется число
    \begin{equation*}
        \beta_2 = \E \left(
            \frac{\xi - a}{\sigma}
        \right) - 3,
    \end{equation*}
    где $a = \E \xi$,  а $\sigma = \sqrt{D \xi}$. 
\end{to_def}

Для нормального распределения $\beta_2 = 0$, при $\beta_2 > 0$ плотность распределения имеет более острую вершину, чем у нормального распределения. 


