Сформулируем теорему о приближенном вычислении вероятности иметь $k$ успехов в большом числе испытаний Бернулли с маленькой вероятностью успеха $p$.

\green{
\begin{to_thr}[теорема Пуассона]
    Пусть $n \to \infty$ и $p_n \to 0$ так, что $n p_n \to \lambda > 0$. Тогда для любого $k \geq 0$ вероятность получить $k$ успехов в $n$ испытаниях схемы Бернулли с вероятностью успеха $p_n$
    \begin{equation}
        \P (\nu_n = k) = C_n^k p_n^k (1 - p_n)^{n-k} \ \ \to \ \ 
        \frac{\lambda^k}{k!} e^{-\lambda}.
    \end{equation}
     то есть стремится к величине $\lambda^k e^{-\lambda} / k!$.
\end{to_thr}
}

\begin{to_def}[$\mathfrak D$]
    Набор чисел
    \begin{equation*}
         \left\{\dfrac{\lambda^k}{k!} e^{-\lambda} \ \bigg| \ k = 0, 1, 2, \ldots\right\}
    \end{equation*} 
    называется \textit{распределением Пуассона} с параметром $\lambda > 0$.
\end{to_def}

 \red{\texttt{Для всех этих распределений можно посчитать вектора средних и матрицы ковариации.}}