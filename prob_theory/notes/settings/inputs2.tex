

% \section{Сходимость последовательностей случайных величин}

% \subsection{Определение и примеры}
% Плотность многомерного нормального распределения:
\begin{equation*}
    f_\xi (\vc{x}) = \left[
        (\sqrt{2\pi})^n \sqrt{\det \Sigma}
    \right]^{-1} \cdot \exp\left(
        - \frac{1}{2} (\vc{x}-\vc{\mu})\T \Sigma^{-1} (\vc{x} - \vc{\mu})
    \right),
\end{equation*}
где $\Sigma$ -- симметричная, положитеьно определенная матрица. 




% Прямо сейчас работает всё работает convert.
% Хочется научить onvert правильно оценивать результаты скиллов.
% Есть кучка диалогов -- неразмеченны. 

% Реальное качество -- frontend -- максимум. 
% Добавить кнопку -- проблема в диалоге. 


% пнуть федора про сайт
% доучить convert на наших данных
% у конверт проблема -- все ответы (я не знаю, я не понял, не уверен) -- проблема
% 
% 


% \newpage

\section{Статистика}

Гвоорим, что $T(\vc{X})$ -- достаточная для параметра $\theta$ статистика, если $\forall  D$ 
\begin{equation*}
    P\left(\vc{X} \in D \mid T(X)\right) \neq P(\theta).
\end{equation*}
Получается, что зная $T(X)$ мы всё узнаем о $\theta$ по выборке. 

Есть критерий о достаточности статистики. 
\begin{to_thr}[Критерий факторизации]
    Статичтика $T(\vc{X})$ достаточна для $\theta$ тогда и только тогда, когда
    \begin{equation*}
        f_{\smallvc{X}} (\vc{x}, \theta) = g(T(\vc{x}), \theta) \cdot h(\vc{x}),
    \end{equation*}
    где $f_{\smallvc{x}}$ -- плотность распределения. 
\end{to_thr}

% \begin{proof}[$\triangle$]
% Пусть известно, что плотность факторизована, тогда вероятность
% \begin{align*}
%     P(\vc{X} &= \vc{x} \mid T = t) = \frac{P(\vc{X} = \vc{x},\, T=t)}{P(T=t)} = I_{T=t} \frac{P(\vc{X} = \vc{x})}{P(T=t)} = I_{T(\smallvc{x})=t} \frac{g(T(\vc{x}), \theta) h(\vc{x})}{\sum\limits_{\smallvc{x}\colon T(\smallvc{x})=t} P(\vc{X} = \vc{x})} 
%     =  \\ &= 
%     I_{T(\smallvc{x})=t} 
%     \frac{g(T(\vc{x}), \theta) h(\vc{x})}{\sum\limits_{\smallvc{x}\colon T(\smallvc{x})=t} g(T(\vc{x}), \theta)h(\vc{x})} = 
% \end{align*}
% \end{proof}
