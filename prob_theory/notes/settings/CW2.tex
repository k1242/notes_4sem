
\subsection*{Первая задача}
Вероятность выпадения решки равна $p_0 = 0.42$. Монетка подброшена 1000 раз, и решка выпала 360 раз. Сколько раз необходимо подрбросить такую же монетку, чтобы доля выпавших решек отличалась от $p_0$ менее, чем в первые 1000 бросков с вероятностью $p_1 = 0.95$.

\begin{to_thr}[ЦПТ Ляпунова]
    Пусть $\xi_1$, $\xi_2$, $\ldots$ -- независимые и одинакоово распределенные случайные величины с конечной и ненулевой дисперсией: $0 < \D \xi_1 < \infty$. Тогда имеет место слабая сходимость 
    \begin{equation*}
        \frac{S_n - n \E \xi_1}{\sqrt{n \D \xi_1}} \underset{n\to\infty}{\to} \textnormal{N}_{0, 1},
        \hspace{5 mm} 
        S_n = \xi_1 + \ldots + \xi_n.
    \end{equation*}
    последовательности центрированных и нормированных сумм случайных величин к стандартному нормальному распределению.
\end{to_thr}

Точнее $S_n$ стремится к $\textnormal{N}_{a, \sigma^2}$, где в пределах данной задачи верно, что $A = n \E \xi_1 = 0.42 n $, а $\sigma = \sqrt{n \D \xi_1} = \sqrt{n 0.42 (1-0.42)} = \sqrt{n p q}$. 

По условиям задачи требуется попадание в интервал $[a, b] = [0.36n,\, 0.48n]$. Тогда
\begin{equation*}
    \int_a^b \frac{1}{\sqrt{2\pi} \sigma} \exp\left(-\frac{(x-A)^2}{2 \sigma^2}\right) = \Phi\left(\sqrt{n}\frac{A-a}{\sqrt{2 pq}}\right) = 0.95, 
    \hspace{0.5cm} \Rightarrow \hspace{0.5cm}
    n = \frac{2 X^2 p q}{(A-a)^2} = 260,
\end{equation*}
где $X = 1.38$ можно найти по таблице. 




\subsection*{Вторая задача}
Известно, что ковариационная матрица случайного вектора $(X,\, Y,\, Z)\T$  равна
\begin{equation*}
    M = \begin{pmatrix}
        2 & -1 & \lambda \\
        -1 & 2 & 1 \\
        \lambda & 1 & 3 \\
    \end{pmatrix}.
\end{equation*}
Хочется найти все возможные значения $\lambda$, а также $\lambda$, соответсвующий минимальной вариации величины $\xi = X + \lambda Y - 2 Z$. 


Для начала поймём возможные значения $\lambda$: матрица неотрицательно определена, а значит, по критерию Сильвестра:
\begin{equation*}
    \det M = 7 - 2 \lambda - 2 \lambda^2 > 0,
    \hspace{0.5cm} \Rightarrow \bigg/
        \lambda_{1,2} = \frac{1}{2}\left(-1\pm \sqrt{15}\right)
    \bigg/
    \Rightarrow
    \hspace{5 mm} 
    \lambda \in \left[
        -\frac{1}{2}-\frac{\sqrt{15}}{2};\ 
        -\frac{1}{2}+\frac{\sqrt{15}}{2}
    \right].
\end{equation*}
Теперь можем найти оптимальное значение $\lambda$ для $\vc{\xi} = X + \lambda Y - 2 Z = (1,\, \lambda,\, -2)\T$ в базисе $(X, Y, Z)$:
\begin{equation*}
    \vc{\xi}\T M \xi = 2\lambda^2 - 10 \lambda + 14 = 2\left(\lambda- \frac{5}{2}\right)^2 + \frac{3}{2},
    \hspace{0.5cm} \Rightarrow \hspace{0.5cm}
    \lambda = \frac{5}{2}.
\end{equation*}
Однако можно заметить, что верхняя граница $(-1 + \sqrt{15})/2 \approx 1.44 < 2.5$, следовательно минимум достигается на правой границе $\lambda_{\text{opt}} = \frac{1}{2}\left(-1 + \sqrt{15}\right)$. 





% \subsection*{Третья задача}
% Известно, что $X$ и $Z$ независимы их плотность вероятности может быть задана, как
% \begin{equation*}
%     f_X (x) = 5 I_{x>0} e^{-5x}, \hspace{5 mm} 
%     f_Z (z) = 5 I_{z>0} e^{-5z}.
% \end{equation*}
% Считая $U = \min(X; Z)$, $V = \max(X, Z)$, найдём ковариацию $U$ и $V$.







\subsection*{Четвертая задача}
Известно, что плотность распределения переменной $Y$ дана
\begin{equation*}
    f_Y(y) = C \exp\left(
        - y^2 + 4y - 10 
    \right), \hspace{5 mm} y \in \mathbb{R}.
\end{equation*}
Найдём константу $C$, а также матожидание и дисперсию $Y$. 

Заметим, что $f_Y (y) =  C \exp\left(
        - (y-2)^2 - 6
    \right) =  \frac{C}{e^6} \exp\left(
        - (y-2)^2
    \right)$, -- нормальное распределение с $\E Y = a = 2$. 
Осталось найти
\begin{equation*}
    \int_{-\infty}^{+\infty} \frac{C}{e^6} \exp\left(
        - (y-2)^2
    \right) \d y = \sqrt{\pi} C e^{-6} = 1,
    \hspace{0.5cm} \Rightarrow \hspace{0.5cm}
    C = \frac{e^6}{\sqrt{\pi}}.
\end{equation*}
Итого, распределение перепишется в виде
\begin{equation*}
    f_\xi(y) = \frac{1}{\sqrt{2\pi} \sqrt{\frac{1}{2}}} \exp\left(
        - \frac{(y-2)^2}{2 (\frac{1}{\sqrt{2}})^2}
    \right),
\end{equation*}
собственно $\D Y = \sigma^2 = 1/2$. 







\subsection*{Пятая задача}



Случайные величины $X_1, \ldots, X_{100}$ независимы и одинаково распределены, в частности с $N(0, \sigma^2)$, $\sigma=2$. Найдём распределение вектора $(Y,\, Z)\T$, где $Y = X_{61} + X_{62} + \ldots + X_{100}$, и $Z = X_1 + X_2 + \ldots + X_{80}$.

 
% Запишем распределение некоторой величины $f_\xi$ в 100-мерном пространстве:
% \begin{equation*}
%     f_\xi(\vc{x}) = \frac{1}{\sqrt{2\pi}^{100} 2^{100}} \exp\left(
%         -\frac{1}{2} \vc{x}\T \Sigma^{-1} \vc{x}
%     \right), \hspace{5 mm} \Sigma = 4E.
% \end{equation*}

\begin{to_lem}
    Если две случайные величины $\xi \in \text{N}_{a_1, \sigma_1^2}$, $\eta \in \text{N}_{a_2, \sigma_2^2}$ независимы, то $\xi + \eta \in \text{N}_{a_1 + a_2, \sigma_1^2 + \sigma_2^2}$.
\end{to_lem}

\begin{to_lem}
    Пусть есть две величины $\xi_1 \in \text{N}_{a_1, \sigma_1^2}$ и $\xi_2 \in \text{N}_{a_2, \sigma_2^2}$ с ковариациоей $\sigma_{12}$, тогда $\det \Sigma = \sigma_1^2 \sigma_2^2 (1-\rho^2)$, где $\rho = \sigma_{12}/(\sigma_1 \sigma_2)$ -- коэффицент корреляции. Плотность двумерного нормального распределения в этом случае:
    \begin{equation*}
        f(x_1, x_2) = \frac{1}{\sqrt{2\pi}^2 \sigma_1 \sigma_2 \sqrt{1-\rho^2}} \exp\left(
            - \frac{1}{2(1-\rho^2)} \left[
                \frac{(x_1-a_1)^2}{\sigma_1^2} - \rho \frac{2 (x_1 - a_1)(x_2-a_2)}{\sigma_1 \sigma_2} + \frac{(x_2-a_2)^2}{\sigma_2^2}
            \right]
        \right).
    \end{equation*}
\end{to_lem}

И это замечательно, ведь по условию $X_i$ и $X_j$ независимы при $i \neq j$, а тогда $Y$ и $Z$ распределены нормально. Получается, остается найти ковариацию $\rho(Y, Z)$, и мы найдём искомое распределение.

Для начала, $\D Y = 40 \sigma^2$, $\D Z = 80 \sigma^2$, также по условию $a_1 = a_2 = 0$. Ковариация $Y$ и $Z$ по линейности:
\begin{equation*}
    \cov \left(\textstyle \sum_{i=61}^{100} X_i,\, \textstyle \sum_{j=1}^{80} X_j\right) = 
    \sum_{i=61}^{100} \sum_{j=1}^{80} \cov(X_i, X_j) = 20 \sigma^2,
    \hspace{0.5cm} \Rightarrow \hspace{0.5cm}
    \rho = \frac{20}{\sqrt{40 \cdot 80}} = \frac{1}{2\sqrt{2}} = 2^{-3/2}.
\end{equation*}
где учтено, что
\begin{equation*}
    \cov(X_i,\, X_i) = \E X_i^2 - \E^2 X_i = \D X_i = \sigma^2.
\end{equation*}
Итого, искомое распределение:
\begin{equation*}
    f(Y, Z) = \frac{1}{2\pi \sigma^2 \cdot 20 \sqrt{7}} \exp\left(
        - \frac{1}{140 \sigma^2} \left[
            2 Y^2 -  Y Z +  Z^2
        \right]
    \right).
\end{equation*}