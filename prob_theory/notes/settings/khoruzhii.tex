\section{Основные понятия теории вероятностей}

\subsection{Элементы комбинаторики}
Для начала подружимся с комбинаторикой, взяв некоторую её проекцию на теорвер

\begin{to_thr}[]
    Пусть множества $A = \{a_1, \ldots, a_k\}$ состоит из $k$ элементов, а множество $B = \{b_1, \ldots, b_m\}$ -- из $m$ элементов. Тогда можно образовать равно $km$ пар $(a_i, b_j)$.
\end{to_thr}

\begin{to_thr}[]
    Общее количество различных наборов при выборе $k$ элементов из $n$ без возвращения и с учётом порядка равняется
    \begin{equation*}
        A_n^k = n \cdot (n-1) \cdot \ldots \cdot (n-k+1) = \frac{n!}{(n-k)!},
    \end{equation*}
    где $A_n^k$ называется \textit{числом размещений} из $n$ элементов по $k$ элементов. 
\end{to_thr}

\begin{to_thr}[]
    Общее количество различных наборов при выборе $k$ элементов из $n$ без возвращения и без учета порядка равняется
    \begin{equation*}
        C_n^k = \frac{A_n^k}{k!} = \frac{n!}{k! (n-k)!},
    \end{equation*}
    где число $C_n^k$ называется \textit{числом сочетаний} из $n$ элементов по $k$ элементов. 
\end{to_thr}

\green{
\begin{to_thr}[]
% вот тут не понял почему, но ладно
    Общее количество различных наборов при выборе $k$ элементов из $n$ с возвращением и без учёта порядка равняется
    \begin{equation*}
        C_{n+k-1}^k = C_{n+k-1}^{n-1}.
    \end{equation*}
\end{to_thr}
}

\subsection{События и операции над ними}
\begin{to_def}
    \textit{Пространством элементарных исходов} называют множество $\Omega$, содержащее все возможные взаимоисключающие результаты данного случайного эксперимента. Элементы множества $\Omega$ называются \textit{элементарными исходами} и обозначаются $\omega$.
\end{to_def}

\begin{to_def}
    \textit{Событиями} называются подмножества $\Omega$. Говорят, что \textit{произошло событие} $A$, если эксперимент завершился одним из элементарных исходов, входящих в множество $A$. 
\end{to_def}

Вообще в силу таких определений события и множества оказываются очень похожими, так что определены операции \textit{объединения}, \textit{пересечения}, \textit{дополнения}, а также взятия \textit{противоположеного} $\bar{A} = \Omega \backslash A$. Также можно выделить достоверное событие $\Omega$ и невозможное $\varnothing$. 

События $A$ и $B$ называются \textit{несовместными}, если они не могут произойти одновременно: $A \cap B = \varnothing$. События $A_1, \ldots, A_n$ называются \textit{попарно несовместными}, если несовместны любые два из них: $A_i \cap A_j = \varnothing, \ \forall i \neq j$. Говорят, что событие $A$ \textit{влечет} событие $B$ ($A \subseteq B$), если $A \Rightarrow B$. 



\subsection{Дискретное пространство элементарных исходов}
Пространство элементарных исходов назовём дискретным, если множество $\Omega$ конечно или счётно: $\Omega = \{\omega_1, .., \omega_n, \ldots\}$. 

\begin{to_def}
    Сопоставим каждому элементарному исходу $\omega_i$ число $p_i \in [0, 1]$ так, чтобы $\sum p_i = 1$. Вероятностью события $A$ называют число
    \begin{equation*}
        \P(A) = \sum_{\omega_i \in A} p_i,
    \end{equation*}
    где с случае $A = \varnothing$ считаем $\P(A) = 0$. 
\end{to_def}


\begin{to_def}[Классическое определение вероятности]
    Говорят, что эксперимент описывается \textit{классической вероятностной моделью}, если пространство его элементарных исходов состоит из конечного числа равновозможных исходов. Для любого события верно, что
    \begin{equation}
        \P (A) = \frac{\card A}{\card \Omega}.
    \end{equation}
    Эту формулу называют \textit{классическим определением вероятности}.
\end{to_def}

Тут стоит вспомнить три схемы из модели с урнами: 
схема выбора с возвращением и с учётом порядка ($n^k$), 
выбора без возвращения и с учётом порядка ($A_n^k$), а также выбора
без возвращения и без учёта порядка ($C_n^k$), 
описываются классической вероятностной моделью. 
А вот схема выбора с возвращением и без учёта порядкауже не описывается классической вероятностью. 


\subsubsection*{Пример с гипергеометрическим распределением}
 Из урны, в которой $K$ белых и $N - K$ чёрных шаров, наудачу и без возвращения вынимают $n$ шаров, где $n \leq N$.
Термин «наудачу» означает, что появление любого набора из $n$ шаров равновозможно. Найти вероятность того, что будет выбрано $k$ белых и $n - k$ чёрных шаров.

Результат -- набор из $n$ шаров. Общее число $\card \Omega = C_N^n$. Пусть $A_k$ -- событие, состоящее в том, что в наборе окажется $k$ белых и $n-k$ черных. Есть ровно $C_K^k$ способов выбрать $k$ белых шаров из $K$, и 
$C_{N-K}^{n-k}$ способов выбрать $n-k$ черных шаров из $N-K$. Тогда $\card A_k = C_K^k C_{N_K}^{n-k}$,
\begin{equation*}
    \P (A_k) = \frac{\card A_k}{\card \Omega} = \frac{C_K^k C_{N_K}^{n-k}}{C_N^n}.
\end{equation*}
Этот набор вероятностей называется \textit{гипергеометрическим распределением} вероятностей. 





\subsection{Дискретное пространство элементарных исходов}
Пространство элементарных исходов назовём дискретным, если множество $\Omega$ конечно или счётно: $\Omega = \{\omega_1, .., \omega_n, \ldots\}$. 

\begin{to_def}
    Сопоставим каждому элементарному исходу $\omega_i$ число $p_i \in [0, 1]$ так, чтобы $\sum p_i = 1$. Вероятностью события $A$ называют число
    \begin{equation*}
        \P(A) = \sum_{\omega_i \in A} p_i,
    \end{equation*}
    где с случае $A = \varnothing$ считаем $\P(A) = 0$. 
\end{to_def}


\begin{to_def}[Классическое определение вероятности]
    Говорят, что эксперимент описывается \textit{классической вероятностной моделью}, если пространство его элементарных исходов состоит из конечного числа равновозможных исходов. Для любого события верно, что
    \begin{equation}
        \P (A) = \frac{\card A}{\card \Omega}.
    \end{equation}
    Эту формулу называют \textit{классическим определением вероятности}.
\end{to_def}

Тут стоит вспомнить три схемы из модели с урнами: 
схема выбора с возвращением и с учётом порядка ($n^k$), 
выбора без возвращения и с учётом порядка ($A_n^k$), а также выбора
без возвращения и без учёта порядка ($C_n^k$), 
описываются классической вероятностной моделью. 
А вот схема выбора с возвращением и без учёта порядкауже не описывается классической вероятностью. 


\subsubsection*{Пример с гипергеометрическим распределением}
 Из урны, в которой $K$ белых и $N - K$ чёрных шаров, наудачу и без возвращения вынимают $n$ шаров, где $n \leq N$.
Термин «наудачу» означает, что появление любого набора из $n$ шаров равновозможно. Найти вероятность того, что будет выбрано $k$ белых и $n - k$ чёрных шаров.

Результат -- набор из $n$ шаров. Общее число $\card \Omega = C_N^n$. Пусть $A_k$ -- событие, состоящее в том, что в наборе окажется $k$ белых и $n-k$ черных. Есть ровно $C_K^k$ способов выбрать $k$ белых шаров из $K$, и 
$C_{N-K}^{n-k}$ способов выбрать $n-k$ черных шаров из $N-K$. Тогда $\card A_k = C_K^k C_{N_K}^{n-k}$,
\begin{equation*}
    \P (A_k) = \frac{\card A_k}{\card \Omega} = \frac{C_K^k C_{N_K}^{n-k}}{C_N^n}.
\end{equation*}
Этот набор вероятностей называется \textit{гипергеометрическим распределением} вероятностей. 





\subsection{Геометрическая вероятность}
\begin{to_def}
    Пусть некоторая область $\Omega \subset \mathbb{R}^k$ такая, что $\mu(\Omega)$ конечна. Пусть эксперимент состоит из равновероятного выбора случайной точки в области $\Omega$. \textit{Геометрическое определение вероятности}:
    \begin{equation*}
        \P (A) = \frac{\mu(A)}{\mu (\Omega)}.
    \end{equation*}
    Если для точки выполнены условия геометрического определения, то говорят, что точка \textit{равномерно распределена} в $\Omega$. 
\end{to_def}







\section{Аксиоматика теории вероятностей}

\subsection{Алгебра и \texorpdfstring{$\sigma$}{сигма}-алгебра событий}
\begin{to_def}
    Множество $\mathcal A$, элементами которого являются некоторые подмножества $\Omega$ называют \textit{алгеброй}, если оно удовлетворяет следующим условиям:
    \begin{itemize}
        \item[А1)] $\Omega \in \mathcal A$ 
        (алгебра содержит достоверные события);
        \item[А2)] если $A \in \mathcal A$, то $\bar{A} \in \mathcal A$ 
        (вместе с любым множеством алгебра содержит противоположное к нему);
        \item[А3)] если $A \in \mathcal A$ и $B \in \mathcal A$, то $A \cup B \in \mathcal A$ 
        (вместе с любыми двумя множествами алгебра содержит их объединение).
    \end{itemize}
\end{to_def}


Вообще из А1 и А2 следует, что $\varnothing = \bar{\Omega} \in \mathcal A$. Пункт А3 экстраполируется на любой конечный набор. Кстати, объединение можно заменить (в силу закона де Моргана) на пересечение:
\begin{equation*}
    xy \in \mathcal A
    \hspace{0.5cm} \Leftrightarrow \hspace{0.5cm}
    \overline{xy} \in \mathcal A
    \hspace{0.5cm} \Leftrightarrow \hspace{0.5cm}
    \overline{x} + \overline{y} \in \mathcal A.
\end{equation*}

\begin{to_thr}[закон де Моргана]
    Для множеств $x$, $y$ верно, что
    \begin{equation*}
        \overline{x + y} = \overline{x} \cdot \overline{y},
        \hspace{1 cm}
        \overline{xy} = \overline{x} + \overline{y},
    \end{equation*}
    где $xy = x \cap y$, $x + y = x \cup y$. 
\end{to_thr}

В случае счётного пространства элементарных исходов А3 алгебры оказывается недостаточно, так приходим к $\sigma$-алгебре:
\begin{to_def}
    Множество $\mathcal F$, элементами которого являются некоторые подмножества $\Omega$ называется $\sigma$-алгеброй, если выполнены следующий условия:
    \begin{itemize}
        \item[S1)] $\Omega \in \mathcal F$ 
        (алгебра содержит достоверные события);
        \item[S2)] если $A \in \mathcal F$, то $\bar{A} \in \mathcal F$ 
        (вместе с любым множеством алгебра содержит противоположное к нему);
        \item[S3)] если $\{A_i\} \in \mathcal F$, то $\cup_i A_i \in \mathcal F$  
        (вместе с любым \textit{счетным} набором событий $\sigma$-алгебра содержит их объединение).
    \end{itemize}
\end{to_def}

\begin{to_def}
    Минимальной $\sigma$-алгеброй, содержащей набор множеств $\mathcal U$, называется пересечение всех $\sigma$-алгебр, содержащих $\mathcal U$. 
\end{to_def}


\begin{to_def}
    Минимальная $\sigma$-алгебра, содержащая множество $\mathcal U$ всех интервалов на вещественной прямой называется \textit{борелевской} \textit{сигма}-\textit{алгеброй} в $\mathbb{R}$ и обозначается $\mathfrak B (\mathbb{R})$.
\end{to_def}

Итак, оказался определен специальный класс $\mathcal F$ подмножеств $\Omega$, названный $\sigma$-алгеброй событий. Применение счетного числа любых операция к множествам из $\mathcal F$ снова дает множество из $\mathcal F$. \textit{Событиями} будем называть только множества $A \in \mathcal F$.

\subsection{Мера и вероятностная мера}
\begin{to_def}
    Пусть $\Omega$ -- некоторое непустое множество $\mathcal F$ -- $\sigma$-алгебра его подмножеств. Функция
    \begin{equation*}
        \mu \colon  \mathcal F \mapsto \mathbb{R} \cap [0, +\infty) \cup \{+\infty\}
    \end{equation*}
    называется \textit{мерой} на $(\Omega, \, \mathcal F)$, если она удовлетворяет условиям
    \begin{itemize}
        \item[$\mu 1$)] $\mu(A) \geq 0$ для любого множества $A \in \mathcal F$;
        \item[$\mu 2$)] $\forall$ счетного $\{A_i\} \in \mathcal F$ таких, что $A_i \cap A_j = \varnothing, \ \forall i \neq j$ мера их объединения равна сумме их мер:
        \begin{equation*}
            \mu\left(
                \bigcup_{i=1}^{\infty} A_i
            \right) = \sum_{i=1}^{\infty} \mu (A_i).
        \end{equation*}
    \end{itemize}
\end{to_def}

Последнее свойство называют \textit{счётное аддитивностью} или $\sigma$-аддитивностью меры. 


\begin{to_thr}[свойство непрерывности меры]
    Пусть дана убывающая последовательность $B_1 \supseteq B_2 \supseteq B_2 \supset B_3 \supset \ldots$ множеств из $\mathcal F$, причем $\mu(B_1) < \infty$. Пусть $B = \cap_i^{\infty} B_i$. Тогда $\mu(B) = \lim\limits_{n\to\infty} \mu (B_n)$.
\end{to_thr}

\begin{to_def}
    Пусть $\Omega$ -- непустое множество, $\mathcal F$ -- $\sigma$-алгебра его подмножеств. Мера $\mu \colon  \mathcal F \mapsto \mathbb{R}$ называется \textit{нормитрованной}, если $\mu(\Omega)=1$. Другое название нормированной меры -- \textit{вероятность}. 
\end{to_def}

\begin{to_def}
    Пусть $\Omega$ -- пространство элементарных исходов, $\mathcal F$ -- $\sigma$-алгебра его подмножеств (событий). \textit{Вероятностью} или \textit{вероятностной мерой} на $(\Omega, \mathcal F)$ называется функция 
    \begin{equation*}
        \P \colon  \mathcal F \mapsto \mathbb{R}
    \end{equation*}
    обладающая свойствами
    \begin{itemize}
        \item[P1)] $\P (A) \geq 0$ для любого события $A \in \mathcal F$;
        \item[P2)] для любого счётного набора \textit{попарно несовместных} событий $\{A_i\} \in \mathcal F$ имеет равенство
        \begin{equation*}
            \P \left(
                \bigcup_{i=1}^{\infty} A_i
            \right) = \sum_{k=1}^{\infty} \P (A_i);
        \end{equation*}
        \item[P3)] вероятность достоверного события равна единице: $\P(\Omega) = 1$.
    \end{itemize}
\end{to_def}

Свойства (P1) -- (P3) называют \textit{аксиомами вероятности}.

\begin{to_def}
    Тройка $\langle \Omega, \mathcal F, \P \rangle$, в которой $\Omega$ -- пространство элементарных исходов, $\mathcal F$ -- $\sigma$-алгебра его подмножеств и $\P$ -- вероятная мера на $\mathcal F$, называется \textit{вероятностным пространством}. 
\end{to_def}

Вообще, для вероятности верны следующие свойства
\begin{enumerate}
    \item $\P(\varnothing) = 0$.
    \item Для любого конечного набора попарно несовместных событий $A_1, \ldots, A_n \in \mathcal F$ имеет место равенство \\ $\P(A_1 \cup \ldots \cup A_n) = \P(A_1) + \ldots + \P(A_n)$.
    \item $\P (\bar{A}) = 1 - \P(A)$.
    \item Если $A \subseteq B$, то $\P(B \backslash A) = \P(B) - \P(A)$.
    \item $A \subseteq B$, то $\P(A) \leq \P(B).$
    \item $\P(A_1 \cup \ldots \cup A_n) \leq \sum_{i=1}^n P(A_i)$.
\end{enumerate}
И это всё, конечно, хорошо, но если мы хотим что-то посчитать, то

\begin{to_thr}[Формула включения-исключения]
    Для вероятности, в частности для двух событий, верно, что
    \begin{equation*}
        \P (A \cup B) = P(A) + P(B) - P(A\cap B)
    \end{equation*}
    и, обобщая, для объединения $n$ множеств
    \begin{equation*}
        \P(A_1 \cup \ldots \cup A_n) = \sum_{i=1}^{n} \P(A_i) - \sum_{i < j} \P (A_i A_j) + 
        \sum_{i < j < m} \P(A_i A_j A_m) - \ldots + (-1)^{n-1} \P(A_1 A_2 \ldots A_n).
    \end{equation*}
\end{to_thr}






\section{Условная вероятность и независимость}

\subsection{Условная вероятность}
\begin{to_def}
    Условной вероятностью события $A$ при условии, что произошло событие $B$, называется число
    \begin{equation*}
        \P(A|B) = \frac{\P(A \cap B)}{\P (B)},
    \end{equation*}
    которое само собой определено только при $\P(B) = 0$.
\end{to_def}

\begin{to_thr}[]
    Если $\P (B) > 0$ и $\P (A) > 0$, то
    \begin{equation*}
        \P (A \cap B) = \P(B) \P (A | B) = \P (A) \P (B | A).
    \end{equation*}
\end{to_thr}

\begin{to_thr}[]
    Для любых событий $A_1, \ldots, A_n$ верно равенство:
    \begin{align*}
        \P (A_1 \ldots A_n) = \P (A_1) \cdot \P(A_2 | A_1) \cdot \P(A_3 | A_1 A_2) \cdot \ldots \cdot
        \P(A_n | A_1 \ldots A_{n-1}), 
        % \P (A_1 \ldots A_n) &= \P (A_1) \cdot \P(A_2 \mid A_1) \cdot \P(A_3 \mid A_1 A_2) \cdot \ldots \cdot
        % \P(A_n \mid A_1 \ldots A_{n-1}). \\
    \end{align*}
    если все участвующие в нём условные вероятности определены.
\end{to_thr}

\subsection{Независимость событий}
\begin{to_def}
    События $A$ и $B$ называются \textit{независимыми}, если $\P(A \cap B) = \P(A) \P(B)$.
\end{to_def}

Из этого определения вытекает следующие леммы.

\begin{to_lem}
    Пусть $\P(B) > 0$. Тогда события $A$ и $B$ независимы тогда и только, когда $\P(A|B) = \P(A)$. 
\end{to_lem}

\begin{to_lem}
    Пусть $A$ и $B$ несовместны. Тогда независимыми они будут только в том случае, если $\P(A) = 0$ или $\P(B) = 0$.
\end{to_lem}

Другими словами несовместные события не могут быть независимыми. Зависимость между ними -- просто причинно-следственная: если $A\cap B = \varnothing$, то $A \subseteq \bar{B}$, т.е. при выполнении $A$ события $B$  \textit{не происходит}.


\begin{to_lem}
    % свойство 6
    Если события $A$ и $B$ независимы, то независимы и события $A$ и $\bar{B}$, $\bar{A}$ и $B$, $\bar{A}$ и $\bar{B}$.
\end{to_lem}


\begin{to_def}
    События $A_1, \ldots, A_n$ называются \textit{независимыми в совокупности}, если для любого $1 \leq k \leq n$ и любого набора различных меж собой индекс $1 \leq i_1 < \ldots < i_k \leq n$ имеет место равенство
    \begin{equation*}
        \P(A_{i_1} \cap \ldots \cap A_{i_k}) = \P (A_{i_1}) \cdot \ldots \cdot \P(A_{i_k}).
    \end{equation*}
\end{to_def}

\subsection{Формула полной вероятности}
\begin{to_def}
    Конечный или счётный набор попарно несовместных событий $\{H_i\}$ таких, что $\P(H_i)>0$ $\forall i$ и 
    $\cup_i H_i = \Omega$, называется \textit{полной группой событий} или разбиением пространства $\Omega$. 
    Также события, образующие полную группу событий, часто называют \textit{гипотезами}.
\end{to_def}

При подходящем выборе гипотез для любого события $A$ могут быть сравнительно просто вычислены $\P(A | H_i)$ и, собственно, $\P(H_i)$. Как посчитать вероятность события $A$?

\begin{to_thr}[формула полной вероятности]
    Пусть дана полная группа событий $\{H_i\}$. Тогда вероятность любого события $A$ может быть вычислена по формуле
    \begin{equation*}
        \P(A) = \sum_{i=1}^{\infty} \P(H_i) \cdot \P(A | H_i).
    \end{equation*}
\end{to_thr}

\subsection{Формула Байеса}
\begin{to_thr}[формула Байеса]
    Пусть $\{H_i\}$ -- полная группа событий, и $A$ -- некоторое событие, $\P (A) > 0$. Тогда условная вероятность того, что имело место событие $H_k$, если в результате эксперимента наблюдалось событие $A$, может быть вычислена по формуле
    \begin{equation}
        \P (H_k | A) = \frac{
        \P (H_k) \cdot \P (A | H_k)
        }{
        \sum\limits_{i=1}^{\infty} \P(H_i)  \cdot \P (A | H_i)
        }.
    \end{equation}
\end{to_thr}

\begin{to_def}
    Вероятности $\P (H_i)$, вычисленные заранее, до проведения эксперимента, называют\footnote{
        \textit{a'priori}  -- << до опыта >>.
    }  \textit{априорными} вероятностями. Условные вероятности $\P(H_i | A)$ называют\footnote{
        \textit{a'priori}  -- << после опыта >>.
    }  \textit{апостериорными} вероятностями.
\end{to_def}

Формула Байеса позволяет переоценить заранее известные вероятности после того, как получено знание о результате эксперимента. 
\texttt{Эта формула находит многочисленные применения в экономике, статистике, соци- логии и т.п}






\section{Схема Бернулли}

\subsection{Распределение числа успехов в \texorpdfstring{$n$}{n} испытаниях}
\begin{to_def}
    \textit{Схемой Бернулли} называется последовательность независимых в совокупности испытаний, в каждом из 
\end{to_def}

\subsection{Номер первого успешного испытания}
% Далее, для схемы Бернулли, введем величину $\tau \in \mathbb{Z}_+ \cap [1, +\infty)$ равную номеру перого успешного испытания. 

\begin{to_thr}[]
    Вероятность того, что первый успех произойдёт в испытании с номером $k \in \mathbb{N} \cap [1, +\infty)$, равна
    \begin{equation*}
        \P(\tau = k) = p q^{k-1}.
    \end{equation*}
\end{to_thr}

\begin{to_def}[$\mathfrak D$]
    Набор чисел $\{p q^{k-1} \mid k = 1, 2, \ldots\}$ называется \textit{геометрическим} распределением вероятностей.
\end{to_def}

\green{
\begin{to_thr}[<<Нестарение>> геометрического распределения]
    Пусть $\P(\tau = k) = p q^{k-1}$ $\forall k \in \mathbb{N}$. Тогда для любых неотрицательных целых $n$ и $k$ имеет место равенство:
    \begin{equation*}
        \P(\tau > n + k \mid \tau > n) = \P(\tau > k).
    \end{equation*}
    Другими название -- свойство отсутствия последствия.
\end{to_thr}
}

% \subsection{Распределение числа успехов в $n$ испытаниях}
% Теперь рассмотрим схему независимых испытаний независимых испытаний уже не с двумя, а с болбшим количество возможных результатов в каждом испытании. 

Пусть возможны $m$ исходов, $i$-й исход в одном испытании случается с вероятностью $p_i$, где $\sum_i p_i = 1$. Через $\P(n_1, \ldots, n_m)$ обозначим вероятность того, что в $n$ независимых испытаниях первый исход случится $n_1$ раз, \ldots, $m$-исход -- $n_m$ раз.

\green{
\begin{to_thr}
    Для любого $n$ и любых неотрицательных целых чисел $\{n_i\}$, сумма которых равна $n$, верна формула
    \begin{equation*}
        \P(n_1, \ldots, n_m) = \frac{n!}{n_1! \ldots n_m!} p_1^{n_1} \cdot \ldots \cdot p_m^{n_m}.
    \end{equation*}
\end{to_thr}
}

\begin{to_def}[$\mathfrak D$]
    Набор чисел
    \begin{equation*}
         \left\{
         \frac{n!}{n_1! \ldots n_m!} p_1^{n_1} \cdot \ldots \cdot p_m^{n_m}
         \ \bigg| \ n = 1, 2,  \ldots\right\}
    \end{equation*} 
    называется \textit{мультиномиальным (полиномиaльным)} распределением.
\end{to_def}



% \subsection{Распределение числа успехов в $n$ испытаниях}
% Сформулируем теорему о приближенном вычислении вероятности иметь $k$ успехов в большом числе испытаний Бернулли с маленькой вероятностью успеха $p$.

\green{
\begin{to_thr}[теорема Пуассона]
    Пусть $n \to \infty$ и $p_n \to 0$ так, что $n p_n \to \lambda > 0$. Тогда для любого $k \geq 0$ вероятность получить $k$ успехов в $n$ испытаниях схемы Бернулли с вероятностью успеха $p_n$
    \begin{equation}
        \P (\nu_n = k) = C_n^k p_n^k (1 - p_n)^{n-k} \ \ \to \ \ 
        \frac{\lambda^k}{k!} e^{-\lambda}.
    \end{equation}
     то есть стремится к величине $\lambda^k e^{-\lambda} / k!$.
\end{to_thr}
}

\begin{to_def}[$\mathfrak D$]
    Набор чисел
    \begin{equation*}
         \left\{\dfrac{\lambda^k}{k!} e^{-\lambda} \ \bigg| \ k = 0, 1, 2, \ldots\right\}
    \end{equation*} 
    называется \textit{распределением Пуассона} с параметром $\lambda > 0$.
\end{to_def}

 \red{\texttt{Для всех этих распределений можно посчитать вектора средних и матрицы ковариации.}}