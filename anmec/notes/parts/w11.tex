Будем считать $\varphi_1 (0) = 0$, $\varphi_2 (0) = 0$, а также
\begin{equation*}
    H = I_1 \omega_1 + I_2 \omega_2 + \varepsilon (I_1 + \sum_{k_1, k_2} a_k \sin(k_1 \varphi_1 + k_2 \varphi_2)).
\end{equation*}
Тогда можем посчитать $\dot{\varphi}_1 = \omega_1 + \varepsilon$, $\varphi_2 = \omega_2$. Также $I_i$ такие, что
\begin{equation*}
    \dot{I}_i - \varepsilon \sum a_k k_i \cos(k_1 \varphi_1 + k_2 \varphi_2) = - \varepsilon \sum_{k_1, k_2} a_k k_i \cos (k_1 (\omega_1 + \varepsilon) + k_2 \omega_2)
\end{equation*}
Воот, невырожденно, но если $\omega_1/\omega_2 + \varepsilon/\omega_2$ -- рационально, то существуют $k_1$, $k_2$ такие, что  $I_i \sim - \varepsilon a_k k_i t$.

% страница 224, Арнольд Гельштадт


\subsection{Сечение Пуанкаре при \texorpdfstring{$n \geq 2$}{}}

Пусть $n=2$, на $M^4$, $H(I_1, I_2) = h$, также $I_1, I_2 = \const$. Тогда можем рассмотреть
\begin{equation*}
    I_2 = I_2 (h, I_1), \hspace{5 mm}
    \dot{\varphi}_1 = \omega_1 (h_1, I_1),
    \hspace{5 mm}
    \dot{\varphi}_2 = \omega_2 (h_1, I_1).
\end{equation*}
Так свели всё к $M^3 =\{(I, \varphi) \mid H = h\}$,  более того, взяв $t_2 = \frac{2\pi}{\omega_2}$ -- период $\varphi_2$, перейдём к 
\begin{equation*}
    \varphi_1 (k t_2) = \varphi_1 (0) + \frac{\omega_1}{\omega_2} 2 \pi k,
\end{equation*}
получается, что каждый раз проходя $\varphi_2$ по периоду, мы получаем набор точек. Ура, перешли к двумерному дискретному отображению.

Полезно посмотреть на $\alpha(I_1) = \omega_1/\omega_2$ -- \textit{число вращений}. Если $\alpha$ -- рационально, то наблюдаем конечный набор точек. Если $\alpha$ иррационально, то увидим несчётный набор точек.
% по теореме Вейля -- плотность ни о чем не говорит.


% Введем \textit{отображение Пуанкаре}
\subsection{Невозмущенное отображение Пуанкаре}

\begin{to_def}
    \textit{Невозмущенное отображение Пуанкаре}\footnote{
        Вообще $\Pi_0$ -- сохраняет площадь, то есть $\partial(\varphi', I')/\partial(\varphi, I) = 1$.
    } :
    \begin{equation*}
    \Pi_0 \colon \left\{\begin{aligned}
        \varphi' &= \varphi + 2 \pi \alpha(I), \\
        I' &= I,
    \end{aligned}\right.
    \hspace{10 mm}
    \frac{d \alpha}{d I} > 0, 
    \text{ \ \ -- \ \ условие невырожденности.}
    \end{equation*}
    Также возьмём три тора $T^-, T, T^+$ такие, что
    \begin{equation*}
        \alpha(T^-) < \alpha(T)= \frac{n}{m} < \alpha(T^+).
    \end{equation*}
\end{to_def}

Совершив конечное число отображений Пуанкаре увидим, что на $T$ траектория замкнется. Тогда логично рассмотреть
\begin{equation*}
    \Pi^m_0 \left(
        \begin{bmatrix}
            \varphi \\ I
        \end{bmatrix}
    \right) = 
    \begin{bmatrix}
        \varphi + 2 \pi \frac{n}{m}m \\ I
    \end{bmatrix}
    = \begin{bmatrix}
        \varphi \\ I
    \end{bmatrix}
\end{equation*}

Вообще торы инварианты, после отображения Пуанкаре внешний тор закрутится против часовой, а внутрениий -- по часовой.


\subsection{Возмущенное отображение Пуанкаре}

\begin{to_def}
    \textit{Возмущенное отображение Пуанкаре}\footnote{
        Аналогично $\Pi_\varepsilon$ -- сохраняет площадь, то есть $\partial(\varphi', I')/\partial(\varphi, I) = 1$.
    } :
    \begin{equation*}
    \Pi_\varepsilon \colon \left\{\begin{aligned}
        \varphi' &= \varphi + 2 \pi \alpha(I) + \varepsilon f(\varphi, I), \\
        I' &= I + \varepsilon g(\varphi, I),
    \end{aligned}\right.
    \hspace{10 mm}
    \frac{d \alpha}{d I} > 0, 
    \text{ \ \ -- \ \ условие невырожденности.}
    \end{equation*}
\end{to_def}

Согласно КАМ-теории $T^- \to T^-_\varepsilon$, $T^+ \to T_\varepsilon^+$, более того они должны получиться инвариантны:
\begin{equation*}
    \Pi_\varepsilon (T^+_\varepsilon) = T_\varepsilon^+,
    \hspace{5 mm}
    v (T_\varepsilon^-) = T^-_\varepsilon.
\end{equation*}
Вообще претендуем ещё на то что, не только торы не разрушатся, но и вращение будет в тех же направлениях.

Если сами резонансы разрушились, то, ну, бывает, но они всё ещё зажаты нерезонансными торами. Резонансному тору резко поплохело, образуется что-то вроде стохастического слоя. 


Также всё существует, возможно единственная, точка на сечение, между внешним и внутренним тором,  такая, что $\varphi_1 = \varphi$.  Получается, можно выделить множество
\begin{equation*}
    C = \{(I, \varphi) \in \text{\,Сечение Пунакаре\,}\colon \varphi' = \varphi\}.
\end{equation*}
Для него, однако, неверно, что $\Pi_\varepsilon^m (C) = C'$. Более того, $C'$ и $C$ пересекаются, пересекаются чётное число раз.




\begin{to_thr}[Последняя теорема Пуанкаре или теорема Пуанкаре-Биркгофа]
    Если есть некоторое преобразование, которой совершает автоморфизм кольца, то существует как минимум две неподвижные точки. Вообще $C \cap C' = 2 k m$. 
\end{to_thr}

\texttt{Пуанкаре действительно прислал эту теорему, бездоказательно, всё как обычно, и умер..}

 Есть неподвижные точки, это либо седло, либо центр. И вот в окрестностях центров возникают новые торы, фрактально самовоспроизводящиеся.

 % https://en.wikipedia.org/wiki/Arnold_diffusions

 Давайте ещё скажем, что
 \begin{equation*}
     \bigg|
         \frac{\omega_1}{\omega_2} - \frac{n}{m}
     \bigg| \geq \frac{C(\varepsilon)}{m^\gamma},
     \hspace{5 mm} \gamma > 2, \hspace{5 mm} C(\varepsilon) \to_{\varepsilon \to 0} 0.
 \end{equation*}
 Тогда
 \begin{equation*}
     \vv{\omega} = \begin{bmatrix}
         \omega_1 \\ \omega_2
     \end{bmatrix}
 \end{equation*}
 называется диофантовый вектор частот. Торы с таким отношением частот выживают, а другие разрушаются. 

 Если это всё просуммировать
 \begin{equation*}
     \sum_{m=1}^{\infty} \frac{C(\varepsilon)}{m^\gamma} = C(\varepsilon) \sum_{m=1}^{\infty} \frac{1}{m^{\gamma-1}} \to 0, \hspace{5 mm} \varepsilon \to 0.
 \end{equation*}

 Так получается континуальное множество нулевой меры.


 \subsection{Размерность}
 
 Можем взять сетку, покрыть её 2d-объект, тогда
 \begin{equation*}
     N(\varepsilon) \sim \frac{1}{\varepsilon^2}.
 \end{equation*}
Аналогично для куба\footnote{
    Так для канторовой лестницы размерность $D = \frac{\ln 2}{\ln 3}$.
}  $N(\varepsilon) \sim \varepsilon^{-3}$, для прямой $N(\varepsilon) \sim \varepsilon^{-1}$, так можем прийти к
\begin{equation*}
    N(\varepsilon) \sim \frac{1}{\varepsilon^D},
    \hspace{5 mm}
    \ln N = - D \ln \varepsilon, \ \ \varepsilon \to 0.
\end{equation*}

\begin{to_def}
    \textit{Размерностью} множества будем считать
    \begin{equation*}
        D = \lim_{\varepsilon \to 0} \left(
            - \frac{\ln N(\varepsilon)}{\ln \varepsilon}
        \right).
    \end{equation*}
\end{to_def}

\subsection{Итоги}

\texttt{То есть что такое интегрируемость? Это про то что мы можем разделить переменные, можем систему\\ представить как набор независимых маятников.} 

\texttt{Да, стоит сказать, что у Гамильтоновых систем аттракторов быть не может, а вот у диссипативных систем возникают странные аттракторы и т.д.} 

МатАн немного плохо начинает работать на фракталах и пространствах дробной размерности. \texttt{МатАна больше нет. Остался ТеорВер, Статистика, DataScience.} \texttt{Таков мир за пределами интегрируемости.} 
