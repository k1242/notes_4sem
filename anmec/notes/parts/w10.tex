Частенько непонятно, как перейти к переменным действие-угол $(I, \varphi)$, когда есть некоторые $(p, q)$. Иногда нам везёт, переменные разделяются, тогда
\begin{equation*}
    H \equiv H(F_1(q_1, p_1), \ldots, F_n(q_n, p_n)),
    \hspace{0.5cm} \Rightarrow \hspace{0.5cm}
    S = - h t + \sum_{k} S_k (q_k, \alpha_k),
    \hspace{0.5cm} \Rightarrow \hspace{0.5cm}
    p_k = \frac{\partial S}{\partial q_k} = p_k (q_k, \alpha)
\end{equation*}
более того
\begin{equation*}
    I_k = \frac{1}{2\pi} \oint_{C_k} p_k (q_k, \alpha) \d q_k,
    \hspace{5 mm}
    \varphi_k \mod 2 \pi,
\end{equation*}
где $k = 1, \ldots, n$, а $C_k$ -- $k$-й цикл на торе. 
\texttt{Пуанкаре пишет, что системы бывают более интегрируемые и менее интегрируемые}.

\subsubsection*{Задача}

Пусть у нас $n=2>1$. Системы из двух грузиков на пружинке в коробке
\begin{equation*}
    L = \frac{m}{2} (\dot{x}_1^2 + \dot{x}_2^2) - \frac{c}{2} (x_1^2 + x_1^2 + (x_1-x_2)^2).
\end{equation*}
Хочется перейти к переменным действие угол. Давайте вспомним про переход к главным координатам, где
\begin{equation*}
    L = \frac{\dot{\theta}_1^2 - \lambda_1 \theta_1^2}{2} + \frac{\dot{\theta}_2^2 - \lambda_2 \theta^2}{2},
    \hspace{0.5cm} \Rightarrow \hspace{0.5cm}
    H = \frac{p_1^2 + \omega_1^2 \theta_1^2}{2} + \frac{p_2^2 + \omega_2^2 \theta_2^2}{2},
\end{equation*}
получается система разделилась на два первых интеграла $F_1$ и $F_2$, тогда
\begin{equation*}
    I_1  =\frac{1}{2\pi} \oint_{C_1} p_1 (\theta_1, F_1) \d \theta_1 = \frac{1}{2\pi} \frac{2 \pi F_1}{\omega_1} = \frac{F_1}{\omega_1},
\end{equation*}
где в конце мы просто вспомнили площадь эллипса, тогда
\begin{equation*}
    \tilde{H} = I_1 \omega_1 + I_2 \omega_2 = h.
\end{equation*}
Приходим к вырожденной (в некотором смысле) системе
\begin{align*}
    &\dot{\varphi}_1 = \omega_1, \hspace{5 mm} &\dot{\varphi}_2 = \omega_2, \\
    &\dot{I}_1 = 0, \hspace{5 mm}   &\dot{I}_2 = 0.
\end{align*}


\subsection{Немного о теории возмущений}

Пусть мы немного возмущаем хорошую систему, рассмотрим случай $n=1$.
Возмущение -- переход к $n=1.5$ (\texttt{ну, типа, ...})
\begin{equation*}
    H \to H(q, p, \lambda(\varepsilon t)), \hspace{5 mm} 0 < \varepsilon \ll 1.
\end{equation*}
Выберем $\tau = \varepsilon t$, например, маятник -- который медленно тянут за ниточку. 

\begin{to_def}
    Функция $I(q, p, \lambda)$ -- называется \textit{адиабатическим инвариантом} системы, если
    \begin{equation*}
        \forall \varkappa > 0, \ \exists \varepsilon_0 (\varkappa) \colon  \forall \varepsilon < \varepsilon_0 
        \ \ 
        \bigg|
            I(q(t), p(t), \lambda(\varepsilon t)) - I(q(0), p(0), \lambda(0))
        \bigg| < \varkappa,
        \hspace{5 mm}
        0 \leq t \leq \frac{1}{\varepsilon}.
    \end{equation*}
\end{to_def}


Делаем всё как раньше, рассмотрим производяющую функцию
\begin{equation*}
    S(q, I, \lambda),
    \hspace{5 mm}
    p = \frac{\partial S}{\partial q},
    \hspace{5 mm}
    \varphi = \frac{\partial S}{\partial I} ,
    \hspace{0.5cm} \Rightarrow \hspace{0.5cm}
    q \equiv q(\varphi, I, \lambda).
\end{equation*}
Итого, находим
\begin{equation*}
    \hat{H} = H + \frac{\partial S}{\partial t} = H + \frac{\partial S}{\partial \lambda} \frac{d \lambda}{d \tau} \frac{d \tau}{d t} = H + \varepsilon \lambda'_\tau \frac{\partial S}{\partial \lambda} = H_0 + \varepsilon H_1.
\end{equation*}
Теперь можем уравнения Гамильтона и записать
\begin{equation*}
    \dot{I} = - \varepsilon \lambda_\tau' \frac{\partial }{\partial \varphi} \left(
        \frac{\partial S}{\partial \lambda} 
    \right) = - \varepsilon \frac{\partial H-1}{\partial \varphi}
    ,
    \hspace{5 mm}
    \dot{\varphi} = \frac{\partial H_0}{\partial \lambda} + \varepsilon \lambda_\tau' \frac{\partial }{\partial I} \left(
        \frac{\partial S}{\partial \lambda} 
    \right) = \frac{\partial H_0}{\partial I} + \varepsilon \frac{\partial H_1}{\partial I},
\end{equation*}
где $\dot{I}$ -- медленная переменная, $\dot{\varphi}$ -- быстрая переменная.

Вообще $H_1$ -- $2\pi$-периодическая функция от $\varphi$, \texttt{будем считать, что} в случае, если 
\begin{equation*}
    \langle \dot{I} \rangle_\varphi = 0,
\end{equation*}
то у усредненной системы есть интеграл движения. Вообще на $t \in [0, 1/\varepsilon]$, верно, что $\textnormal{err}\, I < \const \cdot \varepsilon$.

\subsubsection*{Задача}

Есть некоторая точка массы $m$, стенка удаляется как $l(\varepsilon t)$ -- адиабатически ($v \gg \dot{l}$) медленно расширяется. 

Найдём переменную действия 
\begin{equation*}
    I = \frac{1}{2\pi} \oint p \d x = \frac{m v l}{\pi},
\end{equation*}
что и является адиабатическим инвариантом. 

\subsection{Многомерный случай}

В случае $n>1$ перейдём от $(I, \varphi)$ к $(J, \psi)$ (так, чтобы $H_0(I) \to H_0 (I) + \varepsilon H_1 (I, \varphi, \varepsilon) = \hat{H} (J)$)
преобразованием вида
\begin{equation*}
    S(\varphi, J) = \varphi J +  \varepsilon S_1 (\varphi, J) + \varepsilon^2 S_2 (\varphi, J) + \ldots
\end{equation*}
Тогда, из критерия каноничности,
\begin{equation*}
    I = J + \varepsilon \frac{\partial S_1}{\partial \varphi}  + \ldots,
    \hspace{5 mm}
    \psi = \varphi + \varepsilon \frac{\partial S_1}{\partial J} + \ldots,
\end{equation*}
тогда
\begin{equation*}
    \hat{H} = H_0 \left(J + \varepsilon \frac{\partial S_1}{\partial \varphi} + \ldots \right) + \varepsilon H_1 \left(J + \varepsilon \frac{\partial S_1}{\partial \varphi} + \ldots \right) + \ldots = \hat{H}_0 (J) + \varepsilon \hat{H}_1 (J) + \ldots,
\end{equation*}
так приходим к системе
\begin{align*}
    O(1) &\colon \hat{H}_0 (J) = H_0 (J=I), \\
    O(\varepsilon) &\colon \hat{H}_1 (J) = \frac{\partial H_0}{\partial J} \frac{\partial S_1}{\partial \varphi} + H_1 (J, \varphi, 0)., \\
    O(\varepsilon^2) &\colon \ldots
\end{align*}
что перепишется в виде
\begin{equation*}
    \hat{H}_1 (J) = \omega_0 (J) \cdot \frac{\partial S}{\partial \varphi} + H_1(J, \varphi, 0),
\end{equation*}
раскалдывая в ряд Фурье находим, что
\begin{equation*}
    H_1 = \sum_k H_{1k} \exp(i (\vc{k} \cdot \vc{\varphi}));
    \hspace{5 mm}
    S_1 = \sum_k S_{1k} \exp\left(
        i (\vc{k} \cdot \hat{\varphi})
    \right),
\end{equation*}
что приводит к 
\begin{equation*}
    (\vc{\omega}_0 \cdot \vc{k}) S_{1k} = H_{1k},
    \hspace{0.5cm} \Rightarrow \hspace{0.5cm}
    S_1 = \sum \frac{H_{1k}}{(\vc{\omega}_0 \cdot \vc{k}) } \exp\left(
        i (\vc{k} \cdot \vc{\varphi})
    \right),
\end{equation*}
а это явно демонстрирует \textit{проблему малых знаменателей}.

\texttt{Пуанкаре показал, что такие ряды всегда\footnote{
    \texttt{У Бродского было стихотворение <<Конец прекрасной эпохи>>. } 
}  расходятся и умер в 1912 году. В механике наступил пост- \\модерн -- народ пришел к тому что всё неинтегрируется -- беда.} 


Но в 1954 году \textbf{К}олмогоров написал статейку на 4 странице, его студент \textbf{А}рнольд, в 1963 году её доказал\footnote{
    Арнольд, Нейштат, Табор.
}  страниц на 60. Независимо этим занимался \textbf{М}озер, который пришел к похожим заключениям, но с другой стороны. 

Идея в том, чтобы строить сверхсходящиеся ряды $\varepsilon^{2^n}$, которые очень быстро убывают. Тажке сразу забиваем на рациональные $\omega$, более того важна степень иррациональности числа: когда $\omega$ сильно иррациональное -- есть надежда на успех.

\begin{to_def}
    Для того, чтобы КАМ-теория работала нужно, чтобы система была \textit{невырождена}:
    \begin{equation*}
        \bigg|
            \frac{\partial^2 H_0}{\partial I_i \partial I_j} 
        \bigg| \neq 0, \ \Leftrightarrow \ \bigg|
            \frac{\partial \omega_i}{\partial I_j} 
        \bigg| \neq 0.
    \end{equation*}
    Это всё потому что нам хочется, чтобы мера резонансных торов была бы равна 0. 
\end{to_def}

\begin{to_def}
    Говорят, что \textit{торы иррациональны}, если ($n=2$)
    \begin{equation*}
        \frac{\omega_1 (I_1, I_2)}{\omega_2 (I_1, I_2)} \neq \frac{n}{m}
    \end{equation*}
    что соответствует плотной\footnote{
        Кстати,здесь начинает цвести эргодическая теория.
    }  намотке тора. 
\end{to_def}


\begin{to_def}
    Система называется \textit{изоэнергетически невырожденной}, если
    \begin{equation*}
        \det \begin{bmatrix}
            \dfrac{\partial^2 H_0}{\partial \vc{I}\T \partial \vc{I}} & \vc{\omega} \\
            \vc{\omega}\T & 0
        \end{bmatrix} \neq 0,
\hspace{10 mm}
        \det \begin{bmatrix}
            \dfrac{\partial \vc{\omega} }{\partial \vc{I}\T } & \vc{\omega} \\
            \vc{\omega}\T & 0
        \end{bmatrix} \neq 0.
    \end{equation*}
\end{to_def}



