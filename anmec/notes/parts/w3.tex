\subsection*{Общий подход}


Запишем уравнения Лагранжа
\begin{equation*}
    \frac{d }{d t} \frac{\partial T}{\partial \dot{q}} - \frac{\partial T}{\partial q} = Q,
\end{equation*}
где основная идея Гамильтонова формализма -- всегда уравнения разрешимы относительно ускорений $\ddot{q} = \ddot{q} (q, \dot{q})$. Пусть 
$x_1 = q_1$, $x_2 = \dot{q}_1$, $x_3 = q_2$, $x_4 = \dot{q}_2$, и т.д. Приведем уравнения к \textit{нормальной форме Коши} 
\begin{equation*}
    \vc{\dot{x}} = \vc{f}(\vc{x}), \hspace{1 cm}
    \vc{x} \in M^{2n},
\end{equation*}
где $M^{2n}$ -- фазовое пространство, или пространство состояний. 

Не умоляя общности будем просто рассматривать системы вида $\vc{\dot{x}} = \vc{f}(\vc{x})$, считая, что $\vc{x} \in M^{n}$. Посмотрим на некоторую $\vc{x}_0 \in M^n$, -- начальные условия. Продолжаем считать, что решение охапки диффуров единственно, тогда и через каждую точку конфигурационного многообразия проходит единственная траектория.

\begin{to_def}
    Множество траекторий (интегральных кривых) образует \textit{фазовый портрет}.
    \textit{Бифуркация} -- качественное изменение фазового портрета при
    плавном изменении параметров модели. \textit{Бифуркационная диаграмма} отображает бифуркацию системы.
\end{to_def}


% \subsubsection*{Задача 1}

% Посмотрим на систему
% \begin{equation*}
%     \dot{x} = x (a - x^2),
% \end{equation*}
% где $x \in \mathbb{R}^1$. Заметим, что стационарные точки уравнения
% \begin{equation*}
%     x^* = \left\{\begin{aligned}
%         &0;  \\
%         &\pm \sqrt{a}, \ \ a > 0.
%     \end{aligned}\right.
% \end{equation*}

% Придем вилообразной бифуркации. 

% \subsubsection*{Задача 2}

% Посмотрим на систему
% \begin{equation*}
%     \ddot{x} + 2 a x +  4 x^3 = 0,
%     \hspace{0.5cm} \Leftrightarrow \hspace{0.5cm}
%     \frac{\dot{x}^2}{2} + \Pi(x) = h,
% \end{equation*}
% где $\Pi(x) = a x^2 + x^4$. 

\subsection{Двумерные динамические системы}


Посмотрим ещё на системы на $\mathbb{R}^2$. 

\begin{to_def}
    \textit{Предельный цикл}  -- замкнутая периодическая траектория (ЗПТ) системы дифференциальных уравнений, изолированная от других ЗПТ. Такжа ЗПТ такая, что для всех траекторий из некоторой окрестности периодических траекторий стремится к ней при $t \to + \infty$ (установившийся периодический цикл) \textbf{или} при $t \to - \infty$ (неустановившийся предельный цикл).
\end{to_def}

Другими словами является аттрактором для некоторой своей окрестности. 

% \subsubsection*{Задача 3}

% Посмотрим на систему такую, что
% \begin{equation*}
%     m \ddot{x} + b \dot{x} + c x = V^2 \cdot (a_1 \frac{\dot{x}}{V}-a_3 \frac{\dot{x}^3}{V^3}).
% \end{equation*}
% И пусть $V < b/a$, тогда при $x=0$ асимптотическая устойчивость. При $V > b/a$ соответственно неустойчиво. 

% Здесь хорошая идея перейти в полярные координаты, там все классно сократится. 


