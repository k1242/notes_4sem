\subsection{Вынужденные колебания}

Давайте испортим консервативность так, чтобы
\begin{equation*}
    \frac{d }{d t} \frac{\partial T}{\partial \dot{q}^i} - \frac{\partial T}{\partial q^i} = - \frac{\partial \Pi}{\partial q^i} + Q_i(t).
\end{equation*}
Как выяснили раннее
\begin{equation*}
    q = U  \theta. \hspace{1 cm}
    U\T A U = E, U\T C U = \Lambda.
\end{equation*}
Посчитаем элементарную работу добавленной силы
\begin{equation*}
    \delta A = Q_i \delta q^i = \Theta\T \delta \theta = Q\T U \delta \theta,
\end{equation*}
тогда можно записать, что
\begin{equation*}
    \Theta = U\T Q, \hspace{1 cm} Q = \left(U\T\right)^{-1} \Theta,
\end{equation*}
то есть \textit{преобразование обобщенных сил}. То есть уравнение приходит к виду
\begin{equation*}
    A \ddot{q} + C q = Q(t),
    \hspace{0.5cm} \overset{\text{бог с индексами}}{=}  \hspace{0.5cm}
    \ddot{q}_i + \lambda_i \theta_i = \Theta_i (t).
\end{equation*}
Тогда ответ запишется в виде
\begin{equation*}
    q = \sum_{i=1}^{n} c_i \vc{u}_i \sin \left(
        \sqrt{\lambda_i} \, t + \alpha_i
    \right) + 
    \sum_{i=1}^{n} \vc{u}_i \theta^*_i (t),
\end{equation*}
где вторая сумма соотвествует \textit{вынужденным колебаниям}, а первая свободным гармоническим колебаниям. 

Пусть так вышло, что 
\begin{equation*}
    % \theta_i (t) = a_i \sin (\Omega t),
    % \hspace{0.5cm} \Rightarrow \hspace{0.5cm}
    \left\{\begin{aligned}
        \theta_i^* &= b_i \sin \left(\Omega t\right) \\
        \Theta_i (t) &= a_i \sin \left(\Omega t\right)
    \end{aligned}\right.
    \hspace{0.5cm} \Rightarrow \hspace{0.5cm}
    b_i \left(
        \lambda_i - \Omega^2
    \right) = a_i,
    \hspace{0.5cm} \Rightarrow \hspace{0.5cm}
    \theta_i^* = \frac{a_i}{\lambda_i - \Omega^2} \sin \left(\Omega t\right).
\end{equation*}
В случае же \textit{резонанса} ищем решение в виде
\begin{equation*}
    \theta_i^* (t) = b_i t \cos\left(\Omega t\right),
    \hspace{0.5cm} \Rightarrow \hspace{0.5cm}
    b_i = -\frac{a_i}{2 \Omega}.
\end{equation*}
И здесь мы видим первые звоночки от Пуанкаре, о конце линейной теории.



\subsubsection*{Задача 1 (18.42)}

Есть некоторая платформа, перемещающаяся по закону $a \sin \omega t$. На ней подвешены куча стержней, соединенных пружинами разной упругости, на разных высотах. Вопрос -- на каких $\omega$ возможен резонанс?

Перейдём в СО платформы, тогда возбуждающая сила -- сила инерции, соотвественно для всех стержней возбуждающая сила одинаковая
\begin{equation*}
    \vc{J}_i^e
    = - m \mathrm{\vc{w}}_i^e = m \omega^2 a \sin (\omega t) \vc{e}.
\end{equation*}
Посчитаем обобщенные силы, как
\begin{equation*}
    Q_1^e = \ldots = Q_n^e = \frac{\delta A_i}{\delta \varphi_i} = 
    \frac{
    \left(
        \vc{J}_i^e \cdot \delta \vc{r}_i
    \right)
    }{
        \delta \varphi_i
    } = \frac{1}{2} m \omega^2 a l \sin (\omega t).
\end{equation*}
Получается, что мы посчитали столбец обобщенных сил
\begin{equation*}
    \vc{Q} = \frac{l}{2} \left[
        1, \ 1,\  \ldots, 1
    \right]\T \ a \omega^2 m \sin (\omega t).
\end{equation*}
По крайней мере мы можем сказать, что у нас есть главная частота
\begin{equation*}
    \lambda_1 = \frac{3 g}{2l}, \hspace{1 cm}
    \vc{u}_1 = \left[
        1, \ 1,\  \ldots, 1
    \right]\T.
\end{equation*}
Теперь выпишем матрицу кинетической энергии
\begin{equation*}
    A = \frac{ml^2}{6} E, \hspace{1 cm} U\T A U = E,
    \hspace{0.5cm} \Rightarrow \hspace{0.5cm}
    U U\T = E,
\end{equation*}
с точностью до множителя. Тогда $\vc{u}_1, \ldots, \vc{u}_n$ -- ортогональный базис.

Теперь вспоминаем, что
\begin{equation*}
    \Theta = U\T Q = \begin{pmatrix}
        \vc{u}\T_1 \\ \ldots \\ \vc{u}\T_n
    \end{pmatrix} \vc{u}_1 \ 
    \frac{1}{2} a \omega^2 l m \sin (\omega t) = 
    \begin{pmatrix}
        \vc{u}\T_1 \cdot \vc{u}_1 \\ \ldots \\ \vc{u}\T_n \cdot \vc{u}_1
    \end{pmatrix}\ 
    \frac{1}{2} a \omega^2 l m \sin (\omega t) = 
    \begin{pmatrix}
        n \\ \ldots \\ 0
    \end{pmatrix} \vc{u}_1 \,
    \frac{1}{2} a \omega^2 l m \sin (\omega t).
\end{equation*}
Ура, от сих приходим к приятным уравнениям Лагранжа
\begin{equation*}
    \left\{\begin{aligned}
        \ddot{\theta}_q + \lambda_1 \dot{\theta}_1 &= n a \omega^2 \frac{l}{2} \sin \omega t,
        \ddot{\theta}_2 + \lambda_2 \theta_2  \\
        &\ldots \\
        \ddot{\theta}_n + \lambda_n \theta_n &= 0 \\
    \end{aligned}\right.
    \hspace{0.5cm} \Rightarrow \hspace{0.5cm}
    \omega_{\text{рез}} = \sqrt{\frac{3g}{2l}}.
\end{equation*}


\subsection{Диссипативные системы}

И снова испортим консервативную систему до диссипативной,
\begin{equation*}
    \frac{d }{d t} \frac{\partial T}{\partial \dot{q}^i} - \frac{\partial T}{\partial q^i} = - \frac{\partial \Pi}{\partial q^i} + \tilde Q_i(\dot{q}) = Q_i(q, \dot{q}).
\end{equation*}
С кинетической всё как обычно, тогда
\begin{equation*}
    T = \frac{1}{2} \dot{\vc{q}}\T A \dot{\vc{q}};
    \hspace{1 cm}
    \vc{Q} = \vc{Q}(0) + \frac{\partial \vc{Q}}{\partial \vc{q}\T} \vc{q} + \frac{\partial \vc{Q}(0)}{\partial \dot{\vc{q}}\T} \dot{\vc{q}} = - C \vc{q} - B \dot{\vc{q}}.
\end{equation*}
Где ввели матрицы вида
\begin{equation*}
    C = - \frac{\partial \vc{Q}(0)}{\partial \vc{q}\T};
    \hspace{1 cm}
    B = - \frac{\partial \vc{Q} (0)}{\partial \dot{\vc{q}}\T}.
\end{equation*}
В таком случае уравнение примет вид
\begin{equation}
    A \ddot{\vc{q}} + B \dot{\vc{q}} + C \vc{q} = 0,
\end{equation}
получили \textit{линеаризация уравнений Лагранжа I}.
Но его сходу к каноническом виду не привести.

Вспомним, что энергия системы
\begin{equation*}
    E = \frac{1}{2} \dot{\vc{q}} \cdot A \dot{\vc{q}} + \frac{1}{2} \vc{q} \cdot C \vc{q},
    \hspace{0.5cm} \Rightarrow \hspace{0.5cm}
    \frac{d E}{d t} = A \ddot{\vc{q}} \cdot \dot{\vc{q}} + C \vc{q} \cdot \dot{\vc{q}} =
    \left[
        A \ddot{\vc{q}} + C \vc{q}
    \right] \cdot \dot{\vc{q}} = - B \dot{\vc{q}}^2 = N.
\end{equation*}
И пошла классификация: если $N \equiv 0$, то силы называем \textit{гироскопическими}. Если $N \leq 0$, то силы \textit{диссипативные}. 


\begin{to_def}
    Положение равновесия $\vc{q}^*$ называется \textit{асимптотически устойчивым}, если оно устойчиво \textbf{и} 
    \begin{equation*}
        \exists \delta \colon  \forall\,  |\dot{\vc{q}}| < \delta, \, |\vc{q}| < \delta \ \ 
        \lim_{t \to \infty} \vc{q}(t) = 0, \ 
        \lim_{t \to \infty} \dot{\vc{q}}(t) = 0.
    \end{equation*}
\end{to_def}


Возвращаясь к уравнению, вспомним что решение ищется в виде\footnote{
    В общем случае решение системы вообще сложнее (при кратных $\lambda$), но качественно всё примерно в таком же духе, поэтому, ну, всё хорошо.
} 
\begin{equation*}
    \vc{q} = \sum_{i=1}^{2n} C_i \vc{u}_i \exp\left(
        \lambda_i t
    \right),
    \hspace{0.5cm} \Rightarrow \hspace{0.5cm}
    \left[
        A \lambda^2 + B \lambda + C
    \right] \vc{u} = 0,
    \hspace{0.5cm} \Rightarrow \hspace{0.5cm}
    \det\left[
        A \lambda^2 + B \lambda + C
    \right] = 0,
\end{equation*}
тогда мы находим $2n$ решений $\lambda_1, \ldots, \lambda_{2n}$, и, соответственно, $2n$ амплитудных векторов.


\begin{to_thr}[Достаточное условие асимптотической устойчивости]
    Для того, чтобы решение $\vc{q} = \vc{q}^*$ было асимптотически устойчиво достаточно, чтобы
    \begin{equation*}
        \Re \lambda_i < 0, \ \ \forall i \in \{1, \ldots, 2n\}.
    \end{equation*}
    Если $\exists \lambda_i \colon \Re \lambda_i > 0$, тогда всё не так хорошо.
\end{to_thr}


Как узнать, что ..., для этого достаточно посмотреть на рыбу
\begin{equation*}
    a_m \lambda^m + a_{m-1} \lambda^{m-1} + \ldots + a_0 = 0,
\end{equation*}
и отрежем голову и хвост, получим матрицу
\begin{equation*}
    \Gamma = \begin{pmatrix}
        a_{m-1} & a_{m-3} & \ldots & 0\\
        a_m & a_{m-2} & \ldots & 0 \\ 
        0 & a_{m-1} & \ldots & 0 \\
        0 & 0 & \ldots & a_0 \\
    \end{pmatrix},
\end{equation*}
так получили матрицу Гурвица.

\begin{to_thr}[Критерий Рауса-Гурвица]
    Для того, чтобы $\Re \lambda_i < 0$ необходимо и достаточно, чтобы $a_i > 0$, и
    \begin{equation*}
        \Delta_1, \ \Delta_3, \ \ldots,\ \Delta_{m-1} > 0.
    \end{equation*}
    \red{Есть другие формы}.
\end{to_thr}