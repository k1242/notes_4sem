\begin{to_thr}[Теорема Ли Хуа-Сжуна]
    Был некоторый интегральный инвариант Пуанкаре, так вот, утверждается, что 
    \begin{equation*} 
        J = \oint A(q, p, t) \cdot \delta q + B(q, p, t) \cdot \delta p = \const \cdot J_{\text{П}},
    \end{equation*}
    только вот интегральный инвариант Пуанкаре существует для трубки прямых путей, а сейчас мы обобщаем это на отношения для разных трубок.
\end{to_thr}

Говоря о некоторых следствиях 
\begin{equation*}
    \oint q \delta p = c \oint p \delta q,
    \hspace{0.5cm} \Rightarrow \hspace{0.5cm}
    \oint 
    \underbrace{(q \delta p - c p \delta q)}_{\delta \Phi}
     = 0.
\end{equation*}
Раньше были уравнения Лагранжа
\begin{equation*}
    \frac{d }{d t} \frac{\partial L}{\partial \dot{q}} - \frac{\partial L}{\partial q} = 0,
    \hspace{5 mm} \bigg/
        \tilde{q} = \tilde{q} (q, t)
    \bigg/ \hspace{5 mm}
    \frac{d }{d t} \frac{\partial \tilde{L}}{\partial \dot{\tilde{q}}} -
    \frac{\partial \tilde{L}}{\partial \tilde{q}} = 0.
\end{equation*}
Но, скорости не преобразуются. 

Чуть прикольнее в уравнениях Якоби
\begin{equation}
\label{eqw}
    \left\{\begin{aligned}
        \tilde{q} &= \tilde{q} (q, p, t) \\
        \tilde{p} &= \tilde{p} (q, p, t) \\
    \end{aligned}\right.
    \hspace{5 mm}
    \bigg|
        \frac{\partial (\tilde{q}, \tilde{p})}{\partial (q, p)} 
    \bigg| \neq 0.
\end{equation}
что приводит к ситуации
\begin{equation*}
    \left\{\begin{aligned}
        \dot{q} &= \partial_p H \\
        \dot{p} &= - \partial_q H
    \end{aligned}\right.
    \hspace{0.5cm} \to \hspace{0.5cm}
    \left\{\begin{aligned}
        \dot{\tilde{q}} &= \tilde{Q}(\dot{\tilde{q}}, \dot{\tilde{p}}, t) \\ 
        \dot{\tilde{p}} &= \tilde{P}(\dot{\tilde{q}}, \dot{\tilde{p}}, t) \\ 
    \end{aligned}\right.
\end{equation*}

\begin{to_def}
    Преобразование \eqref{eqw} называется каноническим, если оно переводит любую гамильтонову систему в гамильтонову.
\end{to_def}

Из вариационных принципов умеем получать уравнения Лагранжа, да и уравнения Гамильтона тоже
\begin{equation*}
    \delta \int_{t_0}^{t_1} \l(
        p \cdot \dot{q} - H
    \r) \d t = 0.
\end{equation*}
Как раньше выбираем $\dot{\vc{x}} = [q, \, p]\T$, что приведет к $2n$ переменным.

Но и в новых переменных хочется видеть что-то похожее,
\begin{equation*}
    \delta \int \l(
        \tilde{p} \cdot \dot{\tilde{q}} - \tilde{H}
    \r) \d t = 0,
\end{equation*}
что приводит нас к мысли о том, что 
\begin{equation*}
    \tilde{p} \cdot \dot{\tilde{q}} - \tilde{H} = 
    c (p \cdot \dot{q} - H) - \frac{d \tilde{F}}{d t} (q, p, t).
\end{equation*}
домножая, получаем
\begin{equation*}
    \tilde{p} \cdot \d \tilde{q} - \tilde{H} \d t = c p \d q - H \d t - \d F,
\end{equation*}
тогда
\begin{equation*}
    \d \tilde{q} = \frac{\partial \tilde{q}}{\partial t} \d t + \delta^t \tilde{q},
    \hspace{5 mm}
    \d F = \frac{\partial F}{\partial t} \d t + \delta^t F.
\end{equation*}
так приходим к уравнению
\begin{equation*}
    \tilde{p} \cdot \frac{\partial \tilde{q}}{\partial t}  \d t + 
    \tilde{p} \cdot \partial^t \tilde{q} - \tilde{H} \d t = c p \cdot \d q - c H \d t - \frac{\partial F}{\partial t} \d t - \delta^t F,
\end{equation*}
что приводит к уравнению
\begin{equation}
\label{eqt}
    \boxed{
        \tilde{p} \cdot \delta^t \tilde{q}  -  c p \d q = - \delta^t F
    },
    \text{\ \ --- \ \ критерий канонического преобразования,}
\end{equation}
где $c$ -- \textit{валентность}, соответствующая теореме Ли Хуа-Сжуна, а $F$ -- производящая функция.

\begin{to_thr}[критерий каноничности преобразования]
    Если существует $c$ и $F$ такие, что выполняется \eqref{eqt}, то преобразование $(p, q) \mapsto (\tilde{p}, \tilde{q})$ канонично.
\end{to_thr}

\noindent
Бонусом находим новый Гамильтониан, приравнивая коэффициенты при $\d t$. 
\begin{equation*}
    \tilde{H} = c H + \frac{\partial F}{\partial t} + \tilde{p} \cdot \frac{\partial \tilde{q}}{\partial t}.
\end{equation*}

\subsubsection*{Решим Задачу}
Возьмем такое преобразование
\begin{equation*}
    \left\{\begin{aligned}
        \tilde{q} &= p \tg t \\
        \tilde{p} &= q \ctg t \\
    \end{aligned}\right.
    ,
    \hspace{10 mm}
    H = \frac{qp}{\sin t \, \cos t}.
\end{equation*}
тогда
\begin{equation*}
    \tilde{p} \, \delta^t \tilde{q} = q \ctg t \, \tg t \, \delta^t p = q \delta^t p = \delta^t (qp) - p \delta q,
    \hspace{0.5cm} \Rightarrow \hspace{0.5cm}
    \tilde{p} \cdot \delta^t \tilde{q} - (-1) p \delta q = - \delta (-qp).
\end{equation*}
Теперь можем найти новый гамильтониан
\begin{equation*}
    \tilde{H} = (-1) \frac{pq}{\sin t \, \cos t} + 0 + \frac{q \ctg (t)\,  p}{\cos^2 t} = 0,
\end{equation*}
что очень здорово, ведь в новых переменных $\dot{\tilde{q}} = 0, \, \dot{\tilde{p}} = 0$, что позволило найти первые интегралы системы, а также движение
\begin{equation*}
    \left\{\begin{aligned}
        q &= \beta \tg t, \\
        p &= \alpha \ctg t. \\
    \end{aligned}\right.
\end{equation*}


\subsection{Импульсы не нужны}

Вообще, пусть нам хочется в \textit{$(q, \tilde{q})$ описание} 
\begin{equation*}
    \bigg|
        \frac{\partial \tilde{q}}{\partial p} 
    \bigg| \neq 0,
    \hspace{0.5cm} \Rightarrow \hspace{0.5cm}
    (q, p) \to (q, \tilde{q}).
\end{equation*}
Отличная идея! Теперь $F \to S(q, \tilde{q}, t)$ -- производящая функция. К слову, такие преобразования называются \textit{свободными}. Кусок вывода сразу можем выкинуть и перейти к
\begin{equation*}
    \tilde{p} \cdot d \tilde{q} - \tilde{H} \d t =  c p \cdot \d q - c H \d t - \d S,
\end{equation*}
что дает возможность работать сразу работать с независимыми дифференциалами
\begin{equation*}
    \left\{\begin{aligned}
        \tilde{p} &= - \partial_{\tilde{q}} S \\
        p &= c^{-1} \partial_q S \\
    \end{aligned}\right.
    \text{\ \ --- \ \ критерий каноничности в \textit{$(q, \tilde{q})$ описание}}.
\end{equation*}
Ещё и гамильтониан, как раньше, находим
\begin{equation*}
    \tilde{H} = c H + \partial_t S.
\end{equation*}


\subsubsection*{Задача 2}
Есть преобразование
\begin{equation*}
    \left\{\begin{aligned}
        q &= \sqrt{2 \tilde{p}} \cos \tilde{q} \\
        p &= \sqrt{2 \tilde{p}} \sin \tilde{q} \\ 
    \end{aligned}\right.
    ,
    \hspace{5 mm}
    H = \frac{q^2 + p^2}{2}.
\end{equation*}
Теперь 
\begin{equation*}
    p = q \tg \tilde{q}, \hspace{5 mm}
    \tilde{p} = \frac{q^2}{2 \cos^2 \tilde{q}},
\end{equation*}
и теперь, интегрируя,
\begin{equation*}
    \left\{\begin{aligned}
        \frac{q^2}{2} (1 + \tg^2 \tilde{q}) &= - \partial_{\tilde{q}} S, \\
        q \tg \tilde{q} &= c^{-1} \partial_q S,
    \end{aligned}\right.
    \hspace{0.5cm} \Rightarrow \hspace{0.5cm}
    S = - \l(
        \frac{q^2}{2} \tg \tilde{q}
    \r), \hspace{5 mm} c = -1.
\end{equation*}
И новый гамильтониан
\begin{equation*}
    \tilde{H} = - H = -\tilde{p}.
\end{equation*}
Заметим, что тут уже не восстановить Лагранжиан, -- преобразования Лежандра вырожденно. И уравнения движения
\begin{align*}
    q &= \sqrt{2 \alpha} \cos (-t + \beta) \\
    p &= \sqrt{2 \alpha} \sin (-t + \beta).
\end{align*}


\subsection{Симплектическая геометрия (симплектология)}

% см. Арнольда
% Герман Вель
% Гильберт, Нетер, Вель в Нетингеме 

Гамильтонова система -- набор $(M, \omega, H)$, где $\dim M = 2 n$ -- конфигурационное многообразие, $H$ -- гамильтониан, $\omega$ -- симплектическая 2-форма.

% возможно, форма -- что-то вроде ориентируемой площади.

\begin{to_thr}[Теорема Дарбу]
    Всегда локально есть такие переменные, что 2-форма принимает канонический вид
    \begin{equation*}
        \omega = \sum_{i=1}^{n} \d p_i \wedge \d q_i.
    \end{equation*}
\end{to_thr}

\noindent
К слову, $H\colon M \times \mathbb{R}_t \to \mathbb{R}$. Также точка $x \in M$ такая, что
\begin{equation*}
    \xi \in T_x M, \hspace{5 mm}
    \omega_\xi (\ldots) = \omega(\ldots, \xi) \in T^*_x M,
\end{equation*}
что позволяет задать отображение 
\begin{equation*}
    \omega \colon  T_x M \to T_x^* M.
\end{equation*}
Если сделаем для всех $x \in M$, то 
\begin{equation*}
    \omega \colon TM \mapsto T^* M.
\end{equation*}
Аналогично можем построить
\begin{equation*}
    J \colon T^*M \mapsto TM.
\end{equation*}
Таким образом, считая $dH$ 1-формой,
\begin{equation*}
    J(\d H) \in TM,
    \text{\ \ -- \ \ гамильтоново векторное поле},
\end{equation*}
что позволяет прийти к
\begin{equation*}
    \dot{x} = J(\d H) (x),
    \hspace{10 mm}
    J = \begin{bmatrix}
        0 & E \\
        -E & 0 \\
    \end{bmatrix}.
\end{equation*}
\texttt{Как мы до такой жизни дошли? Было же трёхмерное пространство, время, которое на часах смотрим, наб- -людаемые переменные, но как только мы захотели обобщить, выработать общий метод, пришлось прийти к $n$-мерному конфигурационному многообразию, --- деятельности воспаленного мозга.} 

\red{
Lorem ipsum dolor sit amet, consectetur adipisicing elit, sed do eiusmod
tempor incididunt ut labore et dolore magna aliqua. Ut enim ad minim veniam,
quis nostrud exercitation ullamco laboris nisi ut aliquip ex ea \textit{commodo}
consequat. Duis aute irure dolor in reprehenderit in voluptate velit \textbf{esse}
cillum dolore eu fugiat nulla pariatur. Excepteur sint occaecat cupidatat non
proident, sunt in culpa qui officia deserunt mollit anim id est laborum.
}