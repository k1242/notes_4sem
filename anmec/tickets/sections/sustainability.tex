\sbsnum{1}{Устойчивость в принципе}
% Понятие об устойчивости, неустойчивости и асимптотической устойчивости движения. Формулировка теоремы Ляпунова об устойчивости по первому приближению для установившихся движений. Критерий Рауса-Гурвица. Понятие о критических случаях в теории устойчивости.

\subsubsection*{Возмущенное движение}
что-то

% \subsubsection*{Функции Ляпунова} %x
% Для простоты будем изучать толко установившиеся движения.
В уравнениях возмущенного движения \eqref{231_5} функции $X_i$ будем считать непрерывными в области
\begin{equation}
	|x_i| < H(= \сonst) \ (i = 1,2,\ldots,m),
	\label{232_9}
\end{equation}
и такими, что уравнения \eqref{231_5} при начальных значениях $x_{i0}$ из области \eqref{232_9} допускают единственное решение.

\begin{to_def}[Функция Ляпунова]
	В области $|x_i| < h$, где $h>0$ --- достаточно малое число, будем рассматривать функции \textit{Ляпунова} $V(x_1, x_2, \ldots, x_m)$, предполагая их непрерывно дифференцируемыми, однозначными и обращающимися в нуль в начале координат $x_1 = x_2= \ldots = x_m = 0$.
\end{to_def}
\begin{to_def}
	\textit{Производной} $d V/ d t$ функции $V$ в силу уравнений возмущенного движения \eqref{231_5} называется:
	\begin{equation*}
	 	\frac{d V}{d t} = \sum_{i = 1}^m \frac{\partial V}{\partial x^i} X_i.
	 \end{equation*} 
	 Таким образом производная от функции Ляпунова также $g(x_1,x_2, \ldots, x_m)$, будет непрерывной в той же облачи и обращаться в нуль в начале координат.
\end{to_def}

\begin{to_def}
	$V(x_1, x_2, \ldots, x_m)$ назовём \textit{определенно-положительной} в области $|x_i|<h$, если всюду в этой облачи, кроме начал координат верно: $V >0$.
	Аналогично с определенной отрицательно. В обоих случаях функция $V$ называется \textit{знакоопределенной}.
\end{to_def}

\begin{to_def}
	Если в области $x_i < h$ функция $V$ может принимать значения только одного знака, но может обращаться в нуль не толко в начале координат, то она называется \textit{знакопостоянной}.
\end{to_def}

\begin{to_def}
	Наконец, если функция может принимать в области как значения большие нуля, так и меньшие, она называется \textit{знакопеременной}. 
\end{to_def}

% \subsubsection*{Основные теоремы прямого метода Ляпунова} %x
% % \subsubsection*{Теорема Ляпунова об устойчивости движения}


Здесь и далее для простоты рассматриваем установившееся движение.

\begin{to_thr}[Теорема Ляпунова об устойчивости]
    Если дифференциальные уравнения возмущенного движения таковы, что существует знакоопределенная функция $V$, производная которой $\dot{V}$ в силу этих уравнений является знакопостоянной функцией противоположного знака с $V$, или тождественно равной нулю, то \red{невозмущенное движение} устойчиво.
\end{to_thr}










% \begin{to_thr}[Теорема Ляпунова об асимптотической устойчивости]
    Если дифференциальные уравнения возмущенного движения таковы, что существует знакоопределенная функция $V(x_1, x_2, \ldots, x_m)$, производная которой $\dot{V}$ в силу этих уравнений есть знакоопределенная функция противоположного знака с $V$, то невозмущенное движение асимптотически устойчиво.
\end{to_thr}



% \subsubsection*{Теоремы о неустойчивости}

\begin{to_def}
    Окрестностью положения равновесия, считая, что положение равновесия находится в точке $q^1=\ldots=q^n=0$, назовём область такую, что
    \begin{equation*}
        |q^i| < h, \hspace{1 cm} (i=1, 2,\ldots,m).
    \end{equation*}
\end{to_def}

\begin{to_def}
    Областью $V > 0$ назовём какую-либо область окрестности положения равновесия, в которой $V(x_1, x_2, \ldots, x_m) > 0$. Поверхность $V = 0$ назовём границей области $V >0$.
\end{to_def}

\begin{to_thr}[Теорема Читаева о неустойчивости]
    Если дифференциальные уравнения возмущенного движения таковы, что существует функция $V(x_1, \ldots, x_m)$ такая, что в сколь угодно малой окрестности положения равновесия существует область $V > 0$ и во всех точках области $V > 0$ производная $\dot{V}$ в силу уравнений принимает положительные значения, \textbf{то} невозмущенное движение неустойчиво.
\end{to_thr}

\begin{to_def}
    Функцию $V$, удовлетворяющую теореме Читаева о неустойчивости, называют \textit{функцией} \textit{Читаева}.
\end{to_def}



\begin{to_thr}[I теорема Ляпунова о неустойчивости движения]
    Если дифференциальные уравнения возмущенного движения таковы, что существует функция $V(x_1, \ldots, x_m)$ такая, что ее производная $\dot{V}$ в силу этих уравнений есть функция знакоопределенная, сама функция $V$ не является знакопостоянной, противоположного с $\dot{V}$ знака, то невозмущенное движенние неустойчиво.
\end{to_thr}

\begin{to_thr}[II теорема Ляпунова о неустойчивости движения]
    Если дифференциальные уравнения возмущенного движения таковы, что существует функция $V$ такая, что её производная, в силу этих уравнений, в области положения равновесия может быть представлена в виде
    \begin{equation*}
        \dot{V} = \textnormal{\ae} V + W,
    \end{equation*}
    где \textnormal{\ae} -- положтельная постоянная, а $W$ \textbf{или} тождественно обращается в нуль, \textbf{или} представляет собой знакопостоянную функцию. Если $W$ -- знакопостоянная функция, а $V$ \textbf{не является}  знакопостоянной функцией: $W V < 0$, \textbf{то} невозмущенное движение неустойчиво (ну и если $w \equiv 0$).
\end{to_thr}

\subsubsection*{Усточивость по первому приближению (I)}
Запишем уравнения установившегося возмущенного движения в виде
\begin{equation}
	\frac{d \vc{x}}{d t} = A \vc{x} + \vc{X}(\vc{x}).
	\label{236_1}
\end{equation}
Функции $X_i$ будем считать аналитическими в окрестности начала координат, причем их разложения в ряды начинаются с членов не ниже второго порядка малости относительно $x_1, x_2, \ldots, x_m$.

Вопрос об устойчивости движения очень часто исследуется при помощи уравнений первого приближения:
\begin{equation}
	\frac{d \vc{x}}{d t} = A \vc{x},
	\label{236_2}
\end{equation}
которые получаются из полных уравнений возмущенного движения \eqref{236_1} при отбрасывании в последних нелинейных относительно $x_1, x_2, \ldots, x_m$ членов.

Можно составить характеристическое уравнение
\begin{equation}
	\det(A - \lambda E) = 0,
	\label{236_3}
\end{equation}
которое в общем виде даст решение $\vc{x} = \sum_{j=1}^m c_j \vc{j}_j e^{\lambda_j t}$ (плюс квазимногочлены, но опустим).


Однако как правило уравнения возмущенного движения нелинейны. Поэтому возникает задача об определении условий, при которых выводы об устойчивости, полученные из анализа уравнений первого приближения \eqref{236_2}, справедливы и для полных уравнений возмущенного движения \eqref{236_1} при любых нелинейных членах $X_i$. Эта задача была полностью решена Ляпуновым.

\subsubsection*{Устойчивость по первому приближению (II)}
\begin{to_thr}
	Пусть $\lambda_i$ -- корни уравнения $\det(A - \lambda E) = 0$:
	\begin{itemize*}
		\item[1.] Если $\forall \lambda_i$ $\Re \lambda_i < 0$, то невозмущенное движение асимптотически устойчиво независимо от нелинейных членов.
		% $\frac{d \smallvc{x}}{d t} = A \vc{x} + \vc{X}(\vc{x}).$. 
		\item[2.] Если же $\exists \lambda_i \colon  \Re \lambda_i >0$, то возмущенное движение неустойчиво -- тоже независимо от нелинейных членов. 
		\item[3.] Если же $\exists \lambda_i \colon  \Re \lambda_i = 0$, то подбирая нелинейные члены можно показать, что положение как устойчиво, так и неустойчиво.
	\end{itemize*}
\end{to_thr}

\red{Здесь появится доказательство.}

\subsubsection*{Критерий Рауса-Гурвица}
Запишем уравнение \eqref{236_3} в виде
\begin{equation}
	a_0 \lambda^m + a_1 \lambda^{m-1} + \ldots + a_{m-1}\lambda + a_m = 0.
	\label{238_14}
\end{equation}
Коэффициенты $a_0, a_1,\ldots,a_m$ этого уравнения --- вещественные числа. Без ограничения общности $a_0 >0$.

По теореме Виета имеем:
\begin{gather*}
	\frac{a_1}{a_0} = - (\lambda_1 + \lambda_2 + \ldots + \lambda_m),\\
	\frac{a_2}{a_0} = \lambda_1 \lambda_2 + \ldots + \lambda_{m-1} \lambda_m,\\
	\vdots\\
	\frac{a_m}{a_0} = (-1)^m \lambda_1 \lambda_2 \ldots \lambda_m.
\end{gather*}
Таким образом для отрицательности всех вещественных частей корней $\lambda_1, \lambda_2, \ldots, \lambda_m$ необходимо чтобы все его коэффициенты были положительны.

Однако такого утверждения не достаточно. Необходимое и достаточное условие дается критерием Рауса-Гурвица.

\begin{to_def}
	Назовем \textit{матрицей Гурвица}:
	\begin{equation*}
		\begin{pmatrix}
		    a_1 & a_3 & a_5 & \ldots & 0 \\
		    a_0 & a_2 & a_4 & \ldots & 0 \\
		    0 & a_1 & a_3 & \ldots & 0 \\
		    0 & a_0 & a_2 & \ldots & 0 \\
		    \vdots &  & & \ddots & \\
		    & & & & a_m
		\end{pmatrix}
	\end{equation*} 
	% Рекомендуем читателям самостоятельно разобраться в правилах её построения (мнемонических и нет).
\end{to_def}

Рассмотрим главные миноры матрицы Гурвица (\textit{определители Гурвица}):
\begin{equation*}
	\Delta_1 = a_1, 
	\ 
	\Delta_2 = \begin{vmatrix} a_1& a_3\\ a_0&a_2\end{vmatrix},
	\
	\Delta_3 = \begin{vmatrix}
	    a_1 & a_3 & a_5 \\
	    a_0 & a_2 & a_4 \\
	    0 & a_1 & a_3 \\
	\end{vmatrix},
	\ \ldots \ 
	\Delta_m = a_m \Delta_{m-1}.
\end{equation*}

\begin{to_thr}[Критерий Рауса-Гурвица]
	Для того, чтобы все корни характерестического уравнения с вещественными коэффициентами и положительным старшим $a_0$ имели отрицательные вещественные части, необходимо и достаточно, чтобы выполнялись неравенства:
	\begin{equation}
		\Delta_1 >0,
		\
		\Delta_2 > 0,
		\ \ldots \ 
		\Delta_m >0.
	\end{equation}
	Если же хотя бы одно из неравенств имеет противоположный смысл, то характерестическое уравнение имеет корни, вещественные части которых положительны.
\end{to_thr}

% \subsubsection*{Влиянение диссипативных и гироскопических сил на устойчивость равновесия консервативной системы} %x
% 

\begin{to_thr}[Теорема Томсона-Тэта-Четаева]
    Если в некотором изолированном положении равновесия потенциальная энергия имеет строгий локальный минимум, то при добавлении гироскопических и диссипативных сил с полной диссипацией это положение равновесия становится асимптотически устойчивым.
\end{to_thr}
















% \subsubsection*{Влияние гироскопических и диссипативных сил на
неустойчивое равновесие}

Разложим до квадратичных членов кинетическую и потенциальную энергию системы, и приведем к каноническому виду
\begin{equation*}
    T = \frac{1}{2}\sum_{i=1}^n \dot{\theta}_i^2,
    \hspace{1 cm}
    \Pi = \frac{1}{2} \sum_{i=1}^{n} \lambda_i \theta_i^2.
\end{equation*}
Если $\Pi$ положительно определена, то все величины $\lambda_i$ положительны, и положение устойчиво. Если же присутствуют отрицательные $\lambda_i$, то положение равновесия неустойчиво (\red{по теореме о неустойчивости по первому приближению}). 

\begin{to_def}
    Величины $\lambda_i$ Пуанкаре предложил называть \textit{коэффициентами устойчивости}. Число отрицательных коэффициентов устойчивости называется \textit{степенью неустойчивости}. 
\end{to_def}


\begin{to_thr}[]
    \textbf{Если} среди коэффициентов устойчивости хотя бы один является отрицательным, \textbf{то} изолированное положение равновесия не может быть стабилизировано диссипативными силами с полной диссипацией.
\end{to_thr}

\begin{to_thr}[]
    \textbf{Если} степень неустойчивости изолированного положения равновесия консервативной системы нечетна, \textbf{то} стабилизация его добавлением гироскопических сил невозможна. \textbf{Если} степень неустойчивости четна, \textbf{то} гироскопическая стабилизация возможна.
\end{to_thr}

\begin{to_thr}[]
    \textbf{Если} изолированное положение равновесия консервативной системы имеет отличную от нуля степень неустойчивости, \textbf{то} оно остается неустойчивым при добавлении гироскопических сил и диссипативных сил с полной диссипацией.
\end{to_thr}

\begin{to_def}
    Устойчивость, существующую при одних потенциальных силах, называют \textit{вековой}, а устойчивость, полученную с помощью гироскопических сил, -- \textit{временной}.
\end{to_def}


\sbsnum{2}{Устойчивость консервативной системы  (thr Лагранжа, thr Ляпунова)}
% Теорема Лагранжа об устойчивости положения равновесия консервативной системы. Теоремы Ляпунова об обращении теоремы Лагранжа.

% \subsubsection*{Теорема Лагранжа об устойчивости положения равновесия}
% \subsubsection*{Устойчивость равновесия}


\green{
\begin{to_thr}[Общее уравнение статики\footnote{
    Если с необходимостью всё понятно, то достаточность \red{может быть доказана через уравнения Аппеля (см. п. 158, Маркеев П. А.)}.
}]
    Чтобы некоторое допускаемое идеальными удерживающими связями состояние равновесия системы было состоянием равновесия на интервале $t_0 \leq t \leq t_1$, необходимо и достаточно, чтобы для любого момента времени из этого интервала элементарная работа активных сил на любом виртуальном перемещении равнялась нулю, т.е. чтобы выполнялось
    \begin{equation*}
         \sum_{\nu=1}^N \vc{F}_\nu \cdot \delta \vc{r}_\nu = 0 
         \hspace{1cm}
         (t_0 \leq t \leq t_1).
     \end{equation*} 
     Если система является потенциальной, то уравнения примут вид
     \begin{equation*}
         Q_i = - \frac{\partial \Pi}{\partial q^i} = 0.
     \end{equation*}
\end{to_thr}
}

\begin{to_def} 
    Положение равновесия $q=0$ -- \textit{устойчиво по Ляпунову}, если $\forall \varepsilon > 0 \ \ \exists \delta > 0$, такая что 
\begin{equation}
    \forall \ \ |q(t_0)|<\delta, \ |\dot{q}(t_0)|<\delta \colon
    \hspace{0.5cm} 
    |q(t)|<\varepsilon, \ |\dot{q}(t)| < \varepsilon, \hspace{0.5cm} \forall t \geq t_0.
\end{equation}
\end{to_def}

\begin{to_def} 
    Положение равновесия $q=0$ -- \textit{неустойчиво по Ляпунову}, если $\exists \varepsilon > 0 \ \ \forall \delta > 0$, такая что 
\begin{equation}
    \forall \delta > 0 \ \  \exists |q(t_0)| < \delta, \
    |\dot{q}(t_0)| < \delta, \ \ t^* \colon \hspace{0.5cm} 
    |q(t^*)| > \varepsilon \text{ или } |\dot{q}(t^*)| > \varepsilon.
\end{equation}
\end{to_def}























 %x
\subsubsection*{Теорема Лагранжа}

\begin{to_thr}[Теорема Лагранжа-Дирихле]
     Если в положении равновесия конесервативной системы $\Pi(q)$ имеет строгий локальный минимум, то это положение равновесия устойчиво.
\end{to_thr}

\begin{to_lem} 
    При наличии гироскопических и диссипативных сил положение равновесия сохранится. 
\end{to_lem}




\subsubsection*{Теоремы Ляпунова о неустойчивости положения равновесия консервативной системы}


\begin{to_thr}[Теорема Ляпунова о неустойчивости I]
    Если в положении равновесия $\Pi(q)$ не имеет минимума и это определяется по квадратичной форме её разложения в ряд (в окрестности положения равновесия), то это положение равновесия неустойчиво.     
\end{to_thr}


\begin{to_thr}[Теорема Ляпунова о неустойчивости II]
    Если в положении равновесия $\Pi(q)$  имеет строгий максимум и это определяется по наинизшей степени её разложения в ряд (в окрестности положения равновесия), то это положение равновесия неустойчиво.     
\end{to_thr}










