Рассмотрим консервативную энергию с $n$ степенями свободы. Кинетическая энергия -- $T = \frac{1}{2} a_{ik} \dot{q}^i \dot{q}^k$. Введём в координатном пространстве метрику
\begin{equation*}
    ds^2 = 2 T dt^2 = a_{ik} \d q^i \d q^k,
    \hspace{0.5cm} \Rightarrow \hspace{0.5cm} 
    T = \frac{1}{2} \left(\frac{d s}{d t} \right)^2,
\end{equation*}
т. е. в такой метрике $T$ системы равная изображающей точки в координатном пространстве, если считать, что изображающая точка обладает массой, равной единице.

Если система движется по инерции, т.е. $\Pi = 0$, то из интеграла $T + \Pi = h = \const$ и тогда
\begin{equation*}
    \frac{d s}{d t} = \sqrt{2 h},
    \hspace{0.5cm} \Rightarrow \hspace{0.5cm} 
    W = 2 \int_{t_0}^{t_1} T \d t = 2 h (t_1 - t_0) = \sqrt{2h} l,
\end{equation*}
где $l = \sqrt{2h} (t_1-t_0)$ -- длина кривой, пройденной за время $t_1-t_0$. Из \textit{принципа Якоби} следует, что $\delta l = 0$, т.е. задача свелась к задаче дифгема о поиска геодезической.

Пусть теперь движение в потенциальном поле $(\Pi \neq 0)$. Тогда функция Якоби $P$:
\begin{equation*}
    W = \int_{q^{1}_1}^{q_1^1} P \d q^1 = 
    2 \int_{q^{1}_1}^{q_1^1}
    \sqrt{ (h - \Pi) G} \d q^1 
    =
    \sqrt{2} \int_{q^{1}_1}^{q_1^1} 
    \sqrt{
        (h-\Pi) a_{ik} \d q^i \d q^k
    }.
\end{equation*}
Область движения ограничена $\Pi \leq h$, так что введём новую метрику $d \sigma^2$, по формуле
\begin{equation*}
    d \sigma^2 = (h - \Pi) a_{ik} \d q^i \d q^k,
    \hspace{0.5cm} \Rightarrow \hspace{0.5cm} 
    W = \sqrt{2} \sigma,
\end{equation*}
где $\sigma$ -- длина дуги. Теперь нахождение траекторий снова свелось к нахождению геодезической!)

Далее рассмотрим две задачи, раскрывающих эту тему.













