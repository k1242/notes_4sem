 \begin{enumerate}
 \item Понятие об устойчивости, неустойчивости и асимптотической устойчивости движения. Формулировка теоремы Ляпунова об устойчивости по первому приближению для установившихся движений. Критерий Рауса-Гурвица. Понятие о критических случаях в теории устойчивости.

\item Теорема Лагранжа об устойчивости положения равновесия консервативной системы. Теоремы Ляпунова об обращении теоремы Лагранжа.

\item Линеаризация уравнений движения консервативной системы в окрестности ее положения равновесия. Нормальные координаты и нормальные колебания.

\item Колебания консервативной системы под действием внешних периодических сил. Резонанс в вынужденных колебаниях. Влияние внешних периодических сил на малые колебания склерономной системы.

\item Классификация особых точек на плоскости. Предельные циклы. Понятие об автоколебаниях.

\item Фазовая плоскость консервативной системы с одной степенью свободы. Равновесия, периодические движения, сепаратрисы.

\item Элементы теории бифуркаций: бифуркации «смена устойчивости» и «седло–узел», бифуркация «вилки».

\item Элементы теории бифуркаций: бифуркация Андронова–Хопфа рождения (исчезновения) цикла.

\item Понятие о методе нормальных форм в теории нелинейных колебаний.

\item Понятие о методе усреднения. Построение первого приближения по малому параметру для дифференциальных уравнений в стандартной форме.

\item Канонические уравнения Гамильтона. Физический смысл функции Гамильтона. Обобщенно консервативные системы. Интеграл Якоби.

\item Уравнения Уиттекера для консервативных и обобщенно консервативных систем. Время и энергия как канонически сопряженные переменные.

\item Уравнения Рауса. Понижение порядка системы дифференциальных уравнений движения при помощи уравнений Рауса в случае существования циклических координат. Приведенный потенциал.

\item Скобки Лагранжа. Скобки Пуассона и их свойства. Скобки Пуассона и первые интегралы. Теорема Якоби–Пуассона.

\item Понятие канонического преобразования. Симплектичность (или обобщенная симплектичность) матрицы Якоби преобразования – необходимое и достаточное условие его каноничности. Критерии каноничности преобразования, выраженные через скобки Лагранжа, скобки Пуассона и посредством дифференциальной формы.

\item Канонические преобразования и процесс движения. Теорема Лиувилля о сохранении фазового объема. Инвариантность скобок Пуассона при канонических преобразованиях.

\item Свободное каноническое преобразование и его производящая функция. Каноническое преобразование с производящей функцией, зависящей от старых координат и новых импульсов. Получение новой функции Гамильтона при каноническом преобразовании.

\item Уравнение Гамильтона–Якоби. Полный интеграл. Теорема Якоби.

\item Разделение переменных в уравнении Гамильтона–Якоби. Примеры.

\item Переменные действие–угол для системы с одной степенью свободы. Понятие о переменных действие–угол для систем с несколькими степенями свободы.

\item Понятие интегрируемости гамильтоновых систем. Теорема Лиувилля об интегрируемости гамильтоновых систем в квадратурах. Представление движения на инвариантных торах.

\item Классическая теория возмущений. Нерезонансный и резонансный случаи в теории возмущений. Проблема малых знаменателей.

\item Преобразование Биркгофа.

\item Канонические преобразования, близкие к тождественным и их применение в теории возмущений (на примере маятника, точка подвеса которого совершает периодические вертикальные вибрации).

\item Параметрический резонанс в гамильтоновой системе с одной степенью свободы (на примере уравнения Матье).

\item Понятие адиабатического инварианта. Теорема Арнольда о вечном сохранении адиабатического инварианта в периодической по времени гамильтоновой системе с одной степенью свободы (без доказательства).

\item Понятие интегрального инварианта. Теорема об универсальном интегральном инварианте Пуанкаре и ее обращение. Теорема Ли Хуа-чжуна о единственности интегрального инварианта Пуанкаре (без доказательства).

\item Теорема об интегральном инварианте Пуанкаре–Картана и ее обращение.

\item Регулярные и хаотические аттракторы. Детерминированный хаос. Метод поверхностей сечения Пуанкаре. Понятие о фрактале и фрактальной размерности множеств.

\item Логистическое (квадратичное) отображение. Сценарий перехода к хаосу через каскад бифуркаций удвоения периода. Универсальности Фейгенбаума.

\item Интегрируемые системы Гамильтона. Понятие об их невырожденности и изоэнергетической невырожденности.

\item Формулировка основной теоремы КАМ-теории (теории Колмогорова–Арнольда–Мозера) для гамильтоновых систем, близких к интегрируемым. Понятие о механизме разрушения инвариантных торов.

\item Замена обобщенных координат и времени в уравнениях Лагранжа второго рода. Теорема Э. Нетер.

\item Теорема Э. Нетер. Связь законов сохранения (первых интегралов) со свойствами пространства и времени
\end{enumerate}