Запишем уравнение \eqref{236_3} в виде
\begin{equation}
	a_0 \lambda^m + a_1 \lambda^{m-1} + \ldots + a_{m-1}\lambda + a_m = 0.
	\label{238_14}
\end{equation}
Коэффициенты $a_0, a_1,\ldots,a_m$ этого уравнения --- вещественные числа. Без ограничения общности $a_0 >0$.

По теореме Виета имеем:
\begin{gather*}
	\frac{a_1}{a_0} = - (\lambda_1 + \lambda_2 + \ldots + \lambda_m),\\
	\frac{a_2}{a_0} = \lambda_1 \lambda_2 + \ldots + \lambda_{m-1} \lambda_m,\\
	\vdots\\
	\frac{a_m}{a_0} = (-1)^m \lambda_1 \lambda_2 \ldots \lambda_m.
\end{gather*}
Таким образом для отрицательности всех вещественных частей корней $\lambda_1, \lambda_2, \ldots, \lambda_m$ необходимо чтобы все его коэффициенты были положительны.

Однако такого утверждения не достаточно. Необходимое и достаточное условие дается критерием Рауса-Гурвица.

\begin{to_def}
	Назовем \textit{матрицей Гурвица}:
	\begin{equation*}
		\begin{pmatrix}
		    a_1 & a_3 & a_5 & \ldots & 0 \\
		    a_0 & a_2 & a_4 & \ldots & 0 \\
		    0 & a_1 & a_3 & \ldots & 0 \\
		    0 & a_0 & a_2 & \ldots & 0 \\
		    \vdots &  & & \ddots & \\
		    & & & & a_m
		\end{pmatrix}
	\end{equation*} 
	% Рекомендуем читателям самостоятельно разобраться в правилах её построения (мнемонических и нет).
\end{to_def}

Рассмотрим главные миноры матрицы Гурвица (\textit{определители Гурвица}):
\begin{equation*}
	\Delta_1 = a_1, 
	\ 
	\Delta_2 = \begin{vmatrix} a_1& a_3\\ a_0&a_2\end{vmatrix},
	\
	\Delta_3 = \begin{vmatrix}
	    a_1 & a_3 & a_5 \\
	    a_0 & a_2 & a_4 \\
	    0 & a_1 & a_3 \\
	\end{vmatrix},
	\ \ldots \ 
	\Delta_m = a_m \Delta_{m-1}.
\end{equation*}

\begin{to_thr}[Критерий Рауса-Гурвица]
	Для того, чтобы все корни характерестического уравнения с вещественными коэффициентами и положительным старшим $a_0$ имели отрицательные вещественные части, необходимо и достаточно, чтобы выполнялись неравенства:
	\begin{equation}
		\Delta_1 >0,
		\
		\Delta_2 > 0,
		\ \ldots \ 
		\Delta_m >0.
	\end{equation}
	Если же хотя бы одно из неравенств имеет противоположный смысл, то характерестическое уравнение имеет корни, вещественные части которых положительны.
\end{to_thr}