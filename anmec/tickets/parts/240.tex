\subsubsection*{Влияние гироскопических и диссипативных сил на
неустойчивое равновесие}

Разложим до квадратичных членов кинетическую и потенциальную энергию системы, и приведем к каноническому виду
\begin{equation*}
    T = \frac{1}{2}\sum_{i=1}^n \dot{\theta}_i^2,
    \hspace{1 cm}
    \Pi = \frac{1}{2} \sum_{i=1}^{n} \lambda_i \theta_i^2.
\end{equation*}
Если $\Pi$ положительно определена, то все величины $\lambda_i$ положительны, и положение устойчиво. Если же присутствуют отрицательные $\lambda_i$, то положение равновесия неустойчиво (\red{по теореме о неустойчивости по первому приближению}). 

\begin{to_def}
    Величины $\lambda_i$ Пуанкаре предложил называть \textit{коэффициентами устойчивости}. Число отрицательных коэффициентов устойчивости называется \textit{степенью неустойчивости}. 
\end{to_def}


\begin{to_thr}[]
    \textbf{Если} среди коэффициентов устойчивости хотя бы один является отрицательным, \textbf{то} изолированное положение равновесия не может быть стабилизировано диссипативными силами с полной диссипацией.
\end{to_thr}

\begin{to_thr}[]
    \textbf{Если} степень неустойчивости изолированного положения равновесия консервативной системы нечетна, \textbf{то} стабилизация его добавлением гироскопических сил невозможна. \textbf{Если} степень неустойчивости четна, \textbf{то} гироскопическая стабилизация возможна.
\end{to_thr}

\begin{to_thr}[]
    \textbf{Если} изолированное положение равновесия консервативной системы имеет отличную от нуля степень неустойчивости, \textbf{то} оно остается неустойчивым при добавлении гироскопических сил и диссипативных сил с полной диссипацией.
\end{to_thr}

\begin{to_def}
    Устойчивость, существующую при одних потенциальных силах, называют \textit{вековой}, а устойчивость, полученную с помощью гироскопических сил, -- \textit{временной}.
\end{to_def}