

\subsubsection*{Скобки Пуассона}

\begin{to_def} 
    Пусть $u$, $v$ $\in C^2(q, p, t)$, тогда выражение
    \begin{equation*}
        \{u, v\}  
        =
        \frac{\partial u}{\partial q^i} \frac{\partial v}{\partial p_i} -
        \frac{\partial u}{\partial p_i} \frac{\partial v}{\partial q^i}
    \end{equation*}  
    называют \textit{скобкой Пуассона} функций $u$ и $v$.
\end{to_def}

Вообще, можно было бы ввести алгебры Ли и показать, что пространство гладких функций $f(t, x, p)$ является алгеброй Ли относительно скобки Пуассона. Выражается это в выполнение следующих свойств:
\begin{enumerate}
    \item $\{y, x\} = - \{x, y\}, \ \ \forall x, y \in C^2$ (кососимметричность);
    \item $\{\lambda_1 x_1 + \lambda_2 x_2, y\} = \lambda_1 \{x_1, y_1\} + \lambda_2 \{x_2, y\},
    \ \ \forall \lambda_1, \lambda_2 \in \mathbb{R}
    $ (линейность по первому аргументу);
    \item $\{x, \{y, z\} + \{y, \{z, x\}\} + \{z, \{x, y\}\} = 0$ (тождество Якоби).
\end{enumerate}

\begin{to_def} 
    \textit{Производной} функции $f(t, q, p)$ \textit{в силу гамильтоновой системы} в точке $(t_0, x_0, p^0)$ называется
    \begin{equation*}
         \frac{d f(t_0, x_0, p^0)}{d t} \overset{\mathrm{def}}{=} \frac{d }{d t} 
         \left(
            f(t, q(t), p(t)) \big|_{t=t_0}
         \right),
     \end{equation*} 
     где $q(t)$ и $p(t)$ -- решения гамильтоновой системы с н.у. $q(t_0) = q_0$ и $p(t_0) = p^0$. 
\end{to_def}

Выразим производную в силу системы через скобку Пуассона:
\begin{equation*}
    \frac{d f}{d t} 
    =
     \frac{\partial f}{\partial t} + \frac{\partial f}{\partial q^i} \frac{d q^i}{d t} + \frac{\partial f}{\partial p_i} \frac{d p_i}{d t} 
     =
      \frac{\partial f}{\partial t} + \frac{\partial f}{\partial q^i} \frac{\partial H}{\partial p_i} - \frac{\partial f}{\partial p_i} \frac{\partial H}{\partial q^i} = \frac{\partial f}{\partial t} + \{H, f\},
      \hspace{0.5cm} \Rightarrow \hspace{0.5cm} 
      \frac{d f}{d t} = \frac{\partial f}{\partial t} + \{H, f\}.
\end{equation*}

\begin{to_lem} 
    Уравнение вида 
    \begin{equation*}
         \frac{d f}{d t} = \frac{\partial f}{\partial t} + \{H, f\} = 0
     \end{equation*} 
     является необходимым и достаточным условием того, что $f(t, q, p)$ являлась бы первым интегралом гамильтоновой системы.
\end{to_lem}

\begin{to_thr}[теорема Пуассона]
     Если $f$ и $g$ -- два интеграла движения, то $\{f, g\} = \const$ также является интегралом движения\footnote{
         \red{Докажи!}
     }.
\end{to_thr}

\begin{to_def} 
    \textit{Гамильтоновым полем} для функции $f \in C^1$ называется векторное поле $\vc{f}$, определяемое формулой
    \begin{equation*}
        \omega[ \vc{f}(q, p), \vc{v}] = df(q, p) [\vc{v}], \ \ \forall \vc{v} \in T_{q, p},
        \hspace{1cm} 
        \omega = dq^i \wedge dp_i.
    \end{equation*}
    В координатах это выразится в 
    \begin{equation*}
        \vc{f} = \frac{\partial f}{\partial p_i} \frac{\partial }{\partial q^i} - \frac{\partial f}{\partial q^i} \frac{\partial }{\partial p_i}.
    \end{equation*}
    Более того $\vc{f}(\varphi) = \{f, \varphi\}$, где $\varphi$ -- некоторая гладкая функция.
\end{to_def}

\begin{to_thr}[о связи скобки Пуассона и скобки Ли]
     Пусть $f, g \in C^2$. Тогда гамильтоново поле скобки Пуассона $\{f, g\}$ совпадает со скобкой Ли гамильтоновых полей $\vc{g}$ и $\vc{f}$: 
     \begin{equation*}
         \vv{
         \{f, g\}
         } = [\vc{f}, \vc{g}].
     \end{equation*}
\end{to_thr}

