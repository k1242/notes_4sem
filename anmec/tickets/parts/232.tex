Для простоты будем изучать толко установившиеся движения.
В уравнениях возмущенного движения \eqref{231_5} функции $X_i$ будем считать непрерывными в области
\begin{equation}
	|x_i| < H(= \сonst) \ (i = 1,2,\ldots,m),
	\label{232_9}
\end{equation}
и такими, что уравнения \eqref{231_5} при начальных значениях $x_{i0}$ из области \eqref{232_9} допускают единственное решение.

\begin{to_def}[Функция Ляпунова]
	В области $|x_i| < h$, где $h>0$ --- достаточно малое число, будем рассматривать функции \textit{Ляпунова} $V(x_1, x_2, \ldots, x_m)$, предполагая их непрерывно дифференцируемыми, однозначными и обращающимися в нуль в начале координат $x_1 = x_2= \ldots = x_m = 0$.
\end{to_def}
\begin{to_def}
	\textit{Производной} $d V/ d t$ функции $V$ в силу уравнений возмущенного движения \eqref{231_5} называется:
	\begin{equation*}
	 	\frac{d V}{d t} = \sum_{i = 1}^m \frac{\partial V}{\partial x^i} X_i.
	 \end{equation*} 
	 Таким образом производная от функции Ляпунова также $g(x_1,x_2, \ldots, x_m)$, будет непрерывной в той же облачи и обращаться в нуль в начале координат.
\end{to_def}

\begin{to_def}
	$V(x_1, x_2, \ldots, x_m)$ назовём \textit{определенно-положительной} в области $|x_i|<h$, если всюду в этой облачи, кроме начал координат верно: $V >0$.
	Аналогично с определенной отрицательно. В обоих случаях функция $V$ называется \textit{знакоопределенной}.
\end{to_def}

\begin{to_def}
	Если в области $x_i < h$ функция $V$ может принимать значения только одного знака, но может обращаться в нуль не толко в начале координат, то она называется \textit{знакопостоянной}.
\end{to_def}

\begin{to_def}
	Наконец, если функция может принимать в области как значения большие нуля, так и меньшие, она называется \textit{знакопеременной}. 
\end{to_def}