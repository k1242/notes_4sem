Пусть уравнение движение представлено в виде:
\begin{equation}
	\frac{d y_i}{d t} = Y_i(y_1, y_2, \ldots, y_m, t) \hspace{5 mm}  (i= 1,2,\ldots, m).
	\label{231_1}
\end{equation}
Рассмотрим частное движение --- частное решение этой системы c начальными условиями
\begin{align}
	&y_i^* = f_i(t)  
	&(i = 1,2,\ldots,m), 
	\\
	&y_{i 0} = f_i (t_0)  
	&(i = 1, 2, \ldots, m).
	\label{231_2}
\end{align}
Нас будут интересовать движения системы при отклонении от начальных условий $y_{i 0}$ от значений $f_i(t_0)$.
\begin{to_def}
	Движение системы, описываемое \eqref{231_2} называется \textit{невозмущенным} движением.
	Все другие движения механической системы при тех же силах, что и движение \eqref{231_2} --- \textit{возмущенные} движения. 
\end{to_def}
\begin{to_def} \textit{Возмущениями} назовём разности вида:
	\begin{equation}
		x_i = y_i - f_i(t) \hspace{5 mm}  (i = 1,2,\ldots, m).
		\label{231_4}
	\end{equation}
\end{to_def}
\begin{to_def}
	Теперь, произведя замену по формулам \eqref{231_4} в уравнениях \eqref{231_1} получим \textit{дифференциальные уравнения возмущенного движения}:
	\begin{equation}
		\frac{d x_i}{d t} = X_i (x_1, x_2, \ldots, x_m, t) \hspace{5 mm}  (i = 1,2,\ldots,m).
		\label{231_5}
	\end{equation}
Это уравнение имеет частное решение $x_i \equiv 0$ отвечающее невозмущенному движению.
\end{to_def}

\begin{to_def}[]
	Движение называется \textit{установившимся} (система \textit{автономна}), если $X_i \not \equiv X_i(t)$, в противном же случае движение \textit{неустановившееся}.
\end{to_def}

\begin{to_def}[Устойчивость по Ляпунову]
	Невозмущенное движение называется \textit{устойчивым} по отношению к переменным $y_i$, если $\forall \varepsilon>0 \ \exists \delta(\varepsilon) \colon \forall$ возмущенных движений, для которых 
	\begin{equation}
		|x_i (t_0)| < \delta,\ \forall t>t_0
		\hspace{0.2 cm}
		\text{выполняется}
		\hspace{0.2 cm}
		|x_i (t)| < \varepsilon.
		\label{231_6}
	\end{equation}
\end{to_def}

\begin{to_def}[Асимптотическая устойчивость]
	Невозмущенное движение называется \textit{асимптотически устойчивым} по отношению к переменным $y_i$, если оно устойчиво и $\exists \delta$ такие, что для возмущенных движений удовлетворяющим условиям \eqref{231_6} верно:
	\begin{equation}
		\lim_{t \to \infty} x_i (t) = 0 \ (i = 1, 2, \ldots, m).
		\label{231_8}
	\end{equation}
\end{to_def}
