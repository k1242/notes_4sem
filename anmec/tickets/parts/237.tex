\begin{to_thr}
	Если $\forall \lambda_i$ уравнения \eqref{236_3}: $\Re \lambda_i <0$, то невозмущенное движение асимптотически устойчиво независимо от нелинейных членов в \eqref{236_1}. 

	Если же $\exists \lambda_i: \ \Re \lambda_i >0$, то возмущенное движение неустойчиво --- тоже независимо от нелинейных членов в \eqref{236_1}. 

	Если же $\exists \lambda_i: \ \Re \lambda_i = 0$, то подбирая нелинейные члены можно показать, что положение как устойчиво, так и неустойчиво.
\end{to_thr}

\red{((Когда-нибудь здесь появится доказательство))}