\subsubsection*{Устойчивость равновесия}


\green{
\begin{to_thr}[Общее уравнение статики\footnote{
    Если с необходимостью всё понятно, то достаточность \red{может быть доказана через уравнения Аппеля (см. п. 158, Маркеев П. А.)}.
}]
    Чтобы некоторое допускаемое идеальными удерживающими связями состояние равновесия системы было состоянием равновесия на интервале $t_0 \leq t \leq t_1$, необходимо и достаточно, чтобы для любого момента времени из этого интервала элементарная работа активных сил на любом виртуальном перемещении равнялась нулю, т.е. чтобы выполнялось
    \begin{equation*}
         \sum_{\nu=1}^N \vc{F}_\nu \cdot \delta \vc{r}_\nu = 0 
         \hspace{1cm}
         (t_0 \leq t \leq t_1).
     \end{equation*} 
     Если система является потенциальной, то уравнения примут вид
     \begin{equation*}
         Q_i = - \frac{\partial \Pi}{\partial q^i} = 0.
     \end{equation*}
\end{to_thr}
}

\begin{to_def} 
    Положение равновесия $q=0$ -- \textit{устойчиво по Ляпунову}, если $\forall \varepsilon > 0 \ \ \exists \delta > 0$, такая что 
\begin{equation}
    \forall \ \ |q(t_0)|<\delta, \ |\dot{q}(t_0)|<\delta \colon
    \hspace{0.5cm} 
    |q(t)|<\varepsilon, \ |\dot{q}(t)| < \varepsilon, \hspace{0.5cm} \forall t \geq t_0.
\end{equation}
\end{to_def}

\begin{to_def} 
    Положение равновесия $q=0$ -- \textit{неустойчиво по Ляпунову}, если $\exists \varepsilon > 0 \ \ \forall \delta > 0$, такая что 
\begin{equation}
    \forall \delta > 0 \ \  \exists |q(t_0)| < \delta, \
    |\dot{q}(t_0)| < \delta, \ \ t^* \colon \hspace{0.5cm} 
    |q(t^*)| > \varepsilon \text{ или } |\dot{q}(t^*)| > \varepsilon.
\end{equation}
\end{to_def}























