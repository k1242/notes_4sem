Запишем уравнения установившегося возмущенного движения в виде
\begin{equation}
	\frac{d \vc{x}}{d t} = A \vc{x} + \vc{X}(\vc{x}).
	\label{236_1}
\end{equation}
Функции $X_i$ будем считать аналитическими в окрестности начала координат, причем их разложения в ряды начинаются с членов не ниже второго порядка малости относительно $x_1, x_2, \ldots, x_m$.

Вопрос об устойчивости движения очень часто исследуется при помощи уравнений первого приближения:
\begin{equation}
	\frac{d \vc{x}}{d t} = A \vc{x},
	\label{236_2}
\end{equation}
которые получаются из полных уравнений возмущенного движения \eqref{236_1} при отбрасывании в последних нелинейных относительно $x_1, x_2, \ldots, x_m$ членов.

Можно составить характеристическое уравнение
\begin{equation}
	\det(A - \lambda E) = 0,
	\label{236_3}
\end{equation}
которое в общем виде даст решение $\vc{x} = \sum_{j=1}^m c_j \vc{j}_j e^{\lambda_j t}$.


Однако как правило уравнения возмущенного движения нелинейны. Поэтому возникает задача об определении условий, при которых выводы об устойчивости, полученные из анализа уравнений первого приближения \eqref{236_2}, справедливы и для полных уравнений возмущенного движения \eqref{236_1} при любых нелинейных членах $X_i$. Эта задача была полностью решена Ляпуновым.