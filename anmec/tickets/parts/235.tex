\subsubsection*{Теоремы о неустойчивости}

\begin{to_def}
    Окрестностью положения равновесия, считая, что положение равновесия находится в точке $q^1=\ldots=q^n=0$, назовём область такую, что
    \begin{equation*}
        |q^i| < h, \hspace{1 cm} (i=1, 2,\ldots,m).
    \end{equation*}
\end{to_def}

\begin{to_def}
    Областью $V > 0$ назовём какую-либо область окрестности положения равновесия, в которой $V(x_1, x_2, \ldots, x_m) > 0$. Поверхность $V = 0$ назовём границей области $V >0$.
\end{to_def}

\begin{to_thr}[Теорема Читаева о неустойчивости]
    Если дифференциальные уравнения возмущенного движения таковы, что существует функция $V(x_1, \ldots, x_m)$ такая, что в сколь угодно малой окрестности положения равновесия существует область $V > 0$ и во всех точках области $V > 0$ производная $\dot{V}$ в силу уравнений принимает положительные значения, \textbf{то} невозмущенное движение неустойчиво.
\end{to_thr}

\begin{to_def}
    Функцию $V$, удовлетворяющую теореме Читаева о неустойчивости, называют \textit{функцией} \textit{Читаева}.
\end{to_def}



\begin{to_thr}[I теорема Ляпунова о неустойчивости движения]
    Если дифференциальные уравнения возмущенного движения таковы, что существует функция $V(x_1, \ldots, x_m)$ такая, что ее производная $\dot{V}$ в силу этих уравнений есть функция знакоопределенная, сама функция $V$ не является знакопостоянной, противоположного с $\dot{V}$ знака, то невозмущенное движенние неустойчиво.
\end{to_thr}

\begin{to_thr}[II теорема Ляпунова о неустойчивости движения]
    Если дифференциальные уравнения возмущенного движения таковы, что существует функция $V$ такая, что её производная, в силу этих уравнений, в области положения равновесия может быть представлена в виде
    \begin{equation*}
        \dot{V} = \textnormal{\ae} V + W,
    \end{equation*}
    где \textnormal{\ae} -- положтельная постоянная, а $W$ \textbf{или} тождественно обращается в нуль, \textbf{или} представляет собой знакопостоянную функцию. Если $W$ -- знакопостоянная функция, а $V$ \textbf{не является}  знакопостоянной функцией: $W V < 0$, \textbf{то} невозмущенное движение неустойчиво (ну и если $w \equiv 0$).
\end{to_thr}