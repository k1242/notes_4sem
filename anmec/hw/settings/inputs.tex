% input files

% document's head

\begin{center}
    \LARGE \textsc{Забавные факты по теории вероятностей}
\end{center}

\hrule

\phantom{42}

\begin{flushright}
    \begin{tabular}{rr}
    % written by:
        \textbf{Авторы}: 
        & Хоружий Кирилл \\
        & Примак Евгений \\
        & \\
    % date:
        \textbf{От}: &
        \textit{\today}\\
    \end{tabular}
\end{flushright}

\thispagestyle{empty}
\tableofcontents
\newpage



\section{Первое задание по аналитической механике \texorpdfstring{(\checkmark)}{(ок)}}


\subsection{Малые колебания консервативных систем \texorpdfstring{(\checkmark)}{(ок)}}
% 16.11
% 16.33
% 16.47
% 16.64
% 16.107



\subsubsection*{16.11}

Введём ось $OX$ координат вдоль туннеля, выбрав в качестве $x=0$ положение равновесия. Тогда кинетическая энергия
\begin{equation*}
    T = \frac{1}{2} m \dot{x}^2.
\end{equation*}
Интегрируя силу, действующую на тело, находим потенциальную энергию
\begin{equation*}
    F_x = -\frac{G M(x) m}{r^2(x)} \cdot \frac{x}{r} = - G \kappa x,
    \hspace{1 cm}
    \frac{G \kappa R^3}{R^2} = g,
    \hspace{0.5cm} \Rightarrow \hspace{0.5cm}
    \Pi = \int F \d x = \frac{1}{2}\frac{g}{R} x.
\end{equation*}
Так удачно вышло, что $T$ и $\Pi$ -- квадратичные формы. Запишем вековое уравнение:
\begin{equation*}
     \frac{\partial^2 \Pi}{\partial q^2} - \lambda \frac{\partial^2 T}{\partial \dot{q}^2} = 0,
     \hspace{0.5cm} \Rightarrow \hspace{0.5cm}
     \lambda = \frac{g}{R},
     \hspace{0.5cm} \Rightarrow \hspace{0.5cm}
     T = 2 \pi \sqrt{\frac{R}{g}}.
\end{equation*}


\subsubsection*{16.33}

Выбрав оси, как показано на рисунке, получим систему с 2 степенями свободы.  Кинетическая энергия системы
\begin{equation*}
    T = \frac{m}{2} \left(
        \dot{x}_1^2 + \dot{x}_2^2
    \right).
\end{equation*}
Потенциальная энергия для трёх пружинок (сдвинутая так, чтобы положение равновесия был $0$)
\begin{equation*}
    \Pi = \frac{c}{2} (x_2)^2 + \frac{c}{2} (x_1)^2 + \frac{2c}{2}
    \left(
       x_2 - x_1
    \right)^2.
\end{equation*}
И снова так вышло, что $T$ и $\Pi$ -- квадратичные формы, так что 
\begin{equation*}
     \det \left(
     \frac{\partial^2 \Pi}{\partial q^i \partial q^j} - \lambda \frac{\partial^2 T}{\partial \dot{q}^i \partial \dot{q}^j}\right) = 0,
     \hspace{0.5cm} \Rightarrow \hspace{0.5cm}
     \det\left[
        c\begin{pmatrix}
            3 & 2 \\
            2 & 3 \\
        \end{pmatrix} - 
        \lambda m\begin{pmatrix}
            1 & 0 \\
            0 & 1 \\
        \end{pmatrix}
     \right] = 0,
     \hspace{0.5cm} \Rightarrow \hspace{0.5cm}
     (\lambda m)^2 + 9 c^2 - 6 \lambda m c - 4 c^2 = 0.
\end{equation*}
Соответственно находим квадраты частот
\begin{equation*}
    \lambda^2 - 6 \lambda \frac{c}{m} + 5 \frac{c^2}{m^2} = 
    \left(\lambda_1 -  \frac{c}{m}\right)
    \left(\lambda_2 - 5 \frac{c}{m}\right)
    = 0,
    \hspace{0.5cm} \Rightarrow \hspace{0.5cm}
    \left\{\begin{aligned}
        \lambda_1: & \begin{pmatrix}
            -2c & 2c
        \end{pmatrix}
        \begin{pmatrix}
            x_1 \\ x_2
        \end{pmatrix} = 0
        &\Rightarrow
        \ \ \vc{u}_1 = \begin{pmatrix}
            1 \\ 1
        \end{pmatrix}; \\
        \lambda_1: & \begin{pmatrix}
            2c & 2c
        \end{pmatrix}
        \begin{pmatrix}
            x_1 \\ x_2
        \end{pmatrix} = 0
        &\Rightarrow
        \ \ \vc{u}_2 = \begin{pmatrix}
            1 \\ -1
        \end{pmatrix}.
    \end{aligned}\right.
\end{equation*}
Соответственно, уравнение движения будет иметь вид
\begin{equation*}
    \begin{pmatrix}
        x_1 \\ x_2
    \end{pmatrix} = 
    C_1 
    \begin{pmatrix}
        1 \\ 1
    \end{pmatrix} 
        \sin\left(
            \sqrt{\frac{c}{m}}\,  t + \alpha_1
        \right)
    +
    C_2
    \begin{pmatrix}
        1 \\ -1
    \end{pmatrix}
    \sin\left(
        \sqrt{\frac{5c}{m}} \, t + \alpha_2
    \right).
\end{equation*}



\subsubsection*{16.47}

Запишем с учётом малости колебаний кинетическую энергию системы
\begin{equation*}
    T = \frac{m}{2} l^2 \dot{\varphi}^2 + \frac{m}{2} \left(
        l \dot{\varphi}_2 + l \dot{\varphi}_1
    \right)^2.
\end{equation*}
И, опять же, с учетом малости, потенциальную
\begin{align*}
    \Pi &= \frac{c}{2} \left(
        (l \varphi_1)^2 + (l \varphi_1 + l \varphi_2)^2
    \right) + 
    m g l \cos \varphi_1 + m  g l (\cos \varphi_1 + \cos \varphi_2) = \\
    &= 
    \frac{c}{2} \left(
        (l \varphi_1)^2 + (l \varphi_1 + l \varphi_2)^2
    \right) + 2 m g l \left(1 - \frac{\varphi_1^2}{2}\right) + mgl \left(1 - \frac{\varphi_2^2}{2}\right).
\end{align*}
Как обычно, получив квадратичные формы (хотя бы в малом приближение) радуемся и переходим к поиску частот собственных колебаний
\begin{equation*}
     \det \left(
     \frac{\partial^2 \Pi}{\partial q^i \partial q^j} - \lambda \frac{\partial^2 T}{\partial \dot{q}^i \partial \dot{q}^j}\right) = 0,
     \hspace{0.5cm} \Rightarrow \hspace{0.5cm}
     \det\left[
        \begin{pmatrix}
            2cl^2 - 2 mgl & c l^2 \\
            c l^2 & cl^2  - mgl \\
        \end{pmatrix} - 
        \lambda 
        ml^2\begin{pmatrix}
            2 & 1 \\
            1 & 1 \\
        \end{pmatrix}
     \right] = 0.
\end{equation*}
Раскрыв, получаем уравнение вида
\begin{equation*}
    2 ([cl^2-ml^2 \lambda] - mgl)^2 - [cl^2-ml^2 \lambda]^2 = 0,
    \hspace{0.5cm} \Rightarrow \hspace{0.5cm}
    x = \frac{\sqrt{2} m g l}{\sqrt{2} \pm 1} = [cl^2-ml^2 \lambda],
    \hspace{0.5cm} \Rightarrow \hspace{0.5cm}
    \lambda_{1, 2} = \frac{c}{m} -  2\frac{g}{l} \mp \sqrt{2} \frac{g}{l}.
\end{equation*}
Теперь подставляем известные $\lambda$, и находим амплитудные векторы
\begin{align*}
    &\lambda_1 \, : \ \
    \begin{pmatrix}
        2  +2 \sqrt{2} & 2 + \sqrt{2}
    \end{pmatrix}
    \begin{pmatrix}
        x_1 \\ x_2
    \end{pmatrix} = 0
    \hspace{0.5cm} \Rightarrow \hspace{0.5cm}
    \vc{u}_1 = \begin{pmatrix}
        1 \\ - \sqrt{2}
    \end{pmatrix}; \\
    &\lambda_2 \, : \ \
    \begin{pmatrix}
        2  -2 \sqrt{2} & 2 - \sqrt{2}
    \end{pmatrix}
    \begin{pmatrix}
        x_1 \\ x_2
    \end{pmatrix} = 0
    \hspace{0.5cm} \Rightarrow \hspace{0.5cm}
    \vc{u}_2 = \begin{pmatrix}
        1 \\ \sqrt{2}
    \end{pmatrix}.
\end{align*}
Это позволяет нам записать уравнение движения малых колебаний (при $c/m > (2+\sqrt{2}) g / l$)
\begin{equation*}
    \begin{pmatrix}
        \varphi_1 \\
        \varphi_2
    \end{pmatrix} = 
    C_1 \begin{pmatrix}
        1 \\ - \sqrt{2}
    \end{pmatrix}
    \sin \left(
        \sqrt{\frac{c}{m} - \left(2 + \sqrt{2}\right) \frac{g}{l}} \, t + \alpha_1
    \right) + 
    C_2 \begin{pmatrix}
        1 \\ \sqrt{2}
    \end{pmatrix}
    \sin \left(
        \sqrt{\frac{c}{m} - \left(2-\sqrt{2}\right) \frac{g}{l}} \, t + \alpha_2
    \right).
\end{equation*}


\subsubsection*{16.64}
Запишем кинетическую энергию системы
\begin{equation*}
    T = \frac{m}{2} \left(
        \dot{x}_1^2 + \dot{x}_3^2
    \right) + \frac{nm}{2} \dot{x}_2^2.
\end{equation*}
И, считая $0$ в положении равновесия, потенциальную энергию системы, запасенную в сжатых пружинах
\begin{equation*}
    \Pi = \frac{c}{2} (x_2 - x_1)^2 + \frac{c}{2} (x_3 - x_2)^2.
\end{equation*}
В таком случае
\begin{equation*}
     \det \left(
     \frac{\partial^2 \Pi}{\partial q^i \partial q^j} - \lambda \frac{\partial^2 T}{\partial \dot{q}^i \partial \dot{q}^j}\right) = 0,
     \hspace{0.5cm} \Rightarrow \hspace{0.5cm}
     \det\left[
        c \begin{pmatrix}
            1 & -1 & 0 \\
            -1 & 2 & -1 \\
            0 & -1 & 1 \\
        \end{pmatrix} - \lambda 
        m \begin{pmatrix}
            1 & 0 & 0 \\
            0 & n & 0 \\
            0 & 0 & 1 \\
        \end{pmatrix}
     \right] = 0.
\end{equation*}
Раскрывая, приходим у уравнению на $\lambda$ вида
\begin{equation*}
    \lambda_1 \left(
        \lambda_2 - \frac{c}{m}
    \right)\left(
        \lambda_3 - \frac{(2+n)c}{nm}
    \right) = 0.
\end{equation*}
Соответственно, амплитудные векторы находим, как
\begin{align*}
    &\lambda_1 \, : \ \
    \begin{pmatrix}
        -c & 2c & -c \\
        c & -c & 0 \\
    \end{pmatrix}
    \begin{pmatrix}
        x_1 \\ x_2 \\ x_3
    \end{pmatrix} = 0
    &\Rightarrow \hspace{2cm}
    &\vc{u}_1 = \begin{pmatrix}
        1 \\ 1 \\ 1
    \end{pmatrix}; \\
    % 
    &\lambda_2 \, : \ \
    \begin{pmatrix}
        c & 2c-nc & c \\
        0 & c & 0 \\
    \end{pmatrix}
    \begin{pmatrix}
        x_1 \\ x_2 \\ x_3
    \end{pmatrix} = 0
    &\Rightarrow \hspace{2cm}
    &\vc{u}_2 = \begin{pmatrix}
        -1 \\ 0 \\ 1
    \end{pmatrix}; \\
    % 
    &\lambda_3 \, : \ \
    \begin{pmatrix}
        c & nc & c \\
        0 & c & 2c/n \\
    \end{pmatrix}
    \begin{pmatrix}
        x_1 \\ x_2 \\ x_3
    \end{pmatrix} = 0
    &\Rightarrow \hspace{2cm}
    &\vc{u}_3 = \begin{pmatrix}
        n \\ -2 \\ n
    \end{pmatrix}.
\end{align*}
Что ж, уравнение движения малых колебаний запишется в виде
\begin{equation*}
    \begin{pmatrix}
        x_1 \\ x_2 \\ x_3
    \end{pmatrix} = 
    (C_1 t + \alpha_1) \begin{pmatrix}
        1 \\ 1 \\ 1
    \end{pmatrix} + 
    C_2 \begin{pmatrix}
        1 \\ 0 \\ -1
    \end{pmatrix} 
    \sin \left(
        \sqrt{\frac{c}{m}} \, t + \alpha_2
    \right) + 
    C_3 \begin{pmatrix}
        n \\ -2 \\ n
    \end{pmatrix}
    \sin \left(
        \sqrt{\frac{(n+2)c}{nm}} \, t + \alpha_3
    \right).
\end{equation*}



\subsubsection*{16.107}


Знаем, что кинетическая энергия и обобщенные силы для системы могут быть записаны в виде\footnote{
    С глубоким сожалением вынуждены оставить баланс индексов в рамках этой задачи. Немое суммирование подразумевается, при повторение индексов.
} 
\begin{equation*}
    T = \frac{1}{2} a_{ik} \dot{q}_i \dot{q}_k,
    \hspace{1 cm}
    Q_i = b_{ik} \dot{q}_k,
\end{equation*}
где $a_{ik}$ -- положительно определенная квадратичная форма, а $b_{ik} = - b_{ki}$ -- кососимметричная квадратичная форма. 

Запишем уравнения Лагранжа второго рода
\begin{equation*}
    \frac{d }{d t} \frac{\partial T}{\partial \dot{q}_i} - \frac{\partial T}{\partial q_i} = Q_i,
    \hspace{0.5cm} \Rightarrow \hspace{0.5cm}
    a_{ik} \ddot{q}_k = b_{i\alpha} \dot{q}_\alpha.
\end{equation*}
Осталось этот набор уравнений решить.


Воспользуемся алгоритмом приведения двух квадратичных форм к каноническому виду. Выберем в качестве скалярного произведения $a_{ik}$, в терминах $a_{ik}$ выберем ортогональный базис так, чтобы $a_{ik}$ было равно $\delta_{ik}.$ Повернём через $u_{ik}$ базис, приведя $b_{ik}$ к каноническому виду $b^*_{jl}$, указанному в условии с $m$ блоков $2 \times 2$. 
\begin{equation*}
    \left\{\begin{aligned}
        \delta_{ik} \ddot{q}_k &= b_{i\alpha} \dot{q}_\alpha, \\    
        u_{kj} q^*_j &= q_k
    \end{aligned}\right.
    \hspace{0.5cm} \Rightarrow  \hspace{0.5cm}
    u^{-1}_{li} \cdot \left(\vphantom{\frac{1}{2}}
        \delta_{ik} u_{kj} q^*_j= b_{i\alpha} u_{\alpha \beta} q^*_\beta
    \right)
    \hspace{0.5cm} \overset{\sqsupset\, i = 1}{\Rightarrow}  \hspace{0.5cm}
    \begin{pmatrix}
        1 & 0 \\
        0 & 1 \\
    \end{pmatrix} 
    \begin{pmatrix}
        \ddot{q}_1^* \\ \ddot{q}_2^*
    \end{pmatrix} = 
    \begin{pmatrix}
        0 & -\nu \\
        \nu & 0 \\
    \end{pmatrix}
    \begin{pmatrix}
        \dot{q}_1^* \\ \dot{q}_2^*
    \end{pmatrix}.
\end{equation*}
И таких систем с колебаниями у нас будет $m$ штук
\begin{equation*}
    \left\{\begin{aligned}
        \ddot{q}_1^* &= - \nu \dot{q}_2^* \\
        \ddot{q}_2^* &= - \nu \dot{q}_1^* \\
    \end{aligned}\right.
    \hspace{0.5cm} \Rightarrow \hspace{0.5cm}
    \left\{\begin{aligned}
        \dddot{q}_1^* &= - \nu \ddot{q}_2^* \\
        \dddot{q}_2^* &= - \nu \ddot{q}_1^* \\
    \end{aligned}\right.
    \hspace{0.5cm} \Rightarrow \hspace{0.5cm}
    \left\{\begin{aligned}
        q_1^* &= \frac{A}{\nu} \cos (\nu t + \alpha) + C_1 \\
        q_2^* &= \frac{A}{\nu} \sin (\nu t + \alpha) + C_2. \\
    \end{aligned}\right.
\end{equation*}
Нули же в каноническом виде $b_{ij}$ будут соответствовать трансляциям
\begin{equation*}
    q^* = A t + B.
\end{equation*}
Собирая всё вместе, находим, что 
\begin{equation*}
    q_\alpha = u_{\alpha i} q^*_i,
    \hspace{1 cm}
    q^*_i = 
    \left\{\begin{aligned}
        &({A_j} / {\nu_j}) \cdot \cos (\nu_j t + \alpha_j) + B_{2j - 1}
        & \text{ при } &i = 2j - 1 \leq 2 m; \\
        &({A_j} / {\nu_j}) \cdot \sin (\nu_j t + \alpha_j) + B_{2j}
        & \text{ при } &i = 2j  \leq 2 m; \\
        & (A_j) \cdot t + B_j
        & \text{ при } &i = j  > 2 m. \\
    \end{aligned}\right.
\end{equation*}

\newpage

\subsection{Диссипативные системы и вынужденные колебания \texorpdfstring{(\checkmark)}{(ок)}}



\subsubsection*{17.11 (а)}

Известно, что система описывается, как
\begin{equation*}
    \left\{\begin{aligned}
        \ddot{x} + \ddot{x} + x - \alpha y &= 0 \\
        \ddot{y} + \dot{y} - \beta x + y = 0
    \end{aligned}\right.
    , \hspace{0.5cm} \Rightarrow \hspace{0.5cm}
    A = B = E, \hspace{1 cm}
    C = \begin{pmatrix}
        1 & -\alpha \\
        -\beta & 1 \\
    \end{pmatrix}.
\end{equation*}
Тогда запишем уравнение на собственные числа
\begin{equation*}
    \det\left(
        A \lambda^2 + B\lambda + C
    \right) = \det
    \begin{vmatrix}
        \lambda^2 + \lambda + 1 & -\alpha \\
        -\beta & \lambda^2+\lambda+1 \\
    \end{vmatrix} = 0,
\end{equation*}
Раскрывая,
\begin{equation*}
    (\lambda^2 + \lambda + 1)^2 + \beta \alpha = 
    \left(
        \lambda^2 + \lambda + 1 - i \gamma
    \right)
    \left(
        \lambda^2 + \lambda + 1 + i \gamma
    \right) = 0.
\end{equation*}
Получается, что
\begin{equation*}
    \lambda_{1, 2} = \frac{1}{2} \left(
        \vphantom{\frac{1}{2}}
        -1 \pm \sqrt{
        \pm 4 i \gamma - 3
        }
    \right),
\end{equation*}
где введено обозначение $\gamma = \sqrt{\beta \alpha}$. По теореме об асимптотической устойчивости достаточно, чтобы $\Re \lambda_i < 0$, соответственно найдём все $\gamma$ удовлетворяющие этому условию.

Пусть $\alpha \cdot \beta < 0$, тогда $\gamma = i \sqrt{|\alpha \beta|}$, или
\begin{equation*}
    \lambda_{1, 2} = \frac{1}{2}\left(
    \vphantom{\frac{1}{2}}
    -1 \pm \sqrt{\mp 4 \kappa - 3}
    \right),
    \hspace{0.5cm} \Rightarrow \hspace{0.5cm}
    |4 \kappa - 3| < 1,
    \hspace{0.5cm} \Rightarrow \hspace{0.5cm}
    |\kappa| = |\alpha \beta | < 1,
\end{equation*}
где было введено обозначение $\kappa = |\alpha \beta|$.


При $\alpha \cdot \beta > 0$ верно, что $\gamma = \kappa^2$, тогда
\begin{equation*}
    \Re \sqrt{z} = \Re \left(
        \sqrt{|z|} \cos \left(
            \frac{\varphi}{2} + \pi k
        \right)
    \right) < 0,
    \hspace{0.5cm} \Rightarrow \hspace{0.5cm}
    \sqrt{a^2 + b^2} \ \frac{1}{2} \left(
        1 + \frac{a}{\sqrt{a^2 + b^2}}
    \right) < 1,
\end{equation*}
где комплексное число под корнем было представлено как $a + ib$. Тогда
\begin{equation*}
    \sqrt{9 + \partial \kappa^2} - 3 < 2,
    \hspace{0.5cm} \Rightarrow \hspace{0.5cm}
    9 + 16 \kappa^2 < 5, 
    \hspace{0.5cm} \Rightarrow \hspace{0.5cm}
    |\alpha \beta|  < 1.
\end{equation*}
Получается достаточным условием асимптотической устойчивости является условие $|\alpha \beta| < 1$.

\subsubsection*{17.8 (\checkmark)}

Для начала рассмотрим систему, в которой нижний грузик привязан к полу пружинкой жесткости $c_{n+1} = 0$, так матрица для потенциальной энергии станет немного симметричнее. 

Выберем в качестве координат положения грузиков, где $q^i = 0$ соответствует положению равновесия $i$-го груза.  
Запишем потенциальную энергию системы
\begin{equation*}
    2 \Pi = c_1 q_1^2 + c_2(q_1-q_2)^2 + \ldots + c_n (q_n-q_n-1)^2 + c_{n+1} q_{n+1}^2.
\end{equation*}
Тогда матрица потенциальной энергии $C$ примет вид
\begin{equation*}
    C_{ij} = \frac{\partial^2 \Pi}{\partial q^i \partial q^j},
    \hspace{0.5cm} \Rightarrow \hspace{0.5cm}
    C = \begin{pmatrix}
        c_1 + c_2 & -c_2 & 0 &  &  \\
        -c_2 & c_2 + c_3 & -c_3 & 0 &  \\
        0 & -c_3 & c_3 + c_4 &  &   \\
         & 0 &  & \ddots & -c_n \\
         &  &  & -c_n & c_n + c_{n+1}
    \end{pmatrix}
\end{equation*}
Запишем уравнение Лагранжа второго рода, и рассмотрим систему в линейном приближении
\begin{equation*}
    \frac{d }{d t} \frac{\partial T}{\partial \dot{q}^i} - \frac{\partial T}{\partial q^i}
     = - \frac{\partial \Pi}{\partial q} + Q_i,
     \hspace{0.5cm} \Rightarrow \hspace{0.5cm}
     A \ddot{\vc{q}} + B \dot{\vc{q}} + C \vc{q} = 0,
     \hspace{0.5cm} \Rightarrow \hspace{0.5cm}
     \frac{d E}{d t} =
     A \ddot{\vc{q}} \cdot \dot{\vc{q}} + C \dot{\vc{q}} \cdot \vc{q} = - B \dot{\vc{q}} \cdot \dot{\vc{q}} = - \beta \dot{q}_n^2.
\end{equation*}
Получается, что диссипация является полной, а значит имеет смысл вспомнить теорему о добавлении в систему диссипативных сил с полной диссипацией.

\begin{to_thr}[Теорема Томсона-Тэта-Четаева]
    Если в некотором изолированном положении равновесия потенциальная энергия имеет строгий локальный минимум, то при добавлении диссипативных сил с полной диссипацией (и/или гироскопических) это положение равновесия становится асимптотически устойчивым.
\end{to_thr}

По теореме Лагранжа-Дирихле положение равновесия $\vc{q} = 0$ устойчиво, если в положение равновесия достигается локальный минимум потенциала $\Pi$. Получается остается показать, что матрица $C$ положительно определена, или, по критерию Сильвестра, что все угловые миноры $\Delta_i$ матрицы $C$ положительны.

Посчитав несколько миноров ручками, приходим к виду $\Delta_i$, которое докажем по индукции.
\begin{align*}
    \text{Предположение: }\hspace{0.3 cm} 
    &
    \Delta_n = \sum_{i=1}^{n+1} \frac{1}{c_i} \prod_{j=1}^{n+1} c_j 
    \\
    \text{База: }\hspace{0.3 cm}  
    &
        \Delta_2 = \det \begin{Vmatrix}
            c_1+c_2 & -c_2 \\
            -c_2 & c_2+c_3 \\
        \end{Vmatrix} = 
        c_1 c_2 + c_2 c_3 + c_1 c_3 = \sum_{i=1}^{2+1} \frac{1}{c_i}\left(
        \prod_{j=1}^{2+1} c_j
    \right)
    \\
    \text{Переход: }\hspace{0.3 cm} 
    &
    \Delta_{n+1} 
    \overset{(\textnormal{I})}{=} %=#1
    (c_{n+1} + c_{n+1})
    \Delta_n - c_{n+1}^2 \Delta_{n-1} 
    = %=#2
    \\
    & 
    \phantom{\Delta_{n+1}} = c_{n+1} \sum_{i=1}^{n+1} \frac{1}{c_i}
    % \left(
        \prod_{j=1}^{n+1} c_j
    % \right) 
    +
     c_{n+2} \sum_{i=1}^{n+1} \frac{1}{c_i}
     % \left(
        \prod_{j=1}^{n+1} c_j
    % \right)
    -
    c_{n+1}^2 \sum_{i=1}^{n} \frac{1}{c_i}
    % \left(
        \prod_{j=1}^{n} c_j
    % \right) 
    = %=#3
    \\
    & 
    \phantom{\Delta_{n+1}} =  
    c_{n+2} \sum_{i=1}^{n+1} \frac{1}{c_i}
    % \left(
        \prod_{j=1}^{n+1} c_j
    % \right) 
    + 
    c_{n+1} 
    \left(\sum_{i=1}^{n} \frac{1}{c_i}
            % \left(
                \prod_{j=1}^{n+1} c_j
            % \right)  
            + 
            \frac{1}{c_{n+1}}
            % \left(
                \prod_{j=1}^{n+1} c_j
            % \right) 
    \right)
    - 
    c_{n+1}^2 \sum_{i=1}^{n} \frac{1}{c_i}
    % \left(
        \prod_{j=1}^{n} c_j
    % \right) 
    = 
    \\
    & 
    \phantom{\Delta_{n+1}} 
    \overset{(\textnormal{II})}{=}  %=#4
    \sum_{i=1}^{n+1} \frac{1}{c_i} \prod_{j=1}^{n+2} c_j
    + 
    \frac{1}{c_{n+2}} \prod_{j=1}^{n+2} c_j 
    = 
    \sum_{i=1}^{n+2} \frac{1}{c_i} \prod_{j=1}^{n+2} c_j,
    \hspace{1 cm}
    \textnormal{Q. E. D.}
\end{align*}
Действительно, первый переход (I) получается, раскрытием определителя $\Delta_{n+1}$ по нижней строчке. В переходе (II) были сделаны замены, вида
\begin{equation*}
        \sum_{i=1}^{n} \frac{1}{c_i}
        % \left(
            \prod_{j=1}^{n+1} c_j
        % \right) 
        = 
        c_{n+1} \sum\limits_{i=1}^{n} \frac{1}{c_i}
        % \big(
            \prod\limits_{j=1}^{n} c_j
        % \big) 
        ; \hspace{0.5 cm}
        \prod_{j=1}^{n+1} c_j = 
        \frac{1}{c_{n+2}} \prod_{j=1}^{n+2} c_j
        ; \hspace{0.5 cm}
        c_{n+2} \sum_{i=1}^{n+1} \frac{1}{c_i} \prod_{j=1}^{n+1}
        =
        \sum_{i=1}^{n+1} \frac{1}{c_i} \prod_{j=1}^{n+2} c_j.        
\end{equation*}
Полученная формула для $\Delta_n$ ясно даёт понять, что $\Delta_i > 0$ для $i = 1, \ldots, n$, что доказывает положительную определенность $C$, а значит и локальный минимум потенциала $\Pi$ достигается в положение равновесия $\vc{q}=0$. 

Таким образом выполняются условия теоремы Лагранжа-Дирихле, как и условия теоремы Томсона-Тэта-Четаева, а значит положение равновесия $\vc{q}=0$ является асимптотически устойчивым.





\subsubsection*{17.20}

Запишем систему в матричном виде
\begin{equation*}
    A \ddot{\vc{q}} + B \dot{\vc{q}} + C \vc{q} = 0,
\end{equation*}
и воспользуемся теоремой Ляпунова об асимптотической устойчивости. Действительно, существует функция, такая, что
\begin{equation*}
    V = E = T + \Pi = \frac{1}{2} a_{ij} \dot{q}^i \dot{q}^j + \frac{1}{2} c_{\alpha \beta} q^{\alpha} q^{\beta} > 0.
\end{equation*}
В силу уравнений движения
\begin{equation*}
    \frac{d E}{d t} = a_{ij} \ddot{q}^i \dot{q}^j + c_{\alpha \beta} \dot{q}^\alpha q^\beta = - b_\gamma (\dot{q}^\gamma) < 0,
\end{equation*}
из чего следует асимптотическая устойчивость системы.


\subsubsection*{17.28}

Есть некоторая система такая, что
\begin{equation*}
    \left\{\begin{aligned}
        \dot{x}^1 &= \alpha_1 (x^2 - x^1), \\
        \dot{x}^2 &= \alpha_2 (x^3 - x^2), \\
        &\ldots\\
        \dot{x}^n &= \alpha_n (x^1 - x^n)
    \end{aligned}\right.
\end{equation*}
и снова найдём функцию Ляпунова, например, $V$ вида
\begin{equation*}
    2 V = \frac{1}{\alpha_1}(x_1 - a)^2 + \frac{1}{\alpha_2} (x_2 - a)^2 + \ldots + \frac{1}{\alpha_n}(x_n - a)^2,
\end{equation*}
тогда, в силу уравнений системы,
\begin{align*}
    \dot{V} &= \frac{\dot{x}_1}{\alpha_1} (x_1 - a) + \ldots + 
    \frac{\dot{x}_n}{\alpha_n}(x_n - a) = 
    (x_1 - a) (x_2 - x_1) + \ldots + (x_n - a) (x_1 - x_n) = \\
    &= - \sum_{i=1}^{n} x_i^2 + 
    \sum_{i=1}^{n-1} x_i x_{i+1} + x_n x_1 = 
    - \frac{1}{2}(x_n^2 - 2 x_n x_1 + x_1^2) - \frac{1}{2} 
    \sum_{i=1}^{n} (x_i - x_{i+1})^2 < 0,
\end{align*}
аналогично №17.20,
по теореме Ляпунова об асимптотической устойчивости,
положение равновесия системы асимптотически устойчиво.



\subsubsection*{№18.17}


Известно что на груз действуют две силы
\begin{equation*}
    F_1 (t) = A_1 \sin \omega_1 t,
    \hspace{1 cm}
    F_2 (t) = A_2 \cos \omega_2 t,
\end{equation*}
и сопротивление среды $F = - \beta v$. 

Запишем кинетическую и потенциальную энергию системы
\begin{equation*}
    T = \frac{m}{2} \dot{q}^2, \hspace{1 cm}
    \Pi = \frac{c}{2}q^2.
\end{equation*}
Из уравнений Лагранжа второго рода находим
\begin{equation*}
    m \ddot{q} + \beta \dot{q} + c q = F_1 + F_2 = A \sin (\omega_1 t) + B \cos (\omega_2 t).
\end{equation*}
Для начала найдём собственные колебания системы
\begin{equation*}
    m \lambda^2 + \beta \lambda + c = 0,
    \hspace{0.5cm} \Rightarrow \hspace{0.5cm}
    \lambda_{1, 2} = \frac{-\beta \pm \sqrt{\beta^2 - 4 mc}}{2m}.
\end{equation*}
Найдём теперь частные решения для вынужденных колебаний, в виде
\begin{align*}
    q = \alpha_1 \sin (\omega_1 t + \varphi_1) + \alpha_2 \sin (\omega_2 t + \varphi_2),
\end{align*}
подставляя в уравнения движения получам, что (рассмотрим $\omega_1$, для $\omega_2$ рассуждения аналогичны)
\begin{equation*}
    \sin (\omega_1 t + \varphi_1) (x - m \omega_1^2) + \cos (\omega_1 t + \varphi_1) \omega_1 \beta = \frac{A}{\alpha_1}\sin \omega_1 t,
    \hspace{0.5cm} \Rightarrow \hspace{0.5cm}
    \sin(\omega_1 t + \varphi_1 + \kappa) = \frac{A}{\alpha_1} \frac{\sin \omega_1 t}{\sqrt{
    (c-m \omega_1)^2 + \beta^2 \omega_1^2
    }},
\end{equation*}
где $\kappa$ такая, что
\begin{equation*}
    \cos \kappa = \frac{c - m \omega_1^2}{\sqrt{
        (\omega_1 \beta)^2 + (c-m \omega_1)^2
    }}.
\end{equation*}
Сравнивая выражения, находим константы
\begin{equation*}
\left\{\begin{aligned}
        \varphi_1 &= - \kappa_1 \\
        \varphi_2 &= \frac{\pi}{2} - \kappa_2
\end{aligned}\right.
\hspace{1 cm}
    \alpha_i (\omega_i) = \frac{A_i}{\sqrt{(m \omega_i-c)^2+\omega_i^2 \beta^2}},
\end{equation*}
и подставляем в ответ
\begin{equation*}
    q = \alpha_1 \sin (\omega_1 t + \varphi_1) + \alpha_2 \sin (\omega_2 t + \varphi_2).
\end{equation*}
\subsubsection*{18.31}


И снова запишем кинетическую и потенциальную энергию системы, как
\begin{equation*}
    T = \frac{1}{2} J \left(\varphi_1^2 + \varphi_2^2\right),
    \hspace{1 cm}
    \Pi = \frac{c}{2} \varphi_1^2 + \frac{c}{2}(\varphi_2 - \varphi_1)^2.
\end{equation*}
Из уравнений Лагранжа второго рода перейдём к системе
\begin{align*}
    J \ddot{\varphi}_1 + c(\varphi_1 - \varphi_2) &= M_0 \sin \omega t\\
    J \ddot{\varphi} + \beta \dot{\varphi}_2 + c (\varphi_2 - \varphi_1) = 0.
\end{align*}
Искать собственные числа здесь оказалось плохой идеей, так что просто будем искать решение в виде
\begin{equation*}
    \vc{\varphi} = \begin{pmatrix}
        a_1 \\ a_2
    \end{pmatrix} e^{i\omega t} - 
    \begin{pmatrix}
        b_1 \\ b_2
    \end{pmatrix} e^{-i\omega t}.
\end{equation*}
Для первого слагаемого
\begin{equation*}
    \left\{\begin{aligned}
        - J \omega^2 a_1 + c a_1 - c a_2 &= \mathcal M \\
        - J \omega^2 a_2 + \beta i \omega a_2 + c a_2 - c a_1 &= 0
    \end{aligned}\right.
    \hspace{0.5cm} \Rightarrow \hspace{0.5cm}
    \left\{\begin{aligned}
        a_1 (c - J \omega^2) - c a_2 &= \mathcal M \\
        a_2 (c - J \omega^2 + i \beta \omega) &= c a_1
    \end{aligned}\right.
\end{equation*}
Для второго слагаемого
\begin{equation*}
    \left\{\begin{aligned}
        - J \omega^2 b_1 + c b_1 - c b_2 &= - \mathcal M \\
        - J \omega^2 b_2 - \beta i \omega b_2 + c b_2 - c b_1 = 0
    \end{aligned}\right.
    \hspace{0.5cm} \Rightarrow \hspace{0.5cm}
    \left\{\begin{aligned}
        b_1 = \frac{b_2}{c} (c - J \omega^2 - i \beta \omega)
        b_2 (\frac{c - J \omega^2}{c} (c - J \omega^2 + i \beta \omega - c)) = - \mathcal M
    \end{aligned}\right.
    ,
\end{equation*}
где $\mathcal M = M_0 / (2 i)$. Также хочется ввести некоторые постоянные
\begin{equation*}
    \kappa = \frac{c - J \omega^2}{c} (c - J \omega^2 + i \beta \omega) - c,
    \hspace{1 cm}
    \xi = \frac{c - J \omega^2}{c} (c - J \omega^2 + i \beta \omega - c),
    \hspace{1 cm}
    \eta = 
\end{equation*}
тогда получим хорошие выражения для искомых переменных
\begin{equation*}
    \left\{\begin{aligned}
        a_1 &= \frac{\mathcal M}{\kappa} \frac{c - J \omega^2 + i \beta \omega}{c} \\
        a_2 &= \frac{\mathcal M}{\kappa}
    \end{aligned}\right.
    , \hspace{1 cm}
    \left\{\begin{aligned}
         b_1 &= - \frac{\mu}{\xi} \frac{c - J \omega^2 - i \beta \omega}{c} \\
       b_2 &= - \frac{\mu}{\xi}
    \end{aligned}\right. .
\end{equation*}
Теперь их можно поставить в решение уравнения и получить ответ:
\begin{equation*}
    \vc{\varphi} = \begin{pmatrix}
        a_1 \\ a_2
    \end{pmatrix} e^{i\omega t} - 
    \begin{pmatrix}
        b_1 \\ b_2
    \end{pmatrix} e^{-i\omega t}.
\end{equation*}
\subsubsection*{№18.37}

Момент инерции стержня $J = \frac{1}{3} m l^2$, тогда, считая отклонения малыми, кинетическую и потенциальную энергию системы можем записать, как
\begin{equation*}
    T = \frac{1}{2} J \left(\dot{\varphi}^2 + \dot{\psi}^2\right),
    \hspace{1 cm}
    \Pi = \frac{1}{2}c (\varphi a - \psi a)^2 + \left(
        1 - \frac{\varphi^2}{2} + 1 - \frac{\psi^2}{2}
    \right) mg \frac{l}{2}.
\end{equation*}
Переходя в СО движущейся платформы, к системе добавляется инерциальная сила
\begin{equation*}
    M = \frac{mA}{2} \sin(\omega t) \omega^2 l,
\end{equation*}
действующая на центры масс стержней.

С помощью уравнений Лагранжа второго рода переходим к уравнениям вида
\begin{equation*}
    A \ddot{\vc{q}} + C \vc{q} = M,
    \hspace{1 cm}
    A = J \begin{pmatrix}
        1 & 0 \\
        0 & 1 \\
    \end{pmatrix},
    \hspace{1 cm}
    C = \frac{1}{2}\begin{pmatrix}
        2a^2c + mgl & -2ca^2 \\
        -2ca^2 & 2a^2 c + mgl \\
    \end{pmatrix}
\end{equation*}
Из векового уравнения теперь можем найти собственные частоты системы, для получения однородного решения
\begin{equation*}
    \det(C - \lambda A) = 0,
    \hspace{0.25cm} \Rightarrow \hspace{0.25cm}
    \left(
        mg \frac{l}{2} - J \lambda
    \right) \left(
        a^2 c + mg \frac{l}{2} - J \lambda
    \right) = 0,
\end{equation*}
откуда легко находим $\lambda$
\begin{equation*}
    \lambda_1 = \frac{3}{2}\frac{g}{l}, \hspace{1 cm},
    \hspace{0.5 cm} \vc{u}_1 = \begin{pmatrix}
        1 \\ 1
    \end{pmatrix},
    \hspace{1 cm}
    \lambda_2 = \frac{3}{2} \frac{g}{l} + \frac{6ca^2}{ml^2},
    \hspace{0.5 cm} 
    \begin{pmatrix}
        1 \\ -1
    \end{pmatrix}.
\end{equation*}
из которых уже можем составить ФСР.

Теперь перейдём к поиску частного решения\footnote{
    Так как по условию $\varphi$ и $\psi$ малые, то про резонанс говорить не приходится.
} :
\begin{equation*}
    \varphi = \alpha \sin (\omega t), 
    \psi = \beta \sin (\omega t),
    \hspace{0.5cm} \Rightarrow \hspace{0.5cm}
    -A \omega^2 \begin{pmatrix}
        \alpha \\ \beta
    \end{pmatrix} + C \begin{pmatrix}
        \alpha \\ \beta
    \end{pmatrix} = 
    \frac{m A \omega^2 l}{2} \begin{pmatrix}
        1   \\ 1
    \end{pmatrix}.,
\end{equation*}
вводя матрицу
\begin{equation*}
    \Lambda = C - A \omega^2,
    \hspace{0.5cm} \Rightarrow \hspace{0.5cm}
    \Lambda \begin{pmatrix}
        \alpha \\ \beta
    \end{pmatrix} = 
    \frac{m A \omega^2 l}{2} \begin{pmatrix}
        1 \\ 1
    \end{pmatrix},
    \hspace{0.5cm} \Leftrightarrow \hspace{0.5cm}   
    \begin{pmatrix}
        \alpha \\ \beta
    \end{pmatrix} 
    =
     \Lambda^{-1} \,
    \frac{m A \omega^2 l}{2} \begin{pmatrix}
        1 \\ 1
    \end{pmatrix}.
\end{equation*}
Считая $\Lambda^{-1}$, находим частное решение и получаем ответ
\begin{equation*}
    \begin{pmatrix}
        \varphi \\ \psi
    \end{pmatrix} = 
    \frac{3 A \omega^2}{3 g - 2 l \omega^2} \begin{pmatrix}
        1 \\ 1
    \end{pmatrix} \sin (\omega t) + 
    C_1 \begin{pmatrix}
        1 \\ 1
    \end{pmatrix}
    \sin\left(
        \sqrt{\frac{3}{2}\frac{g}{l}} \, t + \alpha_1
    \right) + 
    C_2 \begin{pmatrix}
        1 \\ -1
    \end{pmatrix} 
    \sin\left(
        \sqrt{
        \frac{3}{2} \frac{g}{l} + \frac{6 c a^2}{ml^2}
        } \, t + \alpha_2
    \right).
\end{equation*}








\subsubsection*{18.62}
Известно, что кинетическая и потенциальная энергия системы могут быть записаны, как
\begin{equation*}
    T = \frac{1}{2} a_{ik} \dot{q}^i \dot{q}^k,
    \hspace{1 cm}
    \Pi = \frac{1}{2} c_{ik} q^i q^k.
\end{equation*}
С помощью уравнений Лагранжа второго рода можем перейти  к системе
\begin{equation*}
    A \ddot{\vc{q}} + C \dot{q} = A \vc{u}_1 \gamma \sin (\omega t).
\end{equation*}
Так как $A, \,C$ -- (невырожденные) положительно-определенные симметричные квадратичные формы, то они во-первых обратимы, а во вторых коммутируют (т.к. одновременно приводятся к диагональному виду), а значит и $A^{-1} C$ симметрична, соответственно имеет ортогональный базис.

Собственно, известно, что
\begin{equation*}
    \left\{\begin{aligned}
        \det (C - \lambda_i A) = 0 \\
        (C - \lambda_i A) \vc{u}_i = 0,
    \end{aligned}\right.
    \hspace{0.5cm} \Rightarrow \hspace{0.5cm}   
    A^{-1} C \, \vc{u}_i = \lambda_1 \vc{u}_i.
\end{equation*}
Перейдём к базису из собственных векторов (и переменным $\theta$), тогда уравнения примут вид
\begin{equation*}
    \ddot{\vc{q}} + \begin{pmatrix}
        \lambda_1 &  &  \\
         & \ddots &  \\
         &  & \lambda_n \\
    \end{pmatrix} \dot{q} = \begin{pmatrix}
        1 \\ \vdots \\ 0
    \end{pmatrix} \gamma \sin (\omega t).
\end{equation*}
Так как резонанс возможен только на собственных частотах системы, и $\lambda_1 = \omega_1^2$, то единственная частота, на которой возможен резонанс равна $\omega_1$.


\newpage

\subsection{Элементы теории бифуркаций в нелинейных системах \texorpdfstring{(\checkmark)}{(ок)}}

% \subsubsection*{no ?Т2}
\subsection*{Т2}
\addcontentsline{toc}{subsection}{T2}

Аппроксимируем движение нИСО в моменты времени $t$ и $t+dt$ сопутствующими ИСО $K'$ и $K''$. Пусть $K$ -- лабороторная система отсчета, $K'$ -- сопутствующая ИСО $\vc{v} \overset{\mathrm{def}}{=}  \vc{v}(t)$, а $K''$ -- сопутствующая ИСО движущаяся относительно $K$ со скоростью $\vc{v}(t + \d t)  = \vc{v} + \d \vc{v}$. Далее для удобства будем считать, что $K''$ движется относительно $K'$ со скоростью $\d \vc{v}'$.

Проверим, что последовательное применеие $\Lambda(d \vc{v}') \cdot \Lambda(\vc{v})$ эквивалентно
$R(\varphi) \cdot \Lambda(\vc{v} + \d \vc{v})$, где $R(\varphi)$ -- вращение в $\{xyz\}$. Для этого просто найдём 
\begin{equation*}
    R(\varphi) = \Lambda(d \vc{v}') \cdot \Lambda(\vc{v}) \cdot \Lambda(\vc{v} + \d \vc{v})^{-1}.
\end{equation*}

Пусть ось $x \parallel \vc{v}$, ось $y$ выберем так, чтобы $d\vc{v} \in \{Oxy\}$. Теперь, согласно $\eqref{LORENTS}$, считая $|\vc{v}|=\beta_1$, $d \vc{v}' = (\beta_x',\, \beta_y')\T$ можем записать (пренебрегая слагаемыми $\beta_x', \beta_y'$ второй и выше степени):
\begin{equation*}
    \Lambda(\vc{v}) =
    \left(
        \begin{array}{cccc}
         \gamma_1 & - \beta_1 \gamma_1 & 0 & 0 \\
         -\beta_1 \gamma_1 & \gamma_1 & 0 & 0 \\
         0 & 0 & 1 & 0 \\
         0 & 0 & 0 & 1 \\
        \end{array}
    \right),
    \hspace{5 mm}
    \Lambda(d \vc{v}') = 
\begin{bmatrix}
 1 & -\beta_x' & -\beta_y' & 0 \\
 -\beta_x' & 1 & 0 & 0 \\
 -\beta_y' & 0 & 1 & 0 \\
 0 & 0 & 0 & 1 \\
\end{bmatrix}.
\end{equation*}
Теперь можем выразить $d \vc{v}'$ через $d \vc{v}$, считая $\vc{r}_{\mathrm{f}}$ центром системы $K''$
\begin{equation*}
    \vc{r}_f' = \Lambda(d \vc{v}') \cdot \Lambda(\vc{v}) \vc{r}_f = 
\l(
    c t', \ 0, \ 0, \ 0
\r)\T
\hspace{0.5cm} \Rightarrow \hspace{0.5cm}
\beta(\vc{v}+d \vc{v})_x = \frac{\beta_1 + \beta_x'}{1 + \beta_1 \beta_x'},
\hspace{5 mm}
\beta(\vc{v}+d \vc{v})_y = \frac{\gamma_{\beta_1} \beta_y}{1+\beta_1 \beta_x}.
% 
\end{equation*}
где скорость находим аналогично первому номеру. Тут стоит заметить, что скоростью $\beta_x$ можно было бы пренебречь в сравнении с $\beta_1$, так как скорее всего первый порядок малось $\beta_x$ не войдёт в ответ, однако хотелось бы в этом убедиться.

Зная $d \vc{v}$ можем найти $d \vc{v}'$:
\begin{equation*}
    \beta_x' = \gamma_{\beta_1}^2 \beta_x,
    \hspace{5 mm}
    \beta_y' = \gamma \beta_y.
\end{equation*}
Но это на потом.

Через $\vc{v}, \ d \vc{v}'$ теперь можем найти $\Lambda(\vc{v} + d \vc{v})$, и посчитать обратную матрицу:
\begin{equation*}
    \Lambda^{-1} (\vc{v} + d \vc{v}) = 
    \begin{bmatrix}
         \gamma_{\beta_1} (\beta_1 \beta_x+1) & \gamma_{\beta_1} (\beta_1+\beta_x) & \beta_y & 0 \\
         \gamma_{\beta_1} (\beta_1+\beta_x) & \gamma_{\beta_1} (\beta_1 \beta_x+1) & \frac{\beta_1 \beta_y}{\gamma_{\beta_1}^{-1}+1} & 0 \\
         \beta_y & \frac{\beta_1 \beta_y}{\gamma_{\beta_1}^{-1}+1} & 1 & 0 \\
         0 & 0 & 0 & 1 \\
    \end{bmatrix}
\end{equation*}
Наконец можем посчитать матрицу поворота, которая в первом приближении действительно не содержит $\beta_x$:
\begin{equation*}
    R(\varphi) = \begin{bmatrix}
 1 & 0 & 0 & 0 \\
 0 & 1 & -\frac{\beta_1 \beta_y'}{\sqrt{1-\beta_1^2}+1} & 0 \\
 0 & \frac{\beta_1 \beta_y'}{\sqrt{1-\beta_1^2}+1} & 1 & 0 \\
 0 & 0 & 0 & 1 \\
\end{bmatrix}
\end{equation*}
что дейстительно соответствует повороту в плоскости $\{xy\}$ вокруг оси $z$ с углом $\varphi$ равным
\begin{equation*}
    \varphi = -\frac{\beta_y \beta_1}{\gamma_{\beta_1}^{-2} + \gamma_{\beta_1}^{-1}} = 
    -\frac{\gamma_{\beta_1}^{2}}{\gamma_{\beta_1} + 1} \beta_1 \beta_y,
\end{equation*}
где $\varphi$ малый, в силу малости $\beta_y$. Так вот, в результате поворота координатных осей меняются и любые векторы, неподвижные в неИСО, то есть искомая угловая скорость
\begin{equation*}
    \omega_z = -\frac{\gamma_{\beta_1}^{2}}{\gamma_{\beta_1} + 1} \beta_1 (\beta_y / \Delta t),
    \hspace{5 mm}
    \Leftrightarrow
    \hspace{5 mm}
    \vc{\omega} = -\frac{\gamma_{\beta_1}^{2}}{\gamma_{\beta_1} + 1} \left[
        \vc{\beta} \times  \dot{\vc{\beta}}
    \right] = \frac{\gamma_{\beta_1}^{2}}{\gamma_{\beta_1} + 1} \left[
        \dot{\vc{\beta}} \times  \vc{\beta}
    \right],
\end{equation*}
что и требовалось доказать.

% \subsubsection*{Т3}
% Исследуем систему вида
% \begin{equation*}
%     \left\{\begin{aligned}
%         \dot{x} &= y, \\
%         \dot{y} &= k \left(
%             \frac{b}{a-x} - x
%         \right)
%     \end{aligned}\right.
% \end{equation*}




\subsection{Т4}

Докажем, что функции вида $P(x) e^{-x^2/2}$, где $P(x) \in \mathbb{C}[x]$, при преобразовании Фурье переходти в функцию того же вида, причём степень многочлена не повышается. 

Действительно,
\begin{align*}
    F[f](y) 
    &= 
    \int_{-\infty}^{+\infty} \frac{d t}{\sqrt{2\pi}} P_\alpha (t) e^{-t^2/2} e^{-iyt} 
    = 
    \int_{-\infty}^{+\infty} \frac{d t}{\sqrt{2\pi}} e^{-t^2/2} P_\alpha \left(
        \frac{\partial }{\partial (-iy)}  e^{-yt}
    \right) 
    = \\ &= 
    P_\alpha \left(
        i \frac{\partial }{\partial y} 
    \right) \int_{-\infty}^{+\infty} \frac{d t}{\sqrt{2\pi}}
    e^{-t^2/2} e^{-iyt}
    \overset{17.8(2)}{=} 
    P_\alpha \left(
        - \frac{\partial }{\partial y} 
    \right) e^{-y^2/2}.
\end{align*}
Осталось показать, что степень многочлена не увеличилась, для этого достаточно рассмотреть
\begin{equation*}
    F[f](y) = p_\alpha \cdot \left(
        i \frac{\partial }{\partial y} 
    \right)^n e^{-y^2/2} + P_{\alpha-1} \left(
        i \frac{\partial }{\partial y} 
    \right) e^{-y^2/2}
    = 
    p_\alpha i^{\alpha} (-y)^\alpha e^{-y^2/2} + Q_{\alpha-1} (y) e^{-y^2/2} + P_{\alpha-1} \left(
        i \frac{\partial }{\partial y}
    \right)e^{-y^2/2},
\end{equation*}
поэтому степень не повышается.


\subsection{Т5}

Вычислим интегралы Лапласа с помощью образения преобразования Фурье:
\begin{equation*}
    I(y) = \int_{0}^{+\infty}  \d x \frac{\cos (yx)}{1 + x^2},
    \hspace{5 mm}
    K(y) = \int_{0}^{+\infty} \d x \frac{x \sin (yx)}{1 + x^2}.
\end{equation*}
В частности рассмотрим функцию $f(x) = e^{-\alpha |x|}$, где $\alpha > 0$, тогда
\begin{equation*}
    F[f](y) = \sqrt{\frac{2}{\pi}} \frac{\alpha}{\alpha^2 + y^2},
    \hspace{10 mm}
    F^{-1}[g] (x) = \int_{-\infty}^{+\infty} 
    \frac{\d y}{\sqrt{2\pi}} g(y) e^{ixy}.
\end{equation*}
Теперь воспользуемся формулой образения, и найдём
\begin{align*}
    f(x) 
    =
    F^{-1} \left[
        F[f]
    \right](x) 
    =
    \frac{2}{\pi}
     \int_{0}^{+\infty} 
     \d y \frac{\alpha \cos (2 y)}{\alpha^2 + y^2} = e^{-\alpha |x|}, \hspace{5 mm} \alpha > 0.
\end{align*}
Соответственно, при $\alpha = 1$, найдём
\begin{equation*}
    \int_{0}^{+\infty} \d y \frac{\cos(xy)}{1 + y^2} = \frac{\pi}{2} e^{-|x|}.
\end{equation*}

Аналогично находим $K(\alpha)$, а именно $F[f'](y) = iy F[f](y)$
\begin{equation*}
    F^{-1} [F[f']](x) = f'(x) = F^{-1}\left[
        iy F[f]
    \right] (x) = 2 \int_0^{+\infty} \frac{\d y}{\sqrt{2\pi}} i \sin (xy) \sqrt{\frac{2}{\pi}} \frac{\alpha i y}{\alpha^2 + y^2} = - \frac{2}{\pi} \int_0^\infty \d y \frac{\alpha y \sin (xy)}{\alpha^2 + y^2} = - \alpha \sign x e^{-\alpha |x|},
\end{equation*}
что при $\alpha = 1$ перейдёт в интеграл вида
\begin{equation*}
    \int_0^{+\infty} \frac{y \sin (xy)}{1 + y^2} \d y = \frac{\pi}{2} \sign (x) e^{-|x|}.
\end{equation*}
\subsubsection*{Т6}

Рассмотрим систему вида
\begin{align*}
    \dot{x} &= - y + \mu x - x y^2,\\
    \dot{y} &= \mu y + x - y^3.
\end{align*}



Аналогично Т5 перейдём к полярным координатам, и выразим $\dot{\varphi}$ и $\dot{r}$, так вышло, что и здесь всё хорошо, и
\begin{equation*}
    \left\{\begin{aligned}
        r \dot{\varphi} &= r \\
        \dot{r} &= r \mu - r^3 \sin^2 (\varphi)
    \end{aligned}\right.
    \hspace{0.5cm} \Rightarrow \hspace{0.5cm}
    \frac{\dot{r}}{\dot{\varphi}} = \frac{d r}{d \varphi} = 
    r (\mu - r^2 \sin^2 \varphi).
\end{equation*}
Найдём значения $r = r_*$, где $\dot{r}$ меняет знак
\begin{equation*}
    r_*^2 = \mu \sin^{-2} \varphi,
\end{equation*}
что возможно только при $\mu > 0$. Аналогично предыдущей задаче рассмотрим $\sign \dot{r}$, и получим
\begin{equation*}
    \sign \dot{r} = 
    \left\{\begin{aligned}
        1 & \ \  r < r_*   \\
        -1 & \ \ r > r_*
    \end{aligned}\right.
\end{equation*}

Подробнее рассмотрим положение равновесия $x=y=0$, которое в силу постоянства $\dot{\varphi}$ единственное. В линейном приближение, 
\begin{equation*}
    J = \begin{pmatrix}
        \dot{x}'_x & \dot{x}'_y \\
        \dot{y}'_x & \dot{y}'_y \\
    \end{pmatrix}
    = 
    \begin{pmatrix}
        \mu-y^2 & -1-2xy \\
        1 & \mu - 3 y^2 \\
    \end{pmatrix} = \begin{pmatrix}
        \mu & -1 \\
        1 & \mu \\
    \end{pmatrix}.
\end{equation*}
Тогда 
\begin{equation*}
    \det(J - \lambda E) = (\mu - \lambda)^2 + 1 = 0,
    \hspace{0.5cm} \Rightarrow \hspace{0.5cm}
    \lambda_{1, 2} = \mu \pm 1.
\end{equation*}
Тогда при $\mu < 0$, по теореме Ляпунова об устойчивости в линейном приближение, $x=y=0$ -- устойчивый фокус, при $\mu = 0$ верно, что $\Re (\lambda) = 0$, следовательно это центр, а при $\mu > 0$ фокус становится неустойчивым. Это позволяет прийти к фазовы портретам при различным значениям $\mu$, изображенным на рисунке \ref{T6}.

\begin{figure}[ht]
    \centering
    \includegraphics{D:\\Kami\\WM12_Workspace\\AnMec\\tmp0.pdf}
    \hspace{0.2cm}
    \includegraphics{D:\\Kami\\WM12_Workspace\\AnMec\\tmp1.pdf}
    \hspace{0.2cm}
    \includegraphics{D:\\Kami\\WM12_Workspace\\AnMec\\tmp2.pdf}
    % \hspace{0.2cm}
    % \includegraphics{D:\\Kami\\WM12_Workspace\\AnMec\\tmp.pdf}
    \caption{Бифуркация Пуанкаре-Андронова-Хопфа}
    \label{T6}
\end{figure}



\newpage

\subsection{Метод усреднения и метод нормальных форм в теории нелинейных колебаний \texorpdfstring{(\checkmark)}{(ок)}}
Вспомним волновое уравнение
\begin{equation*}
    \nabla^2 E - \frac{\varepsilon \mu}{c^2} \frac{\partial^2 E}{\partial t^2} = 0,
\end{equation*}
пусть $\varepsilon \neq 1$ и $\mu=1$, считая волну монохроматической всегда можем получить уравнение Гельмгольца
\begin{equation*}
    E = f(r) \exp(-i \omega t),
    \hspace{0.5cm} \Rightarrow \hspace{0.5cm}
    \nabla^2 f + \varepsilon k^2 f = 0,
\end{equation*}
где $k_0^2 = \omega^2/c^2$. Есть решение в виде плоской волны $f_0 e^{- k_0 \cdot r}$, решение в виде $A r^{-1} e^{i k_0 \cdot r}$. Можно рассматривать также параболическое приближение. 

Выберем некоторую ось $z$. Есть два места, где встречается $r$ -- в числителе и аргументе экспоненты. Известно, что $r^2 = \rho^2 + z^2$, тогда
\begin{equation*}
    r = \sqrt{\rho^2 + z^2} = z \left(1 + \frac{\rho^2}{z^2}\right)^{1/2} \approx z + \frac{\rho^2}{2z}.
\end{equation*}
Говоря об аргументе хочется, чтобы всё работало, для этого
\begin{equation*}
    \frac{2\pi}{\lambda} \left(z + \frac{\rho^2}{2z}+\ldots\right) = \frac{2\pi}{\lambda} z + \frac{2\pi}{\lambda} \frac{\rho^2}{2z} + \ldots,
    \hspace{0.5cm} \Rightarrow \hspace{0.5cm}
    \frac{2\pi}{\lambda} \frac{\rho^2}{2z} \ll \pi.
\end{equation*}
Так\footnote{
    Видно, что входит $n$ зон Френеля.
}  и пришли к \textit{параболическому} \textit{приближению}, вида
\begin{equation*}
    f = \frac{A}{z} \exp\left(
        i k z + i k \frac{\rho^2}{2z}
    \right).
\end{equation*}


\subsection{Параболическое приближение}

Подробнее посмотрим на
\begin{equation*}
    f(\vc{r}) = A(\vc{r}) \exp(ikz).
\end{equation*}
Точнее нас интересует некоторая модуляция сигнала
\begin{align*}
    \frac{\partial A}{\partial z} \lambda \ll A
    \hspace{0.5cm} \Rightarrow \hspace{0.5cm}
    \frac{\partial A}{\partial z} \ll \frac{A}{\lambda}
    \hspace{0.5cm} \Rightarrow \hspace{0.5cm}
    \frac{\partial A}{\partial z} \ll A \cdot k, \\
    \frac{\partial^2 A}{\partial z^2} \cdot \lambda \ll \frac{\partial A}{\partial z} 
    \hspace{0.5cm} \Rightarrow \hspace{0.5cm}
    \ldots
    \hspace{0.5cm} \Rightarrow \hspace{0.5cm}
    \frac{\partial^2 A}{\partial z^2} \ll \frac{\partial A}{\partial z} \cdot k.
\end{align*}
Считая $k^2 = \varepsilon \omega^2 / c^2$, можем записать, что
\begin{equation*}
    \nabla^2 f + \frac{\varepsilon}{c^2}\omega^2 f = 0,
    \hspace{0.5cm} \Rightarrow \hspace{0.5cm}
    \nabla^2 f + k^2 f = 0,
    \hspace{0.5cm} \Rightarrow \hspace{0.5cm}
    \frac{\partial f}{\partial z} = \frac{\partial A}{\partial z} e^{ikz} + A ik e^{ikz}
\end{equation*}
где 
\begin{equation*}
    \nabla^2 = \frac{\partial^2 }{\partial z^2} + \nabla^2_{\bot}.
\end{equation*}
Для второй производной
\begin{equation*}
    \frac{\partial^2 f}{\partial z^2} = \frac{\partial^2 A}{\partial z^2}  e^{ikz} + 2 \frac{\partial A}{\partial z}  i k e^{ikz} - A k^2 e^{ikz}.
\end{equation*}
Подставляя всё в уравнение находим, что
\begin{equation*}
    \frac{\partial^2 A}{\partial z^2} + 2 \frac{\partial A}{\partial z} ik - \cancel{A k^2} + \nabla^2_{\bot} A + \cancel{A k^2} = 0.
\end{equation*}
Вспоминая малость второй производной, получаем
\begin{equation}
    \boxed{
    \nabla^2_\bot A + 2 i k \frac{\partial A}{\partial z} = 0
    }
    \hspace{0.5cm}
    \text{--- параксиальное приближение уравнения Гельмгольца.}
\end{equation}
\red{Возможно, тут минус. На всякий случай хочется проверить, что параболическая волна это решение.} 


Однако, мы будем подробнее работать конкретно с решением 
\begin{equation*}
    f(r) = \frac{A}{z} \exp\left(-i k z - ik \frac{\rho^2}{2z}\right),
    \hspace{0.5cm} \Rightarrow \hspace{0.5cm}
    f(r) = A(r) e^{-ikz},
    \hspace{1 cm}
    A(r) \overset{\mathrm{def}}{=}  \frac{A}{z} \exp\left(-i k \frac{\rho^2}{2z}\right).
\end{equation*}
Здесь хочется сделать некоторый сдвиг
\begin{equation*}
    z \longrightarrow q(z) \overset{\mathrm{def}}{=}  z + i z_0,
\end{equation*}
где $z_0 = \const$ (Рэлеевская длина), $q(z)$ -- $q$-параметр. Тогда уравнение придет к виду
\begin{equation*}
    f(r) = \frac{A}{z + i z_0} \exp\left(
        -i k z - i k \frac{\rho^2}{2 (z+i z_0)}
    \right).
\end{equation*}
Далее заметим, что
\begin{equation*}
    \frac{1}{q(z)} = \frac{1}{z + i z_0} = \frac{z- iz_0}{z^2 + z_0^2},
    \hspace{0.5cm} \Rightarrow \hspace{0.5cm}
    f(r) = A\left(
        \frac{z}{z^2+z_0^2} - i \frac{z_0}{z^2 + z_0^2}
    \right) \exp\left(
        -i k z- ik \frac{\rho^2}{2} \left(
            \frac{z}{z^2+z_0^2} - i \frac{z_0}{z^2+z_0^2}
        \right)
    \right).
\end{equation*}
Тогда получается
\begin{equation*}
    f(r) = A \left(
        \frac{z}{z^2+z_0^2} - i \frac{z_0}{z^2+z_0^2}
    \right) \exp\left(
        - \frac{k \rho^2}{2} \frac{z_0}{z^2+z_0^2}
    \right) \exp\left(
        - ikz - ik \frac{\rho^2 z}{2(z^2 + z_0^2)}
    \right).
\end{equation*}
Внимательно оглядев выражение в экспоненте, понимаем что хочется переписать его в виде
\begin{equation}
    - \frac{2 \pi}{2} \frac{\rho^2 z_0}{\lambda (z^2+z_0^2)} = - \frac{\rho^2}{\frac{\lambda}{z_0 \pi}(z^2+z_0^2)} = 
    - \frac{\rho^2}{W^2(z)},
    \hspace{1 cm}
    W^2(z) \overset{\mathrm{def}}{=} \frac{\lambda}{z_0 \pi}(z^2+z_0^2).
\end{equation}
Другим переобозначением будет
\begin{equation}
    \frac{z}{z^2+z_0^2} = \frac{1}{z\left(
        1 + \frac{z_0^2}{z^2}
    \right)},
    \hspace{1 cm}
    R(z) \overset{\mathrm{def}}{=}  z \left(1 + \frac{z_0^2}{z^2}\right).
\end{equation}
Тогда исходное уравнение перепишется в виде
\begin{equation*}
    f(r) = A \left(
        \frac{1}{R(z)} - i \frac{\lambda}{\pi W^2(z)}
    \right) \exp\left(
        - \frac{\rho^2}{W^2(z)}
    \right) \exp\left(
        - i k z - ik \frac{\rho^2}{2 R(z)}
    \right).
\end{equation*}
Приводя к удобной форме комплексную амплитуду, получим
\begin{equation}
    f(r) = \frac{A}{i z_0} \frac{W_0}{W(z)} \exp\left(
        - \frac{\rho^2}{W^2(z)}
    \right) \exp\left(
        - ik z - ik \frac{\rho^2}{2 R(z)} + i \zeta(z)
    \right),
    \hspace{0.5 cm}
    \zeta \overset{\mathrm{def}}{=}  \arctan \frac{z}{z_0},
    \hspace{0.5 cm} W_0 \overset{\mathrm{def}}{=} W(0).
\end{equation}


\subsection{Интенсивность}

\begin{to_def}
    Интенсивность есть
    \begin{equation*}
        I = \langle |\vc{S}| \rangle_t = \left\langle \frac{c}{4\pi} \sqrt{\varepsilon} E^2\right\rangle = \frac{cn}{4\pi} \left\langle E^2\right\rangle = \frac{cn}{8\pi} E_0^2,
        \hspace{1 cm}
        \left[I\right] = \frac{\text{Дж}}{\text{с}\cdot\text{см}^2} = \frac{\text{Вт}}{\text{см}^2}.
    \end{equation*}
\end{to_def}
Но далее $I \sim E_0^2$ превращается $I = E^2$. Вспоминая, что всё хорошо, и $I = E E^*$, находим, что
\begin{equation}
    I = A_0^2 \frac{W_0^2}{W^2(z)} \exp\left(
        - \frac{2\rho^2}{W^2(z)}
    \right),
\end{equation}
и именно поэтому пучки называются Гауссовыми. Получается, что при увеличение $z$ наш пучок размывается (см. $I(\rho)$). Если мы задумаемся, что есть $I_{\text{центр}} (z) \sim 1 / z^2$. 

Если нас интересует мощность, то
\begin{equation*}
    P = \int_{0}^{\infty} I(\rho) 2 \pi \rho \d \rho = \frac{1}{2} I_0 \pi W_0^2,
    \hspace{0.5 cm} I_0 = A_0^2.
\end{equation*}
Если посчитать
\begin{equation*}
    \alpha = \frac{1}{P} \int_{0}^{\rho_0} I(\rho, z) 2 \pi \rho \d \rho = 1 - \exp\left(
        - \frac{2\rho_0^2}{W^2(z)}
    \right),
\end{equation*}
так, например, $\rho_0 = W(z)$ приводит к величине $\alpha \approx 0.86$, а при $\rho_0 = \frac{3}{2} W(z)$ получим $\alpha \approx 0.99$. Поэтому $W(r)$ называется \textit{радиусом (диаметром)} \textit{пучка}. 

Вспомним зависимость радиуса пучка от $z$
\begin{equation*}
    W(z) = \sqrt{
        \frac{\lambda z_0}{\pi} \left(
            1 + \frac{z^2}{z_0^2}
        \right)
    },
\end{equation*}
где в $W_0 = W(0) = \sqrt{ \lambda z_0 / \pi}$, а при больших $z$ 
\begin{equation*}
     \lim_{z \to \infty} W(z) \approx z\, \sqrt{\frac{\lambda}{\pi z_0}},
     \hspace{1 cm}
     \theta = \sqrt{\frac{\lambda}{\pi z_0}} = \frac{W_0}{z_0} = 
     \lambda \sqrt{\frac{1}{\lambda \pi z_0}} \approx \frac{\lambda}{W_0}
     .
 \end{equation*} 
Также можно указать $2 z_0$ -- \textit{глубина резкости}. 

Если взять гелий-неоновый лазер при длине волны $\lambda_0 = 633$ нм, получится из $2 W_0 = 2$ см, то $2 z_0 = 1$ км, а при $2 W_0 = 200$ мкм будет $2 z_0 = 1$ мм.

Тот момент, что фаза набегает на $\pi$ -- эффект Гюйи. Говоря о волновом фронте,
\begin{equation*}
    k \left(z + \frac{\rho^2}{2R}\right) + \zeta(z) = 2 \pi m,
    \hspace{0.5cm} \Rightarrow \hspace{0.5cm}
    z + \frac{\rho^2}{2R} = m \lambda + \frac{\zeta \lambda}{2 \pi},
\end{equation*}
что приводит нас к тому, что $\rho^2 / 2 R$ -- \textit{радиус кривизны}.




% 

% \section{Теоретический минимум ко второму заданию}
% \begin{to_def}
    Определим гамильтониан, как
    \begin{equation*}
        H(q, p, t) \overset{\mathrm{def}}{=}  p \cdot \dot{q} (q, p, t) - L(q, \dot{q} (q, p, t), t).
    \end{equation*}
    Уравнения Гамильтона могут быть записаны в виде
    \begin{equation*}
        \left\{\begin{aligned}
            \dot{q} &= \partial_p H \\
            \dot{p} &= -\partial_q H.
        \end{aligned}\right.
    \end{equation*}
\end{to_def}


\section{Второе задание по аналитической механике}

\subsection{Функция Гамильтона и канонические уравнения}
\subsubsection*{19.9}

Найдём гамильтониан системы, и составим канонические уравнения движения механической системы, с лагранжианом вида
\begin{equation*}
    L = \frac{1}{2} \left(
        (\dot{q}_1^2 + \dot{q}_2^2)^2 + a \dot{q}_1^2 t^2
    \right) - a \cos q_2.
\end{equation*}
Что ж, выразим импульсы, через обобщенные скорости
\begin{equation*}
    \left\{\begin{aligned}
        p_1 &= \partial_{\dot{q}_1} L &= \dot{q}_1 - \dot{q}_2 + q \dot{q}_1 t^2, \\
        p_2 &= \ldots &= \dot{q}_2 - \dot{q}_1
    \end{aligned}\right.
    \hspace{0.5cm} \Rightarrow \hspace{0.5cm}
    \dot{q}_1 = \frac{p_1+p_2}{at^2},
    \hspace{5 mm} 
    \dot{q}_2 = \frac{p_1 + p_2(1+at^2)}{at^2}.
\end{equation*}
И получим функцию Гамильтона
\begin{equation*}
    H = p_1 \dot{q}_1 + p_2 \dot{q}_2 - L = 
    \frac{(p_1+p_2)^2}{2 at^2} + a \cos q_2 - \frac{p_2^2}{2}.
\end{equation*}
Канонические уравнения системы:
\begin{align*}
    \dot{q}_1 &= \frac{p_1+p_2}{a t^2}, \\
    \dot{q}_2 &= \frac{p_1+p_2}{a t^2}-p_2, \\
    \dot{p}_1 &= 0, \\
    \dot{p}_2 &= a \sin (q_2) .
\end{align*}




\subsubsection*{19.15}

Решим обраьтную задачу, попробуем найти лагранжиан механической системы, гамильтониан которой имеет вид
\begin{equation*}
    H = \frac{1}{2} \frac{p_1^2 + p_2^2}{q_1^2 + 1_2^2} + a(q_1^2 + q_2^2).
\end{equation*} 
Для начала перейдём к обобщенным скоростям
\begin{equation*}
    \dot{q}_1 = \frac{p_1}{q_1^2+q_2^2},
    \hspace{5 mm} 
    \dot{q}_2 = \frac{p_2}{q_1^2+q_2^2},
    \hspace{0.5cm} \Rightarrow \hspace{0.5cm}
    \left\{\begin{aligned}
        p_1 &= (q_1^2+q_2^2) \dot{q}_1, \\
        p_2 &= (q_1^2+q_2^2) \dot{q}_2.
    \end{aligned}\right.
\end{equation*}
Лагранжиан же примет вид
\begin{equation*}
    L = p_1 \dot{q}_1 + p_2 \dot{q}_2 - H = 
    \left[
        \frac{1}{2} \left(
            \dot{q}_1^2 + \dot{q}_2^2
        \right) - a
    \right] (q_1^2 + q_2^2).
\end{equation*}


\subsubsection*{19.32}

Найдём гамильтониан для двойного маятника, состоящего из двух одинаковых стержней массы $m$ и для $l$. 

Координаты центар масс стержней:
\begin{equation*}
    \left.\begin{aligned}
        h_1 &= \textstyle \frac{l}{2} \cos \alpha_1, \\
        h_2 &= \textstyle l \cos \alpha_1 + \frac{l}{2} \cos \alpha_2,\\
    \end{aligned}\right.
    \hspace{5 mm} 
    \left.\begin{aligned}
        x_1 &= \textstyle \frac{l}{2} \sin \alpha_1, \\
        x_2 &= \textstyle l \sin \alpha_1 + \frac{l}{2} \sin \alpha_2.
    \end{aligned}\right.
\end{equation*}
Тогда кинетическая и потенциальная энергия системы
\begin{equation*}
    \Pi = mg(h_1 + h_2),
    \hspace{5 mm} 
    T = \frac{m}{2} l^2 \left(
        \dot{\alpha}_1^2 + \dot{\alpha}_2^2
    \right) + 
    \frac{m}{24} \left(
        \dot{x}_1^2 + \dot{h}_1^2 + \dot{x}_2^2 + \dot{h}_2^2
    \right).
\end{equation*}
Подставляя координаты, находим
\begin{align*}
    T &= \frac{1}{6} l^2 m \left(4 \alpha _1'{}^2+\alpha _2'{}^2+3 \alpha _2' \alpha _1' \cos \left(\alpha _1-\alpha _2\right)\right), \\
    \Pi &= g m \left(\frac{3}{2} l \cos \left(\alpha _1\right)+\frac{1}{2} l \cos \left(\alpha _2\right)\right),
\end{align*}
что не так плохо, как могло бы быть.
Для консервативной системы с адекваными связями
\begin{equation*}
    H = E = T + \Pi, 
    \hspace{5 mm} 
    L = T - \Pi,
    \hspace{5 mm} 
    \left.\begin{aligned}
        p_1 &= \partial_{\dot{q}_1} L, \\
        p_2 &= \partial_{\dot{q}_2} L. \\
    \end{aligned}\right.
\end{equation*}
Ну, подставляя координаты, находим
\begin{equation*}
    \left.\begin{aligned}
        p_1 &= 
        \frac{1}{6} l^2 m \left(8 \dot{\alpha}_1 +3 \dot{\alpha}_2  \cos \left(\alpha _1-\alpha _2\right)\right), \\
        p_2 &= 
        \frac{1}{6} l^2 m \left(
        2 \dot{\alpha}_2  +
        3 \dot{\alpha}_1  \cos \left(\alpha _1-\alpha _2\right)
        \right),
        \\
    \end{aligned}\right.
    \hspace{10 mm} 
    \left.\begin{aligned}
        \dot{\alpha}_1 &=  \frac{12 \left(3 p_2 \cos \left(\alpha _1-\alpha _2\right)-2 p_1\right)}{l^2 m \left(9 \cos \left(2 \left(\alpha _1-\alpha _2\right)\right)-23\right)},\\
        \dot{\alpha}_2 &= \frac{12 \left(3 p_1 \cos \left(\alpha _1-\alpha _2\right)-8 p_2\right)}{l^2 m \left(9 \cos \left(2 \left(\alpha _1-\alpha _2\right)\right)-23\right)}. \\
    \end{aligned}\right.
\end{equation*}
Возможно, после подтстановки станет в гамильтониан станет лучше:
\begin{equation*}
    H = - m g \frac{l}{2} 
    \left[3 \cos \left(\alpha _1\right)- \cos \left(\alpha _2\right)\right]
    +
    \frac{1}{ml^2}
    \frac{
        6 \left(p_1^2-3 p_1 p_2 \cos (q_1-q_2)+4 p_2^2\right)
    }{
        \left(9 \sin ^2(q_1-q_2)+7\right)
    }
\end{equation*}
И канонические уравнения ($\alpha_1 = q_1, \, \alpha_2 = q_2$):
\begin{align*}
\dot{q}_1 &= \frac{12 (3 p_2 \cos (q_1-q_2)-2 p_1)}{l^2 m (9 \cos (2 (q_1-q_2))-23)}, \\
\dot{q}_2 &= \frac{12 (3 p_1 \cos (q_1-q_2)-8 p_2)}{l^2 m (9 \cos (2 (q_1-q_2))-23)}, \\
\dot{p}_1 &= -\frac{3}{2} g l m \sin (q_1)-\frac{36 \sin (q_1-q_2) \left(p_1 p_2 (9 \cos (2 (q_1-q_2))+41)-12 \left(p_1^2+4 p_2^2\right) \cos (q_1-q_2)\right)}{l^2 m (23-9 \cos (2 (q_1-q_2)))^2}, \\
\dot{p}_2 &= -\frac{1}{2} g l m \sin (q_2)-\frac{36 \sin (q_1-q_2) \left(12 \left(p_1^2+4 p_2^2\right) \cos (q_1-q_2)-p_1 p_2 (9 \cos (2 (q_1-q_2))+41)\right)}{l^2 m (23-9 \cos (2 (q_1-q_2)))^2}. 
\end{align*}
Гамильтониан сходится с приведенным в ответах, что не может не радовать. 


\subsubsection*{19.69}

Тяжелое колечко массы $m$ скользит по гладкой проволочной окружности массы $M$ и радуса $r$, которая может вращаться вокруг своего вертикального диаметра. Составим уравнения движения системы в форме уравнений Рауса (выбрав такие обобщенные скорости, чтобы всё сошлось):
\begin{equation*}
    \Pi = m g R \cos \psi,
    \hspace{5 mm} 
    T = \frac{m}{2} \left(
        (\dot{\varphi} R \sin \psi)^2 + (\dot{\psi} R)^2
    \right) + \frac{M R^2}{2} \frac{\dot{\varphi}^2}{2}.
\end{equation*}
Лагранжиан знаем, как $L = T - \Pi$, функцию Рауса через
\begin{equation*}
    R(\psi, \varphi, \dot{\psi}, p_\varphi) = p_\varphi \dot{\varphi} - L(\psi, \varphi, \dot{\psi}, p_\varphi),
\end{equation*}
Находя из $p_\varphi = \partial_{\dot{\varphi}} L$ значение $\dot{\varphi}$, находим функцию Рауса
\begin{equation*}
    \dot{\varphi} = \frac{2 p_\varphi}{r^2 (M + 2 m \sin^2 \psi)},
    \hspace{0.5cm} \Rightarrow \hspace{0.5cm}
    R = m g R \cos \psi - \frac{m R^2}{2} \dot{\psi}^2 + \frac{p_\varphi^2}{R^2(M+2m \sin^2 \psi)}.
\end{equation*}



\subsubsection*{19.70}

В сферических координатах лагранжиан релятивистской частицы в поле тяготения имеет вид
\begin{equation*}
    L = - m_0 c^2 \sqrt{1 - \frac{\dot{r}^2 + r^2 \dot{\theta}^2 + r^2 \dot{\varphi}^2 \sin^2 \theta}{c^2}} + \frac{G}{r}.
\end{equation*}
Найдём соответствующую функцию Рауса. Аналогично предыдущей задачи, из $p_\varphi = \partial_{\dot{\varphi}} L$ находим $\dot{\varphi}$:
\begin{equation*}
    \dot{\varphi} = \pm \frac{p_\varphi}{r \sin \theta} \sqrt{
    \frac{c^2 - \dot{r}^2 - \dot{\theta}^2 r^2}{p_\varphi^2 + c^2 m^2 r^2 \sin^2 \theta}
    },
\end{equation*}
что подставляем в выражение для функции Рауса $R = p_\varphi \dot{\varphi} - L$, откуда получаем
\begin{equation*}
    R = m_0 c^2 \sqrt{\left(
        1 - \frac{\dot{r}^2 + r^2 \dot{\theta}^2}{c^2}
    \right)\left(
        1 + \frac{p_\varphi^2}{m_0^2 c^2 r^2 \sin^2 \theta}
    \right)} - \frac{G}{r}.
\end{equation*}


\subsubsection*{19.72}

Для системы с лагранжианом вида
\begin{equation*}
    L = \frac{\dot{q}^2_1 + \dot{q}^2_2 + {q}^2_1 \dot{q}^2_3}{2} - \frac{q_1^2+q_2^2}{2}.
\end{equation*}
Сразу перепишем это в терминах $(q, p)$ разбив на кинетическую и потенциальную энергии:
\begin{equation*}
    T = \frac{p_1^2 + p_2^2 + p_3^2/q_1^2}{2},
    \hspace{5 mm} 
    \Pi = \frac{q_1^2+q_2^2}{2}.
\end{equation*}
Осталось сказать, что $T + \Pi = H = h = \const$, и найти
\begin{equation*}
    p_3 = K = \pm q_1 \sqrt{2 h - (p_1^2+p_2^2)-(q_1^2+q_2^2)}.
\end{equation*}
Теперь выпешем уравнения Гамильтона
\begin{align*}
    \dot{q}_1 &= \partial_{p_1} H, \hspace{5 mm} \dot{p}_1 = - \partial_{q_1} H,
    \dot{q}_i &= \partial_{p_i} H, \hspace{5 mm} \dot{p}_i = - \partial_{q_i} H.
\end{align*}
Тогда
\begin{align*}
    \frac{d q_2}{d q_1} &= \frac{\partial K}{\partial p_2} = -\frac{p_2 q_1}{\sqrt{2 h-p_1^2-p_2^2-q_1^2-q_2^2}},
    % \hspace{5 mm} 
    % \frac{d q_3}{d q_1} = \frac{\partial K}{\partial p_3} = 0, 
    \\
    \frac{d p_2}{d q_1} &= - \frac{\partial K}{\partial q_2} = \frac{q_1 q_2}{\sqrt{2 h-p_1^2-p_2^2-q_1^2-q_2^2}}.
    % \hspace{5 mm} 
    % \frac{d p_3}{d q_3} = 0.
\end{align*}



% \subsection{Первые интегралы и теорема Э. Нётер}
% Есть система Гамильтона
\begin{equation*}
    \left\{\begin{aligned}
        \dot{q} &= \partial_p H \\
        \dot{p} &= - \partial_q H \\
    \end{aligned}\right.
\end{equation*}
и для них существуют первые интегралы -- $\varphi(q, p, t)$ -- сохранение на любых траектория движения системы.

Как их получать? Во-первых, до тех пор, пока гамильтонин явно от времени не зависит -- это первый интеграл:
\begin{equation*}
    \partial_t H = 0, \hspace{0.5cm} \Rightarrow \hspace{0.5cm} d_t H = 0.
\end{equation*}
Аналогично
\begin{equation*}
    \frac{\partial H}{\partial q^i} = 0,
    \hspace{0.5cm} \Rightarrow \hspace{0.5cm}
    p_i = \const.
\end{equation*}

\begin{to_def}
    \textit{Скобкой Пуассона} для функции гамильтоновых переменных может быть определена, как
    \begin{equation*}
        \{\varphi, \psi\} = \frac{\partial \varphi}{\partial q} \frac{\partial \psi}{\partial p} -
        \frac{\partial \varphi}{\partial p} \frac{\partial \psi}{\partial q}.
    \end{equation*}
\end{to_def}

Что происходит и почему
\begin{equation*}
    \frac{d \varphi}{d t} = \frac{\partial \varphi}{\partial t} + \frac{\partial \varphi}{\partial q} \dot{q} + \frac{\partial \varphi}{\partial p} \dot{p} = \frac{\partial \varphi}{\partial t} \{\varphi, H\} = 0,
\end{equation*}
соответственно скобки пуассона -- вплоне логичный критерий первого интеграла.

\begin{to_thr}[]
    Если $\varphi, \psi$ -- первые интагралы, то $\{\varphi, \psi\}$ -- это первый интеграл или число.
\end{to_thr}

Первые интегралы бывают зависимы, так для $\varphi_1, \ldots, \varphi_m$ можем составить
\begin{equation*}
    \rg \frac{\partial (\varphi_1, \ldots, \varphi_m)}{ \partial(q^i, p_j, t)} = m.
\end{equation*}

\begin{to_thr}[Теорема Э. Нетер]
    Пусть есть некоторое однопараметрическое семейство $\tilde{q} = \tilde{q}(q, t, \alpha)$ и $\tilde{t} = \tilde{t} (q, t, \alpha)$ где $\alpha \in \mathbb{R}^1$ такое, что дифференцируемо, $\alpha =0 \ \sim \ $ тождественное преобразование, и
    \begin{equation*}
        L \left(\tilde{q}, \frac{d \tilde{q}}{d \tilde{t}}, t\right) \d \tilde{t} = L\left(q, \frac{d q}{d t}, t\right) \d t.
    \end{equation*}
    Тогда в системе есть первый интеграл, который вычисляется так:
    \begin{equation*}
        \varphi(q, p, t) = \tilde{p} \cdot \left(
            \frac{\partial \tilde{q}}{\partial \alpha} 
        \right)_{\alpha=0} - H \cdot \left(
            \frac{\partial \tilde{t}}{\partial \alpha} 
        \right)_{\alpha=0}.
    \end{equation*}
\end{to_thr}

Например, 
\begin{align*}
    \tilde{q} &= q + \alpha, \hspace{0.5cm} \Rightarrow \hspace{0.5cm} p_q = \const \\
    \tilde{t} &= t + \alpha, \hspace{0.5cm} \Rightarrow \hspace{0.5cm} H = \const.
\end{align*}


\subsubsection*{Задача 1}
Пусть есть некоторая точка в радиальном потенциальном поле. Лагранжиан
\begin{equation*}
    L = \frac{m}{2} (
        \dot{r}^2 + r^2 \dot{\theta}^2 + r^2 \sin^2 \theta \varphi^2
    ) - \Pi(r).
\end{equation*}
Тогда вполне логично рассмотреть $\tilde{\varphi} = \varphi + \alpha$, тогда
\begin{equation*}
    I = p_\varphi \left(
        \frac{\partial \tilde{\varphi}}{\partial \alpha}_{\alpha=0} = p_\varphi = \frac{\partial L}{\partial \dot{\varphi}} = m r^2 \sin^2 \theta \dot{\varphi},
    \right)
\end{equation*}
так что момент сохраняется. 
\texttt{ Вопрос: если есть первый интеграл, то существует ли симметрия для этого первого интеграла?}
% прецессия перегелия меркурия, -- классно описывается в ото
% если человек во что-то верит, то он это докажет

\subsection{Интегральные инварианты}

\begin{to_def}
    \textit{Интегральный инвариант} -- интегральное выражение, от гамильтоновых переменных, сохраняющееся на некоторой области траектории прямых путей.
\end{to_def}


Скажем, что $N$ -- конфигурационное многообразие, $(q, \dot{q}) \in TN$, также введем $\vc{x} = (q, p)\T$, где
\begin{equation*}
    \vc{x} \in M^{2n} \equiv T^* N.
\end{equation*}
Продолжим итерации, перейдем к
\begin{equation*}
    (\vc{x}, \dot{\vc{x}}) \in TM^{4n}.
\end{equation*}
Теперь введем некоторый 
\begin{equation*}
    L(\vc{x}, \dot{\vc{x}}, t) \equiv L(q, p, \dot{q}, \dot{p}, t) = p \cdot \dot{q} - H(q, p, t).
\end{equation*}
Также мы знаем, что
\begin{equation*}
    \delta \int L \d t = 0,
    \hspace{0.5cm} \Rightarrow \hspace{0.5cm}   
    \frac{d }{d t} \frac{\partial L}{\partial \dot{\vc{x}}} - \frac{\partial L}{\partial \vc{x}} = 0,
    \hspace{0.5cm} \Rightarrow \hspace{0.5cm}
    \dot{p} = - \frac{\partial H}{\partial q},
    \hspace{5mm}
    \dot{q} = \frac{\partial H}{\partial p}.
\end{equation*}
что верно для задачи варьирования за закрепленными концами.

Тогда
\begin{equation*}
    \delta \int_{t_0(\alpha)}^{t_1(\alpha)} L \d t
    = (p \delta q - H \delta t) \bigg|_{t_0}^{t_1} -
    \int_{t_0}^{t_1} \left[
        \left(
            \dot{p} + \frac{\partial H}{\partial q} 
        \right) \cdot \delta q + 
        \left(
            \dot{q} - \frac{\partial H}{\partial p} 
        \right) \cdot \delta p
    \right] \d t.
\end{equation*}
Это приводит нас к \textbf{трубке прямых путей}. Вводим согласованные контуры по $\alpha$.

Вспоминаем, что 
\begin{equation*}
    \int_{\alpha=0}^{\alpha=1} \delta S (\alpha) = S(1) - S(0) \equiv 0.
\end{equation*}
Тогда
\begin{equation*}
    \oint_{C_0} (p \delta q - H \delta t) - 
    \oint_{C_1} (p \delta q - H \delta t)  = 0,
\end{equation*}
что в силу произвольности выбранных контуров
\begin{equation*}
    J_{\text{ПК}} = \oint_{C} (p \delta q - H \delta t) = \const
\end{equation*}
что приводит нас к \textit{интегралу Пуанкаре-Картана}.

В изохронном случае
\begin{equation*}
    I_{\text{П}} = \oint p \delta q = \const
\end{equation*}
что приводит к униврсальному интегральному инварианту Пуанкаре. \texttt{Прикол в том, что он не особо зависит от $H$.}

\subsubsection*{Пример}

Пусть $L = \frac{1}{2} (\dot{q}^2 - q^2)$, и в качестве $C_0$ выберем $q = \cos \alpha$ и $\dot{q} =\sin \alpha$, при $t \equiv 0$. Хочется найти вид трубки прямых путей и посчитать интегральный инвариант:
\begin{equation*}
    \left\{\begin{aligned}
        q &= A \cos (t + \alpha) \\
        p &= \dot{q} = -A \sin(t + \alpha)
    \end{aligned}\right.
\end{equation*}
Тогда
\begin{equation*}
    q^2 + p^2 = A^2,
\end{equation*}
что соответсвует окружности, или, в случае с движением по времени, цилиндру. Интеграл Пуанкаре тогда
\begin{equation*}
    I_\Pi = \oint p \delta q = \int_0^{2\pi} p \frac{\partial q}{\partial \alpha} \delta \alpha = 
    \int_0^{2\pi} \cos^2 \alpha \d \alpha \overset{\mathrm{!}}{=}  A^2 \pi.
\end{equation*}
то есть пока $n=1$ интеграл Пуанкаре -- это просто фазовый объем, который для всех гамильтоновых систем сохраняется.

\subsection{Обратные теоремы теории интегральных инвариантов}

Пока что мы сформулировали, что если система Гамильтонова, то у нее сохраняется интегральный инвариант Пуанкаре и интегральный инвариант Пуанкаре Картана.

Но верно и обратно, если $\forall \bar{c}$  
\begin{equation*}
    I_\Pi = \oint p \delta q = \const, 
    \hspace{0.5cm} \Rightarrow \hspace{0.5cm} \exists H(q, p, t).
\end{equation*}
Если же сохранятся для некоторой $F$ интеграл $I_{\text{ПК}}$, то $H = F(q, p, t) + f(t)$.

% \subsection{Интегральные инварианты}
% Вспомним кинематическое уравнение:
\begin{equation*}
    \dot{n}_2 = - A n_2 = -\frac{n_2}{\tau_{\text{Б}}},
    \hspace{10 mm}
    \dot{n}_2 =  - \frac{\tau_{\text{Б}}}{\tau_{\text{сп}}}
\end{equation*}
где $\tau_{\text{Б}}$ -- характерное время безизлучательного перехода, $\tau_{\text{сп}}$ -- спонтанного перехода.
Также есть поглощение:
\begin{equation*}
    \dot{n}_1 = - \sigma F n_1,
\end{equation*}
и вынужденное излучение
\begin{equation*}
    \dot{n}_2 = - \sigma F n_2.
\end{equation*}
Вообще $A, B, \sigma F$ -- коэффициенты Эйнштейна. 


Вспомним про <<просветление>>, запишем набор кинематических уравнений
\begin{equation*}
    \left\{\begin{aligned}
        \dot{n}_1 &=  - n_1 \sigma F + n_2 \sigma F + n_2/\tau  \\
         N &= n_1 + n_2
    \end{aligned}\right.
    \hspace{0.5cm} \Rightarrow \hspace{0.5cm}
    \dot{n}_1 = \dot{n}_2 = 0,
    \hspace{5 mm} \text{--- \ \ стационарное приближение}.
\end{equation*}
решая эту систему находим, что
\begin{equation*}
    n_1 (F) = N \frac{F \sigma \tau + 1}{1 + 2 F \sigma \tau},
    \hspace{5 mm}
    n_2 (F) = N \frac{F \sigma \tau}{1 + 2 F \sigma \tau}.
\end{equation*}
Также, вспоминая про поглощение, понимая как происходит поглощение фотонов, 
\begin{equation*}
    \d F = D (\sigma n_2 - \sigma n_1) \d z,
\end{equation*}
где $\alpha \overset{\mathrm{def}}{=}  \sigma n_1 - \sigma n_2 = \sigma N$, так приходим к стандартному закону
\begin{equation*}
    d F = - F \alpha \d z,
    \hspace{0.5cm} \Rightarrow \hspace{0.5cm}
    F = F_0 \exp \left(- \alpha h\right),
\end{equation*}
где $h$ -- толщина образца. И вообще можно найти $\alpha (F)$, который просто монотонно убывает.





\subsection{Затемнение}

Добавим к системе третий уровень. Договоримся, что электроны умеют переходить с 1 на 2 уровень, то на самом деле он запрыгивает чуть выше второго, и сваливается, но не с частотой $h \nu$, поэтому вынужденного излучения тут не будет. 

Также за $\tau_2$ происходит излучение с 3 на 2 не с $h \nu$. Итого остаются процессы с $\sigma_2$, $\sigma_1$, $\tau_1$ и $\tau_2$. На практике $\tau_2 \ll \tau_1$, примерно как 1 пс $\ll$ 1 нс.

Запишем кинематические уравнения:
\begin{align*}
    \dot{n}_1 &= - n_1 \varphi_1 F +  \frac{n_2}{\tau_1}, \\
    \dot{n}_2 &= n_1 \sigma_1 F - n_2 \sigma_2 F - \frac{n_2}{\tau_1} + \frac{n_3}{\tau_2}, \\
    n_1 + n_2 + n_3 = N,
\end{align*}
где мы сразу перейдём к стационарному приближнию $\dot{n}_1 = \dot{n}_2 = \dot{n}_3 = 0$. 
Считая
\begin{equation*}
    \frac{1}{\sigma_1 \tau_1} = F_1, \hspace{5 mm} 
    \frac{1}{\sigma_2 \tau_2} = F_2, \hspace{5 mm}
    n_2 = n_1 \frac{F}{F_1}, \hspace{5 mm}
    n_3 = n_1 \frac{F^2}{F_1 F_2}
\end{equation*}
и, наконец, получаем
\begin{equation}
    n_1 = N \left(
        1 + \frac{F}{F_1} + \frac{F^2}{F_1 F_2}
    \right)^{-1}.
\end{equation}
Что с $n_1$ -- она затухает, также $n_2$ слегка растёт, а потом затухает, а $n_3$ выходит на $N$. 
Если внимательно посмотреть на $\alpha (F) = \sigma_1 n_1 \red{\pm} \sigma_2 n_2$,
\begin{equation*}
    \frac{d F}{d z} = - \alpha F.
\end{equation*}
Так как $\tau_2$ мааленький, то $n_3$ растёт оочень медленно. Для $\alpha$ получается немонотонная зависимость. 
На деле $n_3$ всегда ноль, т.к. $F \ll \sqrt{F_1 F_2}$. В итоге $n_2$ идёт к $N$, $n_1$ к 0. 

Так вот, $\alpha = \sigma (n_1 - n_2)$. Также $\d F = - \alpha F \d z$. Тогда
\begin{align*}
    \textnormal{if} \ \ n_1 > n_2,
    \hspace{0.5cm} \Rightarrow \hspace{0.5cm}
    \alpha > 0, \hspace{5 mm} \frac{d F}{d z} < 0, \\
    \textnormal{if} \ \ n_1 < n_2,
    \hspace{0.5cm} \Rightarrow \hspace{0.5cm}
    \alpha < 0, \hspace{5 mm} \frac{d F}{d z} > 0.
\end{align*}
Если $n_2 > n_1$, то переходим к ситуации с инверсией населенности. 

Теперь устроим её. Пусть со второго на третий уровень происходит очень быстрое сваливание (третий ниже второго). 
Снова запищем кинематические уравнения
\begin{gather*}
    n_1 + n_3 = N \\
    \dot{n}_1 = - n_1 W + \frac{n_3}{\tau} + n_3 \sigma F = 0.
\end{gather*}
Но такие схемы уже редки, чаще сейчас используют четырех уровневые системы.

Есть уровни $1,4,3,2$ -- $E_{14} \gg k T$. Также $\tau_{23} = 10$пс. Также с $\tau_{41} = 10$пс. Итого очень легко добиться инверсии населенности между третьим и четвертым уровнем. Накачка происходит с первого на второй уровень. 


\subsection*{Лазер}

Есть некоторая среда, есть накачка, с помощью которой происходит переброс с $1$ на $4$ уровень. Пока что это только усилитель. Теперь добавим зеркало 100\% слева и 50\% справа. 

И тут на сцену выходит произвольное излучение. Как только один полетит в удачном направлении, захватит с собой остальных -- лавина, успех. 




% \subsection{Канонические преобразования и теория возмущений}
% Были разные критерии каноничности, главное, чтобы Якобиан был невырожденный. В общем план такой, если мы научимся приводить все гамильтонианы к $0$ -- было бы здорово.

\begin{equation*}
    \tilde{H} = 0, \hspace{0.5cm} \Rightarrow \hspace{0.5cm}
    \left\{\begin{aligned}
        \dot{\tilde{q}} = 0 \\
        \dot{\tilde{p}} = 0 \\
    \end{aligned}\right.
    \hspace{0.5cm} \Rightarrow \hspace{0.5cm}
    \left\{\begin{aligned}
        \tilde{p} = \alpha \\
        \tilde{p} = - \beta
    \end{aligned}\right.
\end{equation*}
Осталось найти хорошую $S$, которую можем найти из уравнения
\begin{equation*}
    H(q, p, t) + \frac{1}{c} \frac{\partial S}{\partial t} (q, \alpha, t) = 0,
\end{equation*}
только $p$ здесь явно не в тему ($p = \partial_q S$), так что, считая далее $S = c S$, перейдём к уравнению
\begin{equation}
    \frac{\partial S}{\partial t} + H\left(q, \frac{\partial S}{\partial q}, t\right) = 0,
\end{equation}
что называаем \textit{уравнением Гамильтона-Якоби}.

Из плюсов: решаем это уравнение -- находим уравнение движения. Минусы: это нелинейное уравнение в частных производных. 

\begin{to_def}
    Функция $S(q, \alpha, t)$ назывется \textit{полным интегралом} уравнения Гамильтона-Якоби, если она является решением  и 
    \begin{equation*}
        \bigg|
            \frac{\partial^2 S}{\partial q\T \partial \alpha} \neq 0
        \bigg|.
    \end{equation*}
\end{to_def}

Так вот, полный алгоритм: находим Гамильтониан, переходим к уравнению Гамильтона-Якоби, из него вытаскиваем производящую функцию, приводящую к нулевому решению, и, обратными заменами, находим уравнения движения
\begin{equation*}
    \left\{\begin{aligned}
        p &= \partial_q S (q, \alpha, t) \\
        \beta &= \partial_\alpha S (q, \alpha, t) \\
    \end{aligned}\right.
    .
\end{equation*}

Единственная $\pm$ алгоритмичная надежда -- метод разделения переменных. Давайте попробуем найти полный интеграл в виде
\begin{equation*}
    S = S_0 (t, \alpha) + S_1 (q_1, \alpha) + \ldots + S_n (q_n, \alpha).
\end{equation*}

\subsubsection*{Задача}

Посмотрим на некоторую точку $m$ в потенциале 
\begin{equation*}
    \Pi = \frac{\vc{a} \cdot \vc{r}}{r^3}, \hspace{5 mm} \vc{a} = \const.
\end{equation*}
Вообще для уравнения Гамильтона-Якоби крайне важно хорошим образом выбрать координаты, выражающим геометрию задачи.

Далее для простоты будем считать $\vc{a} \parallel Oz$, введем сферические координаты $(r, \theta, \varphi)$. 
\begin{equation*}
    T = \frac{m}{2} \left(
        \dot{r}^2 + r^2 \dot{\theta}^2 + r^2 \sin^2 (\theta) \dot{\varphi}^2
    \right),
    \hspace{5 mm}
    \Pi = \frac{a \cos \theta}{r^2}.
\end{equation*}
\texttt{Арнольд пишет, что хрен там вы докажите, что полного интеграла там нет.}

\subsection{Немного магии}

Посмотрим на полную полную производную полного интеграла системы
\begin{equation}
    \frac{d S}{d t}  (q, \alpha, t) = \frac{\partial S}{\partial t}  + \frac{\partial S}{\partial q}  \cdot \dot{q} = p \cdot \dot{q} - H = L (q, \dot{q}, t),
    \hspace{0.5cm} \Rightarrow \hspace{0.5cm}
    S = \int L \d t.
\end{equation}
Теормех не единственная теория, \texttt{возможно и не самая верная, но красота в глазах смотрящего}, так вот, уравнения Гамильтона-Якоби -- наиболее близкое к квантмеху уравнение, а-ля квазиклассическое приближение. Выглядит эта конструкция так:
\begin{equation*}
    i \hbar \frac{\partial }{\partial t} \Psi = \hat{H} \Psi,
\end{equation*}
где $\Psi$ -- волновая функция, а $\Psi^2$ -- плотность вероятности нахождения где-нибудь. 

Рассмотрим Гамильтониан вида
\begin{equation*}
    i \hbar \frac{\partial \Psi}{\partial t} = \left[
        - \frac{\hbar^2}{2m} \frac{\partial^2 }{\partial x^2} + \Pi(x, t)
    \right] \Psi.
\end{equation*}
Понятно, что где-нибудь частица да находится. Вообще, давайте считать, что
\begin{equation*}
    \Psi \sim \exp\left(\frac{i}{\hbar} S\right),
    \hspace{0.5cm} \Rightarrow \hspace{0.5cm}
    \frac{\partial \psi}{\partial t} = \frac{i}{\hbar} \frac{\partial S}{\partial t}  \exp\left(\frac{i}{\hbar} S\right),
    \hspace{5 mm}
    \frac{\partial^2 \psi}{\partial x^2} = \frac{i}{\hbar} \left[
        \frac{\partial^2 S}{\partial x^2} + \frac{i}{\hbar} \left(
            \frac{\partial S}{\partial x} 
        \right)^2
    \right] \exp\left(\frac{i}{\hbar} S\right).
\end{equation*}
Подставляя это в уравненеие Шредингера находим, что
\begin{equation*}
    - \frac{\partial S}{\partial t} = \frac{i \hbar}{2 m} \frac{\partial^2 S}{\partial x^2} + \frac{1}{2m} 
    \left(\frac{\partial S}{\partial x} \right)^2 + \Pi
\end{equation*}
что, если отбросить мнимую часть, перейдет в 
\begin{equation*}
    \frac{\partial S}{\partial t} + \frac{1}{2m} \left(
        \frac{\partial S}{\partial x} 
    \right)^2 + \Pi(x, t) = 0.
\end{equation*}



\subsection{Каниническая теория возмущений (к Т12)}

Давайте перейдём к $(q, \tilde{p})$ -- описане. Вообще критерий тогда получится вида
\begin{equation*}
    S(q, \tilde{p}, t) 
    \hspace{5 mm} \to  \hspace{5 mm}
    \left\{\begin{aligned}
        \tilde{q} &= \partial_{\tilde{p}} S \\
        p &= c^{-1} \partial_q S
    \end{aligned}\right.
    \hspace{0.5cm} \Rightarrow \hspace{0.5cm}
    \tilde{H} = c H + \frac{\partial S}{\partial t}.
\end{equation*}
Ещё раз используем идею о том, чтобы что-то испортить
\begin{equation*}
    H = H_0 + \varepsilon H, \hspace{5 mm} \varepsilon \ll 1.
\end{equation*}
Посмотрим на производящую функцию вида
\begin{equation*}
    S = S_0 = q \tilde{p},
    \hspace{5 mm}
    \left\{\begin{aligned}
        \tilde{q} = q, \\
        p = \tilde{p}
    \end{aligned}\right.
\end{equation*}
которая переведет гамильтониан в себя. Но нам нужно немного систему возмутить, соответсвенно выберем
\begin{equation*}
    S = q \tilde{p} + \varepsilon S,
    \hspace{5 mm}
    \left\{\begin{aligned}
        \tilde{q} = q + \varepsilon \partial_{\tilde{p}} S_1, \\
        p = \tilde{p} + \varepsilon \partial_q S_1.
    \end{aligned}\right.
\end{equation*}
Так мы , подбирая $S_1$, сделаем для $H_1$ хорошую долгую жизнь
\begin{equation*}
    \tilde{H} = H_0 + O(\varepsilon^2).
\end{equation*}


% Т13 --- смотри Маркеева, преобразования Биркгофа


 


% 


% \subsection{Уравнения Гамильтона-Якоби}
% \subsection{Модуляция добротности}

% 4-ltvtnbkfvbyf-lbnbk,typjkf

Есть \textit{активная} и \textit{пассивная} модуляция добротности. 

Рассмотрим синхронизацию мод. Имея одинаковые фазы на всех модах из фабри перо можно получать мощные короткие импульсы. Если добавить в лазер амплитудный модулятор
\begin{equation*}
    A = A_0 (1 + m \cos \Omega t) \cos \omega t,
\end{equation*}
то это приведет, при $\Omega = \omega$ (активная самосинхронизация), и в конечном итоге получить пики со временем порядка пикисекунд. 



Пассивная самосинхронизация мод возможная с помощью \textit{красителя} -- двухуровневой системой. Характерное время отдельного выброса $\tau \sim 1/\Delta \nu$. Один из механизмов -- пусть у красителя $\tau$  очень мало (время релаксации), но и $I_{\text{sat}}$ также очень мало (возможно при эффективном сечение $I \sim 1/\sigma\tau$ достаточно большом). 





\subsection{Фотонные кристаллы}

\textit{Фотонные кристаллы} -- бываем $\{1, 2, 3\}$-мерные. Например в одномерном случае можно воспринимать это как набор пластинок толшины $h_i$ и коэффициентом преломления $n_i$. 


Построим теорию матриц для фотонных кристаллов. Есть два типа матриц: $M$-типа и $M$-типа. Пусть в матрицу $M$-типа свящывается $u_2$ и $u_1$:
\begin{equation*}
    \begin{pmatrix}
        u_2^+ \\ u_2^-
    \end{pmatrix} = 
    \begin{pmatrix}
        A & B \\
        C & D \\
    \end{pmatrix}
    \begin{pmatrix}
        u_1^+ \\ u_1^-
    \end{pmatrix}.
\end{equation*}
Приятный момент в том, что $M = M_6 M_5 \ldots M_1$. 

Другой тип -- \textit{матрица рассеяния} или $S$-матрица, которая связывает
-типа свящывается $u_2$ и $u_1$:
\begin{equation*}
    \red{\begin{pmatrix}
        u_1^- \\ u_2^+
    \end{pmatrix}} = 
    \begin{pmatrix}
        t_{12} & r_{21} \\
        r_{12} & t_{21} \\
    \end{pmatrix}
    \begin{pmatrix}
        u_1^+ \\ u_2^-
    \end{pmatrix}.
\end{equation*}
Но их перемножить не получится. 

Хотелось бы понять переход от $M$ матриц, к $S$ матрицам и назад. Рубрика <<занимательная арифметика>>:
\begin{align*}
    \frac{\det M}{D}  = \frac{AD-BC}{D} = t_{12}, \ldots, \text{\ \ ну, СЛУ, ..}
\end{align*}
В общем получается связь вида
\begin{equation}
    M = \begin{pmatrix}
        A & B \\
        C & D \\
    \end{pmatrix} = 
    \frac{1}{t_{21}} \begin{pmatrix}
        t_{12} t_{21} & r_2 \\
        -r_1 & 1 \\ 
    \end{pmatrix}.
\end{equation}
И, аналогично,
\begin{equation}
    S = \begin{pmatrix}
        t_{12} & r_2 \\
        r_1 & t_{21} \\
    \end{pmatrix} = 
    \frac{1}{D} \begin{pmatrix}
        AD-BC & B  \\
        -C & 1
    \end{pmatrix}.
\end{equation}



\subsubsection*{Пример №1: однородная среда}
Так в однородной среде с $n_0, h$ и для волны $E = E_0 \exp(i \omega t - i k z)$  верно, что
\begin{equation*}
    u_1^- = u_2^- e^{-i \varphi},
    \hspace{0.5cm} \Rightarrow \hspace{0.5cm}
    S = \begin{pmatrix}
        e^{-i \varphi} & 0 \\
        0 & e^{-i\varphi}
    \end{pmatrix},
\end{equation*}
откуда уже можем найти $M$-матрицу
\begin{equation*}
    M = \begin{pmatrix}
        e^{i \varphi} & 0 \\
        0 & e^{i \varphi} \\
    \end{pmatrix}
\end{equation*}





\subsubsection*{Пример №2: системы без потерь}

Рассмотрим системы без потерь
\begin{equation*}
    |u_1^+|^2 + |u_2^-|^2 = |u_1^-|^2 + |u_2^+|^2.
\end{equation*}
Если система без потерь, то $|t|^2 + |r|^2 = 1$, также $|t_{12}| = |t_{21}| = |t|$ и то же для $r$. Также верно, что
\begin{equation*}
    \frac{t_{12}}{t_{21}^*} = - \frac{r_1}{r_2^*}.
\end{equation*}
В случае, если мы говорим про взаимные системы, то
\begin{equation*}
    t_{12} = t_{21} = t, \hspace{5 mm} r_{21} = r_{12} = r. 
\end{equation*}
Также для $S$ матриц можем так получить условия
\begin{equation*}
    A = D^*, \hspace{5 mm} B=C^*, \hspace{5 mm}  |A|^2 - |B|^2 = 1.
\end{equation*}
Так $S$ и $M$ матрицы запишутся в виде
\begin{equation*}
    S = \begin{pmatrix}
        t & r  \\ 
        r & t  \\
    \end{pmatrix},
    \hspace{5 mm} 
    M = \begin{pmatrix}
        1/t^*, & r/t \\
        r^*/t, & 1/t
    \end{pmatrix}.
\end{equation*}




\subsubsection*{Пример №3: граница}

Рассмотрим падение на границу
\begin{equation*}
    S = \begin{pmatrix}
    \vphantom{\dfrac{1}{2}}
        \frac{2n_1}{n_1 + n_2} & \frac{n_2-n_1}{n_1 + n_2} \\ 
    \vphantom{\dfrac{1}{2}}
        \frac{n_1-n_2}{n_1+n_2} & \frac{2 n_2}{n_1 + n_2},
    \end{pmatrix}
\end{equation*}
по Френелю. Забавно выглядит $M$-матрица:
\begin{equation*}
    M  = \frac{1}{2n_2} \begin{pmatrix}
        n_2 + n_1 & n_2-n_1 \\
        n_2-n & n_2 + n_1
    \end{pmatrix}.
\end{equation*}
Если добавим распространение в среде, перемножим, а также добавим второе преломление, то получим
\begin{equation*}
    M = \frac{1}{2n_2} \begin{pmatrix}
        n_2 + n_1 & n_2-n_1 \\
        n_2-n & n_2 + n_1
    \end{pmatrix} 
    \cdot 
    \begin{pmatrix}
        e^{- i \varphi} & 0 \\
        0 & e^{i \varphi} \\
    \end{pmatrix}
    \cdot \ldots
    = 
    \frac{1}{4n_1 n_2} \begin{pmatrix}
        \tilde{A} & \tilde{B} \\
        \tilde{C} & \tilde{D} \\
    \end{pmatrix}.
\end{equation*}
Так находим, что
\begin{equation*}
    t = \frac{AD - BC}{D} = e^{-i \varphi_1 - i \varphi_2} \frac{4 n_1 n_2}{(n_1+n_2)^2-(n)}
\end{equation*}

% \subsection{Адиабатические инварианты}
% Частенько непонятно, как перейти к переменным действие-угол $(I, \varphi)$, когда есть некоторые $(p, q)$. Иногда нам везёт, переменные разделяются, тогда
\begin{equation*}
    H \equiv H(F_1(q_1, p_1), \ldots, F_n(q_n, p_n)),
    \hspace{0.5cm} \Rightarrow \hspace{0.5cm}
    S = - h t + \sum_{k} S_k (q_k, \alpha_k),
    \hspace{0.5cm} \Rightarrow \hspace{0.5cm}
    p_k = \frac{\partial S}{\partial q_k} = p_k (q_k, \alpha)
\end{equation*}
более того
\begin{equation*}
    I_k = \frac{1}{2\pi} \oint_{C_k} p_k (q_k, \alpha) \d q_k,
    \hspace{5 mm}
    \varphi_k \mod 2 \pi,
\end{equation*}
где $k = 1, \ldots, n$, а $C_k$ -- $k$-й цикл на торе. 
\texttt{Пуанкаре пишет, что системы бывают более интегрируемые и менее интегрируемые}.

\subsubsection*{Задача}

Пусть у нас $n=2>1$. Системы из двух грузиков на пружинке в коробке
\begin{equation*}
    L = \frac{m}{2} (\dot{x}_1^2 + \dot{x}_2^2) - \frac{c}{2} (x_1^2 + x_1^2 + (x_1-x_2)^2).
\end{equation*}
Хочется перейти к переменным действие угол. Давайте вспомним про переход к главным координатам, где
\begin{equation*}
    L = \frac{\dot{\theta}_1^2 - \lambda_1 \theta_1^2}{2} + \frac{\dot{\theta}_2^2 - \lambda_2 \theta^2}{2},
    \hspace{0.5cm} \Rightarrow \hspace{0.5cm}
    H = \frac{p_1^2 + \omega_1^2 \theta_1^2}{2} + \frac{p_2^2 + \omega_2^2 \theta_2^2}{2},
\end{equation*}
получается система разделилась на два первых интеграла $F_1$ и $F_2$, тогда
\begin{equation*}
    I_1  =\frac{1}{2\pi} \oint_{C_1} p_1 (\theta_1, F_1) \d \theta_1 = \frac{1}{2\pi} \frac{2 \pi F_1}{\omega_1} = \frac{F_1}{\omega_1},
\end{equation*}
где в конце мы просто вспомнили площадь эллипса, тогда
\begin{equation*}
    \tilde{H} = I_1 \omega_1 + I_2 \omega_2 = h.
\end{equation*}
Приходим к вырожденной (в некотором смысле) системе
\begin{align*}
    &\dot{\varphi}_1 = \omega_1, \hspace{5 mm} &\dot{\varphi}_2 = \omega_2, \\
    &\dot{I}_1 = 0, \hspace{5 mm}   &\dot{I}_2 = 0.
\end{align*}


\subsection{Немного о теории возмущений}

Пусть мы немного возмущаем хорошую систему, рассмотрим случай $n=1$.
Возмущение -- переход к $n=1.5$ (\texttt{ну, типа, ...})
\begin{equation*}
    H \to H(q, p, \lambda(\varepsilon t)), \hspace{5 mm} 0 < \varepsilon \ll 1.
\end{equation*}
Выберем $\tau = \varepsilon t$, например, маятник -- который медленно тянут за ниточку. 

\begin{to_def}
    Функция $I(q, p, \lambda)$ -- называется \textit{адиабатическим инвариантом} системы, если
    \begin{equation*}
        \forall \varkappa > 0, \ \exists \varepsilon_0 (\varkappa) \colon  \forall \varepsilon < \varepsilon_0 
        \ \ 
        \bigg|
            I(q(t), p(t), \lambda(\varepsilon t)) - I(q(0), p(0), \lambda(0))
        \bigg| < \varkappa,
        \hspace{5 mm}
        0 \leq t \leq \frac{1}{\varepsilon}.
    \end{equation*}
\end{to_def}


Делаем всё как раньше, рассмотрим производяющую функцию
\begin{equation*}
    S(q, I, \lambda),
    \hspace{5 mm}
    p = \frac{\partial S}{\partial q},
    \hspace{5 mm}
    \varphi = \frac{\partial S}{\partial I} ,
    \hspace{0.5cm} \Rightarrow \hspace{0.5cm}
    q \equiv q(\varphi, I, \lambda).
\end{equation*}
Итого, находим
\begin{equation*}
    \hat{H} = H + \frac{\partial S}{\partial t} = H + \frac{\partial S}{\partial \lambda} \frac{d \lambda}{d \tau} \frac{d \tau}{d t} = H + \varepsilon \lambda'_\tau \frac{\partial S}{\partial \lambda} = H_0 + \varepsilon H_1.
\end{equation*}
Теперь можем уравнения Гамильтона и записать
\begin{equation*}
    \dot{I} = - \varepsilon \lambda_\tau' \frac{\partial }{\partial \varphi} \left(
        \frac{\partial S}{\partial \lambda} 
    \right) = - \varepsilon \frac{\partial H-1}{\partial \varphi}
    ,
    \hspace{5 mm}
    \dot{\varphi} = \frac{\partial H_0}{\partial \lambda} + \varepsilon \lambda_\tau' \frac{\partial }{\partial I} \left(
        \frac{\partial S}{\partial \lambda} 
    \right) = \frac{\partial H_0}{\partial I} + \varepsilon \frac{\partial H_1}{\partial I},
\end{equation*}
где $\dot{I}$ -- медленная переменная, $\dot{\varphi}$ -- быстрая переменная.

Вообще $H_1$ -- $2\pi$-периодическая функция от $\varphi$, \texttt{будем считать, что} в случае, если 
\begin{equation*}
    \langle \dot{I} \rangle_\varphi = 0,
\end{equation*}
то у усредненной системы есть интеграл движения. Вообще на $t \in [0, 1/\varepsilon]$, верно, что $\textnormal{err}\, I < \const \cdot \varepsilon$.

\subsubsection*{Задача}

Есть некоторая точка массы $m$, стенка удаляется как $l(\varepsilon t)$ -- адиабатически ($v \gg \dot{l}$) медленно расширяется. 

Найдём переменную действия 
\begin{equation*}
    I = \frac{1}{2\pi} \oint p \d x = \frac{m v l}{\pi},
\end{equation*}
что и является адиабатическим инвариантом. 

\subsection{Многомерный случай}

В случае $n>1$ перейдём от $(I, \varphi)$ к $(J, \psi)$ (так, чтобы $H_0(I) \to H_0 (I) + \varepsilon H_1 (I, \varphi, \varepsilon) = \hat{H} (J)$)
преобразованием вида
\begin{equation*}
    S(\varphi, J) = \varphi J +  \varepsilon S_1 (\varphi, J) + \varepsilon^2 S_2 (\varphi, J) + \ldots
\end{equation*}
Тогда, из критерия каноничности,
\begin{equation*}
    I = J + \varepsilon \frac{\partial S_1}{\partial \varphi}  + \ldots,
    \hspace{5 mm}
    \psi = \varphi + \varepsilon \frac{\partial S_1}{\partial J} + \ldots,
\end{equation*}
тогда
\begin{equation*}
    \hat{H} = H_0 \left(J + \varepsilon \frac{\partial S_1}{\partial \varphi} + \ldots \right) + \varepsilon H_1 \left(J + \varepsilon \frac{\partial S_1}{\partial \varphi} + \ldots \right) + \ldots = \hat{H}_0 (J) + \varepsilon \hat{H}_1 (J) + \ldots,
\end{equation*}
так приходим к системе
\begin{align*}
    O(1) &\colon \hat{H}_0 (J) = H_0 (J=I), \\
    O(\varepsilon) &\colon \hat{H}_1 (J) = \frac{\partial H_0}{\partial J} \frac{\partial S_1}{\partial \varphi} + H_1 (J, \varphi, 0)., \\
    O(\varepsilon^2) &\colon \ldots
\end{align*}
что перепишется в виде
\begin{equation*}
    \hat{H}_1 (J) = \omega_0 (J) \cdot \frac{\partial S}{\partial \varphi} + H_1(J, \varphi, 0),
\end{equation*}
раскалдывая в ряд Фурье находим, что
\begin{equation*}
    H_1 = \sum_k H_{1k} \exp(i (\vc{k} \cdot \vc{\varphi}));
    \hspace{5 mm}
    S_1 = \sum_k S_{1k} \exp\left(
        i (\vc{k} \cdot \hat{\varphi})
    \right),
\end{equation*}
что приводит к 
\begin{equation*}
    (\vc{\omega}_0 \cdot \vc{k}) S_{1k} = H_{1k},
    \hspace{0.5cm} \Rightarrow \hspace{0.5cm}
    S_1 = \sum \frac{H_{1k}}{(\vc{\omega}_0 \cdot \vc{k}) } \exp\left(
        i (\vc{k} \cdot \vc{\varphi})
    \right),
\end{equation*}
а это явно демонстрирует \textit{проблему малых знаменателей}.

\texttt{Пуанкаре показал, что такие ряды всегда\footnote{
    \texttt{У Бродского было стихотворение <<Конец прекрасной эпохи>>. } 
}  расходятся и умер в 1912 году. В механике наступил пост- \\модерн -- народ пришел к тому что всё неинтегрируется -- беда.} 


Но в 1954 году \textbf{К}олмогоров написал статейку на 4 странице, его студент \textbf{А}рнольд, в 1963 году её доказал\footnote{
    Арнольд, Нейштат, Табор.
}  страниц на 60. Независимо этим занимался \textbf{М}озер, который пришел к похожим заключениям, но с другой стороны. 

Идея в том, чтобы строить сверхсходящиеся ряды $\varepsilon^{2^n}$, которые очень быстро убывают. Тажке сразу забиваем на рациональные $\omega$, более того важна степень иррациональности числа: когда $\omega$ сильно иррациональное -- есть надежда на успех.

\begin{to_def}
    Для того, чтобы КАМ-теория работала нужно, чтобы система была \textit{невырождена}:
    \begin{equation*}
        \bigg|
            \frac{\partial^2 H_0}{\partial I_i \partial I_j} 
        \bigg| \neq 0, \ \Leftrightarrow \ \bigg|
            \frac{\partial \omega_i}{\partial I_j} 
        \bigg| \neq 0.
    \end{equation*}
    Это всё потому что нам хочется, чтобы мера резонансных торов была бы равна 0. 
\end{to_def}

\begin{to_def}
    Говорят, что \textit{торы иррациональны}, если ($n=2$)
    \begin{equation*}
        \frac{\omega_1 (I_1, I_2)}{\omega_2 (I_1, I_2)} \neq \frac{n}{m}
    \end{equation*}
    что соответствует плотной\footnote{
        Кстати,здесь начинает цвести эргодическая теория.
    }  намотке тора. 
\end{to_def}


\begin{to_def}
    Система называется \textit{изоэнергетически невырожденной}, если
    \begin{equation*}
        \det \begin{bmatrix}
            \dfrac{\partial^2 H_0}{\partial \vc{I}\T \partial \vc{I}} & \vc{\omega} \\
            \vc{\omega}\T & 0
        \end{bmatrix} \neq 0,
\hspace{10 mm}
        \det \begin{bmatrix}
            \dfrac{\partial \vc{\omega} }{\partial \vc{I}\T } & \vc{\omega} \\
            \vc{\omega}\T & 0
        \end{bmatrix} \neq 0.
    \end{equation*}
\end{to_def}





\subsection{Элементы теории детерменированного хаоса и КАМ-теории}
\Tsec{Т19}

Пусть электрон движется вдоль оси $x$, конденсатор вызывает колебания вдоль оси $y$. 
Предполагаем, что колебания примерно не влияют на движения электрона с большой скоростью вдоль оси конденсатора. 

Перейдём в систему $K'$, движущуюся относительно лабораторной со скоростью $\langle v_x\rangle$. 4-вектор стоячей волны в конденсаторе имеет вид $k^\mu_{\text{ст}} = (\omega_0/c,\, 0,\, 0,\, 0)\T$.  В системе $K'$ можем найти, что
\begin{equation*}
   k^\mu_{\text{ст}}\vphantom{1}' = \Lambda(\langle v_x\rangle,\, Ox)_\nu^{\mu} k^\nu_{\text{ст}} =  (\gamma \omega_0/c,\, -\beta \gamma \omega_0,\, 0,\, 0)\T,
\end{equation*}
с $\beta = v/c$. Если в системе $K'$ электроны остаются нерелятивисткими, частота частота излучения не зависит от направления и совпадает с вынуждающей сильной $k^0_{\text{ст}} \vphantom{1}' = \gamma \omega_0 /c$. 

Рассмотрим волновой 4-вектор излучения в системе $K'$. Считая, что излучение происходит $(x, y)$, можем записать 
\begin{equation*}
    \tilde{k}^\mu = \frac{1}{c} \left(
        \gamma \omega_0,\,  \gamma \omega_0 \cos \theta',\, \gamma \omega_0 \sin \theta',\, 0
    \right),
\end{equation*}
где $\theta'$ -- угол между волновым вектором $\vc{k}'$ и осью $x$. После обратного преобразования Лоренца переходим к выражению, вида
\begin{equation*}
    k^\mu = \Lambda(-\langle v_x\rangle,\, Ox)_{\nu}^{\mu} \tilde{k}^\nu = \frac{1}{c} \begin{pmatrix}
        \gamma^2 \omega_0 (1 + \beta \cos \theta') \\ 
        \gamma^2 \omega_0 (\beta + \cos \theta') \\
        \gamma \omega_0 \sin \theta' \\ 
        0
    \end{pmatrix} = \frac{1}{c} \begin{pmatrix}
        \omega \\ 
        \omega \cos \theta \\ 
        \omega \sin \theta \\ 
        0
    \end{pmatrix},
\end{equation*}
внимательно посмотрев на которое, находим
\begin{equation*}
    \cos \theta = \frac{k^1}{k^0} = \frac{\beta + \cos \theta'}{1 + \beta \cos \theta'},
    \hspace{5 mm} 
    \cos \theta' = \frac{-\beta + \cos \theta}{1 - \beta \cos \theta},
    \hspace{5 mm} 
    \omega = \gamma^2 \omega_0 (1 + \beta \cos \theta') = \frac{\omega_0}{1-\beta \cos \theta}. 
\end{equation*}
Таким образом излучение происходяет в формате узкого конуса вперед по движению $(\theta \approx 0)$. 
\Tsec{Т20}

Имеем что? $\vc{E}_0 \parallel \vc{e}_x$ и $\vc{H}_0 \parallel \vc{e}_y$. Запишем вектор Пойтинга для такой рассейянной волны:
\begin{equation*}
    \vc{S} = \frac{c}{4 \pi} |H|^2 \vc{n} = \frac{c}{4 \pi} |\vc{E}|^2 \vc{n}.
\end{equation*}
Вдали от источника, как мы обсуждали выполняется $\vc{E} \perp \vc{H}$ и равны по модулю.

Для интенсивности имеем
\begin{equation*}
    d I = \vc{n} \vc{S} r^2 d \Omega = S_0 d \sigma,
\end{equation*}
где ввели дифференциальное сечение рассеяния $d \sigma$, а $S_0 = \frac{c}{4 \pi} |\vc{E}_0|^2$.

Внутри идеально проводящего шара $\vc{E} = \vc{0}$ и $\vc{H} = \vc{0}$.
Рассмотрим конкретно электрическое поле в центре шара с плотностью заряда $\rho(\theta,\phi)$, взяв закон Кулона:
\begin{equation*}
    \vc{E}_0 \cos (\omega t - \vc{k} \cdot \vc{r}) + \int \frac{\rho (\theta,\varphi) (-  \vc{r})}{r^3}d S = 0
    \hspace{1 cm}
    \Rightarrow
    \hspace{1 cm}
    \vc{E}_0 \cos (\omega t) - \frac{1}{r^3} \underbrace{\int \rho(\theta,\varphi) \vc{r} d S}_{\vc{d}} = 0.
\end{equation*}
Соответственно получаем: $\vc{d} = \vc{E}_0 r^3 \cos (\omega t)$.

Теперь берем Био-Савара
\begin{equation*}
    \vc{H}_0 \cos (\omega t) + \int \frac{[\vc{J} \times (- \vc{r})]}{c r^3} dS = 0
    \hspace{1 cm}
    \Rightarrow
    \hspace{1 cm}
    \vc{H}_0 \cos (\omega t) + \frac{2}{r^3} \int \frac{\vc{r} \times \vc{J}}{2 c} d S = 0.
\end{equation*}
И аналогично $\vc{\mu} = - \frac{\smallvc{H}_0 r^3}{2} \cos (\omega t)$.

В волновой (зоне) будет верно, что
\begin{equation*}
    \vc{H}_d = \frac{\vc{\ddot{d}} \times \vc{n}}{c^2 r},
    \hspace{1 cm}
    \vc{H}_\mu = \frac{\vc{n} (\vc{n} \cdot \vc{\ddot{\mu}}) - \vc{\ddot{\mu}}}{c^2 r}.
\end{equation*}
Таким образом вектор Пойтинга:
\begin{equation*}
    \vc{S} = \frac{c}{4 \pi} |\vc{H}_d + \vc{H}_\mu|^2 \vc{n}
    =
    \frac{c}{4 \pi c^4 r^2} \big( |[\vc{\ddot{d}} \times \vc{n}]|^2 + (\vc{n} \cdot \vc{\ddot{\mu}})^2 + |\vc{\ddot{\mu}}|^2 + 2 ([\vc{\ddot{d}} \times \vc{n}] \cdot \vc{n}) (\vc{n} \cdot \vc{\ddot{\mu}}) - 2 (\vc{\ddot{\mu}} \cdot [\vc{\ddot{d}} \times \vc{n}]) - 2 (\vc{n} \cdot \vc{\ddot{\mu}})^2 \big) \vc{n}.
\end{equation*}

Будем разбираться по очереди: $[\vc{\ddot{d}} \times \vc{n}] \cdot \vc{n} = 0$.
Далее:
\begin{equation*}
    [\vc{\ddot{d}} \times \vc{n}]_\alpha [\vc{\ddot{d}} \times \vc{n}]^\alpha = |\vc{\ddot{d}}|^2 - (\vc{n} \cdot \vc{\ddot{d}})^2.
\end{equation*}
Теперь вроде как немного упростилось:
\begin{equation*}
    \vc{S} = \frac{1}{4 \pi c^3 r^2} \big( |[\vc{\ddot{d}} \times \vc{n}]|^2 - (\vc{n} \cdot \vc{\ddot{\mu}})^2 + |\vc{\ddot{\mu}}|^2 - (\vc{n} \cdot \vc{\ddot{\mu}})^2 - 2 (\vc{\ddot{\mu}} \cdot [\vc{\ddot{d}} \times \vc{n}]) \big) \vc{n}.
\end{equation*}
И теперь по формулам выше найдём сечение:
\begin{align*}
    d \sigma &= \frac{\omega^4 r^6}{c^4} \cos^2 (\omega t)\big( |\vc{e}_x|^2 - (\vc{n} \cdot \vc{e}_x)^2 + \frac{1}{4}|\vc{e}_y|^2 - \frac{1}{4} (\vc{n} \cdot \vc{e}_y)^2 - (\vc{e}_y [\vc{e}_x \times \vc{n}])\big) d \Omega 
    = \\ &= 
    \frac{\omega^4 r^6}{2 c^4}
    \left( \frac{5}{4} - \sin^2 \theta \cos^2 \varphi - \frac{1}{4} \sin^2 \theta \sin^2 \varphi + \cos \theta\right) d \Omega.
\end{align*}
А теперь, как нас просят  задаче, мы это возьмём и проинтегрируем!
\begin{equation*}
    \sigma = 2 \pi \frac{\omega^4 r^6}{2 c^4} (\frac{5}{4} -\frac{1}{3} - \frac{1}{12}) =  \frac{5 \pi}{3} \frac{\omega^4}{2 c^4} r^6.
\end{equation*}

% \newpage

\subsubsection*{Т22}

Вычислим хаусдорфову размерность для обобщения кривой Коха, с углом между линиями $\theta$. Примеры построения кривой на разных шагах приведены на рисунке \ref{fig:koch}.
\begin{figure}[ht]
    \centering
    \includegraphics[width=0.7\textwidth]{figures/koch_line.pdf}
    \caption{Кривая из Т22 при разных параметрах $\theta, n$}
    \label{fig:koch}
\end{figure}
Для начала поймём, что на $n$-ном шаге всего будет $N = 4^n$ звеньев, длины $\rho$ каждый. Понятно, что
\begin{equation*}
    \sin \frac{\theta}{2} = \left(\frac{\rho_n - 2 \rho_{n+1}}{2}\right) \bigg/ \rho = \frac{\rho_n}{2 \rho_{n+1}} - 1,
    \hspace{0.5cm} \Rightarrow \hspace{0.5cm}
    \rho_n = \frac{\rho_0}{\left[
        2 +2 \sin(\theta/2)
    \right]^n}.
\end{equation*}
Длину кривой мы можем найти, как $N(\varepsilon)$ отрезков длины $\varepsilon = \rho$, тогда искомая размерность кривой
\begin{equation*}
    \textnormal{dim}(\theta) = \lim_{n \to \infty} \frac{
    \ln N(\varepsilon)
    }{
    \ln(1/\varepsilon)
    } = \lim_{n \to \infty} 
    \frac{\ln 4^n}{\ln\left[2(1+\sin \theta/2)\right]^n} = \frac{\ln 4}{\ln 2 + \ln \left[1+\sin \frac{\theta}{2}\right]},
\end{equation*}
что любопытно рассмотреть на некоторых частных случаях. 

В частности, что также видно из построения, при $\theta = 0$ кривая превратиться в некоторое покрытие плоскости умельчающейся сеткой, и $\textnormal{dim}(\theta=0) = 2$.
При $\theta = \pi/3$ мы придём к кривой Коха, с размерностью $\textnormal{dim}(\theta= \pi/3) = \ln 4/ \ln 3 \approx 1.26$, рисунок которой приведен посередине \eqref{fig:koch}. 
Наконец, при $\theta = \pi$ мы после каждой итерации будем получать прямую, и $\textnormal{dim}(\theta= \pi) = 1$. 

В случае, если мы будем говорить о размерности фигуры под рассматриваемой кривой, то обнаружим, что площадь на $n$-ной итерации может быть найдена, как
\begin{equation*}
    S_n = \frac{1}{2} \rho_0^2 \sin \theta \left[
        \frac{1}{4^2} + \frac{1}{4^3} + \ldots + \frac{1}{4^{n}}
    \right],
    \hspace{10 mm} 
    \lim_{n \to \infty} S_n = \frac{1}{24} \rho_0^2 \sin \theta,
\end{equation*}
таким образом нас интересует размерность плоской фигуры конечной площади, $N(\varepsilon) \sim \varepsilon^{-2}$ $\Rightarrow$ $\textnormal{dim} = 2$.

