\subsubsection*{17.8}

\red{Ниже представлено решение прикольной задачи по линейной алгебре, и отсутствует доказательное решение. По-хорошему можно просто записать функцию Ляпунова, как в ответах, и всё. Диссипация не является полной в этой системе.}

Для начала рассмотрим систему, в которой нижний грузик привязан к полу пружинкой жесткости $c_{n+1} = 0$, так матрица для потенциальной энергии станет немного симметричнее. 

Выберем в качестве координат положения грузиков, где $q^i = 0$ соответствует положению равновесия $i$-го груза.  
Запишем потенциальную энергию системы
\begin{equation*}
    2 \Pi = c_1 q_1^2 + c_2(q_1-q_2)^2 + \ldots + c_n (q_n-q_{n-1})^2 + c_{n+1} q_{n+1}^2.
\end{equation*}
Тогда матрица потенциальной энергии $C$ примет вид
\begin{equation*}
    C_{ij} = \frac{\partial^2 \Pi}{\partial q^i \partial q^j},
    \hspace{0.5cm} \Rightarrow \hspace{0.5cm}
    C = \begin{pmatrix}
        c_1 + c_2 & -c_2 & 0 &  &  \\
        -c_2 & c_2 + c_3 & -c_3 & 0 &  \\
        0 & -c_3 & c_3 + c_4 &  &   \\
         & 0 &  & \ddots & -c_n \\
         &  &  & -c_n & c_n + c_{n+1}
    \end{pmatrix}
\end{equation*}
Запишем уравнение Лагранжа второго рода, и рассмотрим систему в линейном приближении
\begin{equation*}
    \frac{d }{d t} \frac{\partial T}{\partial \dot{q}^i} - \frac{\partial T}{\partial q^i}
     = - \frac{\partial \Pi}{\partial q} + Q_i,
     \hspace{0.5cm} \Rightarrow \hspace{0.5cm}
     A \ddot{\vc{q}} + B \dot{\vc{q}} + C \vc{q} = 0,
     \hspace{0.5cm} \Rightarrow \hspace{0.5cm}
     \frac{d E}{d t} =
     A \ddot{\vc{q}} \cdot \dot{\vc{q}} + C \dot{\vc{q}} \cdot \vc{q} = - B \dot{\vc{q}} \cdot \dot{\vc{q}} = - \beta \dot{q}_n^2.
\end{equation*}
Получается, что диссипация является полной, а значит имеет смысл вспомнить теорему о добавлении в систему диссипативных сил с полной диссипацией.

\begin{to_thr}[Теорема Томсона-Тэта-Четаева]
    Если в некотором изолированном положении равновесия потенциальная энергия имеет строгий локальный минимум, то при добавлении диссипативных сил с полной диссипацией (и/или гироскопических) это положение равновесия становится асимптотически устойчивым.
\end{to_thr}

По теореме Лагранжа-Дирихле положение равновесия $\vc{q} = 0$ устойчиво, если в положение равновесия достигается локальный минимум потенциала $\Pi$. Получается остается показать, что матрица $C$ положительно определена, или, по критерию Сильвестра, что все угловые миноры $\Delta_i$ матрицы $C$ положительны.

Посчитав несколько миноров ручками, приходим к виду $\Delta_i$, которое докажем по индукции.
\begin{align*}
    \text{Предположение: }\hspace{0.3 cm} 
    &
    \Delta_n = \sum_{i=1}^{n+1} \frac{1}{c_i} \prod_{j=1}^{n+1} c_j 
    \\
    \text{База: }\hspace{0.3 cm}  
    &
        \Delta_2 = \det \begin{Vmatrix}
            c_1+c_2 & -c_2 \\
            -c_2 & c_2+c_3 \\
        \end{Vmatrix} = 
        c_1 c_2 + c_2 c_3 + c_1 c_3 = \sum_{i=1}^{2+1} \frac{1}{c_i}\left(
        \prod_{j=1}^{2+1} c_j
    \right)
    \\
    \text{Переход: }\hspace{0.3 cm} 
    &
    \Delta_{n+1} 
    \overset{(\textnormal{I})}{=} %=#1
    (c_{n+1} + c_{n+1})
    \Delta_n - c_{n+1}^2 \Delta_{n-1} 
    = %=#2
    \\
    & 
    \phantom{\Delta_{n+1}} = c_{n+1} \sum_{i=1}^{n+1} \frac{1}{c_i}
    % \left(
        \prod_{j=1}^{n+1} c_j
    % \right) 
    +
     c_{n+2} \sum_{i=1}^{n+1} \frac{1}{c_i}
     % \left(
        \prod_{j=1}^{n+1} c_j
    % \right)
    -
    c_{n+1}^2 \sum_{i=1}^{n} \frac{1}{c_i}
    % \left(
        \prod_{j=1}^{n} c_j
    % \right) 
    = %=#3
    \\
    & 
    \phantom{\Delta_{n+1}} =  
    c_{n+2} \sum_{i=1}^{n+1} \frac{1}{c_i}
    % \left(
        \prod_{j=1}^{n+1} c_j
    % \right) 
    + 
    c_{n+1} 
    \left(\sum_{i=1}^{n} \frac{1}{c_i}
            % \left(
                \prod_{j=1}^{n+1} c_j
            % \right)  
            + 
            \frac{1}{c_{n+1}}
            % \left(
                \prod_{j=1}^{n+1} c_j
            % \right) 
    \right)
    - 
    c_{n+1}^2 \sum_{i=1}^{n} \frac{1}{c_i}
    % \left(
        \prod_{j=1}^{n} c_j
    % \right) 
    = 
    \\
    & 
    \phantom{\Delta_{n+1}} 
    \overset{(\textnormal{II})}{=}  %=#4
    \sum_{i=1}^{n+1} \frac{1}{c_i} \prod_{j=1}^{n+2} c_j
    + 
    \frac{1}{c_{n+2}} \prod_{j=1}^{n+2} c_j 
    = 
    \sum_{i=1}^{n+2} \frac{1}{c_i} \prod_{j=1}^{n+2} c_j,
    \hspace{1 cm}
    \textnormal{Q. E. D.}
\end{align*}
Действительно, первый переход (I) получается, раскрытием определителя $\Delta_{n+1}$ по нижней строчке. В переходе (II) были сделаны замены, вида
\begin{equation*}
        \sum_{i=1}^{n} \frac{1}{c_i}
        % \left(
            \prod_{j=1}^{n+1} c_j
        % \right) 
        = 
        c_{n+1} \sum\limits_{i=1}^{n} \frac{1}{c_i}
        % \big(
            \prod\limits_{j=1}^{n} c_j
        % \big) 
        ; \hspace{0.5 cm}
        \prod_{j=1}^{n+1} c_j = 
        \frac{1}{c_{n+2}} \prod_{j=1}^{n+2} c_j
        ; \hspace{0.5 cm}
        c_{n+2} \sum_{i=1}^{n+1} \frac{1}{c_i} \prod_{j=1}^{n+1}
        =
        \sum_{i=1}^{n+1} \frac{1}{c_i} \prod_{j=1}^{n+2} c_j.        
\end{equation*}
Полученная формула для $\Delta_n$ ясно даёт понять, что $\Delta_i > 0$ для $i = 1, \ldots, n$, что доказывает положительную определенность $C$, а значит и локальный минимум потенциала $\Pi$ достигается в положение равновесия $\vc{q}=0$. 

Таким образом выполняются условия теоремы Лагранжа-Дирихле, как и условия теоремы Томсона-Тэта-Четаева, а значит положение равновесия $\vc{q}=0$ является асимптотически устойчивым.


