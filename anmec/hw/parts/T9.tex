\subsubsection*{№Т9}


Посмотрим на маятник
\begin{equation*}
    \ddot{x} + \sin x = 0,
    \hspace{0.5cm} \Rightarrow \hspace{0.5cm}
    \left\{\begin{aligned}
        \dot{x}_1 &= x_2, \\
        \dot{x}_2 &= -x_1 + x_1^2/6,
    \end{aligned}\right.
\end{equation*}
тогда
\begin{equation*}
    A = \begin{pmatrix}
        0 & 1 \\
        -1 & 0 \\
    \end{pmatrix},
    \hspace{0.5cm} \Rightarrow \hspace{0.5cm}
    \Lambda = \begin{pmatrix}
        -i & 0 \\
        0 & i \\
    \end{pmatrix} = 
    \frac{1}{2}
    \begin{pmatrix}
        -i & 1 \\
        i & 1 \\
    \end{pmatrix}
    A
    \begin{pmatrix}
        i & -i \\
        1 & 1 \\
    \end{pmatrix}.
\end{equation*}
Посмотрим на замену $\vc{x} = S \vc{u}$, тогда
\begin{equation*}
    \dot{\vc{u}} = \Lambda \vc{u} + S^{-1} \begin{pmatrix}
        0 \\ X_1^3/6
    \end{pmatrix},
\end{equation*}
или, в координатах,
\begin{equation*}
    \left\{\begin{aligned}
        \dot{u}_1 &= - i u_1 - i (u_1-u_2)^3/12 \\
        \dot{u}_2 &= i u_2 - i (u_1 - u_2)^3/12
    \end{aligned}\right.
    \hspace{1 cm}
    \vc{g} = \frac{-i}{12}
    \begin{pmatrix}
        u_1^3 - 3 u_1^2 u_2 + 3 u_1 u_2^2 - u_2^3 \\
        u_1^3 - 3 u_1^2 u_2 + 3 u_1 u_2^2 - u_2^3 
    \end{pmatrix}.
\end{equation*}
Во имя упрощения уравнений, перейдём к переменным 
\begin{equation*}
    \vc{u} = \vc{y} + \vc{p}(\vc{y}),
    p_i = 
    \frac{g_i^k}{k_1 \lambda_1 + \ldots + k_n \lambda_n - \lambda_i},
\end{equation*}
тогда коэффициенты многочлена
\begin{align*}
    &p_{03}^1 = \frac{1}{48}, 
    &p_{12}^1 = -\frac{1}{8},
    &&p_{21}^1 = \frac{\neq 0}{0},
    &&p_{30}^1 = \frac{1}{24}, \\
    &p_{03}^2 = \frac{1}{24}, 
    &p_{12}^2 = \frac{\neq 0}{0},
    &&p_{21}^2 = \frac{1}{8},
    &&p_{30}^2 = \frac{1}{48}.
\end{align*}
Получается, остались только резонансные слагаемые,
\begin{equation*}
    \begin{pmatrix}
        \dot{y}_1 \\ \dot{y}_2
    \end{pmatrix} = 
    \begin{pmatrix}
        -iy_1 + i y_1^2 y_2/4 \\
        i y_2 - i y_1 y_2^2 / 4
    \end{pmatrix}
    \hspace{0.5cm} \Rightarrow \hspace{0.5cm}
    \left\{\begin{aligned}
        y_1 &= C_1(t) e^{it}, \\
        y_2 &= C_2(t) e^{-it},
    \end{aligned}\right.
    \hspace{0.5cm} \Rightarrow \hspace{0.5cm}
    \left\{\begin{aligned}
        y_1 &= \gamma_2 \exp\left(i(\gamma_1-4)t/4\right), \\
        y_2 &= \gamma_1/\gamma_2 \cdot e^{-i(\gamma_1-4)t/4}.
    \end{aligned}\right.
\end{equation*}
Теперь подставим начальные условия $t=0$ и $x_1=a$, $x_2=0$. Пусть
\begin{equation*}
    \left\{\begin{aligned}
        u &= x_2 + i x_1 \\
        \bar{u} &= x_2 - i x_1
    \end{aligned}\right.
    \hspace{0.5cm} \Rightarrow \hspace{0.5cm}
    \left\{\begin{aligned}
        x_2 &= u_1 + u_2 \\
        x_1 &= i(u_1-u_2)
    \end{aligned}\right.
    \hspace{0.5cm} \Rightarrow \hspace{0.5cm}
    \left\{\begin{aligned}
        \Re u &= u_1 + u_2 \\
        \Im u &= u_1 - u_2
    \end{aligned}\right.
\end{equation*}
Тогда
\begin{equation*}
    u \bar{u} (t=0) = a^2 = \Re^2 u + \Im^2 u = 2 (u_1^2 + u_2^2),
\end{equation*}
также
\begin{equation*}
    \tg \text{Arg} u = \frac{\Re u}{\Im u} = 0,
    \hspace{0.5cm} \Rightarrow \hspace{0.5cm}
    u_1^2 - u_2^2 = 0.
\end{equation*}
Пренебрегая членам $O(y_1^3)$ находим, что
\begin{equation*}
    u_1 = y_1,
    \hspace{0.5cm} \Rightarrow \hspace{0.5cm}
    u_1^2 \approx y_1^2,
\end{equation*}
аналогчино $u_2^2 \approx y_2^2$, другими словами
\begin{equation*}
    y_1^2 + y_2^2 = \frac{a^2}{2},
    \hspace{0.5cm} \Rightarrow \hspace{0.5cm}
    \gamma_2^2 + \frac{\gamma_1^2}{\gamma_2^2} = \frac{a^2}{2}.
\end{equation*}
также верно, что
\begin{equation*}
    y_1^2 - y_2^2 = 0,
    \hspace{0.5cm} \Rightarrow \hspace{0.5cm}   
    \gamma_1^2 = \gamma_2^4,
    \hspace{0.5cm} \Rightarrow \hspace{0.5cm}
    \gamma_1 = \gamma_2^2,
    \hspace{0.5cm} \Rightarrow \hspace{0.5cm}
    \gamma_1 = \frac{a^2}{4}.
\end{equation*}
Вспомнив, как выглядит $y_1 (t)$ находим, что
\begin{equation*}
    \omega = \bigg|1-\frac{\gamma}{4}\bigg| = \bigg|
    1 - \frac{a^2}{16}
    \bigg|,
\end{equation*}
что очень сильно похоже на правду.