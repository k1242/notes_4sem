


\subsubsection*{17.11 (а)}

Известно, что система описывается, как
\begin{equation*}
    \left\{\begin{aligned}
        \ddot{x} + \ddot{x} + x - \alpha y &= 0 \\
        \ddot{y} + \dot{y} - \beta x + y = 0
    \end{aligned}\right.
    , \hspace{0.5cm} \Rightarrow \hspace{0.5cm}
    A = B = E, \hspace{1 cm}
    C = \begin{pmatrix}
        1 & -\alpha \\
        -\beta & 1 \\
    \end{pmatrix}.
\end{equation*}
Тогда запишем уравнение на собственные числа
\begin{equation*}
    \det\left(
        A \lambda^2 + B\lambda + C
    \right) = \det
    \begin{vmatrix}
        \lambda^2 + \lambda + 1 & -\alpha \\
        -\beta & \lambda^2+\lambda+1 \\
    \end{vmatrix} = 0,
\end{equation*}
Раскрывая,
\begin{equation*}
    (\lambda^2 + \lambda + 1)^2 + \beta \alpha = 
    \left(
        \lambda^2 + \lambda + 1 - i \gamma
    \right)
    \left(
        \lambda^2 + \lambda + 1 + i \gamma
    \right) = 0.
\end{equation*}
Получается, что
\begin{equation*}
    \lambda_{1, 2} = \frac{1}{2} \left(
        \vphantom{\frac{1}{2}}
        -1 \pm \sqrt{
        \pm 4 i \gamma - 3
        }
    \right),
\end{equation*}
где введено обозначение $\gamma = \sqrt{\beta \alpha}$. По теореме об асимптотической устойчивости достаточно, чтобы $\Re \lambda_i < 0$, соответственно найдём все $\gamma$ удовлетворяющие этому условию.

Пусть $\alpha \cdot \beta < 0$, тогда $\gamma = i \sqrt{|\alpha \beta|}$, или
\begin{equation*}
    \lambda_{1, 2} = \frac{1}{2}\left(
    \vphantom{\frac{1}{2}}
    -1 \pm \sqrt{\mp 4 \kappa - 3}
    \right),
    \hspace{0.5cm} \Rightarrow \hspace{0.5cm}
    |4 \kappa - 3| < 1,
    \hspace{0.5cm} \Rightarrow \hspace{0.5cm}
    |\kappa| = |\alpha \beta | < 1,
\end{equation*}
где было введено обозначение $\kappa = |\alpha \beta|$.


При $\alpha \cdot \beta > 0$ верно, что $\gamma = \kappa^2$, тогда
\begin{equation*}
    \Re \sqrt{z} = \Re \left(
        \sqrt{|z|} \cos \left(
            \frac{\varphi}{2} + \pi k
        \right)
    \right) < 0,
    \hspace{0.5cm} \Rightarrow \hspace{0.5cm}
    \sqrt{a^2 + b^2} \ \frac{1}{2} \left(
        1 + \frac{a}{\sqrt{a^2 + b^2}}
    \right) < 1,
\end{equation*}
где комплексное число под корнем было представлено как $a + ib$. Тогда
\begin{equation*}
    \sqrt{9 + \partial \kappa^2} - 3 < 2,
    \hspace{0.5cm} \Rightarrow \hspace{0.5cm}
    9 + 16 \kappa^2 < 5, 
    \hspace{0.5cm} \Rightarrow \hspace{0.5cm}
    |\alpha \beta|  < 1.
\end{equation*}
Получается достаточным условием асимптотической устойчивости является условие $|\alpha \beta| < 1$.

\subsubsection*{17.8}

\red{Ниже представлено решение прикольной задачи по линейной алгебре, и отсутствует доказательное решение. По-хорошему можно просто записать функцию Ляпунова, как в ответах, и всё. Диссипация не является полной в этой системе.}

Для начала рассмотрим систему, в которой нижний грузик привязан к полу пружинкой жесткости $c_{n+1} = 0$, так матрица для потенциальной энергии станет немного симметричнее. 

Выберем в качестве координат положения грузиков, где $q^i = 0$ соответствует положению равновесия $i$-го груза.  
Запишем потенциальную энергию системы
\begin{equation*}
    2 \Pi = c_1 q_1^2 + c_2(q_1-q_2)^2 + \ldots + c_n (q_n-q_{n-1})^2 + c_{n+1} q_{n+1}^2.
\end{equation*}
Тогда матрица потенциальной энергии $C$ примет вид
\begin{equation*}
    C_{ij} = \frac{\partial^2 \Pi}{\partial q^i \partial q^j},
    \hspace{0.5cm} \Rightarrow \hspace{0.5cm}
    C = \begin{pmatrix}
        c_1 + c_2 & -c_2 & 0 &  &  \\
        -c_2 & c_2 + c_3 & -c_3 & 0 &  \\
        0 & -c_3 & c_3 + c_4 &  &   \\
         & 0 &  & \ddots & -c_n \\
         &  &  & -c_n & c_n + c_{n+1}
    \end{pmatrix}
\end{equation*}
Запишем уравнение Лагранжа второго рода, и рассмотрим систему в линейном приближении
\begin{equation*}
    \frac{d }{d t} \frac{\partial T}{\partial \dot{q}^i} - \frac{\partial T}{\partial q^i}
     = - \frac{\partial \Pi}{\partial q} + Q_i,
     \hspace{0.5cm} \Rightarrow \hspace{0.5cm}
     A \ddot{\vc{q}} + B \dot{\vc{q}} + C \vc{q} = 0,
     \hspace{0.5cm} \Rightarrow \hspace{0.5cm}
     \frac{d E}{d t} =
     A \ddot{\vc{q}} \cdot \dot{\vc{q}} + C \dot{\vc{q}} \cdot \vc{q} = - B \dot{\vc{q}} \cdot \dot{\vc{q}} = - \beta \dot{q}_n^2.
\end{equation*}
Получается, что диссипация является полной, а значит имеет смысл вспомнить теорему о добавлении в систему диссипативных сил с полной диссипацией.

\begin{to_thr}[Теорема Томсона-Тэта-Четаева]
    Если в некотором изолированном положении равновесия потенциальная энергия имеет строгий локальный минимум, то при добавлении диссипативных сил с полной диссипацией (и/или гироскопических) это положение равновесия становится асимптотически устойчивым.
\end{to_thr}

По теореме Лагранжа-Дирихле положение равновесия $\vc{q} = 0$ устойчиво, если в положение равновесия достигается локальный минимум потенциала $\Pi$. Получается остается показать, что матрица $C$ положительно определена, или, по критерию Сильвестра, что все угловые миноры $\Delta_i$ матрицы $C$ положительны.

Посчитав несколько миноров ручками, приходим к виду $\Delta_i$, которое докажем по индукции.
\begin{align*}
    \text{Предположение: }\hspace{0.3 cm} 
    &
    \Delta_n = \sum_{i=1}^{n+1} \frac{1}{c_i} \prod_{j=1}^{n+1} c_j 
    \\
    \text{База: }\hspace{0.3 cm}  
    &
        \Delta_2 = \det \begin{Vmatrix}
            c_1+c_2 & -c_2 \\
            -c_2 & c_2+c_3 \\
        \end{Vmatrix} = 
        c_1 c_2 + c_2 c_3 + c_1 c_3 = \sum_{i=1}^{2+1} \frac{1}{c_i}\left(
        \prod_{j=1}^{2+1} c_j
    \right)
    \\
    \text{Переход: }\hspace{0.3 cm} 
    &
    \Delta_{n+1} 
    \overset{(\textnormal{I})}{=} %=#1
    (c_{n+1} + c_{n+1})
    \Delta_n - c_{n+1}^2 \Delta_{n-1} 
    = %=#2
    \\
    & 
    \phantom{\Delta_{n+1}} = c_{n+1} \sum_{i=1}^{n+1} \frac{1}{c_i}
    % \left(
        \prod_{j=1}^{n+1} c_j
    % \right) 
    +
     c_{n+2} \sum_{i=1}^{n+1} \frac{1}{c_i}
     % \left(
        \prod_{j=1}^{n+1} c_j
    % \right)
    -
    c_{n+1}^2 \sum_{i=1}^{n} \frac{1}{c_i}
    % \left(
        \prod_{j=1}^{n} c_j
    % \right) 
    = %=#3
    \\
    & 
    \phantom{\Delta_{n+1}} =  
    c_{n+2} \sum_{i=1}^{n+1} \frac{1}{c_i}
    % \left(
        \prod_{j=1}^{n+1} c_j
    % \right) 
    + 
    c_{n+1} 
    \left(\sum_{i=1}^{n} \frac{1}{c_i}
            % \left(
                \prod_{j=1}^{n+1} c_j
            % \right)  
            + 
            \frac{1}{c_{n+1}}
            % \left(
                \prod_{j=1}^{n+1} c_j
            % \right) 
    \right)
    - 
    c_{n+1}^2 \sum_{i=1}^{n} \frac{1}{c_i}
    % \left(
        \prod_{j=1}^{n} c_j
    % \right) 
    = 
    \\
    & 
    \phantom{\Delta_{n+1}} 
    \overset{(\textnormal{II})}{=}  %=#4
    \sum_{i=1}^{n+1} \frac{1}{c_i} \prod_{j=1}^{n+2} c_j
    + 
    \frac{1}{c_{n+2}} \prod_{j=1}^{n+2} c_j 
    = 
    \sum_{i=1}^{n+2} \frac{1}{c_i} \prod_{j=1}^{n+2} c_j,
    \hspace{1 cm}
    \textnormal{Q. E. D.}
\end{align*}
Действительно, первый переход (I) получается, раскрытием определителя $\Delta_{n+1}$ по нижней строчке. В переходе (II) были сделаны замены, вида
\begin{equation*}
        \sum_{i=1}^{n} \frac{1}{c_i}
        % \left(
            \prod_{j=1}^{n+1} c_j
        % \right) 
        = 
        c_{n+1} \sum\limits_{i=1}^{n} \frac{1}{c_i}
        % \big(
            \prod\limits_{j=1}^{n} c_j
        % \big) 
        ; \hspace{0.5 cm}
        \prod_{j=1}^{n+1} c_j = 
        \frac{1}{c_{n+2}} \prod_{j=1}^{n+2} c_j
        ; \hspace{0.5 cm}
        c_{n+2} \sum_{i=1}^{n+1} \frac{1}{c_i} \prod_{j=1}^{n+1}
        =
        \sum_{i=1}^{n+1} \frac{1}{c_i} \prod_{j=1}^{n+2} c_j.        
\end{equation*}
Полученная формула для $\Delta_n$ ясно даёт понять, что $\Delta_i > 0$ для $i = 1, \ldots, n$, что доказывает положительную определенность $C$, а значит и локальный минимум потенциала $\Pi$ достигается в положение равновесия $\vc{q}=0$. 

Таким образом выполняются условия теоремы Лагранжа-Дирихле, как и условия теоремы Томсона-Тэта-Четаева, а значит положение равновесия $\vc{q}=0$ является асимптотически устойчивым.





\subsubsection*{17.20}

Запишем систему в матричном виде
\begin{equation*}
    A \ddot{\vc{q}} + B \dot{\vc{q}} + C \vc{q} = 0,
\end{equation*}
и воспользуемся теоремой Ляпунова об асимптотической устойчивости. Действительно, существует функция, такая, что
\begin{equation*}
    V = E = T + \Pi = \frac{1}{2} a_{ij} \dot{q}^i \dot{q}^j + \frac{1}{2} c_{\alpha \beta} q^{\alpha} q^{\beta} > 0.
\end{equation*}
В силу уравнений движения
\begin{equation*}
    \frac{d E}{d t} = a_{ij} \ddot{q}^i \dot{q}^j + c_{\alpha \beta} \dot{q}^\alpha q^\beta = - b_\gamma (\dot{q}^\gamma) < 0,
\end{equation*}
из чего следует асимптотическая устойчивость системы.


\subsubsection*{17.28}

Есть некоторая система такая, что
\begin{equation*}
    \left\{\begin{aligned}
        \dot{x}^1 &= \alpha_1 (x^2 - x^1), \\
        \dot{x}^2 &= \alpha_2 (x^3 - x^2), \\
        &\ldots\\
        x^n &= \alpha_n (x^1 - x^n)
    \end{aligned}\right.
\end{equation*}
и снова найдём функцию Ляпунова, например, $V$ вида
\begin{equation*}
    2 V = \frac{1}{\alpha_1}(x_1 - a)^2 + \frac{1}{\alpha_2} (x_2 - a)^2 + \ldots + \frac{1}{\alpha_n}(x_n - a)^2,
\end{equation*}
тогда, в силу уравнений системы,
\begin{align*}
    \dot{V} &= \frac{\dot{x}_1}{\alpha_1} (x_1 - a) + \ldots + 
    \frac{\dot{x}_n}{\alpha_n}(x_n - a) = 
    (x_1 - a) (x_2 - x_1) + \ldots + (x_n - a) (x_1 - x_n) = \\
    &= - \sum_{i=1}^{n} x_i^2 + 
    \sum_{i=1}^{n-1} x_i x_{i+1} + x_n x_1 = 
    - \frac{1}{2}(x_n^2 - 2 x_n x_1 + x_1^2) - \frac{1}{2} 
    \sum_{i=1}^{n} (x_i - x_{i+1})^2 < 0,
\end{align*}
аналогично №17.20,
по теореме Ляпунова об асимптотической устойчивости,
положение равновесия системы асимптотически устойчиво.



\subsubsection*{18.17}


Известно что на груз действуют две силы
\begin{equation*}
    F_1 (t) = A_1 \sin \omega_1 t,
    \hspace{1 cm}
    F_2 (t) = A_2 \cos \omega_2 t,
\end{equation*}
и сопротивление среды $F = - \beta v$. 

Запишем кинетическую и потенциальную энергию системы
\begin{equation*}
    T = \frac{m}{2} \dot{q}^2, \hspace{1 cm}
    \Pi = \frac{c}{2}q^2.
\end{equation*}
Из уравнений Лагранжа второго рода находим
\begin{equation*}
    m \ddot{q} + \beta \dot{q} + c q = F_1 + F_2 = A \sin (\omega_1 t) + B \cos (\omega_2 t).
\end{equation*}
Для начала найдём собственные колебания системы
\begin{equation*}
    m \lambda^2 + \beta \lambda + c = 0,
    \hspace{0.5cm} \Rightarrow \hspace{0.5cm}
    \lambda_{1, 2} = \frac{-\beta \pm \sqrt{\beta^2 - 4 mc}}{2m}.
\end{equation*}
Найдём теперь частные решения для вынужденных колебаний, в виде
\begin{align*}
    q = \alpha_1 \sin (\omega_1 t + \varphi_1) + \alpha_2 \sin (\omega_2 t + \varphi_2),
\end{align*}
подставляя в уравнения движения получам, что (рассмотрим $\omega_1$, для $\omega_2$ рассуждения аналогичны)
\begin{equation*}
    \sin (\omega_1 t + \varphi_1) (x - m \omega_1^2) + \cos (\omega_1 t + \varphi_1) \omega_1 \beta = \frac{A}{\alpha_1}\sin \omega_1 t,
    \hspace{0.5cm} \Rightarrow \hspace{0.5cm}
    \sin(\omega_1 t + \varphi_1 + \kappa) = \frac{A}{\alpha_1} \frac{\sin \omega_1 t}{\sqrt{
    (c-m \omega_1)^2 + \beta^2 \omega_1^2
    }},
\end{equation*}
где $\kappa$ такая, что
\begin{equation*}
    \cos \kappa = \frac{c - m \omega_1^2}{\sqrt{
        (\omega_1 \beta)^2 + (c-m \omega_1)^2
    }}.
\end{equation*}
Сравнивая выражения, находим константы
\begin{equation*}
\left\{\begin{aligned}
        \varphi_1 &= - \kappa_1 \\
        \varphi_2 &= \frac{\pi}{2} - \kappa_2
\end{aligned}\right.
\hspace{1 cm}
    \alpha_i (\omega_i) = \frac{A_i}{\sqrt{(m \omega_i-c)^2+\omega_i^2 \beta^2}},
\end{equation*}
и подставляем в ответ
\begin{equation*}
    q = \alpha_1 \sin (\omega_1 t + \varphi_1) + \alpha_2 \sin (\omega_2 t + \varphi_2).
\end{equation*}
\subsubsection*{№18.31}


И снова запишем кинетическую и потенциальную энергию системы, как
\begin{equation*}
    T = \frac{1}{2} J \left(\varphi_1^2 + \varphi_2^2\right),
    \hspace{1 cm}
    \Pi = \frac{c}{2} \varphi_1^2 + \frac{c}{2}(\varphi_2 - \varphi_1)^2.
\end{equation*}
Из уравнений Лагранжа второго рода перейдём к систем\footnote{
    Тут при решении была потеряна \red{двойка}, выделенная красным цветом, но перерешивать как-то грустно.
} 
\begin{align*}
    J \ddot{\varphi}_1 + c(\text{\red{2}}\varphi_1 - \varphi_2) &= M_0 \sin \omega t\\
    J \ddot{\varphi}_2 + \beta \dot{\varphi}_2 + c (\varphi_2 - \varphi_1) &= 0.
\end{align*}
Искать собственные числа здесь оказалось плохой идеей, так что просто будем искать решение в виде
\begin{equation*}
    \vc{\varphi} = \begin{pmatrix}
        a_1 \\ a_2
    \end{pmatrix} e^{i\omega t} - 
    \begin{pmatrix}
        b_1 \\ b_2
    \end{pmatrix} e^{-i\omega t}.
\end{equation*}
Для первого слагаемого
\begin{equation*}
    \left\{\begin{aligned}
        - J \omega^2 a_1 + c a_1 - c a_2 &= \mathcal M \\
        - J \omega^2 a_2 + \beta i \omega a_2 + c a_2 - c a_1 &= 0
    \end{aligned}\right.
    \hspace{0.5cm} \Rightarrow \hspace{0.5cm}
    \left\{\begin{aligned}
        a_1 (c - J \omega^2) - c a_2 &= \mathcal M \\
        a_2 (c - J \omega^2 + i \beta \omega) &= c a_1
    \end{aligned}\right.
\end{equation*}
Для второго слагаемого
\begin{equation*}
    \left\{\begin{aligned}
        - J \omega^2 b_1 + c b_1 - c b_2 &= - \mathcal M \\
        - J \omega^2 b_2 - \beta i \omega b_2 + c b_2 - c b_1 &= 0
    \end{aligned}\right.
    \hspace{0.5cm} \Rightarrow \hspace{0.5cm}
    \left\{\begin{aligned}
        &b_1 = \frac{b_2}{c} (c - J \omega^2 - i \beta \omega) \\
        &b_2 \left(\frac{c - J \omega^2}{c} (c - J \omega^2 + i \beta \omega - c)\right) = - \mathcal M
    \end{aligned}\right.
    ,
\end{equation*}
где $\mathcal M = M_0 / (2 i)$. Также хочется ввести некоторые постоянные
\begin{equation*}
    \kappa = \frac{c - J \omega^2}{c} (c - J \omega^2 + i \beta \omega) - c,
    \hspace{1 cm}
    \xi = \frac{c - J \omega^2}{c} (c - J \omega^2 + i \beta \omega - c),
    \hspace{1 cm}
    \eta = 
\end{equation*}
тогда получим хорошие выражения для искомых переменных
\begin{equation*}
    \left\{\begin{aligned}
        a_1 &= \frac{\mathcal M}{\kappa} \frac{c - J \omega^2 + i \beta \omega}{c} \\
        a_2 &= \frac{\mathcal M}{\kappa}
    \end{aligned}\right.
    , \hspace{1 cm}
    \left\{\begin{aligned}
         b_1 &= - \frac{\mu}{\xi} \frac{c - J \omega^2 - i \beta \omega}{c} \\
       b_2 &= - \frac{\mu}{\xi}
    \end{aligned}\right. .
\end{equation*}
Теперь их можно поставить в решение уравнения и получить ответ:
\begin{equation*}
    \vc{\varphi} = \begin{pmatrix}
        a_1 \\ a_2
    \end{pmatrix} e^{i\omega t} - 
    \begin{pmatrix}
        b_1 \\ b_2
    \end{pmatrix} e^{-i\omega t}.
\end{equation*}
\subsubsection*{18.37}

Момент инерции стержня $J = \frac{1}{3} m l^2$, тогда, считая отклонения малыми, кинетическую и потенциальную энергию системы можем записать, как
\begin{equation*}
    T = \frac{1}{2} J \left(\dot{\varphi}^2 + \dot{\psi}^2\right),
    \hspace{1 cm}
    \Pi = \frac{1}{2}c (\varphi a - \psi a)^2 + \left(
        1 - \frac{\varphi^2}{2} + 1 - \frac{\psi^2}{2}
    \right) mg \frac{l}{2}.
\end{equation*}
Переходя в СО движущейся платформы, к системе добавляется инерциальная сила
\begin{equation*}
    M = \frac{mA}{2} \sin(\omega t) \omega^2 l,
\end{equation*}
действующая на центры масс стержней.

С помощью уравнений Лагранжа второго рода переходим к уравнениям вида
\begin{equation*}
    A \ddot{\vc{q}} + C \vc{q} = M,
    \hspace{1 cm}
    A = J \begin{pmatrix}
        1 & 0 \\
        0 & 1 \\
    \end{pmatrix},
    \hspace{1 cm}
    C = \frac{1}{2}\begin{pmatrix}
        2a^2c + mgl & -2ca^2 \\
        -2ca^2 & 2a^2 c + mgl \\
    \end{pmatrix}
\end{equation*}
Из векового уравнения теперь можем найти собственные частоты системы, для получения однородного решения
\begin{equation*}
    \det(C - \lambda A) = 0,
    \hspace{0.25cm} \Rightarrow \hspace{0.25cm}
    \left(
        mg \frac{l}{2} - J \lambda
    \right) \left(
        a^2 c + mg \frac{l}{2} - J \lambda
    \right) = 0,
\end{equation*}
откуда легко находим $\lambda$
\begin{equation*}
    \lambda_1 = \frac{3}{2}\frac{g}{l}, \hspace{1 cm},
    \hspace{0.5 cm} \vc{u}_1 = \begin{pmatrix}
        1 \\ 1
    \end{pmatrix},
    \hspace{1 cm}
    \lambda_2 = \frac{3}{2} \frac{g}{l} + \frac{6ca^2}{ml^2},
    \hspace{0.5 cm} 
    \begin{pmatrix}
        1 \\ -1
    \end{pmatrix}.
\end{equation*}
из которых уже можем составить ФСР.

Теперь перейдём к поиску частного решения\footnote{
    Так как по условию $\varphi$ и $\psi$ малые, то про резонанс говорить не приходится.
} :
\begin{equation*}
    \varphi = \alpha \sin (\omega t), 
    \psi = \beta \sin (\omega t),
    \hspace{0.5cm} \Rightarrow \hspace{0.5cm}
    -A \omega^2 \begin{pmatrix}
        \alpha \\ \beta
    \end{pmatrix} + C \begin{pmatrix}
        \alpha \\ \beta
    \end{pmatrix} = 
    \frac{m A \omega^2 l}{2} \begin{pmatrix}
        1   \\ 1
    \end{pmatrix}.,
\end{equation*}
вводя матрицу
\begin{equation*}
    \Lambda = C - A \omega^2,
    \hspace{0.5cm} \Rightarrow \hspace{0.5cm}
    \Lambda \begin{pmatrix}
        \alpha \\ \beta
    \end{pmatrix} = 
    \frac{m A \omega^2 l}{2} \begin{pmatrix}
        1 \\ 1
    \end{pmatrix},
    \hspace{0.5cm} \Leftrightarrow \hspace{0.5cm}   
    \begin{pmatrix}
        \alpha \\ \beta
    \end{pmatrix} 
    =
     \Lambda^{-1} \,
    \frac{m A \omega^2 l}{2} \begin{pmatrix}
        1 \\ 1
    \end{pmatrix}.
\end{equation*}
Считая $\Lambda^{-1}$, находим частное решение и получаем ответ
\begin{equation*}
    \begin{pmatrix}
        \varphi \\ \psi
    \end{pmatrix} = 
    \frac{3 A \omega^2}{3 g - 2 l \omega^2} \begin{pmatrix}
        1 \\ 1
    \end{pmatrix} \sin (\omega t) + 
    C_1 \begin{pmatrix}
        1 \\ 1
    \end{pmatrix}
    \sin\left(
        \sqrt{\frac{3}{2}\frac{g}{l}} \, t + \alpha_1
    \right) + 
    C_2 \begin{pmatrix}
        1 \\ -1
    \end{pmatrix} 
    \sin\left(
        \sqrt{
        \frac{3}{2} \frac{g}{l} + \frac{6 c a^2}{ml^2}
        } \, t + \alpha_2
    \right).
\end{equation*}








\subsubsection*{18.62}
Известно, что кинетическая и потенциальная энергия системы могут быть записаны, как
\begin{equation*}
    T = \frac{1}{2} a_{ik} \dot{q}^i \dot{q}^k,
    \hspace{1 cm}
    \Pi = \frac{1}{2} c_{ik} q^i q^k.
\end{equation*}
С помощью уравнений Лагранжа второго рода можем перейти  к системе
\begin{equation*}
    A \ddot{\vc{q}} + C \dot{q} = A \vc{u}_1 \gamma \sin (\omega t).
\end{equation*}
Так как $A, \,C$ -- (невырожденные) положительно-определенные симметричные квадратичные формы, то они во-первых обратимы, а во вторых коммутируют (т.к. одновременно приводятся к диагональному виду), а значит и $A^{-1} C$ симметрична, соответственно имеет ортогональный базис.

Собственно, известно, что
\begin{equation*}
    \left\{\begin{aligned}
        \det (C - \lambda_i A) = 0 \\
        (C - \lambda_i A) \vc{u}_i = 0,
    \end{aligned}\right.
    \hspace{0.5cm} \Rightarrow \hspace{0.5cm}   
    A^{-1} C \, \vc{u}_i = \lambda_1 \vc{u}_i.
\end{equation*}
Перейдём к базису из собственных векторов (и переменным $\theta$), тогда уравнения примут вид
\begin{equation*}
    \ddot{\vc{q}} + \begin{pmatrix}
        \lambda_1 &  &  \\
         & \ddots &  \\
         &  & \lambda_n \\
    \end{pmatrix} \dot{q} = \begin{pmatrix}
        1 \\ \vdots \\ 0
    \end{pmatrix} \gamma \sin (\omega t).
\end{equation*}
Так как резонанс возможен только на собственных частотах системы, и $\lambda_1 = \omega_1^2$, то единственная частота, на которой возможен резонанс равна $\omega_1$.
