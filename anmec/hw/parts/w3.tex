
% \subsubsection*{no ?Т2}
\subsection{Т2}

\textbf{Интеграл Дирихле}. Вычислим \textit{интеграл Дирихле}
\begin{equation}
    I(\alpha) = \int_0^\infty \frac{\sin \alpha x}{x} \d x
\end{equation}
Для начала вычислим некоторый другой интеграл:
\begin{equation*}
    \Phi(\alpha, \beta) = \int_0^{\infty} e^{-\beta x} \frac{\sin (\alpha x)}{x} \d x,
    \hspace{10 mm}
    \Phi'_\alpha (\alpha, \beta) = 
    \int_0^\infty e^{-\beta x} \cos (\alpha x) \d x = \frac{\beta}{\alpha^2 + \beta^2}.
\end{equation*}
Действительно, считая $f'_\alpha (x, \alpha) = e^{-\beta x} \cos (\alpha x)$, заметим, что $f$ и $f'_\alpha$ непрерывны на $E$, $\int_0^{\infty} f(x, \alpha) \d x$ сходится $\forall \alpha \in \mathbb{R}$ по Дирихле:
\begin{equation*}
    \bigg|
        \int_0^{\infty} \sin (\alpha x) \d x
    \bigg| = \bigg|
        \frac{\cos (\alpha t) - 1}{\alpha}
    \bigg| \leq \frac{2}{|\alpha|}, \hspace{5 mm} \alpha \neq 0,
\end{equation*}
а функция $x^{-1} e^{-\beta x}$ убывает на промежутке $(0, +\infty)$, также верно, что $\int_0^{\infty} f'_\alpha (x, \alpha) \d x$ сходится равномерно по признаку Вейерштрассе, следовательно можем дифференцировать под знаком интеграла. 

Теперь, интегриря $\alpha$ на отрезке $[0, \alpha]$ находим
\begin{equation*}
    \Phi(\alpha, \beta) - \Phi(0, \beta) = \beta \int_{0}^{\alpha} \frac{d t}{t^2 + \beta^2} = 
    \arctg \frac{\alpha}{\beta} 
    % \overset{\mathrm{?}}{\to} I(\alpha)
    .
\end{equation*}
Понятно, что $I(\alpha) = - I(\alpha)$, так что далее считаем $\alpha > 0$. Имеем право рассмотреть $\beta \in [0, 1]$, точнее предел
\begin{equation*}
    \lim_{\beta \to +0} \int_0^{+\infty} e^{-\beta x} \frac{\sin(\alpha x)}{x} \d x = I(\alpha) =
    \lim_{\beta \to +0} \arctg \frac{\alpha}{\beta} = \frac{\pi}{2}.
\end{equation*}
Таким образом для произвольного $\alpha$ верно, что
\begin{equation}
    \int_0^{\infty} \frac{\sin(\alpha x)}{x} \d x = \frac{\pi}{2} \sign (\alpha).
\end{equation}

\textbf{Интеграл Лапласа}. Вычислим интегралы Лапласа
\begin{equation*}
    I(\alpha) = \int_0^\infty \frac{\cos \alpha x}{1 + x^2} \d x = \int_0^\infty f(x, \alpha) \d x, 
    \hspace{10 mm}
    K(\alpha) = \int_0^\infty \frac{x \sin \alpha x}{1 + x^2} \d x.
\end{equation*}
Без ограничения общности рассмотрим $\alpha > 0$. Проверим, что можем дифференцировать под знаком интеграла: $f(x, \alpha)$ непрерывна $\forall \alpha, \, x$, интеграл
\begin{equation*}
    \int_0^{+\infty} f'_\alpha \d x = - \int_0^{+\infty} \frac{x \sin \alpha x}{1 + x^2} \d x,
\end{equation*}
сходится равномерно по  $\alpha$ на $[a_0, +\infty)$ для $\forall \alpha_0 > 0$, получается верно, что
\begin{equation*}
    I'(\alpha) = - \int_0^{+\infty}  \frac{x \sin \alpha x}{1 + x^2} = 
    - K(\alpha).
\end{equation*}
Складывая с известным выражением интеграла Дирихле, находим
\begin{equation*}
    I'(\alpha) + \frac{\pi}{2} = \int_0^{+\infty}  \left(
        \frac{\sin \alpha x}{x} - \frac{x \sin \alpha x}{1+ x^2} 
    \right) \d x = 
    \int_0^{+\infty} \frac{\sin \alpha x}{x (1 + x^2)} \d x.
\end{equation*}
Аргумент интеграла непрерывен, как и его производная по $\alpha$, они Лебег-интегрируемы, поэтому, дифференцируя под знаком интеграла, находим
\begin{equation*}
    I''(\alpha) = \int_0^{+\infty}  \frac{\cos \alpha x}{1 + x^2} \d x.
\end{equation*}
Так мы приходим к дифференциальному уравнению на $I(\alpha)$:
\begin{equation*}
    I''(\alpha) - I(\alpha) = 0,
    \hspace{0.5cm} \Rightarrow \hspace{0.5cm}
    I(\alpha) = C_1 e^\alpha + C_2 e^{-\alpha}.
\end{equation*}
Рассматривая пределы $\alpha \to 0$ и $\alpha \to + \infty$, находим константы интегрирования $C_1 = 0$ и $C_2 = \pi/2$. В силу четности $I(\alpha)$ находим
\begin{equation*}
    I(\alpha) = \frac{\pi}{2} e^{-|\alpha|}, \hspace{5 mm} \alpha \in \mathbb{R}.
\end{equation*}
Бонусом находим $K(\alpha) = - I'_\alpha (\alpha)$:
\begin{equation*}
    K(\alpha) = \frac{\pi}{2} e^{-|\alpha|} \cdot \sign \alpha.
\end{equation*}

\textbf{Интегралы Френеля}. Вычислим \textit{интеграл Френеля}
\begin{equation*}
    I = \int_0^{+\infty} \sin^2 x^2 \d x.
\end{equation*}
Для нахождения нам понадобится \textit{интеграл Эйлера-Пуассона} и, возможно, \textit{интеграл Лапласа}:
\begin{equation*}
    \int_0^{+\infty} e^{-x^2} \d x = \frac{\sqrt{\pi}}{2},
    \hspace{10 mm}
    I(\alpha) = \int_0^{+\infty}  e^{-x^2} \cos 2 \alpha x \d x = \frac{\sqrt{\pi}}{2} e^{-\alpha^2}.
    .
\end{equation*}
Полагая $x^2 = t$ запишем интеграл $I$ в виде
\begin{equation*}
    I  = \frac{1}{2} \int_0^{+\infty}  \frac{\sin t}{\sqrt{ t}} \d t.
\end{equation*}
При $t > 0$ справедливо равенство
\begin{equation}
    \int_0^{+\infty} e^{-t u^2} \d u =
    \bigg/
        x = \sqrt{t} u
    \bigg/ = \frac{1}{2} \sqrt{\frac{\pi}{t}},
    \hspace{0.5cm} \Rightarrow \hspace{0.5cm}
    \frac{1}{\sqrt{t}} = \frac{2}{\pi} \int_0^{+\infty}  e^{-t u^2} \d u,
\end{equation}
Так приходим к двойному интегралу
\begin{equation*}
    I = \frac{1}{\sqrt{\pi}} \int_0^{+\infty} \sin t \d t \int_0^{+\infty}  e^{- t u^2} \d u.
\end{equation*}
Меняя порядок интегрирования, получаем
\begin{equation*}
    I = \frac{1}{\sqrt{\pi}} \int_0^{+\infty} \d u \int_0^{+\infty}  e^{-tu^2} \sin t \d t = 
    \frac{1}{\sqrt{\pi}} \int_0^{+\infty} \frac{\d u}{1 + u^4}.
\end{equation*}
Который легко вычисляется, если заметить, что
\begin{equation*}
    \int_0^{+\infty} \frac{x^2 \d x }{1 + x^4} = \int_0^{+\infty} \frac{(1/x^2) \d x}{1 + (1/x)^4} = \int_0^{+\infty} \frac{\d x}{1 + x^4}.
\end{equation*}
Поэтому 
\begin{equation*}
    2 \int_0^{+\infty}  \frac{\d x}{1 + x^4} = \int_0^{+\infty}  \frac{(1 + 1/x^2)\d x}{x^2 + 1/x^2} = \int_0^{+\infty}  \frac{\d (x - 1/x)}{ (x-1/x)^2 + 2} = \frac{1}{\sqrt{2}} \arctg\left(
        \frac{x-1/x}{\sqrt{2}}
    \right) \bigg|_{0}^{\infty} = \frac{\pi}{\sqrt{2}}.
\end{equation*}
Откуда уже и получаем
\begin{equation}
    I = 
    \int_0^{+\infty} \sin^2 x^2 \d x
    =
    \frac{1}{\sqrt{\pi}} \cdot \frac{\pi}{2 \sqrt{2}} = \frac{1}{2} \sqrt{\frac{\pi}{2}}.
\end{equation}




\subsubsection*{Т3}
Исследуем систему вида
\begin{equation*}
    \left\{\begin{aligned}
        \dot{x} &= y, \\
        \dot{y} &= k \left(
            \frac{b}{a-x} - x
        \right)
    \end{aligned}\right.
\end{equation*}
Рассмотрим положение равновесия $\dot{x} = \dot{y} = 0$, при $x \neq a$
\begin{equation*}
    x^* = \frac{1}{2}\left(
        a \pm \sqrt{a^2 - 4b}
    \right),
\end{equation*}
что приводит нас к следующим случаям. 

Пусть $a^2 = 4b$, тогда $x^* = a/2$, попробуем найти фазовый портрет по линейному приближению
\begin{equation*}
    \det(J - \lambda E) = 
     \lambda^2 - k\left(
        \frac{b}{(a-x^*)^2}-1
     \right) = k \cdot 0 = 0,
     \hspace{0.5cm} \Rightarrow \hspace{0.5cm}
     \lambda = 0,
\end{equation*}
следовательно линейным приближением здесь не воспользоваться.

При $a^2 > 4b$, 
\begin{equation*}
    x^* = \frac{a}{2} \pm \sqrt{
    \left(\frac{a}{2}\right)^2 - b
    } = \frac{a}{2} \pm \delta,
\end{equation*}
тогда
\begin{equation*}
    \lambda^2 = \frac{5}{4} (4b - a^2) \pm a \delta,
\end{equation*}
в случае $+ a\delta$ $\Re \lambda_{1, 2} = 0$, следовательно это \textit{центр}, при $-a \delta$ получается $\lambda^2 = 100 b - 9 a^2 > 64 b > 0$, следовательно это \textit{седло}.

При $a^2 < 4b$ не существует положения равновесия, что приводит нас к физовым диаграммам аналогичным задаче Т4. 


\subsection*{Т4}
\addcontentsline{toc}{subsection}{T4}

Теперь рассмотрим реакцию превращения электрона и позитрона в мюон и антимюон:
\begin{equation*}
    e^+ + e^- \to \mu^+ + \mu^-.
\end{equation*}
Хотелось бы зная энергию стакивающихся частиц найти эффективную массу системы и энергии $\mu^{\pm}$.

Для 4-импульса $p^i = (\mathscr{E}/c, \vc{p})$, для которого верно
\begin{equation*}
    c^2 (2 m_\mu)^2 
    \leq(p_1^i + p_2^i)^2 = \bar{p}_1^2 + \bar{p}_2^2 + 2 \bar{p}_1 \cdot \bar{p}_2 = 
    c^2 2 m_e^2 + 2 \l(
        \mathscr{E}_1 \mathscr{E}_2 / c^2 - \vc{p}_1 \cdot \vc{p}_2
    \r).
\end{equation*}
что приводит нас к неравенству
\begin{equation*}
    c^2 (2 m_\mu^2 - m_e^2) \leq \frac{1}{c^2} \mathscr{E}_1 \mathscr{E}_2 - \vc{p}_1 \cdot \vc{p}_2.
\end{equation*}
При равных энергия $\mathscr{E}_1 = \mathscr{E}_2 = \mathscr{E}$ и $\vc{p}_1 = - \vc{p}_2$ верно, что
\begin{equation*}
    \vc{p}_1^2 = \frac{\mathscr{E}^2}{c^2} - m^2 c^2,
\end{equation*}
тогда
\begin{equation*}
    c^2 (2 m_\mu^2 - m_e^2) \leq \frac{1}{c^2} \mathscr{E}^2 + \vc{p}_1^2 = \frac{2}{c^2} \mathscr{E}^2 - m_e^2 c^2,
\end{equation*}
таким образом 
\begin{equation*}
    \mathscr{E} \geq m_\mu c^2,
    \hspace{5 mm}
    T_{\textnormal{порог}} = (m_\mu - m_e) c^2,
\end{equation*}
а эффективной массе системы соответствует ...

При налете на неподвижную частицу $\mathscr{E}_2 = m_e c$ и $\vc{p}_2 = 0$, тогда
\begin{equation*}
    (2 m_\mu^2 - m_e^2) c^2 \leq \mathscr{E}_1 m_e,
    \hspace{0.5cm} \Rightarrow \hspace{0.5cm}
    \mathscr{E}_1 \geq \left(
        2 \frac{m_\mu^2}{m_e} - m_e
    \right) c^2.
\end{equation*}
Соответсвенно для пороговой энергии верно
\begin{equation*}
    T_{\textnormal{порог}} = \frac{2 c^2}{m_e} \left(
        m_\mu^2 - m_e^2
    \right),
\end{equation*}
а эффективной массе значение ...
\subsection*{Т5}
\addcontentsline{toc}{subsection}{T5}

Имеем две частицы, 4-импульсы которых в начальный момент:
\begin{equation*}
	p_\gamma^i = \begin{pmatrix}\varepsilon_\gamma \\ \vc{p}_\gamma \end{pmatrix},
	\hspace{0.5 cm}
	|p_\gamma^i| \approx \varepsilon_\gamma.
	\hspace{2 cm}
	p_e^i = \begin{pmatrix} \varepsilon_e \\ \vc{p}_e\end{pmatrix},
	\hspace{0.5 cm}
	|p_e^i| =\beta_e \varepsilon_e.
\end{equation*}

Перейдём в систему центра инерции двух частиц. Пусть пусть движется со скоростью $\beta$, тогда матрица преобразования для такой пересадки и аберрация угла будут
\begin{equation*}
 	\begin{pmatrix}
 	    \gamma & \gamma \beta & 0 & 0 \\
 	    \gamma \beta & \gamma & 0 & 0 \\
 	    0 & 0 & 1 & 0 \\
 	    0 & 0 & 0 & 1 \\
 	\end{pmatrix},
 	\hspace{1 cm}
 	\cos \theta' = \frac{\cos \theta - \beta}{1 - \cos \theta \beta}.
\end{equation*}
Запишем закон сохранения импульса до и после столкновения, штрихами пометим величины после столкновения.
\begin{equation*}
	p_\gamma^i + p_e^i = p_\gamma'^i + p_e'^i
	\hspace{0.5 cm}
	\Rightarrow
	\hspace{0.5 cm}
	(p_e'^i)^2 = (p_\gamma^i + p_e^i - p_\gamma'^i)^2
	=
	{p_\gamma^2} + p_e^2 + p_\gamma'^2 + 2 p_e p_\gamma - 2 p_e p_\gamma' - 2 p_\gamma p_\gamma',
\end{equation*}
пренебрегая квадратом импульса фотонов получаем
\begin{equation*}
	m_e^2 = m_e^2 + 2 p_e p_\gamma - 2 p_e p_\gamma' - 2 p_\gamma p_\gamma'
	\hspace{0.5 cm}
	\Rightarrow
	\hspace{0.5 cm}
	p_e p_\gamma -  p_e p_\gamma' -  p_\gamma p_\gamma' = 0.
\end{equation*}
Перемножим компоненты 4-импульсов:
\begin{equation*}
	\varepsilon_e \varepsilon_\gamma - \vc{p}_e \cdot \vc{p}_\gamma - \varepsilon_e \varepsilon_\gamma' + \vc{p}_e \cdot \vc{p}_\gamma' - \varepsilon_\gamma \varepsilon_\gamma' + \vc{p}_\gamma \cdot \vc{p}_\gamma' = 0.
\end{equation*}
Пусть частицы разлетелись под углом $\theta$:
\begin{equation*}
	\varepsilon_e \varepsilon_\gamma + \varepsilon_e \varepsilon_\gamma \beta_e - \varepsilon_e \varepsilon_\gamma' + \beta_e \varepsilon_e \varepsilon_\gamma' \cos \theta - \varepsilon_\gamma \varepsilon_\gamma' + \varepsilon_\gamma \varepsilon_\gamma' \cos (\pi-\theta) = 0.
\end{equation*}
Откуда не сложно выразить энергию фотона после столкновения, заметим, что по условию задачи: $\varepsilon_\gamma/\varepsilon_e = 10^{-11}$, такими членами будем пренебрегать:
\begin{equation*}
	\varepsilon_\gamma' = \frac{\varepsilon_e \varepsilon_\gamma(1 + \beta_e)}{\varepsilon_e (1 - \beta_e \cos \theta) + \varepsilon_\gamma (1 + \cos \theta)}
	=
	\frac{\varepsilon_\gamma (1 + \beta_e)}{1 - \beta \cos \theta + \frac{\varepsilon_\gamma}{\varepsilon_e}(1 + \cos \theta)} 
	\approx \frac{\varepsilon_\gamma (1 + \beta_e)}{1 - \beta_e \cos \theta}.
\end{equation*}

Имея формулу плюс-минус общую не сложно ответить на вопрос про рассеяние назад:
\begin{equation*}
	\varepsilon_\gamma' (cos \theta = -1) \approx \varepsilon_\gamma = 2 \text{ эВ}. 
\end{equation*}
в то время, как вперед пролетает:
\begin{equation*}
	\varepsilon_\gamma' (cos \theta = 1) \approx \frac{\varepsilon_\gamma \varepsilon_e^2}{m_e^2} \approx 320 \text{ ГэВ}. 
\end{equation*}
\subsection*{Т6}
\addcontentsline{toc}{subsection}{T6}
 
 Пион распадается на нейтрино и мюон: $\pi \to \mu + \nu$. Будем работать в система центра инерции.
 \begin{equation*}
 	p_{0 \mu}^i = p_{0 \mu}^i + p_{0 \nu}^i
 	\hspace{0.5 cm}
 	\Rightarrow
 	\hspace{0.5 cm}
 	(p_{0\mu}^i)^2 = (p_{0 \pi}^i - p_{0 \nu}^i)^2
 	\hspace{0.5 cm}
 	\Rightarrow
 	\hspace{0.5 cm}
 	m_\mu^2 c^2 = c^2 (m_\pi^2 + m_\nu^2) - 2 p_{0\pi}^i p_{i0\nu}  = c^2 m_\pi^2 - 2 m_\pi \varepsilon_{0 \nu}.
 \end{equation*}
Откуда получаем
\begin{equation*}
	\varepsilon_{o \nu} = \frac{m_\pi^2 - m_\mu^2}{2 m_\pi}c^2 = \frac{140^2 - 105^2}{2 \cdot 140} \cdot 1^2 = 31 \text{ МэВ}.
\end{equation*}

Переходя в лабораторную систему отсчёта:	
\begin{equation*}
	\varepsilon_\nu = \gamma(v) \varepsilon_{0 \nu} \left(1 - \frac{v}{c} \cos \theta_0\right).
\end{equation*}
Подставляя углы $\theta_0$ найдём минимальную и максимальную энергии:
\begin{equation*}
	\varepsilon_{\text{min}}^\nu = \varepsilon_{0 \nu} \gamma(v) \left(1 - \frac{v}{c}\right) \approx 0.4 \text{ МэВ},
	\hspace{1 cm}
	\varepsilon_{\text{max}}^\nu = \varepsilon_{0 \nu} \gamma(v) \left(1 + \frac{v}{c}\right) \approx 2666 \text{ МэВ}.
\end{equation*}

Для определения среднего значения сначала нужно задаться вопросом распределения по углу отклонения, пока в системе покоя $\pi$:
\begin{equation*}
	\varepsilon_\nu = \gamma(v) \varepsilon_{0 \nu} \left(1 - \frac{v}{c} \cos \theta_0\right)
	\hspace{0.5 cm}
	\overset{d}{\Rightarrow}
	\hspace{0.5 cm}
	d \varepsilon = \frac{v}{c} \gamma \varepsilon_0 \d \cos \theta_0.
\end{equation*}
Из всех частиц $N_0$ в телесном угле $d \Omega_0$ заключено:
\begin{equation*}
	\frac{d N}{N_0} = \frac{d \Omega_0}{4 \pi} = \frac{1}{2} (d \cos \theta) \frac{d \varphi}{2 \pi}
	\hspace{1 cm}
	\Rightarrow
	\hspace{1 cm}
	\frac{d N}{N_0} = \frac{1}{4 \pi} \frac{c \d \varepsilon}{v \gamma \varepsilon_0} (2 \pi) = \frac{d \varepsilon}{\varepsilon_{\text{max}} - \varepsilon_{\text{min}}}.
\end{equation*}

Но это всё было в системе центра инерции, нужно перейти в лабораторную, а тогда произойдёт аберрация:
\begin{equation*}
	\cos \theta = 
	\frac{\cos \theta - \beta}{1 - \beta \cos \theta}.
\end{equation*}
Таким образом
\begin{equation*}
	\frac{d N}{d \cos \theta} = \left(\frac{d N}{d \cos \theta'}\right) \frac{d \cos \theta'}{d \cos \theta} 
	=
	\left(\frac{d N}{d \cos \theta'}\right) \frac{1 - \beta^2}{(\beta \cos \theta - 1)^2},
\end{equation*}
где $d N/d \cos \theta'$ --- распределение по углу в системе центра инерции, которое в силу изотропности пространства постоянно. Так как в правой части отсутствует энергия, то распределение энергии по углу -- постоянно, тогда 
\begin{equation*}
	\langle \varepsilon^\nu\rangle = 1333 \text{\ МэВ}.
\end{equation*}

