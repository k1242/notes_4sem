
% \subsubsection*{no ?Т2}
\subsection*{Т2}
\addcontentsline{toc}{subsection}{T2}

Аппроксимируем движение нИСО в моменты времени $t$ и $t+dt$ сопутствующими ИСО $K'$ и $K''$. Пусть $K$ -- лабороторная система отсчета, $K'$ -- сопутствующая ИСО $\vc{v} \overset{\mathrm{def}}{=}  \vc{v}(t)$, а $K''$ -- сопутствующая ИСО движущаяся относительно $K$ со скоростью $\vc{v}(t + \d t)  = \vc{v} + \d \vc{v}$. Далее для удобства будем считать, что $K''$ движется относительно $K'$ со скоростью $\d \vc{v}'$.

Проверим, что последовательное применеие $\Lambda(d \vc{v}') \cdot \Lambda(\vc{v})$ эквивалентно
$R(\varphi) \cdot \Lambda(\vc{v} + \d \vc{v})$, где $R(\varphi)$ -- вращение в $\{xyz\}$. Для этого просто найдём 
\begin{equation*}
    R(\varphi) = \Lambda(d \vc{v}') \cdot \Lambda(\vc{v}) \cdot \Lambda(\vc{v} + \d \vc{v})^{-1}.
\end{equation*}

Пусть ось $x \parallel \vc{v}$, ось $y$ выберем так, чтобы $d\vc{v} \in \{Oxy\}$. Теперь, согласно $\eqref{LORENTS}$, считая $|\vc{v}|=\beta_1$, $d \vc{v}' = (\beta_x',\, \beta_y')\T$ можем записать (пренебрегая слагаемыми $\beta_x', \beta_y'$ второй и выше степени):
\begin{equation*}
    \Lambda(\vc{v}) =
    \left(
        \begin{array}{cccc}
         \gamma_1 & - \beta_1 \gamma_1 & 0 & 0 \\
         -\beta_1 \gamma_1 & \gamma_1 & 0 & 0 \\
         0 & 0 & 1 & 0 \\
         0 & 0 & 0 & 1 \\
        \end{array}
    \right),
    \hspace{5 mm}
    \Lambda(d \vc{v}') = 
\begin{bmatrix}
 1 & -\beta_x' & -\beta_y' & 0 \\
 -\beta_x' & 1 & 0 & 0 \\
 -\beta_y' & 0 & 1 & 0 \\
 0 & 0 & 0 & 1 \\
\end{bmatrix}.
\end{equation*}
Теперь можем выразить $d \vc{v}'$ через $d \vc{v}$, считая $\vc{r}_{\mathrm{f}}$ центром системы $K''$
\begin{equation*}
    \vc{r}_f' = \Lambda(d \vc{v}') \cdot \Lambda(\vc{v}) \vc{r}_f = 
\l(
    c t', \ 0, \ 0, \ 0
\r)\T
\hspace{0.5cm} \Rightarrow \hspace{0.5cm}
\beta(\vc{v}+d \vc{v})_x = \frac{\beta_1 + \beta_x'}{1 + \beta_1 \beta_x'},
\hspace{5 mm}
\beta(\vc{v}+d \vc{v})_y = \frac{\gamma_{\beta_1} \beta_y}{1+\beta_1 \beta_x}.
% 
\end{equation*}
где скорость находим аналогично первому номеру. Тут стоит заметить, что скоростью $\beta_x$ можно было бы пренебречь в сравнении с $\beta_1$, так как скорее всего первый порядок малось $\beta_x$ не войдёт в ответ, однако хотелось бы в этом убедиться.

Зная $d \vc{v}$ можем найти $d \vc{v}'$:
\begin{equation*}
    \beta_x' = \gamma_{\beta_1}^2 \beta_x,
    \hspace{5 mm}
    \beta_y' = \gamma \beta_y.
\end{equation*}
Но это на потом.

Через $\vc{v}, \ d \vc{v}'$ теперь можем найти $\Lambda(\vc{v} + d \vc{v})$, и посчитать обратную матрицу:
\begin{equation*}
    \Lambda^{-1} (\vc{v} + d \vc{v}) = 
    \begin{bmatrix}
         \gamma_{\beta_1} (\beta_1 \beta_x+1) & \gamma_{\beta_1} (\beta_1+\beta_x) & \beta_y & 0 \\
         \gamma_{\beta_1} (\beta_1+\beta_x) & \gamma_{\beta_1} (\beta_1 \beta_x+1) & \frac{\beta_1 \beta_y}{\gamma_{\beta_1}^{-1}+1} & 0 \\
         \beta_y & \frac{\beta_1 \beta_y}{\gamma_{\beta_1}^{-1}+1} & 1 & 0 \\
         0 & 0 & 0 & 1 \\
    \end{bmatrix}
\end{equation*}
Наконец можем посчитать матрицу поворота, которая в первом приближении действительно не содержит $\beta_x$:
\begin{equation*}
    R(\varphi) = \begin{bmatrix}
 1 & 0 & 0 & 0 \\
 0 & 1 & -\frac{\beta_1 \beta_y'}{\sqrt{1-\beta_1^2}+1} & 0 \\
 0 & \frac{\beta_1 \beta_y'}{\sqrt{1-\beta_1^2}+1} & 1 & 0 \\
 0 & 0 & 0 & 1 \\
\end{bmatrix}
\end{equation*}
что дейстительно соответствует повороту в плоскости $\{xy\}$ вокруг оси $z$ с углом $\varphi$ равным
\begin{equation*}
    \varphi = -\frac{\beta_y \beta_1}{\gamma_{\beta_1}^{-2} + \gamma_{\beta_1}^{-1}} = 
    -\frac{\gamma_{\beta_1}^{2}}{\gamma_{\beta_1} + 1} \beta_1 \beta_y,
\end{equation*}
где $\varphi$ малый, в силу малости $\beta_y$. Так вот, в результате поворота координатных осей меняются и любые векторы, неподвижные в неИСО, то есть искомая угловая скорость
\begin{equation*}
    \omega_z = -\frac{\gamma_{\beta_1}^{2}}{\gamma_{\beta_1} + 1} \beta_1 (\beta_y / \Delta t),
    \hspace{5 mm}
    \Leftrightarrow
    \hspace{5 mm}
    \vc{\omega} = -\frac{\gamma_{\beta_1}^{2}}{\gamma_{\beta_1} + 1} \left[
        \vc{\beta} \times  \dot{\vc{\beta}}
    \right] = \frac{\gamma_{\beta_1}^{2}}{\gamma_{\beta_1} + 1} \left[
        \dot{\vc{\beta}} \times  \vc{\beta}
    \right],
\end{equation*}
что и требовалось доказать.


\subsubsection*{№Т3}
Исследуем систему вида
\begin{equation*}
    \left\{\begin{aligned}
        \dot{x} &= y, \\
        \dot{y} &= k \left(
            \frac{b}{a-x} - x
        \right)
    \end{aligned}\right.
\end{equation*}
Рассмотрим положение равновесия $\dot{x} = \dot{y} = 0$, при $x \neq a$
\begin{equation*}
    x^* = \frac{1}{2}\left(
        a \pm \sqrt{a^2 - 4b}
    \right),
\end{equation*}
что приводит нас к следующим случаям. 

Пусть $a^2 = 4b$, тогда $x^* = a/2$, попробуем найти фазовый портрет по линейному приближению
\begin{equation*}
    \det(J - \lambda E) = 
     \lambda^2 - k\left(
        \frac{b}{(a-x^*)^2}-1
     \right) = k \cdot 0 = 0,
     \hspace{0.5cm} \Rightarrow \hspace{0.5cm}
     \lambda = 0,
\end{equation*}
следовательно линейным приближением здесь не воспользоваться.

При $a^2 > 4b$, 
\begin{equation*}
    x^* = \frac{a}{2} \pm \sqrt{
    \left(\frac{a}{2}\right)^2 - b
    } = \frac{a}{2} \pm \delta,
\end{equation*}
тогда
\begin{equation*}
    \lambda^2 = \frac{5}{4} (4b - a^2) \pm a \delta,
\end{equation*}
в случае $+ a\delta$ $\Re \lambda_{1, 2} = 0$, следовательно это \textit{центр}, при $-a \delta$ получается $\lambda^2 = 100 b - 9 a^2 > 64 b > 0$, следовательно это \textit{седло}.

При $a^2 < 4b$ не существует положения равновесия, что приводит нас к фазовым диаграммам аналогичным задаче Т4. 


\subsection{Т4}

Докажем, что функции вида $P(x) e^{-x^2/2}$, где $P(x) \in \mathbb{C}[x]$, при преобразовании Фурье переходти в функцию того же вида, причём степень многочлена не повышается. 

Действительно,
\begin{align*}
    F[f](y) 
    &= 
    \int_{-\infty}^{+\infty} \frac{d t}{\sqrt{2\pi}} P_\alpha (t) e^{-t^2/2} e^{-iyt} 
    = 
    \int_{-\infty}^{+\infty} \frac{d t}{\sqrt{2\pi}} e^{-t^2/2} P_\alpha \left(
        \frac{\partial }{\partial (-iy)}  e^{-yt}
    \right) 
    = \\ &= 
    P_\alpha \left(
        i \frac{\partial }{\partial y} 
    \right) \int_{-\infty}^{+\infty} \frac{d t}{\sqrt{2\pi}}
    e^{-t^2/2} e^{-iyt}
    \overset{17.8(2)}{=} 
    P_\alpha \left(
        - \frac{\partial }{\partial y} 
    \right) e^{-y^2/2}.
\end{align*}
Осталось показать, что степень многочлена не увеличилась, для этого достаточно рассмотреть
\begin{equation*}
    F[f](y) = p_\alpha \cdot \left(
        i \frac{\partial }{\partial y} 
    \right)^n e^{-y^2/2} + P_{\alpha-1} \left(
        i \frac{\partial }{\partial y} 
    \right) e^{-y^2/2}
    = 
    p_\alpha i^{\alpha} (-y)^\alpha e^{-y^2/2} + Q_{\alpha-1} (y) e^{-y^2/2} + P_{\alpha-1} \left(
        i \frac{\partial }{\partial y}
    \right)e^{-y^2/2},
\end{equation*}
поэтому степень не повышается.


\subsection{Т5}

Вычислим интегралы Лапласа с помощью образения преобразования Фурье:
\begin{equation*}
    I(y) = \int_{0}^{+\infty}  \d x \frac{\cos (yx)}{1 + x^2},
    \hspace{5 mm}
    K(y) = \int_{0}^{+\infty} \d x \frac{x \sin (yx)}{1 + x^2}.
\end{equation*}
В частности рассмотрим функцию $f(x) = e^{-\alpha |x|}$, где $\alpha > 0$, тогда
\begin{equation*}
    F[f](y) = \sqrt{\frac{2}{\pi}} \frac{\alpha}{\alpha^2 + y^2},
    \hspace{10 mm}
    F^{-1}[g] (x) = \int_{-\infty}^{+\infty} 
    \frac{\d y}{\sqrt{2\pi}} g(y) e^{ixy}.
\end{equation*}
Теперь воспользуемся формулой образения, и найдём
\begin{align*}
    f(x) 
    =
    F^{-1} \left[
        F[f]
    \right](x) 
    =
    \frac{2}{\pi}
     \int_{0}^{+\infty} 
     \d y \frac{\alpha \cos (2 y)}{\alpha^2 + y^2} = e^{-\alpha |x|}, \hspace{5 mm} \alpha > 0.
\end{align*}
Соответственно, при $\alpha = 1$, найдём
\begin{equation*}
    \int_{0}^{+\infty} \d y \frac{\cos(xy)}{1 + y^2} = \frac{\pi}{2} e^{-|x|}.
\end{equation*}

Аналогично находим $K(\alpha)$, а именно $F[f'](y) = iy F[f](y)$
\begin{equation*}
    F^{-1} [F[f']](x) = f'(x) = F^{-1}\left[
        iy F[f]
    \right] (x) = 2 \int_0^{+\infty} \frac{\d y}{\sqrt{2\pi}} i \sin (xy) \sqrt{\frac{2}{\pi}} \frac{\alpha i y}{\alpha^2 + y^2} = - \frac{2}{\pi} \int_0^\infty \d y \frac{\alpha y \sin (xy)}{\alpha^2 + y^2} = - \alpha \sign x e^{-\alpha |x|},
\end{equation*}
что при $\alpha = 1$ перейдёт в интеграл вида
\begin{equation*}
    \int_0^{+\infty} \frac{y \sin (xy)}{1 + y^2} \d y = \frac{\pi}{2} \sign (x) e^{-|x|}.
\end{equation*}
\subsubsection*{Т6}

Рассмотрим систему вида
\begin{align*}
    \dot{x} &= - y + \mu x - x y^2,\\
    \dot{y} &= \mu y + x - y^3.
\end{align*}



Аналогично Т5 перейдём к полярным координатам, и выразим $\dot{\varphi}$ и $\dot{r}$, так вышло, что и здесь всё хорошо, и
\begin{equation*}
    \left\{\begin{aligned}
        r \dot{\varphi} &= r \\
        \dot{r} &= r \mu - r^3 \sin^2 (\varphi)
    \end{aligned}\right.
    \hspace{0.5cm} \Rightarrow \hspace{0.5cm}
    \frac{\dot{r}}{\dot{\varphi}} = \frac{d r}{d \varphi} = 
    r (\mu - r^2 \sin^2 \varphi).
\end{equation*}
Найдём значения $r = r_*$, где $\dot{r}$ меняет знак
\begin{equation*}
    r_*^2 = \mu \sin^{-2} \varphi,
\end{equation*}
что возможно только при $\mu > 0$. Аналогично предыдущей задаче рассмотрим $\sign \dot{r}$, и получим
\begin{equation*}
    \sign \dot{r} = 
    \left\{\begin{aligned}
        1 & \ \  r < r_*   \\
        -1 & \ \ r > r_*
    \end{aligned}\right.
\end{equation*}

Подробнее рассмотрим положение равновесия $x=y=0$, которое в силу постоянства $\dot{\varphi}$ единственное. В линейном приближение, 
\begin{equation*}
    J = \begin{pmatrix}
        \dot{x}'_x & \dot{x}'_y \\
        \dot{y}'_x & \dot{y}'_y \\
    \end{pmatrix}
    = 
    \begin{pmatrix}
        \mu-y^2 & -1-2xy \\
        1 & \mu - 3 y^2 \\
    \end{pmatrix} = \begin{pmatrix}
        \mu & -1 \\
        1 & \mu \\
    \end{pmatrix}.
\end{equation*}
Тогда 
\begin{equation*}
    \det(J - \lambda E) = (\mu - \lambda)^2 + 1 = 0,
    \hspace{0.5cm} \Rightarrow \hspace{0.5cm}
    \lambda_{1, 2} = \mu \pm 1.
\end{equation*}
Тогда при $\mu < 0$, по теореме Ляпунова об устойчивости в линейном приближение, $x=y=0$ -- устойчивый фокус, при $\mu = 0$ верно, что $\Re (\lambda) = 0$, следовательно это центр, а при $\mu > 0$ фокус становится неустойчивым. Это позволяет прийти к фазовы портретам при различным значениям $\mu$, изображенным на рисунке \ref{T6}.

\begin{figure}[ht]
    \centering
    \includegraphics{D:\\Kami\\WM12_Workspace\\AnMec\\tmp0.pdf}
    \hspace{0.2cm}
    \includegraphics{D:\\Kami\\WM12_Workspace\\AnMec\\tmp1.pdf}
    \hspace{0.2cm}
    \includegraphics{D:\\Kami\\WM12_Workspace\\AnMec\\tmp2.pdf}
    % \hspace{0.2cm}
    % \includegraphics{D:\\Kami\\WM12_Workspace\\AnMec\\tmp.pdf}
    \caption{Бифуркация Пуанкаре-Андронова-Хопфа}
    \label{T6}
\end{figure}

