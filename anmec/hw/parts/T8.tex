
\subsubsection*{№Т8}
Сразу подставим $\lambda_2 = 2 \lambda_1$ и приведем к нормальной форме Коши уравнения вида
\begin{align*}
    \dot{x}_1 &= \lambda_1 x_1 + c_{20} x_2^2 + c_{11} x_1 x_2 + c_{1, 2} x_2^2 \\
    \dot{x}_2 &= 2 \lambda_1 x_2 + d_{20} x_1^2 + d_{11} x_1 x_2 +  d_{02} x_2^2,
\end{align*}
диагональный вид уже получен, остается подобрать многочлен $\vc{p}$ такой, что
\begin{equation*}
    \vc{x} = \vc{y} + \vc{p}(\vc{y}),
    \hspace{1 cm}
    \dot{\vc{y}} = \Lambda \vc{y} + \Lambda \vc{p} + \vc{g}^2 (\vc{y}) -
    \frac{\partial \vc{p}}{\partial \vc{y}\T}  \Lambda \vc{y} + O(y^3)
\end{equation*}
что возможно, при
\begin{equation*}
     p_i = \frac{g_i^2}{k_1 \lambda_1 + \ldots + k_n \lambda_n - \lambda_i}.
\end{equation*}
Из этого находим
\begin{align*}
    &p_{20}^1 =\frac{c_{20}}{\lambda_1},
    &p_{11}^1 = \frac{c_{11}}{\lambda_2},
    &&p_{02}^1 = \frac{c_{02}}{3 \lambda_1}, \\
    &p_{20}^2 = \frac{\neq 0}{0}, 
    &p_{11}^2 = \frac{d_{11}}{\lambda_1}, 
    &&p_{02}^2 = \frac{d_{02}}{2\lambda_1}, \\
\end{align*}    
и записываем нормальную форму
\begin{equation*}
    \dot{\vc{y}} = \begin{pmatrix}
        \lambda_1 & 0 \\
        0 & 2 \lambda_1 \\
    \end{pmatrix} \vc{y} + 
    \begin{pmatrix}
        0 \\ d_{20} y_1^2
    \end{pmatrix} + O(|\vc{y}|^3).
\end{equation*}