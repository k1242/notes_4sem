\subsubsection*{№18.37}

Момент инерции стержня $J = \frac{1}{3} m l^2$, тогда, считая отклонения малыми, кинетическую и потенциальную энергию системы можем записать, как
\begin{equation*}
    T = \frac{1}{2} J \left(\dot{\varphi}^2 + \dot{\psi}^2\right),
    \hspace{1 cm}
    \Pi = \frac{1}{2}c (\varphi a - \psi a)^2 + \left(
        1 - \frac{\varphi^2}{2} + 1 - \frac{\psi^2}{2}
    \right) mg \frac{l}{2}.
\end{equation*}
Переходя в СО движущейся платформы, к системе добавляется инерциальная сила
\begin{equation*}
    M = \frac{mA}{2} \sin(\omega t) \omega^2 l,
\end{equation*}
действующая на центры масс стержней.

С помощью уравнений Лагранжа второго рода переходим к уравнениям вида
\begin{equation*}
    A \ddot{\vc{q}} + C \vc{q} = M,
    \hspace{1 cm}
    A = J \begin{pmatrix}
        1 & 0 \\
        0 & 1 \\
    \end{pmatrix},
    \hspace{1 cm}
    C = \frac{1}{2}\begin{pmatrix}
        2a^2c + mgl & -2ca^2 \\
        -2ca^2 & 2a^2 c + mgl \\
    \end{pmatrix}
\end{equation*}
Из векового уравнения теперь можем найти собственные частоты системы, для получения однородного решения
\begin{equation*}
    \det(C - \lambda A) = 0,
    \hspace{0.25cm} \Rightarrow \hspace{0.25cm}
    \left(
        mg \frac{l}{2} - J \lambda
    \right) \left(
        a^2 c + mg \frac{l}{2} - J \lambda
    \right) = 0,
\end{equation*}
откуда легко находим $\lambda$
\begin{equation*}
    \lambda_1 = \frac{3}{2}\frac{g}{l}, \hspace{1 cm},
    \hspace{0.5 cm} \vc{u}_1 = \begin{pmatrix}
        1 \\ 1
    \end{pmatrix},
    \hspace{1 cm}
    \lambda_2 = \frac{3}{2} \frac{g}{l} + \frac{6ca^2}{ml^2},
    \hspace{0.5 cm} 
    \begin{pmatrix}
        1 \\ -1
    \end{pmatrix}.
\end{equation*}
из которых уже можем составить ФСР.

Теперь перейдём к поиску частного решения\footnote{
    Так как по условию $\varphi$ и $\psi$ малые, то про резонанс говорить не приходится.
} :
\begin{equation*}
    \varphi = \alpha \sin (\omega t), 
    \psi = \beta \sin (\omega t),
    \hspace{0.5cm} \Rightarrow \hspace{0.5cm}
    -A \omega^2 \begin{pmatrix}
        \alpha \\ \beta
    \end{pmatrix} + C \begin{pmatrix}
        \alpha \\ \beta
    \end{pmatrix} = 
    \frac{m A \omega^2 l}{2} \begin{pmatrix}
        1   \\ 1
    \end{pmatrix}.,
\end{equation*}
вводя матрицу
\begin{equation*}
    \Lambda = C - A \omega^2,
    \hspace{0.5cm} \Rightarrow \hspace{0.5cm}
    \Lambda \begin{pmatrix}
        \alpha \\ \beta
    \end{pmatrix} = 
    \frac{m A \omega^2 l}{2} \begin{pmatrix}
        1 \\ 1
    \end{pmatrix},
    \hspace{0.5cm} \Leftrightarrow \hspace{0.5cm}   
    \begin{pmatrix}
        \alpha \\ \beta
    \end{pmatrix} 
    =
     \Lambda^{-1} \,
    \frac{m A \omega^2 l}{2} \begin{pmatrix}
        1 \\ 1
    \end{pmatrix}.
\end{equation*}
Считая $\Lambda^{-1}$, находим частное решение и получаем ответ
\begin{equation*}
    \begin{pmatrix}
        \varphi \\ \psi
    \end{pmatrix} = 
    \frac{3 A \omega^2}{3 g - 2 l \omega^2} \begin{pmatrix}
        1 \\ 1
    \end{pmatrix} \sin (\omega t) + 
    C_1 \begin{pmatrix}
        1 \\ 1
    \end{pmatrix}
    \sin\left(
        \sqrt{\frac{3}{2}\frac{g}{l}} \, t + \alpha_1
    \right) + 
    C_2 \begin{pmatrix}
        1 \\ -1
    \end{pmatrix} 
    \sin\left(
        \sqrt{
        \frac{3}{2} \frac{g}{l} + \frac{6 c a^2}{ml^2}
        } \, t + \alpha_2
    \right).
\end{equation*}
