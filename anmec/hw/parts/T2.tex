\subsubsection*{№Т2}


Рассмотрим уравнение вида
\begin{equation*}
    \dot{x} = (x-a)(x^2-a),
\end{equation*}
найдём положения равновесия
\begin{equation*}
    \dot{x}=0,
    \hspace{0.5cm} \Rightarrow \hspace{0.5cm}
    \left[\begin{aligned}
        x &= a             & \\
        x &= \pm \sqrt{a},  &a > 0
    \end{aligned}\right.
\end{equation*}
\begin{figure}[ht]
    \centering
    \includegraphics{D:\\Kami\\WM12_Workspace\\AnMec\\T2_0.pdf}
    \caption{Зависимость положения равновесия $x^*$ от параметра $a$ к  №Т2}
\end{figure}
Соответственно, при неположительных $a$ существует единственное положение равновесия $x^*=a$, при положительных $a \neq 1$ существует три положения равновесия $x^* \in \{a, +\sqrt{a}, -\sqrt{a}\}$, и при $a=1$ существует два положения равновесия $x^* \in \{+1, -1\}$. Соответствующие зависимости $\dot{x}(x, a)$ приведены на рисунке \ref{IT2}.
\begin{figure}[ht]
    \centering
    \hspace{0.3cm}
    \includegraphics{D:\\Kami\\WM12_Workspace\\AnMec\\T2_1.pdf}
    \hspace{0.3cm}
    \includegraphics{D:\\Kami\\WM12_Workspace\\AnMec\\T2_2.pdf}
    \hspace{0.3cm}
    \includegraphics{D:\\Kami\\WM12_Workspace\\AnMec\\T2_3.pdf}
    \caption{Зависимость $\dot{x}(x)$ при различных $a$ к  №Т2}
    \label{IT2}
\end{figure}