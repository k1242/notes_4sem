Исследуем параметрический резонанс в уравнении Матьё
\begin{equation*}
    \ddot{x} + (a + \varepsilon \cos t) x = 0,
\end{equation*}
где $0 < \varepsilon \ll 1$. Будем считать $a = 1 + c \varepsilon^2$, и рассматривать задачу относительно медленного времени $\tau = t$ и быстрого времени $T = \varepsilon^2 t$. Пусть также $a = 1 + c \varepsilon^2$, где $c = )(1)$. Величинами порядка $o(\varepsilon^2)$ пренебрежем.

Для начала перепишем дифференцирование по времени в терминах $T, \tau$:
\begin{align*}
    d_t &= \partial_\tau + \varepsilon^2 \partial_T, \\
    d^2_{t, t} &= 
    \partial^2_{\tau, \tau} + 2 \varepsilon^2 \partial_\tau \partial_T,
\end{align*}
где слагаемым $\varepsilon^4 \partial^2_{T, T}$ пренебрегли. 

Для поиска решения воспользуемся естественным анзацем, вида
\begin{equation*}
    x = x_0(\tau, T) + \varepsilon x_1 (\tau, T) + \varepsilon^2 x_2 (\tau, T),
\end{equation*}
тогда, после подстановки в уравнение Матье и группировки по степеням\footnote{
    Коэффициент при каждой степени должен быть нулевым.
}  $\varepsilon$, получим набор условий. При $\varepsilon^0$
\begin{equation*}
    \varepsilon^0 \colon \ \ \ddot{x}_0(\tau, T) +  x_0(\tau, T) = 0,
\end{equation*}
следовательно
\begin{equation*}
    x_0 = A(T) e^{i\tau} + B(T) e^{-i\tau}.
\end{equation*}
При $\varepsilon^1$, подставляя значение $x_0$ находим, что
\begin{equation*}
    \varepsilon^1 \colon \ \ \ddot{x}_1(\tau, T) +  x_1(\tau, T) 
    = -\frac{1}{2}\left(
        A+B + A e^{2 i \tau} + B e^{- 2 i \tau}
    \right)
    ,
\end{equation*}
решая это дифференциальное уравнение относительно $x_1(\tau, T)$ находим, что
\begin{equation*}
    x_1(\tau, T) = 
\end{equation*}