% 16.11
% 16.33
% 16.47
% 16.64
% 16.107



\subsubsection*{16.11}

Введём ось $OX$ координат вдоль туннеля, выбрав в качестве $x=0$ положение равновесия. Тогда кинетическая энергия
\begin{equation*}
    T = \frac{1}{2} m \dot{x}^2.
\end{equation*}
Интегрируя силу, действующую на тело, находим потенциальную энергию
\begin{equation*}
    F_x = -\frac{G M(x) m}{r^2(x)} \cdot \frac{x}{r} = - G \kappa x,
    \hspace{1 cm}
    \frac{G \kappa R^3}{R^2} = g,
    \hspace{0.5cm} \Rightarrow \hspace{0.5cm}
    \Pi = \int F \d x = \frac{1}{2}\frac{g}{R} x.
\end{equation*}
Так удачно вышло, что $T$ и $\Pi$ -- квадратичные формы. Запишем вековое уравнение:
\begin{equation*}
     \frac{\partial^2 \Pi}{\partial q^2} - \lambda \frac{\partial^2 T}{\partial \dot{q}^2} = 0,
     \hspace{0.5cm} \Rightarrow \hspace{0.5cm}
     \lambda = \frac{g}{R},
     \hspace{0.5cm} \Rightarrow \hspace{0.5cm}
     T = 2 \pi \sqrt{\frac{R}{g}}.
\end{equation*}


\subsubsection*{16.33}

Выбрав оси, как показано на рисунке, получим систему с 2 степенями свободы.  Кинетическая энергия системы
\begin{equation*}
    T = \frac{m}{2} \left(
        \dot{x}_1^2 + \dot{x}_2^2
    \right).
\end{equation*}
Потенциальная энергия для трёх пружинок (сдвинутая так, чтобы положение равновесия был $0$)
\begin{equation*}
    \Pi = \frac{c}{2} (x_2)^2 + \frac{c}{2} (x_1)^2 + \frac{2c}{2}
    \left(
       x_2 - x_1
    \right)^2.
\end{equation*}
И снова так вышло, что $T$ и $\Pi$ -- квадратичные формы, так что 
\begin{equation*}
     \det \left(
     \frac{\partial^2 \Pi}{\partial q^i \partial q^j} - \lambda \frac{\partial^2 T}{\partial \dot{q}^i \partial \dot{q}^j}\right) = 0,
     \hspace{0.5cm} \Rightarrow \hspace{0.5cm}
     \det\left[
        c\begin{pmatrix}
            3 & 2 \\
            2 & 3 \\
        \end{pmatrix} - 
        \lambda m\begin{pmatrix}
            1 & 0 \\
            0 & 1 \\
        \end{pmatrix}
     \right] = 0,
     \hspace{0.5cm} \Rightarrow \hspace{0.5cm}
     (\lambda m)^2 + 9 c^2 - 6 \lambda m c - 4 c^2 = 0.
\end{equation*}
Соответственно находим квадраты частот
\begin{equation*}
    \lambda^2 - 6 \lambda \frac{c}{m} + 5 \frac{c^2}{m^2} = 
    \left(\lambda_1 -  \frac{c}{m}\right)
    \left(\lambda_2 - 5 \frac{c}{m}\right)
    = 0,
    \hspace{0.5cm} \Rightarrow \hspace{0.5cm}
    \left\{\begin{aligned}
        \lambda_1: & \begin{pmatrix}
            -2c & 2c
        \end{pmatrix}
        \begin{pmatrix}
            x_1 \\ x_2
        \end{pmatrix} = 0
        &\Rightarrow
        \ \ \vc{u}_1 = \begin{pmatrix}
            1 \\ 1
        \end{pmatrix}; \\
        \lambda_1: & \begin{pmatrix}
            2c & 2c
        \end{pmatrix}
        \begin{pmatrix}
            x_1 \\ x_2
        \end{pmatrix} = 0
        &\Rightarrow
        \ \ \vc{u}_2 = \begin{pmatrix}
            1 \\ -1
        \end{pmatrix}.
    \end{aligned}\right.
\end{equation*}
Соответственно, уравнение движения будет иметь вид
\begin{equation*}
    \begin{pmatrix}
        x_1 \\ x_2
    \end{pmatrix} = 
    C_1 
    \begin{pmatrix}
        1 \\ 1
    \end{pmatrix} 
        \sin\left(
            \sqrt{\frac{c}{m}}\,  t + \alpha_1
        \right)
    +
    C_2
    \begin{pmatrix}
        1 \\ -1
    \end{pmatrix}
    \sin\left(
        \sqrt{\frac{5c}{m}} \, t + \alpha_2
    \right).
\end{equation*}



\subsubsection*{16.47}

Запишем с учётом малости колебаний кинетическую энергию системы
\begin{equation*}
    T = \frac{m}{2} l^2 \dot{\varphi}^2 + \frac{m}{2} \left(
        l \dot{\varphi}_2 + l \dot{\varphi}_1
    \right)^2.
\end{equation*}
И, опять же, с учетом малости, потенциальную
\begin{align*}
    \Pi &= \frac{c}{2} \left(
        (l \varphi_1)^2 + (l \varphi_1 + l \varphi_2)^2
    \right) + 
    m g l \cos \varphi_1 + m  g l (\cos \varphi_1 + \cos \varphi_2) = \\
    &= 
    \frac{c}{2} \left(
        (l \varphi_1)^2 + (l \varphi_1 + l \varphi_2)^2
    \right) + 2 m g l \left(1 - \frac{\varphi_1^2}{2}\right) + mgl \left(1 - \frac{\varphi_2^2}{2}\right).
\end{align*}
Как обычно, получив квадратичные формы (хотя бы в малом приближение) радуемся и переходим к поиску частот собственных колебаний
\begin{equation*}
     \det \left(
     \frac{\partial^2 \Pi}{\partial q^i \partial q^j} - \lambda \frac{\partial^2 T}{\partial \dot{q}^i \partial \dot{q}^j}\right) = 0,
     \hspace{0.5cm} \Rightarrow \hspace{0.5cm}
     \det\left[
        \begin{pmatrix}
            2cl^2 - 2 mgl & c l^2 \\
            c l^2 & cl^2  - mgl \\
        \end{pmatrix} - 
        \lambda 
        ml^2\begin{pmatrix}
            2 & 1 \\
            1 & 1 \\
        \end{pmatrix}
     \right] = 0.
\end{equation*}
Раскрыв, получаем уравнение вида
\begin{equation*}
    2 ([cl^2-ml^2 \lambda] - mgl)^2 - [cl^2-ml^2 \lambda]^2 = 0,
    \hspace{0.5cm} \Rightarrow \hspace{0.5cm}
    x = \frac{\sqrt{2} m g l}{\sqrt{2} \pm 1} = [cl^2-ml^2 \lambda],
    \hspace{0.5cm} \Rightarrow \hspace{0.5cm}
    \lambda_{1, 2} = \frac{c}{m} -  2\frac{g}{l} \mp \sqrt{2} \frac{g}{l}.
\end{equation*}
Теперь подставляем известные $\lambda$, и находим амплитудные векторы
\begin{align*}
    &\lambda_1 \, : \ \
    \begin{pmatrix}
        2  +2 \sqrt{2} & 2 + \sqrt{2}
    \end{pmatrix}
    \begin{pmatrix}
        x_1 \\ x_2
    \end{pmatrix} = 0
    \hspace{0.5cm} \Rightarrow \hspace{0.5cm}
    \vc{u}_1 = \begin{pmatrix}
        1 \\ - \sqrt{2}
    \end{pmatrix}; \\
    &\lambda_2 \, : \ \
    \begin{pmatrix}
        2  -2 \sqrt{2} & 2 - \sqrt{2}
    \end{pmatrix}
    \begin{pmatrix}
        x_1 \\ x_2
    \end{pmatrix} = 0
    \hspace{0.5cm} \Rightarrow \hspace{0.5cm}
    \vc{u}_2 = \begin{pmatrix}
        1 \\ \sqrt{2}
    \end{pmatrix}.
\end{align*}
Это позволяет нам записать уравнение движения малых колебаний (при $c/m > (2+\sqrt{2}) g / l$)
\begin{equation*}
    \begin{pmatrix}
        \varphi_1 \\
        \varphi_2
    \end{pmatrix} = 
    C_1 \begin{pmatrix}
        1 \\ - \sqrt{2}
    \end{pmatrix}
    \sin \left(
        \sqrt{\frac{c}{m} - \left(2 + \sqrt{2}\right) \frac{g}{l}} \, t + \alpha_1
    \right) + 
    C_2 \begin{pmatrix}
        1 \\ \sqrt{2}
    \end{pmatrix}
    \sin \left(
        \sqrt{\frac{c}{m} - \left(2-\sqrt{2}\right) \frac{g}{l}} \, t + \alpha_2
    \right).
\end{equation*}


\subsubsection*{16.64}
Запишем кинетическую энергию системы
\begin{equation*}
    T = \frac{m}{2} \left(
        \dot{x}_1^2 + \dot{x}_3^2
    \right) + \frac{nm}{2} \dot{x}_2^2.
\end{equation*}
И, считая $0$ в положении равновесия, потенциальную энергию системы, запасенную в сжатых пружинах
\begin{equation*}
    \Pi = \frac{c}{2} (x_2 - x_1)^2 + \frac{c}{2} (x_3 - x_2)^2.
\end{equation*}
В таком случае
\begin{equation*}
     \det \left(
     \frac{\partial^2 \Pi}{\partial q^i \partial q^j} - \lambda \frac{\partial^2 T}{\partial \dot{q}^i \partial \dot{q}^j}\right) = 0,
     \hspace{0.5cm} \Rightarrow \hspace{0.5cm}
     \det\left[
        c \begin{pmatrix}
            1 & -1 & 0 \\
            -1 & 2 & -1 \\
            0 & -1 & 1 \\
        \end{pmatrix} - \lambda 
        m \begin{pmatrix}
            1 & 0 & 0 \\
            0 & n & 0 \\
            0 & 0 & 1 \\
        \end{pmatrix}
     \right] = 0.
\end{equation*}
Раскрывая, приходим у уравнению на $\lambda$ вида
\begin{equation*}
    \lambda_1 \left(
        \lambda_2 - \frac{c}{m}
    \right)\left(
        \lambda_3 - \frac{(2+n)c}{nm}
    \right) = 0.
\end{equation*}
Соответственно, амплитудные векторы находим, как
\begin{align*}
    &\lambda_1 \, : \ \
    \begin{pmatrix}
        -c & 2c & -c \\
        c & -c & 0 \\
    \end{pmatrix}
    \begin{pmatrix}
        x_1 \\ x_2 \\ x_3
    \end{pmatrix} = 0
    &\Rightarrow \hspace{2cm}
    &\vc{u}_1 = \begin{pmatrix}
        1 \\ 1 \\ 1
    \end{pmatrix}; \\
    % 
    &\lambda_2 \, : \ \
    \begin{pmatrix}
        c & 2c-nc & c \\
        0 & c & 0 \\
    \end{pmatrix}
    \begin{pmatrix}
        x_1 \\ x_2 \\ x_3
    \end{pmatrix} = 0
    &\Rightarrow \hspace{2cm}
    &\vc{u}_2 = \begin{pmatrix}
        -1 \\ 0 \\ 1
    \end{pmatrix}; \\
    % 
    &\lambda_3 \, : \ \
    \begin{pmatrix}
        c & nc & c \\
        0 & c & 2c/n \\
    \end{pmatrix}
    \begin{pmatrix}
        x_1 \\ x_2 \\ x_3
    \end{pmatrix} = 0
    &\Rightarrow \hspace{2cm}
    &\vc{u}_3 = \begin{pmatrix}
        n \\ -2 \\ n
    \end{pmatrix}.
\end{align*}
Что ж, уравнение движения малых колебаний запишется в виде
\begin{equation*}
    \begin{pmatrix}
        x_1 \\ x_2 \\ x_3
    \end{pmatrix} = 
    (C_1 t + \alpha_1) \begin{pmatrix}
        1 \\ 1 \\ 1
    \end{pmatrix} + 
    C_2 \begin{pmatrix}
        1 \\ 0 \\ -1
    \end{pmatrix} 
    \sin \left(
        \sqrt{\frac{c}{m}} \, t + \alpha_2
    \right) + 
    C_3 \begin{pmatrix}
        n \\ -2 \\ n
    \end{pmatrix}
    \sin \left(
        \sqrt{\frac{(n+2)c}{nm}} \, t + \alpha_3
    \right).
\end{equation*}



\subsubsection*{16.107}


Знаем, что кинетическая энергия и обобщенные силы для системы могут быть записаны в виде\footnote{
    С глубоким сожалением вынуждены оставить баланс индексов в рамках этой задачи. Немое суммирование подразумевается, при повторение индексов.
} 
\begin{equation*}
    T = \frac{1}{2} a_{ik} \dot{q}_i \dot{q}_k,
    \hspace{1 cm}
    Q_i = b_{ik} \dot{q}_k,
\end{equation*}
где $a_{ik}$ -- положительно определенная квадратичная форма, а $b_{ik} = - b_{ki}$ -- кососимметричная квадратичная форма. 

Запишем уравнения Лагранжа второго рода
\begin{equation*}
    \frac{d }{d t} \frac{\partial T}{\partial \dot{q}_i} - \frac{\partial T}{\partial q_i} = Q_i,
    \hspace{0.5cm} \Rightarrow \hspace{0.5cm}
    a_{ik} \ddot{q}_k = b_{i\alpha} \dot{q}_\alpha.
\end{equation*}
Осталось этот набор уравнений решить.


Воспользуемся алгоритмом приведения двух квадратичных форм к каноническому виду. Выберем в качестве скалярного произведения $a_{ik}$, в терминах $a_{ik}$ выберем ортогональный базис так, чтобы $a_{ik}$ было равно $\delta_{ik}.$ Повернём через $u_{ik}$ базис, приведя $b_{ik}$ к каноническому виду $b^*_{jl}$, указанному в условии с $m$ блоков $2 \times 2$. 
\begin{equation*}
    \left\{\begin{aligned}
        \delta_{ik} \ddot{q}_k &= b_{i\alpha} \dot{q}_\alpha, \\    
        u_{kj} q^*_j &= q_k
    \end{aligned}\right.
    \hspace{0.5cm} \Rightarrow  \hspace{0.5cm}
    u^{-1}_{li} \cdot \left(\vphantom{\frac{1}{2}}
        \delta_{ik} u_{kj} q^*_j= b_{i\alpha} u_{\alpha \beta} q^*_\beta
    \right)
    \hspace{0.5cm} \overset{\sqsupset\, i = 1}{\Rightarrow}  \hspace{0.5cm}
    \begin{pmatrix}
        1 & 0 \\
        0 & 1 \\
    \end{pmatrix} 
    \begin{pmatrix}
        \ddot{q}_1^* \\ \ddot{q}_2^*
    \end{pmatrix} = 
    \begin{pmatrix}
        0 & -\nu \\
        \nu & 0 \\
    \end{pmatrix}
    \begin{pmatrix}
        \dot{q}_1^* \\ \dot{q}_2^*
    \end{pmatrix}.
\end{equation*}
И таких систем с колебаниями у нас будет $m$ штук
\begin{equation*}
    \left\{\begin{aligned}
        \ddot{q}_1^* &= - \nu \dot{q}_2^* \\
        \ddot{q}_2^* &= - \nu \dot{q}_1^* \\
    \end{aligned}\right.
    \hspace{0.5cm} \Rightarrow \hspace{0.5cm}
    \left\{\begin{aligned}
        \dddot{q}_1^* &= - \nu \ddot{q}_2^* \\
        \dddot{q}_2^* &= - \nu \ddot{q}_1^* \\
    \end{aligned}\right.
    \hspace{0.5cm} \Rightarrow \hspace{0.5cm}
    \left\{\begin{aligned}
        q_1^* &= \frac{A}{\nu} \cos (\nu t + \alpha) + C_1 \\
        q_2^* &= \frac{A}{\nu} \sin (\nu t + \alpha) + C_2. \\
    \end{aligned}\right.
\end{equation*}
Нули же в каноническом виде $b_{ij}$ будут соответствовать трансляциям
\begin{equation*}
    q^* = A t + B.
\end{equation*}
Собирая всё вместе, находим, что 
\begin{equation*}
    q_\alpha = u_{\alpha i} q^*_i,
    \hspace{1 cm}
    q^*_i = 
    \left\{\begin{aligned}
        &({A_j} / {\nu_j}) \cdot \cos (\nu_j t + \alpha_j) + B_{2j - 1}
        & \text{ при } &i = 2j - 1 \leq 2 m; \\
        &({A_j} / {\nu_j}) \cdot \sin (\nu_j t + \alpha_j) + B_{2j}
        & \text{ при } &i = 2j  \leq 2 m; \\
        & (A_j) \cdot t + B_j
        & \text{ при } &i = j  > 2 m. \\
    \end{aligned}\right.
\end{equation*}