\subsubsection*{Т21}

Для отображения Мандельброта 
$
    z_{n+1} = z_n^2 + c, \hspace{5 mm} z, c \in \mathbb{C}
$
найдём стационарные точки и исследуем их на устойчивость.
Будем считать точку стационарной, если $z_{n+1} = z_n$, тогда
\begin{equation*}
    z^2 - z + c = 0,
    \hspace{0.5cm} \Rightarrow \hspace{0.5cm}
    z = \frac{1 \pm \sqrt{1-4 c}}{2} = \left\{\begin{aligned}
        &(1 \pm 1)/2, &c=0; \\
        &1/2, & c={1}/{4}.
    \end{aligned}\right.
\end{equation*}
Осталось проверить их на устойчивость. 

\textbf{Первый случай}. При $c = 0$ удобно представить 
\begin{equation*}
    z_1 = |z_1| e^{i \varphi}, 
    \hspace{0.5cm} \Rightarrow \hspace{0.5cm}
    |z_n| = |z_1|^{2^{n-1}} e^{i 2^{n-1} \varphi} < |z_1|^{2^{n-1}}
\end{equation*}
то есть для стационарной точки $z_1 = 0$ $\forall \varepsilon > 0 \ \exists N \colon  \forall n > N \ z_n < \varepsilon$, то есть вернется к нулю сколь угодно близко (асимптотически устойчиво). Можно также сказать, что точка $z_1 = 0$ устойчива по Ляпунову: $\forall \varepsilon >  0 \ \exists \delta > 0$ такая, что 
\begin{equation*}
    \forall |z_1| < \delta\colon |z_n| < \varepsilon, \ \forall n \geq 1,
\end{equation*}
при $\delta = \varepsilon$.

Стационарная точка $z_1 = 1$ очевидно не обладает этими свойствами, так как при $|z_1|>1$ точка неограниченно растёт по модулю .



\textbf{Второй случай}. При $c = 1$ проще всего показать, что $1/2$ -- неустойчивая точка, рассмотрев $z_1 = 1/2 + \varepsilon$:
\begin{equation*}
    z_n = z_1 + n \varepsilon + o(\varepsilon),
\end{equation*}
таким образом при $\varepsilon > 0$ точка будет неограниченно расти $n$.
