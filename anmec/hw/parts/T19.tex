\subsubsection*{Т19}



Для гамильтоновой системы $H_0 = 7 I_1 - I_2$ найдём уравнения движения $x(t) = \sqrt{2 I_1} \cos \varphi_1$, $y = \sqrt{2 I_1} \sin \varphi_1$, при наложении возмущения $\varepsilon H_1$, с 
\begin{align*}
    &1) \ \ H_1 = \sqrt{2 I_1} \cos \varphi_1 \cos 6 \varphi_2 = \frac{\sqrt{2 I_1}}{2} \left[
        \cos(\varphi_1 + 6 \varphi_2) + \cos(\varphi_1-6\varphi_2)
    \right] \\
    &2) \ \ H_1 = \sqrt{2 I_1} \cos \varphi_1 \cos 7 \varphi_2 = \frac{\sqrt{2 I_1}}{2} \left[
        \cos(\varphi_1 + 7 \varphi_2) + \cos(\varphi_1-7\varphi_2)
    \right].
\end{align*}

Система является вырожденной в терминах
\begin{equation*}
        \bigg|
            \frac{\partial^2 H_0}{\partial I_i \partial I_j} 
        \bigg| = 0,
\end{equation*}
т.к. $\omega_1 = \partial_{I_1} H_0 = 7 = \const$, $\omega_2 = \partial_{I_2} H_0 = -1 = \const$.
По аналогичным причинам
\begin{equation*}
            \det \begin{bmatrix}
        \vphantom{\dfrac{1}{2}}
            \dfrac{\partial^2 H_0}{\partial \vc{I}\T \partial \vc{I}} & \vc{\omega} \\
        \vphantom{\dfrac{1}{2}}
            \vc{\omega}\T & 0
        \end{bmatrix} = 0,
\end{equation*}
то есть $\omega_2/\omega_1$ не зависит $I_1, \ I_2$.

\textbf{Первый случай}. Так как возмущение явно периодично, то можем попробовать найти новые хорошие переменные через производящую функцию $S$, вида
\begin{equation*}
    S = \vc{I} \cdot \vc{\varphi} + \varepsilon S_1,
    \hspace{0.5cm} \Rightarrow \hspace{0.5cm}
    \left\{\begin{aligned}
        \psi &= \varphi + \varepsilon\, \partial_I S \\
        J &= I + \varepsilon\, \partial_\varphi S
    \end{aligned}\right.
\end{equation*}
Саму функцию ищем в виде
\begin{equation*}
    S = \sum_i S_i \exp(i \vc{k}_i \cdot \vc{\varphi}) = S_1 \sin(\varphi_1 + 6 \varphi_2) + S_2 \sin (\varphi - 6 \varphi_2).
\end{equation*}
Вспоминая, что $\partial_I H_0 \cdot \partial_\varphi S = -\left[H_1(I, \varphi)\right]_{\text{период}}$, находим\footnote{
    Заодно переобозначая $\varepsilon \to - \varepsilon$, так как в этом месте потерялся знак.
} 
\begin{equation*}
    S_1 = \frac{\sqrt{2 I_1}}{2 (7-6)}, \hspace{5 mm} 
    S_2 = \frac{\sqrt{2 I_1}}{2 (7+6)}.
\end{equation*}
Теперь можем найти новые переменные
\begin{align*}
    \psi_1 &= \varphi_1 + \frac{\varepsilon}{2 \sqrt{2 I_1}} \sin(\varphi_1 + 6 \varphi_2) + \varepsilon \frac{\sqrt{2 I_1}}{26} \sin(\varphi_1 - 6 \varphi_2),\\
    \psi_2 &= \varphi_2, \\
    J_1 &= I_1 + \frac{\varepsilon \sqrt{2 I_1}}{2} \cos(\varphi_1 + 6 \varphi_2) + \frac{\varepsilon \sqrt{2 I_1}}{26} \cos(\varphi_1 - 6 \varphi_2), 
\end{align*}
с новым гамильтонианом $\hat{H} = 7 J_1 - J_2 + o(\varepsilon)$, $J_1, J_2 = \const$, и $\psi_1 = 7 t + C_1$, $\psi_2 = - t + C_2$. Выражение для второго интеграла системы содержит $I_2$, поэтому нам неинтересно.

Осталось выразить $I_1$ и $\varphi_1$ в терминах $x, y$:
\begin{equation*}
    \varphi_1 = \arctg \frac{y}{x}, \hspace{5 mm} x^2 + y^2 = 2 I_1,
\end{equation*}
а также (считая константы нулевыми), подставляя $x,\, y$ в уравнения
\begin{equation*}
    \psi_1 = \varphi_1 + \frac{\varepsilon}{4 I_1}\left(
        \frac{14}{13} y \cos 6 t - \frac{12}{13} x \sin 6 t
    \right) = 
    \arctg \frac{y}{x} + \frac{\varepsilon}{13} \frac{1}{x^2+y^2} \left[
        7 y \cos(6t) - 6 x \sin(6t)
    \right].
\end{equation*}
Аналогично для выражения $J_1$:
\begin{equation*}
    J_1 = I_1 + \frac{\varepsilon}{13}\left[
        7 x \cos (6 t) + 6 y \sin (6t)
    \right].
\end{equation*}
Итогвая система для $x(t)$ и $y(t)$ получилась вида
\begin{align*}
    J_1 &= \frac{x^2+y^2}{2} + \frac{\varepsilon}{13}\left[
        7 x \cos (6 t) + 6 y \sin (6t)
    \right], \\
    7 t &= \arctg \frac{y}{x} + \frac{\varepsilon}{13} \frac{1}{x^2+y^2} \left[
        7 y \cos(6t) - 6 x \sin(6t)
    \right],
\end{align*}
которую можно уже и не разрешать.


\textbf{Второй случай}. Если повторить рассуждения первого случая, то увидим деление на $0$ в выражение для $S_1$, однако производящую функцию вида
\begin{equation*}
    S = \vc{I} \cdot \vc{\varphi} + S_2,
    \hspace{10 mm} 
    S_2 = \frac{\sqrt{2I_1}}{28},
\end{equation*}
всё можем рассмотреть, чтобы избавиться от одной из возмущающих гармоник.

Новые переменные будут вида
\begin{align*}
    \psi_1 &= \varphi_1 + \frac{\varepsilon}{28 \sqrt{2I_1}} \sin(\varphi_1 - 7 \varphi_2), \\
    \psi_2 &= \varphi_2, \\
    J_1 &= I_1 + \frac{\varepsilon \sqrt{2I_1}}{28} \cos\left(\varphi_1 - 7 \varphi_2\right), \\
    J_2 &= I_2 - \frac{\varepsilon \sqrt{2 I_1}}{4} \cos\left(\varphi_1 - 7 \varphi_2\right),
\end{align*}
с новым гамильтонианом вида
\begin{equation*}
    \hat{H} = 7 J_1 - J_2 + \frac{\varepsilon \sqrt{2 J_1}}{2} \cos(\psi_1 + 7 \psi_2) + o(\varepsilon).
\end{equation*}
В таком случае уравнения движения могут быть записаны, как
\begin{align*}
    \dot{\psi}_1 &= \partial_{J_1} \hat{H} = 7 + \frac{\varepsilon}{2 \sqrt{2 J_1}} \cos (\psi_1 + 7 \psi_2), \\
    \dot{\psi}_2 &= -1, \\
    \dot{J}_1 &= \varepsilon \frac{\sqrt{2 I_1}}{2} \sin(\psi_1 + 7 \psi_2),
\end{align*}
а $J_2$ нас как и раньше не интересует. 

Пристально взглянув на систему понимаем, что хорошая затея ввеси переменные $\gamma = \psi_1 + 7 \psi_2$ и $\varkappa = \sqrt{2 J_1}$, что приводит к системе замечательного вида
\begin{equation*}
    \left\{\begin{aligned}
        \dot{\gamma} = \textstyle \frac{\varepsilon}{2 \varkappa} \cos \gamma, \\
        \dot{\kappa} = \textstyle \frac{\varepsilon}{2} \sin \gamma,
    \end{aligned}\right.
    \hspace{0.25cm} \Rightarrow \hspace{0.25cm}
    \frac{d \gamma}{d \varkappa} = \frac{1}{\varkappa \tg \gamma},
    \hspace{0.25cm} \Rightarrow \hspace{0.25cm}
    \ln \varkappa = - \ln \cos \gamma + \tilde{C},
    \hspace{0.25cm} \Rightarrow \hspace{0.25cm}
    \boxed{\varkappa = \frac{C}{\cos \gamma}}.
\end{equation*}
Подставляя это в уравнения движения, находим
\begin{equation*}
    \varkappa = C \sqrt{1 + \textstyle \left(\frac{\varepsilon}{2C}t\right)^2},
    \hspace{10 mm} 
    \tg \gamma = \left(\frac{\varepsilon}{2C}t\right).
\end{equation*}
Теперь можно пройти обратную цепочку $\{\gamma, \varkappa\} \to \{J_1, \psi_1\} \to \{I_1, \varphi_1\} \to \{x(t), y(t)\}$, и найти искомые решения уравнений движения.