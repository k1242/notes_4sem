\subsubsection*{Т4}

Запишем уравнения Бине для движения в метрике Шварцшильда:
\begin{equation*}
    u'' + u = \frac{a}{2c^2} v^2 + \frac{3}{2}a u^2,
\end{equation*}
где $r^2 \dot{\varphi} = c, \ u=1/r$. Перейдём к системе
\begin{equation*}
    \left\{\begin{aligned}
        u' &= y \\
        y' &= \frac{3}{2} a u^2 + \frac{a}{2c^2} v^2 - u
    \end{aligned}\right.
\end{equation*}
положение равновесия которой находится в точке
\begin{equation*}
    y^* = 0, \hspace{1 cm}
    u^* = 
    \frac{1\pm \sqrt{1-3 \frac{v^2 a^2}{c^2}}}{3a}.
\end{equation*}
Зависимость $u^*(c)$ представлена на рисунке, соответственно положение равновесия существует только при $c \geq \sqrt{3} v a$, при чём при равенстве оно единственно.
\begin{figure}[ht]
    \centering
    \includegraphics{D:\\Kami\\WM12_Workspace\\AnMec\\T4_plot.pdf}
    \caption{Бифуркационная диаграмма стационарных точек уравнения Бине к №Т4}
\end{figure}

\noindent
Посмотрим на устойчивость положения равновесия
\begin{equation*}
    J = \begin{pmatrix}
        0 & 1 \\
        3au-1 & 0 \\
    \end{pmatrix},
    \hspace{0.5cm} \Rightarrow \hspace{0.5cm}
    \lambda^2 = 3 a u - 1 = \pm \sqrt{1 - 3 \frac{v^2 a^2}{c^2}},
\end{equation*}
получается плюсу соответствует седл, а минусу центр. Соответствующие фазовые портреты представлены на рисунке, при $a,v=1$.
\begin{figure}[ht]
    \centering
    \includegraphics{D:\\Kami\\WM12_Workspace\\AnMec\\T4_1.pdf}
    \hspace{0.2cm}
    \includegraphics{D:\\Kami\\WM12_Workspace\\AnMec\\T4_2.pdf}
    \hspace{0.2cm}
    \includegraphics{D:\\Kami\\WM12_Workspace\\AnMec\\T4_3.pdf}
    \caption{Фазовый портрет системы №Т4 (бифуркация при плавном изменение $c$)}
\end{figure}



