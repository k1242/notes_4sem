\subsubsection*{№Т6}

% \red{По хорошему это нужно сделать через нормальную форму, а то что написано ниже -- неточно.} 
Рассмотрим систему вида
\begin{align*}
    \dot{x} &= - y + \mu x - x y^2,\\
    \dot{y} &= \mu y + x - y^3.
\end{align*}



Аналогично Т5 перейдём к полярным координатам, и выразим $\dot{\varphi}$ и $\dot{r}$, так вышло, что и здесь всё хорошо, и
\begin{equation*}
    \left\{\begin{aligned}
        r \dot{\varphi} &= r \\
        \dot{r} &= r \mu - r^3 \sin^2 (\varphi)
    \end{aligned}\right.
    \hspace{0.5cm} \Rightarrow \hspace{0.5cm}
    \frac{\dot{r}}{\dot{\varphi}} = \frac{d r}{d \varphi} = 
    r (\mu - r^2 \sin^2 \varphi).
\end{equation*}
Найдём значения $r = r_*$, где $\dot{r}$ меняет знак
\begin{equation*}
    r_*^2 = \mu \sin^{-2} \varphi,
\end{equation*}
что возможно только при $\mu > 0$. Аналогично предыдущей задаче рассмотрим $\sign \dot{r}$, и получим
\begin{equation*}
    \sign \dot{r} = 
    \left\{\begin{aligned}
        1 & \ \  r < r_*   \\
        -1 & \ \ r > r_*
    \end{aligned}\right.
\end{equation*}

Подробнее рассмотрим положение равновесия $x=y=0$, которое в силу постоянства $\dot{\varphi}$ единственное. В линейном приближение, 
\begin{equation*}
    J = \begin{pmatrix}
        \dot{x}'_x & \dot{x}'_y \\
        \dot{y}'_x & \dot{y}'_y \\
    \end{pmatrix}
    = 
    \begin{pmatrix}
        \mu-y^2 & -1-2xy \\
        1 & \mu - 3 y^2 \\
    \end{pmatrix} = \begin{pmatrix}
        \mu & -1 \\
        1 & \mu \\
    \end{pmatrix}.
\end{equation*}
Тогда 
\begin{equation*}
    \det(J - \lambda E) = (\mu - \lambda)^2 + 1 = 0,
    \hspace{0.5cm} \Rightarrow \hspace{0.5cm}
    \lambda_{1, 2} = \mu \pm 1.
\end{equation*}
Тогда при $\mu < 0$, по теореме Ляпунова об устойчивости в линейном приближение, $x=y=0$ -- устойчивый фокус, при $\mu = 0$ верно, что $\Re (\lambda) = 0$, следовательно это центр, а при $\mu > 0$ фокус становится неустойчивым. Это позволяет прийти к фазовы портретам при различным значениям $\mu$, изображенным на рисунке \ref{T6}.

\begin{figure}[ht]
    \centering
    \includegraphics{D:\\Kami\\WM12_Workspace\\AnMec\\tmp0.pdf}
    \hspace{0.2cm}
    \includegraphics{D:\\Kami\\WM12_Workspace\\AnMec\\tmp1.pdf}
    \hspace{0.2cm}
    \includegraphics{D:\\Kami\\WM12_Workspace\\AnMec\\tmp2.pdf}
    % \hspace{0.2cm}
    % \includegraphics{D:\\Kami\\WM12_Workspace\\AnMec\\tmp.pdf}
    \caption{Бифуркация Пуанкаре-Андронова-Хопфа к №Т6}
    \label{T6}
\end{figure}