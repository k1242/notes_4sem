\subsection*{Т2}
\addcontentsline{toc}{subsection}{T2}

Аппроксимируем движение нИСО в моменты времени $t$ и $t+dt$ сопутствующими ИСО $K'$ и $K''$. Пусть $K$ -- лабороторная система отсчета, $K'$ -- сопутствующая ИСО $\vc{v} \overset{\mathrm{def}}{=}  \vc{v}(t)$, а $K''$ -- сопутствующая ИСО движущаяся относительно $K$ со скоростью $\vc{v}(t + \d t)  = \vc{v} + \d \vc{v}$. Далее для удобства будем считать, что $K''$ движется относительно $K'$ со скоростью $\d \vc{v}'$.

Проверим, что последовательное применеие $\Lambda(d \vc{v}') \cdot \Lambda(\vc{v})$ эквивалентно
$R(\varphi) \cdot \Lambda(\vc{v} + \d \vc{v})$, где $R(\varphi)$ -- вращение в $\{xyz\}$. Для этого просто найдём 
\begin{equation*}
    R(\varphi) = \Lambda(d \vc{v}') \cdot \Lambda(\vc{v}) \cdot \Lambda(\vc{v} + \d \vc{v})^{-1}.
\end{equation*}

Пусть ось $x \parallel \vc{v}$, ось $y$ выберем так, чтобы $d\vc{v} \in \{Oxy\}$. Теперь, согласно $\eqref{LORENTS}$, считая $|\vc{v}|=\beta_1$, $d \vc{v}' = (\beta_x',\, \beta_y')\T$ можем записать (пренебрегая слагаемыми $\beta_x', \beta_y'$ второй и выше степени):
\begin{equation*}
    \Lambda(\vc{v}) =
    \left(
        \begin{array}{cccc}
         \gamma_1 & - \beta_1 \gamma_1 & 0 & 0 \\
         -\beta_1 \gamma_1 & \gamma_1 & 0 & 0 \\
         0 & 0 & 1 & 0 \\
         0 & 0 & 0 & 1 \\
        \end{array}
    \right),
    \hspace{5 mm}
    \Lambda(d \vc{v}') = 
\begin{bmatrix}
 1 & -\beta_x' & -\beta_y' & 0 \\
 -\beta_x' & 1 & 0 & 0 \\
 -\beta_y' & 0 & 1 & 0 \\
 0 & 0 & 0 & 1 \\
\end{bmatrix}.
\end{equation*}
Теперь можем выразить $d \vc{v}'$ через $d \vc{v}$, считая $\vc{r}_{\mathrm{f}}$ центром системы $K''$
\begin{equation*}
    \vc{r}_f' = \Lambda(d \vc{v}') \cdot \Lambda(\vc{v}) \vc{r}_f = 
\l(
    c t', \ 0, \ 0, \ 0
\r)\T
\hspace{0.5cm} \Rightarrow \hspace{0.5cm}
\beta(\vc{v}+d \vc{v})_x = \frac{\beta_1 + \beta_x'}{1 + \beta_1 \beta_x'},
\hspace{5 mm}
\beta(\vc{v}+d \vc{v})_y = \frac{\gamma_{\beta_1} \beta_y}{1+\beta_1 \beta_x}.
% 
\end{equation*}
где скорость находим аналогично первому номеру. Тут стоит заметить, что скоростью $\beta_x$ можно было бы пренебречь в сравнении с $\beta_1$, так как скорее всего первый порядок малось $\beta_x$ не войдёт в ответ, однако хотелось бы в этом убедиться.

Зная $d \vc{v}$ можем найти $d \vc{v}'$:
\begin{equation*}
    \beta_x' = \gamma_{\beta_1}^2 \beta_x,
    \hspace{5 mm}
    \beta_y' = \gamma \beta_y.
\end{equation*}
Но это на потом.

Через $\vc{v}, \ d \vc{v}'$ теперь можем найти $\Lambda(\vc{v} + d \vc{v})$, и посчитать обратную матрицу:
\begin{equation*}
    \Lambda^{-1} (\vc{v} + d \vc{v}) = 
    \begin{bmatrix}
         \gamma_{\beta_1} (\beta_1 \beta_x+1) & \gamma_{\beta_1} (\beta_1+\beta_x) & \beta_y & 0 \\
         \gamma_{\beta_1} (\beta_1+\beta_x) & \gamma_{\beta_1} (\beta_1 \beta_x+1) & \frac{\beta_1 \beta_y}{\gamma_{\beta_1}^{-1}+1} & 0 \\
         \beta_y & \frac{\beta_1 \beta_y}{\gamma_{\beta_1}^{-1}+1} & 1 & 0 \\
         0 & 0 & 0 & 1 \\
    \end{bmatrix}
\end{equation*}
Наконец можем посчитать матрицу поворота, которая в первом приближении действительно не содержит $\beta_x$:
\begin{equation*}
    R(\varphi) = \begin{bmatrix}
 1 & 0 & 0 & 0 \\
 0 & 1 & -\frac{\beta_1 \beta_y'}{\sqrt{1-\beta_1^2}+1} & 0 \\
 0 & \frac{\beta_1 \beta_y'}{\sqrt{1-\beta_1^2}+1} & 1 & 0 \\
 0 & 0 & 0 & 1 \\
\end{bmatrix}
\end{equation*}
что дейстительно соответствует повороту в плоскости $\{xy\}$ вокруг оси $z$ с углом $\varphi$ равным
\begin{equation*}
    \varphi = -\frac{\beta_y \beta_1}{\gamma_{\beta_1}^{-2} + \gamma_{\beta_1}^{-1}} = 
    -\frac{\gamma_{\beta_1}^{2}}{\gamma_{\beta_1} + 1} \beta_1 \beta_y,
\end{equation*}
где $\varphi$ малый, в силу малости $\beta_y$. Так вот, в результате поворота координатных осей меняются и любые векторы, неподвижные в неИСО, то есть искомая угловая скорость
\begin{equation*}
    \omega_z = -\frac{\gamma_{\beta_1}^{2}}{\gamma_{\beta_1} + 1} \beta_1 (\beta_y / \Delta t),
    \hspace{5 mm}
    \Leftrightarrow
    \hspace{5 mm}
    \vc{\omega} = -\frac{\gamma_{\beta_1}^{2}}{\gamma_{\beta_1} + 1} \left[
        \vc{\beta} \times  \dot{\vc{\beta}}
    \right] = \frac{\gamma_{\beta_1}^{2}}{\gamma_{\beta_1} + 1} \left[
        \dot{\vc{\beta}} \times  \vc{\beta}
    \right],
\end{equation*}
что и требовалось доказать.