\subsection*{Т2}

Аппроксимируем движение нИСО в моменты времени $t$ и $t+dt$ сопутствующими ИСО $K'$ и $K''$. Пусть $K$ -- лабороторная система отсчета, $K'$ -- сопутствующая ИСО $\vc{v} \overset{\mathrm{def}}{=}  \vc{v}(t)$, а $K''$ -- сопутствующая ИСО движущаяся относительно $K$ со скоростью $\vc{v}(t + \d t)  = \vc{v} + \d \vc{v}$. Далее для удобства будем считать, что $K''$ движется относительно $K'$ со скоростью $\d \vc{v}'$.

Проверим, что последовательное применеие $\Lambda(d \vc{v}') \cdot \Lambda(\vc{v})$ эквивалентно
$R \cdot \Lambda(\vc{v} + \d \vc{v})$, где $R$ -- вращение в $\{xyz\}$.

Пусть ось $x \parallel \vc{v}$, ось $y$ выберем так, чтобы $d\vc{v} \in \{Oxy\}$. Теперь, согласно $\eqref{LORENTS}$, считая $|\vc{v}|=\beta_1$ можем записать
\begin{equation*}
    \Lambda(\vc{v}) =
    \left(
        \begin{array}{cccc}
         \gamma_1 & - \beta_1 \gamma_1 & 0 & 0 \\
         -\beta_1 \gamma_1 & \gamma_1 & 0 & 0 \\
         0 & 0 & 1 & 0 \\
         0 & 0 & 0 & 1 \\
        \end{array}
    \right),
    \Lambda(d \vc{v}) = 
\end{equation*}
