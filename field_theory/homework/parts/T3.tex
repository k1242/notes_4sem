\subsection*{Т3}
\addcontentsline{toc}{subsection}{T3}

Посмотрим на сопутствующую вращающемуся интерферометру в точке рассматриваемого луча. Для луча можем записать волновой вектор, как
\begin{equation*}
    \bar{k}'_{\pm} = \l(
        \frac{\omega}{c}, \ \pm n \frac{\omega}{c}, \ 0, \ 0
    \r),
\end{equation*}
где знак выбирается в соответсвии с направлением обхода. Считая, что ось $Ox$ направлена вдоль вращения интерферометра в рассматриваемой точке
\begin{equation*}
    c k_{\pm} = \begin{bmatrix}
        \gamma & \gamma\beta & 0  & 0 \\
        \gamma\beta & \gamma & 0  & 0 \\
        0 & 0 & 1 & 0 \\
        0 & 0 & 0 & 1 \\
    \end{bmatrix}
    \begin{pmatrix}
        \omega \\ \pm n \omega \\ 0 \\ 0
    \end{pmatrix} = 
    \begin{pmatrix}
        \omega \gamma (1 + \pm n \beta) \\
        \omega \gamma ( n \pm\beta) \\ 0 \\ 0
    \end{pmatrix},
\end{equation*}
откуда
\begin{equation*}
    c k_{x, \pm} = \omega \gamma (n \pm \beta).
\end{equation*}
Можно заметить, что у света также зависит частота от направления двиения, судя по формуле выше, но в силу малости скорости вращения, это приведет только к оооочень медленной осцилляции в интерференции
\begin{equation*}
    I_{\textnormal{инт}} = I_1 + I_2 + \left\langle \left(\vc{E}_{10} \cdot \vc{E}_{20}\right) \cos\left(
                (\omega_2 - \omega_1) t + \ldots
            \right)\right\rangle,
\end{equation*}
так что по идее этим эффектом можно принебречь.

В силу различности $k_+$ и $k_-$ можем найти разность хода
\begin{equation*}
    \Delta \varphi = \varphi_+ - \varphi_- = 2\pi R \frac{\gamma \omega \beta}{c},
\end{equation*}
считая данной угловую скорость вращения интерферометра $\Omega$ приходим к выражению вида
\begin{equation*}
    \Delta \varphi = \frac{2\gamma}{c^2} \omega \Omega \pi R^2 
    \overset{\gamma \sim 1}{\approx} \frac{2\pi}{c^2} \omega \Omega R^2,
\end{equation*}
где $\gamma \approx 1$ для корректности результата, так как при расчете не учитывалось изменение метрики для неИСО.