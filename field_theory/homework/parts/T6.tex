\subsection*{Т6}
\addcontentsline{toc}{subsection}{T6}
 
 Пион распадается на нейтрино и мюон: $\pi \to \mu + \nu$. Будем работать в система центра инерции.
 \begin{equation*}
 	p_{0 \mu}^i = p_{0 \mu}^i + p_{0 \nu}^i
 	\hspace{0.5 cm}
 	\Rightarrow
 	\hspace{0.5 cm}
 	(p_{0\mu}^i)^2 = (p_{0 \pi}^i - p_{0 \nu}^i)^2
 	\hspace{0.5 cm}
 	\Rightarrow
 	\hspace{0.5 cm}
 	m_\mu^2 c^2 = c^2 (m_\pi^2 + m_\nu^2) - 2 p_{0\pi}^i p_{i0\nu}  = c^2 m_\pi^2 - 2 m_\pi \varepsilon_{0 \nu}.
 \end{equation*}
Откуда получаем
\begin{equation*}
	\varepsilon_{o \nu} = \frac{m_\pi^2 - m_\mu^2}{2 m_\pi}c^2 = \frac{140^2 - 105^2}{2 \cdot 140} \cdot 1^2 = 31 \text{ МэВ}.
\end{equation*}

Переходя в лабораторную систему отсчёта:	
\begin{equation*}
	\varepsilon_\nu = \gamma(v) \varepsilon_{0 \nu} \left(1 - \frac{v}{c} \cos \theta_0\right).
\end{equation*}
Подставляя углы $\theta_0$ найдём минимальную и максимальную энергии:
\begin{equation*}
	\varepsilon_{\text{min}}^\nu = \varepsilon_{0 \nu} \gamma(v) \left(1 - \frac{v}{c}\right) \approx 0.4 \text{ МэВ},
	\hspace{1 cm}
	\varepsilon_{\text{max}}^\nu = \varepsilon_{0 \nu} \gamma(v) \left(1 + \frac{v}{c}\right) \approx 2666 \text{ МэВ}.
\end{equation*}

Для определения среднего значения сначала нужно задаться вопросом распределения по углу отклонения, пока в системе покоя $\pi$:
\begin{equation*}
	\varepsilon_\nu = \gamma(v) \varepsilon_{0 \nu} \left(1 - \frac{v}{c} \cos \theta_0\right)
	\hspace{0.5 cm}
	\overset{d}{\Rightarrow}
	\hspace{0.5 cm}
	d \varepsilon = \frac{v}{c} \gamma \varepsilon_0 \d \cos \theta_0.
\end{equation*}
Из всех частиц $N_0$ в телесном угле $d \Omega_0$ заключено:
\begin{equation*}
	\frac{d N}{N_0} = \frac{d \Omega_0}{4 \pi} = \frac{1}{2} (d \cos \theta) \frac{d \varphi}{2 \pi}
	\hspace{1 cm}
	\Rightarrow
	\hspace{1 cm}
	\frac{d N}{N_0} = \frac{1}{4 \pi} \frac{c \d \varepsilon}{v \gamma \varepsilon_0} (2 \pi) = \frac{d \varepsilon}{\varepsilon_{\text{max}} - \varepsilon_{\text{min}}}.
\end{equation*}

Но это всё было в системе центра инерции, нужно перейти в лабораторную, а тогда произойдёт аберрация:
\begin{equation*}
	\cos \theta = 
	\frac{\cos \theta - \beta}{1 - \beta \cos \theta}.
\end{equation*}
Таким образом
\begin{equation*}
	\frac{d N}{d \cos \theta} = \left(\frac{d N}{d \cos \theta'}\right) \frac{d \cos \theta'}{d \cos \theta} 
	=
	\left(\frac{d N}{d \cos \theta'}\right) \frac{1 - \beta^2}{(\beta \cos \theta - 1)^2},
\end{equation*}
где $d N/d \cos \theta'$ --- распределение по углу в системе центра инерции, которое в силу изотропности пространства постоянно. Так как в правой части отсутствует энергия, то распределение энергии по углу -- постоянно, тогда 
\begin{equation*}
	\langle \varepsilon^\nu\rangle = 1333 \text{\ МэВ}.
\end{equation*}