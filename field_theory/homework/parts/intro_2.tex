% \textbf{Электростатика}. 
Запишем действие взаимодействия $S_{\textnormal{int}}$
\begin{equation*}
    \sub{S}{int} = - \frac{e}{c} \int d \tau\, u^\mu A_\mu,
\end{equation*}
учитывая $u^\mu = f x^\mu / d \tau$:
\begin{equation*}
    \sub{S}{int} = - \frac{1}{c} \int \d^3 V\, \rho \int \d \tau\, \frac{d x^\mu}{d \tau}  A_\mu = 
    - \frac{1}{c} \int \d t\, d^3 V\, \rho \frac{d x^\mu}{d t}  A_\mu,
\end{equation*}
что можем переписать в случае неподвижных зарядов ($\frac{d x^\mu}{d t} = (c,\, \vv{0})\T$), как
\begin{equation*}
    \sub{S}{int} = - \int \d t \int d^3 V \cdot \rho \varphi(\vc{r}),
    \hspace{0.5cm} \Rightarrow \hspace{0.5cm}
    \sub{L}{int} = - \int \d^3 V\, \rho A_0(\vc{r}),
    \hspace{0.5cm} \Rightarrow \hspace{0.5cm}
    U = \int d^3 r\, \rho(\vc{r}) A_0 (\vc{r}).
\end{equation*}

\begin{to_thr}[Теорема Адемолло-Гатто]
    Если к исходному действию $S_0$, привлдящему к периодическому движению, и, следовательно, к адиабатическому инварианту $\mathcal I$, добавлено возмущение с малым параметром $\lambda$, так что полной действие $S = S_0 + \lambda \int V(q, \dot{q})\d t$, то инвариант, по-прежнему, сохарняется с точностью до членов второго порядка малости по $\lambda$: $\frac{d }{d t} \mathcal I = O(\lambda^2)$.
\end{to_thr}

Тензор энергии-импульса поля:
\begin{equation*}
    T_\mu^\nu = \frac{1}{4 \pi c} F^{\nu \lambda} F_{\lambda \mu} + \frac{1}{16 \pi c} F^2 \delta_\mu^\nu,
\end{equation*}
где $F^2 = F_{ij} F^{ij}$. В частности, пространственная и временная компоненты
\begin{equation*}
    T_0^0 = \frac{1}{8 \pi c} (\vc{H}^2 + \vc{E}^2),
    \hspace{5 mm} 
    T_0^\alpha = \frac{1}{4\pi c}\left[\vc{E} \times  \vc{H}\right]^\alpha.
\end{equation*}
Баланс энергии можем записать в интегральном и в дифференциальном виде:
\begin{equation*}
    \frac{d }{d t} \left(W_{\text{ч}} + \int d^3 r\, c T_0^0\right) + \int d^3 r \, \div \vc{S} = 0,
    \hspace{5 mm} 
    \vc{S} = \frac{c}{4\pi} \left[\vc{E} \times  \vc{H}\right],
    \vc{E} \cdot \vc{j} + \frac{d }{d t}  c T_0^0 + \div \vc{S} = 0. 
\end{equation*}
Аналогично, баланс импульса:
\begin{equation*}
    \frac{d }{d t} \left(p^\beta + \frac{1}{c^2} \int d^3 r \, S^\beta\right) = \int d^3 r \, c \nabla_\alpha T^\alpha_\beta,
    \hspace{5 mm} 
    \frac{1}{c}\left(
        j_0 \vc{E} + \vc{j} \times  \vc{H}
    \right)^\beta + \frac{1}{c^2} \frac{d S^\beta}{d t} = c \nabla_\alpha T^\alpha_\beta.
\end{equation*}
Для электрического, и магнитного поля, во время \textit{электрического дипольного излучения} верно, что
\begin{equation*}
    \vc{H} = - \frac{1}{c r^2} \vc{n} \times  \dot{\vc{d}} - \frac{1}{c^2 r} \vc{n} \times  \ddot{\vc{d}},
    \hspace{5 mm} 
    \vc{E} = \frac{1}{r^3} \left(
        3 \left(\vc{d} \cdot \vc{n}\right) \vc{n} - \vc{d}
    \right) + \frac{1}{c r^2} \left(
        3 (\dot{\vc{d}} \cdot \vc{n}) \vc{n} - \dot{\vc{d}}
    \right) + \frac{1}{c^2 r} \left(
        (\ddot{\vc{d}} \cdot \vc{n}) \vc{n} - \ddot{\vc{d}}
    \right).
\end{equation*}
Вектор потока энергии в волновой зоне
\begin{equation*}
    \vc{S} = \frac{c}{4\pi} \left[\vc{E} \times \vc{H}\right] = \frac{1}{4\pi c^3 r^2} \vc{n} \ddot{\vc{d}}^2 \sin^2 \theta,
    \hspace{0.5cm} \Rightarrow \hspace{0.5cm}
    \mathcal J = \frac{2}{3 c^3}\left\langle \ddot{\vc{d}}^2\right\rangle.
\end{equation*}
где $\theta$ -- угол между векторами $\vc{n}$ и $\ddot{\vc{d}}$, $\mathcal J$ --полная интенсивность дипольного излучения. 

