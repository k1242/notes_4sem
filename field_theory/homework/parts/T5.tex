\subsection*{Т5}
\addcontentsline{toc}{subsection}{T5}

Имеем две частицы, 4-импульсы которых в начальный момент:
\begin{equation*}
	p_\gamma^i = \begin{pmatrix}\varepsilon_\gamma \\ \vc{p}_\gamma \end{pmatrix},
	\hspace{0.5 cm}
	|p_\gamma^i| \approx \varepsilon_\gamma.
	\hspace{2 cm}
	p_e^i = \begin{pmatrix} \varepsilon_e \\ \vc{p}_e\end{pmatrix},
	\hspace{0.5 cm}
	|p_e^i| =\beta_e \varepsilon_e.
\end{equation*}

Перейдём в систему центра инерции двух частиц. Пусть пусть движется со скоростью $\beta$, тогда матрица преобразования для такой пересадки и аберация угла будут
\begin{equation*}
 	\begin{pmatrix}
 	    \gamma & \gamma \beta & 0 & 0 \\
 	    \gamma \beta & \gamma & 0 & 0 \\
 	    0 & 0 & 1 & 0 \\
 	    0 & 0 & 0 & 1 \\
 	\end{pmatrix},
 	\hspace{1 cm}
 	\cos \theta' = \frac{\cos \theta - \beta}{1 - \cos \theta \beta}.
\end{equation*}
Запишем закон сохранения импульса до и после столкновения, штрихами пометим величины после столкновения.
\begin{equation*}
	p_\gamma^i + p_e^i = p_\gamma'^i + p_e'^i
	\hspace{0.5 cm}
	\Rightarrow
	\hspace{0.5 cm}
	(p_e'^i)^2 = (p_\gamma^i + p_e^i - p_\gamma'^i)^2
	=
	{p_\gamma^2} + p_e^2 + p_\gamma'^2 + 2 p_e p_\gamma - 2 p_e p_\gamma' - 2 p_\gamma p_\gamma',
\end{equation*}
пренебрегая квадратом импульса фотонов получаем
\begin{equation*}
	m_e^2 = m_e^2 + 2 p_e p_\gamma - 2 p_e p_\gamma' - 2 p_\gamma p_\gamma'
	\hspace{0.5 cm}
	\Rightarrow
	\hspace{0.5 cm}
	p_e p_\gamma -  p_e p_\gamma' -  p_\gamma p_\gamma' = 0.
\end{equation*}
Перемножим компоненты 4-импульсов:
\begin{equation*}
	\varepsilon_e \varepsilon_\gamma - \vc{p}_e \cdot \vc{p}_\gamma - \varepsilon_e \varepsilon_\gamma' + \vc{p}_e \cdot \vc{p}_\gamma' - \varepsilon_\gamma \varepsilon_\gamma' + \vc{p}_\gamma \cdot \vc{p}_\gamma' = 0.
\end{equation*}
Пусть частицы разлетелись под углом $\theta$:
\begin{equation*}
	\varepsilon_e \varepsilon_\gamma + \varepsilon_e \varepsilon_\gamma \beta_e - \varepsilon_e \varepsilon_\gamma' + \beta_e \varepsilon_e \varepsilon_\gamma' \cos \theta - \varepsilon_\gamma \varepsilon_\gamma' + \varepsilon_\gamma \varepsilon_\gamma' \cos (\pi-\theta) = 0.
\end{equation*}
Откуда не сложно выразить энергию фотона после столкновения, заметим, что по условию задачи: $\varepsilon_\gamma/\varepsilon_e = 10^{-11}$, такими членами будем пренебрегать:
\begin{equation*}
	\varepsilon_\gamma' = \frac{\varepsilon_e \varepsilon_\gamma(1 + \beta_e)}{\varepsilon_e (1 - \beta_e \cos \theta) + \varepsilon_\gamma (1 + \cos \theta)}
	=
	\frac{\varepsilon_\gamma (1 + \beta_e)}{1 - \beta \cos \theta + \frac{\varepsilon_\gamma}{\varepsilon_e}(1 + \cos \theta)} 
	\approx \frac{\varepsilon_\gamma (1 + \beta_e)}{1 - \beta_e \cos \theta}.
\end{equation*}

Имея формулу плюс-минус общую не сложно ответить на вопрос про рассеяние назад:
\begin{equation*}
	\varepsilon_\gamma' (\cos \theta = -1) \approx \varepsilon_\gamma = 2 \text{ эВ}. 
\end{equation*}
в то время, как вперед пролетает:
\begin{equation*}
	\varepsilon_\gamma' (\cos \theta = 1) \approx \frac{\varepsilon_\gamma \varepsilon_e^2}{m_e^2} \approx 320 \text{ ГэВ}. 
\end{equation*}