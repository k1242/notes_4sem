\Tsec{Т15}

Пусть есть плоскость $Oxz$, и диполь направлен под углом $\theta_d$ к оси $Oz$.
Воспользуемся методом изображений и зеркально под проводящей плоскостью $Oxy$расположим второй диполь, заменяющий её.
\begin{equation*}
    \vc{d}_1 = d (\vc{e}_z\cos \theta_d + \vc{e} \sin \theta_d),
    \hspace{1 cm}
    \vc{d}_2 = d (\vc{e}_z \cos \theta_d  - \vc{e}_x \sin \theta_d).
\end{equation*}
Здесь введены единичные векторы, и также ещё введём на будущее $\vc{r}$ вектор на точку наблюдения из середины координат, $\vc{n}$ -- единичный по этому направлению.
\begin{equation*}
    \vc{r}_1 = \vc{r} - L \vc{e}_z,
    \hspace{1 cm}
    \vc{r}_2 = \vc{r} + L \vc{e}_z.
\end{equation*}
Тут $2L$ -- расстояние между диполями, будем работать в приближении $r \gg L$. Тогда примерно $\vc{r}_1 \parallel \vc{r}_2 \parallel \vc{n}$. 
И соответственно $r = (\vc{r} \vc{n})$, а остальные: \begin{equation*}
    r_1 = (\vc{r}_1 \vc{n}) = r - L (\vc{e}_z \vc{n}),
    \hspace{1 cm}
    r_2 = (\vc{r}_2 \vc{n}) = r + L (\vc{e}_z \vc{n}).
\end{equation*}
И для $d_{1,2} (t - \frac{r_{1,2}}{c})$
\begin{equation*}
    \vc{d}_1 = d (\vc{e}_z\cos \theta_d + \vc{e} \sin \theta_d) \cos (\omega t - k r_1),
    \hspace{1 cm}
    \vc{d}_2 = d (\vc{e}_z \cos \theta_d  - \vc{e}_x \sin \theta_d) \cos (\omega t - k r_2),
\end{equation*}
колеблеющегося гармонически (по условию):
\begin{equation*}
    \vc{H} = \frac{[\vc{\ddot{d}}_1 \times \vc{n}]}{c^2 r_1} + \frac{\vc{\ddot{d}}_2 \times \vc{n}}{c^2 r_2}
    =
    \frac{- \omega^2 d}{c^2 r} \left(( [\vc{e}_z \times \vc{n}] \cos \theta_d 
    + [\vc{e}_x \times \vc{n}] \sin \theta_d) \cos(\omega t - \frac{\omega}{c}r_1)
    +
( [\vc{e}_z \times \vc{n}] \cos \theta_d 
    + [\vc{e}_x \times \vc{n}] \sin \theta_d) \cos(\omega t - \frac{\omega}{c}r_2)
    \right).
\end{equation*}
Очень хочется упростить: 
\begin{equation*}
    \cos(\omega t - k r_1) = \cos (\omega t - k r + k L (\vc{e}_z \vc{n}))
    =
    \cos(\omega t - k r) cos (k L (\vc{e}_z \vc{n}))    - \sin (\omega t - k r) \sin (k L (\vc{e}_z \vc{n})).
\end{equation*}
\begin{equation*}
    \cos(\omega t - k r_2) = \cos (\omega t - k r - k L (\vc{e}_z \vc{n}))
    =
    \cos(\omega t - k r) cos (k L (\vc{e}_z \vc{n})) + \sin (\omega t - k r) \sin (k L (\vc{e}_z \vc{n})).
\end{equation*}
Тогда возвращаемся к выражению для $H$:
\begin{equation*}
    \vc{H} = \frac{-2 \omega^2 d}{c^2 r} \left([\vc{e}_z \times \vc{n}] \cos \theta_d \cos (\omega t - k r) \cos (k L (\vc{e}_z \vc{n})\right)
    -
    [\vc{e}_x \times \vc{n}] \sin \theta_d \sin (\omega t - k r) \sin (k L (\vc{e}_z \vc{n}))).
\end{equation*}
