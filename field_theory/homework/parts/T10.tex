\Tsec{Т10}


Нас просят найти диполный момент двух полусфер.
Так как нас спросили только про дипольный момент, а про распределение зарядов не спросили, то мы последнее и не будем находить.
\begin{equation*}
    \varphi(\vc{r}) = \int \frac{\rho(\vc{r}') d^3 \vc{r}'}{|\vc{r} - \vc{r}'|}
     = \varphi^{(0)} + \varphi^{(1)} + \varphi^{(2)} + \ldots = \sum_{l=0}^\infty \varphi^{(l)}.
\end{equation*}
Известно что (ЛЛII \S 41 
% его лучше прочитать, потому как мне лень техать выражения для всех величин тут
) :
\begin{equation}
    \varphi^{(l)} = \sum_a e_a\sum_{m = -l}^l \sqrt{\frac{4 \pi}{2 l +1}} D_l^{(m)} Y_{l m} (\theta, \varphi) \frac{r_a^l}{R_0^{l+1}}.
    \label{potential_zhest}
\end{equation}
Любой скалярный потенциал мы всегда можем разложить по сферическим гармоникам:
\begin{equation*}
    \varphi(z) = \sum_{l,m} c_{lm} Y_{lm}(\theta,\varphi) R(z).
\end{equation*}
На сфере: $R=z$:
\begin{equation*}
 \varphi(r) =
    \left\{
    \begin{aligned}
        \Phi_0, \ z>0 \\
        -\Phi_0, \ z<0
    \end{aligned}
    \right.
    \hspace{0.5 cm}
    \Rightarrow
    \hspace{0.5 cm}
    \int \varphi(r) Y_{l m} (\theta, \varphi) = \sin \theta d \theta d\varphi
    =
    i \sqrt{\frac{2 l +1}{4 \pi}} \int_0^{2\pi} \int_0^\pi \cos \theta \varphi(r) \sin \theta d \theta \varphi =
\end{equation*}
\begin{equation*}
    = i \sqrt{\frac{2 l +1}{4 \pi}} \left(
    -\int_0^{\pi/2} \Phi_0 \cos \theta d \cos \theta + \int_{\pi/2}^\pi \Phi_0 \cos \theta d \cos \theta
    \right)
    =
    2 \Phi_0 \pi i \sqrt{\frac{3}{4 \pi}}\left(
    - \frac{\cos ^2 \theta}{2}\bigg|_0^{\pi/2} + \frac{\cos^2 \theta}{2}\bigg|_{\pi/2}^\pi 
    \right).
\end{equation*}
Мы получили, что из \eqref{potential_zhest} взяв как и в выводе формулы до $l=1$:
\begin{equation*}
    2 \pi i \Phi_0 \sqrt{\frac{3}{4\pi}} = \frac{4 \pi}{3} \frac{1}{r} D_l^m = \sqrt{\frac{2 \pi}{3}} i d_z \frac{1}{r}
    \hspace{0.5 cm}
    \Rightarrow
    \hspace{0.5 cm}
    d_z = r \frac{3 \Phi_0}{2}.
\end{equation*}

% Занятно, но решая ту же задачу на семинаре четвртой парой 08.04 мы получили ответ:
% \begin{equation*}
%     d_z = \frac{3}{2}R^2 \Phi_0     
%     \hspace{1 cm}
%     \Leftarrow
%     \hspace{1 cm}
%     \varphi^{(l)} = \sum_a e_a\sum_{m = -l}^l \sqrt{\frac{4 \pi}{2 l +1}} D_l^{(m)} Y_{l m} (\theta, \varphi) \frac{1}{R_0^{l+1}},
% \end{equation*}
% думается, что это из-за отличия в вот этой формуле \eqref{potential_zhest}. И вроде бы сейчас он правдивее и совпадает с ЛЛ2.
