\subsection*{Т7}
\addcontentsline{toc}{subsection}{T7}
Начнём с небольшого вступления.
Выберем оси $z$ по $\vc{H}$, ось $y$ так, чтобы в $ \vc{E} \in \text{Oyz}$.
Тогда тензор электромагнитного поля запишется:
\begin{equation*}
	F^{\mu \nu} = 
	\begin{pmatrix}
	    0 & 0 & -E \sin \theta  & -E \cos \theta \\
	    0 & 0 & -H & 0 \\
	    E \sin \theta & H & 0 & 0 \\
	    E \cos \theta & 0 & 0 & 0 \\
	\end{pmatrix},
\end{equation*}
с учетом $\theta = \pi/2$, $E = \alpha H$, $m c^2 / e = K$, запишем уравнение движения
\begin{equation*}
	 \frac{m c^2}{e} \frac{d u^i}{d s} = F^{ik} u_k = F^{ij} g_{jk} u^k,
	 \hspace{0.5cm} \Rightarrow \hspace{0.5cm}
	 K \frac{d \bar{u}}{d s}  = \begin{pmatrix}
	     0 & 0 & \alpha H  & 0 \\
	     0 & 0 & H  & 0 \\
	     \alpha H & -H & 0 & 0 \\
	     0 & 0 & 0 & 0 \\
	 \end{pmatrix} \bar{u},
\end{equation*}
где $\bar{u} = (u_0, \, u_x,\, u_y,\, u_z) = \bar{p} / m c$. 
Линейные дифференциальные уравнения мы решать вроде умеем, так что находм собственные числа, как
\begin{equation*}
	\det (F^{ik} g_{kj} - \lambda \mathbb{E}^i_j) = \lambda^2 (\lambda^2 - H^2 (\alpha^2-1)) = 0,
	\hspace{0.5cm} \Rightarrow \hspace{0.5cm}
	\begin{aligned}
	    \lambda_{1, 2} &= 0, \\
	    \lambda_{3, 4} &= \pm H \sqrt{\alpha^2 - 1}.
	\end{aligned}.
\end{equation*}
И, соответственно, собственные векторы ($\alpha \neq 1$):
\begin{equation*}
	(\bar{v}_1, \bar{v}_2, \bar{v}_3, \bar{v}_4) = 
	\left(
	\begin{array}{cccc}
	 0 & 0 & 0 & 1 \\
	 \frac{1}{\alpha } & 1 & 0 & 0 \\
	 -\frac{\alpha }{\sqrt{\alpha ^2-1}} & -\frac{1}{\sqrt{\alpha ^2-1}} & 1 & 0 \\
	 \frac{\alpha }{\sqrt{\alpha ^2-1}} & \frac{1}{\sqrt{\alpha ^2-1}} & 1 & 0 \\
	\end{array}
	\right).
\end{equation*}
Осталось подставить начальные условия
\begin{equation*}
	\bar{u}(s=0) = (\mathscr{E}_0/c, p_{0x}, p_{0y}, p_{0z})\T / mc,
\end{equation*}
находим уравнения относительно $\bar{u}$ для трёх случаев. При $\alpha \in (0, 1)$:
\begin{align*}
	u_x(s) &= -\frac{(\alpha  p_0-p_{0x}) \cos \left(\frac{\sqrt{1-\alpha ^2} e H s}{c^2 m}\right)}{\left(1-\alpha ^2\right) c m}-\frac{\left(\alpha ^2-1\right) p_{0y} \sin \left(\frac{\sqrt{1-\alpha ^2} e H s}{c^2 m}\right)}{\left(1-\alpha ^2\right)^{3/2} c m}-\frac{\alpha  (\alpha  p_{0x}-p_0)}{\left(1-\alpha ^2\right) c m} ,
	\\
	u_y(s) &= \frac{(\alpha  p_0-p_{0x}) \sin \left(\frac{\sqrt{1-\alpha ^2} e H s}{c^2 m}\right)}{\sqrt{1-\alpha ^2} c m}+\frac{p_{0y} \cos \left(\frac{\sqrt{1-\alpha ^2} e H s}{c^2 m}\right)}{c m}.
\end{align*}
При $\alpha > 1$:
\begin{align*}
	u_x(s) &= \frac{(\alpha  p_0-p_{0x}) \cosh \left(\frac{\sqrt{\alpha ^2-1} e H s}{c^2 m}\right)}{\left(\alpha ^2-1\right) c m}+\frac{p_{0y} \sinh \left(\frac{\sqrt{\alpha ^2-1} e H s}{c^2 m}\right)}{\sqrt{\alpha ^2-1} c m}+\frac{\alpha  (\alpha  p_{0x}-p_0)}{\left(\alpha ^2-1\right) c m},
	\\
	u_y(s) &= \frac{(\alpha  p_0-p_{0x}) \sinh \left(\frac{\sqrt{\alpha ^2-1} e H s}{c^2 m}\right)}{\sqrt{\alpha ^2-1} c m}+\frac{p_{0y} \cosh \left(\frac{\sqrt{\alpha ^2-1} e H s}{c^2 m}\right)}{c m}.
\end{align*}
И при $\alpha = 1$:
\begin{align*}
	u_0(s) &= \frac{e^2 H^2 s^2 (p_0-p_{0x})}{2 c^5 m^3}+\frac{e H p_{0y} s}{c^3 m^2}+\frac{p_0}{c m} \\
	u_x(s) &= \frac{e^2 H^2 s^2 (p_0-p_{0x})}{2 c^5 m^3}+\frac{e H p_{0y} s}{c^3 m^2}+\frac{p_{0x}}{c m}, \\
	u_y(s) &= \frac{e H s (p_0-p_{0x})}{c^3 m^2}+\frac{p_{0y}}{c m}.
\end{align*}
и по оси $z$ движение с $u_z = p_{0z} / mc = \const$, а $s = c \tau$.



% $\Lambda = \frac{e E}{m c}$ и $\omega = \frac{e H}{m c}$, уравнения движения же :
% \begin{equation*}
% 	\frac{d}{d \tau}
% 	\begin{pmatrix}
% 		u_0 \\ u_x \\ u_y \\ u_z
% 	\end{pmatrix}
% 	=
% 	\begin{pmatrix}
% 	    0 & 0 & \Lambda \sin \theta & \Lambda \cos \theta \\
% 	    0 & 0 & \omega & 0 \\
% 	    \Lambda \sin \theta & - \omega & 0 & 0 \\
% 	    \Lambda \cos \theta & 0 & 0 & 0 \\
% 	\end{pmatrix}
% 	\begin{pmatrix}
% 		u_0 \\ u_x \\ u_y \\ u_z
% 	\end{pmatrix}.
% \end{equation*}
% Решаем полученное дифференциальное уравнения, в степенях экспонент:
% \begin{equation*}
% 	\lambda^4 + \lambda^2 (\omega^2 - \Lambda^2) - \Lambda^2 \omega^2 \cos^2 \theta = 0
% 	\hspace{0.5 cm}
% 	\Rightarrow
% 	\hspace{0.5 cm}
% 	\lambda^2 = -\frac{1}{2} \left(\frac{e}{m c}\right)^2 \left[I_1 \mp \sqrt{I_1^2 + I_2^2}\right],
% 	\hspace{0.5 cm}
% 	I_1 = H^2-E^2,
% 	\hspace{0.2 cm}
% 	I_2 = 2 \vc{E} \cdot \vc{H}
% \end{equation*}
% И получаем:
% \begin{equation*}
% 	\left\{
% 	\begin{aligned}
% 		&\lambda_{1,2} = \pm \chi \\
% 		&\lambda_{3,4} = \pm i \Omega
% 	\end{aligned}\right.
% \end{equation*}
% В нашей же задаче мы имеем случай ортогональных полей, тогда $I_2 = 0$.

% Пусть сначала у нас поля не равны друг другу по модулю, и $H>E$. Тогда остаются ненулевыми только мнимые корни $\lambda_{3,4} = \pm i \Omega$. И циклотронная частота: \begin{equation*}
% 	\Omega = \frac{e}{m c} \sqrt{H^2 - E^2} \equiv \frac{e H'}{m c}.
% \end{equation*}
% \red{
% Тут $H'$ -- магнитное поле в системе отсчета, движущейся вдоль оси $x$ со скоростью:
% \begin{equation*}
% 	\vc{v} = c \frac{\vc{E} \times \vc{H}}{\vc{H}^2}.
% \end{equation*}
% }И из уравнения движения выше получаем, что:
% \begin{equation*}
% 	\dot{u}_z = 0,
% 	\hspace{0.5 cm}
% 	u_z(\tau) = \frac{p_{0z}}{m c},
% 	\hspace{0.5 cm}
% 	z(\tau) = \frac{p_{0z}}{m}\tau.
% \end{equation*}
% Для оставшихся компонент имеем:
% \begin{equation*}
% 	\begin{pmatrix}
% 		u_0 (\tau) \\ u_x(\tau) \\ u_y(\tau)
% 	\end{pmatrix}
% 	=
% 	a_0 \begin{pmatrix}
% 		1 \\ \Lambda / \omega \\ 0
% 	\end{pmatrix}
% 	+
% 	a_+ \begin{pmatrix}
% 		1 \\ \omega / \Lambda \\ i \Omega / \Lambda
% 	\end{pmatrix} e^{i \Omega \tau}
% 	+
% 	a_- \begin{pmatrix}
% 		1 \\ \omega/\Lambda \\ - i \Omega/\Lambda
% 	\end{pmatrix} e^{- i \Omega \tau}.
% \end{equation*}
% И если подставить начальные условия:
% \begin{align*}
% 	&u_0(\tau) = \frac{\omega^2(\varepsilon_0 - p_{0x} v)}{m c^2 \Omega^2} - \frac{\Lambda \omega(\varepsilon_0 v - c^2 p_{0x})}{m c^3 \Omega^2} \cos \Omega \tau
% 	- \frac{\Lambda p_{0y}}{m c \Omega} \sin \Omega \tau \\
% 	&u_{y}(\tau) = \frac{\omega}{m c^2 \Omega}\left(\varepsilon_0 \frac{v}{c} - c p_{0x}\right) \sin \Omega \tau + \frac{p_{0y}}{m c} \cos \Omega \tau\\
% 	&u_x(\tau) = \frac{\Lambda \omega (\varepsilon_0 - p_{0x}v)}{m c^2 \Omega^2} - \frac{\omega^2}{m c^2 \Omega^2}\left[
% 	\left(\varepsilon_0 \frac{v}{c} - c p_{0x}\right)\cos \Omega \tau +\frac{\Omega c p_{0y}}{\omega}\sin \Omega \tau
% 	\right]
% \end{align*}
При пристальном взгляде на $u_0$ и $u_x$ 
\begin{equation*}
	u_0 - u_x = \frac{p_0 - p_{0x}}{cm} = 
	\frac{\mathscr{E}_0 - c p_{0x}}{m c^2}
	=
	\const.
\end{equation*}
Для скорости по оси $x$ получили компоненту, независящую от времени ($E > H$) --- это скорость дрейфа:
\begin{equation*}
	v_{\text{др}} = u_x^{\neq f(s)} c  = c
	\frac{\alpha  (\alpha  p_{0x}-p_0)}{\left(\alpha ^2-1\right) c m} = 
	\bigg/
	\begin{aligned}
	    p_0 &\approx m c\\
	    \alpha &\ll 1. \\
	\end{aligned}
	\bigg/ = c \alpha = c \frac{E}{H},
\end{equation*}
что соответствует нерелятивистскому случаю.

Для случая $H > E$:
\begin{equation*}
	v_{\text{др}} = u_x^{\neq f(s)} c  = c
	\frac{\alpha  (\alpha  p_{0x}-p_0)}{\left(\alpha ^2-1\right) c m} = 
	\bigg/
	\begin{aligned}
	    p_0 &\approx m c\\
	    \alpha &\gg 1. \\
	\end{aligned}
	\bigg/ = \frac{p_0}{cm} \frac{c}{\alpha} = 
	 -\frac{c}{\alpha} = -c \frac{H}{E},
\end{equation*}
что уже очень похоже на правду.