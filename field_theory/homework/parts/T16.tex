\Tsec{Т16}

Значит есть разноименные заряды, один будем характеризовать индексами "1", а другой "2".
Будем работать в системе центра инерции:
\begin{equation*}
    \vc{R} = \frac{m_1 \vc{r}_1 + m_2 \vc{r}_2}{m_1 + m_2},
     \ \vc{r} = \vc{r}_2 - \vc{r}_1
    \hspace{0.5 cm}
    \leadsto
    \hspace{0.5 cm}
    \vc{R}_{\text{центра инерции}} = 0.
\end{equation*}
Тогда не сложно вычислить:
\begin{equation*}
    \vc{r}_1 = - \frac{m_2}{m_1} \vc{r}_2,
    \
    \vc{r}_2 = \frac{m_1}{m_1 + m_2} \vc{r},
    \
    \vc{r} = \vc{r}_2 (1 + \frac{m_2}{m_1}),
    \
    \vc{r}_1 = - \frac{m_2}{m_1 + m_2} \vc{r}.
\end{equation*}
Ну и как мы показывали для излучения диполя: $I =2 
|\vc{\ddot{d}}|^2/(3 c^3)$.
В нашем случае:
\begin{equation*}
    \vc{d} = e_1 \vc{r}_1 + e_2 \vc{r}_2 = \left(\frac{e_2 m_1 - e_1 m_2}{m_1 + m_2}\right) \vc{r} = q \vc{r}
    \hspace{1 cm}
    \Rightarrow
    \hspace{1 cm}
    I = \frac{2 q^2 \vc{\ddot{r}}^2}{3 c^3}.
\end{equation*}
Введем $\mu = \frac{m_1 m_2}{m_1 + m_2}$. Посмотрим на энергию, излучаемую за один период:
\begin{equation*}
    \delta \varepsilon = I \cdot T_{\text{период}} = I \frac{2 \pi r}{v}.
\end{equation*}
Будем работать в предположении, что $\delta \varepsilon \ll \varepsilon$. Воспользуемся теоремой Вириала:
\begin{equation*}
    2 \langle T \rangle = n \langle u \rangle,
    \
    T = \frac{\mu v^2}{2},
    \
    \mu v^2 = -\frac{e_1 e_2}{r}
    \hspace{0.5 cm}
    \leadsto
    \hspace{0.5 cm}
    u = \frac{e_1 e_2}{r}.
\end{equation*}
Таким образом $v = \sqrt{|e_1 e_2|/\mu^2}$, тогда опять к энергии:
\begin{equation*}
    \delta \varepsilon = \frac{2 q^2 \ddot{r}^2}{3 c^3} \frac{\pi r^{3/2} \mu^{1/2}}{|e_1 e_2|^{1/2}}
    =
    \frac{2 q^2}{3 c^3} \frac{|e_1 e_2|^{3/2}}{\mu^{3/2} r^{5/2}}\pi 
    =
    \frac{|e_1 e_2}{2 r} \frac{4 \pi q^2}{3 |e_1 e_2|} 
    \underbrace{\frac{|e_1 e_2|^{3/2}}{c^3 \mu^{3/2} r^{3/2}}}_{(v/c)^3}
    =
    \varepsilon\left(\frac{v}{c}\right)^3 \frac{4 \pi q^2}{3 |e_1 e_2|} \ll \varepsilon.
\end{equation*}

Теперь на интересует $r(t)$. Знаем, что $\varepsilon = \frac{e_1 e_2}{2 r}$.
\begin{equation*}
    I = - \frac{d \varepsilon}{d t} = - \frac{d \varepsilon}{ d r} \frac{d r}{d t}
    =
    \frac{e_1 e_2}{2 r^2} \dot{r} = \frac{2 q^2 \ddot{r}^2}{3 c^3},
\end{equation*}
подставляем сюда Кулона $\mu \ddot{r} = \frac{e_1 e_2}{r^2}$, а он выполняется, так как за один оборот не очень много энергии теряется:
\begin{equation*}
    \frac{e_1 e_2}{2 r^2} \dot{r} = \frac{2 q^2 (e_1 e_2)^2}{3 c^3 \mu^2 r^4}
    \hspace{0.5 cm}
    \leadsto
    \hspace{0.5 cm}
    r^2 \dot{r} = \frac{4 q^2 (e_1 e_2}{3 c^3 \mu^2} = \frac{1}{3} \frac{d r^3}{d t}.
\end{equation*}
Не сложно тогда получается:
\begin{equation*}
    r = \left(r_0^3 + \frac{4 \theta^2 (e_1 e_2)}{c^3 \mu^2} t\right)^{1/3},
    \hspace{1 cm}
    t_{\text{пад}} = \frac{r_0^3 \mu^2 c^3}{4 q^2 (e_1 e_2)}.
\end{equation*}
Для атома время падения электрона на него $t \sim 10^{-8}$ секунды, и действительно в классической теории поля атомы с электронами стабильно существовать не могут.
