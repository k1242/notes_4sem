\Tsec{Т8}

Заряд электрона распределен с плотностью
\begin{equation*}
    \rho(r) = \frac{e}{\pi a^3} \exp\left(-\frac{2r}{a}\right).
\end{equation*}
Найдём энергию взаимодействия электронного облака с ядром в случае ядра, как точечного заряда, и в случае ядра, как равномерно заряженного шара  радиуса $r_0$. Точнее найдём значение следующего выражения:
\begin{equation*}
    \sub{S}{int} = - \frac{e}{c} \int d \tau\, u^\mu A_\mu,
    \hspace{0.5cm} \Rightarrow \hspace{0.5cm}
    U = \int d^3 r\, \rho(\vc{r}) A_0 (\vc{r}).
\end{equation*}


\textbf{Ядро, как точечный заряд.} Вспоминая, что $\vc{E} = - \nabla A_0$ и $\div \vc{E} = 4 \pi \rho_N$, тогда $\nabla(-\nabla A_0) = - \Delta A_0 = 4 \pi \rho_N$, тогда плотность заряда ядра
\begin{equation*}
    \rho_N = -e \cdot \delta(\vc{r}).
\end{equation*}
Для электронного облака известно $\rho(\vc{r})$, тогда
\begin{equation*}
    - \Delta A_0 = - 4 \pi e \, \delta(\vc{r}),
    \hspace{0.5cm} \Rightarrow \hspace{0.5cm}
    A_0 = - \frac{e}{r}, 
\end{equation*}
и, соответсвенно,
\begin{equation*}
    U = \int d^3 r\, 
    \frac{e}{\pi a^3} e^{-2r/a} \cdot \left(-\frac{e}{r}\right) \overset{\mathrm{sp.\, c.s.}}{=}
    - \frac{e^2}{\pi a^3} \int_{0}^{2\pi} d \varphi\, \int_{-1}^{+1} \d \cos \theta \int_{0}^{\infty} r^2 \d r \frac{1}{r} e^{-2r/a},
\end{equation*}
упрощая выражение, переходим к интегралу вида
\begin{equation*}
    U = - \frac{e^2}{\pi a^3} \cdot 2 \Pi \cdot 2  \cdot \int_{0}^{\infty}dr\, r e^{-2r/a}
    = - \frac{e^2}{a} \approx 27.2 \text{\ эВ}
    ,
\end{equation*}
где интеграл мы взяли по частям:
\begin{equation*}
    \int_{0}^{\infty} dt\, t^n e^{-t} = e^{-t} t^n \bigg|_{0}^{\infty} + \int_{0}^{\infty} e^{-t} t^{n-1} n \d t = \ldots = n!\ .
\end{equation*}

\textbf{Ядро, как шар.} Здесь стоит разделить пространство на две области:
\begin{equation*}
    A_0 = \left\{\begin{aligned}
        &-e/r, &r \geq r_0, \\
        &\frac{e}{2r_0^3}r^2- \frac{3}{2}\frac{e}{r_0}, & r \leq r_0,
    \end{aligned}\right.
\end{equation*}
где $A_0$ для $r \leq r_0$ находится, как решение уравнения Пуассона ($\rho_N = \const$):
\begin{equation*}
    \int_{0}^{r_0} d^3 r \ \rho_N = - e,
    \hspace{5 mm} 
    \rho_N = - \frac{3}{4\pi}\frac{e}{r_0^3},
    \hspace{5 mm} 
    \Delta A_0 = - 3 \frac{e}{r_0^3}.
    \hspace{5 mm} 
    A_0(r_0) = - \frac{e}{r_0}.
\end{equation*}
Так как садача симметрична относительно любых поворотов, то $A_0 \equiv A_0 (r)$, тогда
\begin{equation*}
    \Delta A_0 = \frac{d^2 A_0}{d r^2} + \frac{2}{r} \frac{d A_0}{d r},
    \hspace{0.5cm} \Rightarrow \hspace{0.5cm}
    A_0'' + \frac{2 A_0'}{r} =
    \frac{(r A_0)''}{r}
    =- 3 \frac{e}{r_0^3}.
\end{equation*}
Интегрируя, находим
\begin{equation*}
    r A_0 = -\frac{3 e}{r_0^3}\left(
        \frac{1}{6}r^3 + c_1 r + c_2
    \right),
    \hspace{0.5cm} \Rightarrow \hspace{0.5cm}
    A_0(r) = -\frac{e}{2 r_0^3} r^2 + \tilde{c}_1 + \frac{\tilde{c}_2}{r}.
\end{equation*}
Подставляя граничное условие, находим
\begin{equation*}
    \tilde{c}_1 = - \frac{3}{2} \frac{e}{r_0},
    \hspace{0.5cm} \Rightarrow \hspace{0.5cm}
    A_0 = \frac{e}{2r_0^3}r^2 - \frac{3}{2} \frac{e}{r_0}.
\end{equation*}
Осталось посчитать интеграл вида
\begin{equation*}
    U = \int d^3 r \ \rho A_0 \overset{sp.\, c.s.}{=}  
    \int_{0}^{2\pi} d \varphi\, \int_{-1}^{+1} d \cos \theta\, 
    \int_{0}^{\infty}  r^2 \d r \cdot \rho A_0 = 4 \pi I,
\end{equation*}
где $I$, соответсвенно, равен
\begin{equation*}
I = \int_{0}^{r_0}  r^2 \d r \cdot \left(
        A_0 - A_0^{\text{точ}} + A_0^{\text{точ}}
    \right) + \int_{0}^{\infty}  r^2 \d r \rho A_0^{\text{точ}} = 
    \int_0^\infty r^2 \d r \rho A_0^{\text{точ}} + 
    \int_{0}^{r_0} r^2 \d r \rho \left(A_0 - A_0^{\text{точ}}\right).
\end{equation*}
Таким образом искомая энергия представилась, как $U = U_{\text{точ}} + \Delta U$, где $\Delta U$ -- некоторая поправка, связанная с ненулевым размером ядра. Она равна
\begin{equation*}
    \Delta U = 4 \pi \int_0^{r_0} r^2 \d r \rho \cdot \left(
        A_0 - A_0^{\text{точ}}
    \right) = \frac{4 e^2}{a^3} \int_{0}^{r_0} \d r e^{-2r/a} \left(
        \frac{e}{2 r_0^3}r^4 - \frac{3}{2} \frac{e}{r_0} r^2 + e r
    \right).
\end{equation*}
Если разложить экспоненту в ряд, то найдём, что $r_0/a \approx 10^{-5} \ll 1$, тогда получится интеграл вида
\begin{equation*}
    \Delta U  = \frac{4 e^2}{a^3} \left(
        \frac{e}{2 r_0^3} \frac{1}{5} r_0^5 - \frac{3}{2} \frac{e}{r_0} \frac{1}{3} r_0^3 + e \frac{r_0^2}{2}
    \right) = \frac{4}{9} \left(
        \frac{e^2}{2a}
    \right) \left(\frac{r_0}{a}\right)^2 = 2.2 \times  10^{-11} U_{\text{точ}}.
\end{equation*}


