\subsection*{Т4}
\addcontentsline{toc}{subsection}{T4}

Теперь рассмотрим реакцию превращения электрона и позитрона в мюон и антимюон:
\begin{equation*}
    e^+ + e^- \to \mu^+ + \mu^-.
\end{equation*}
Хотелось бы зная энергию стакивающихся частиц найти эффективную массу системы и энергии $\mu^{\pm}$.

Для 4-импульса $p^i = (\mathscr{E}/c, \vc{p})$, для которого верно
\begin{equation*}
    c^2 (2 m_\mu)^2 
    \leq(p_1^i + p_2^i)^2 = \bar{p}_1^2 + \bar{p}_2^2 + 2 \bar{p}_1 \cdot \bar{p}_2 = 
    c^2 2 m_e^2 + 2 \l(
        \mathscr{E}_1 \mathscr{E}_2 / c^2 - \vc{p}_1 \cdot \vc{p}_2
    \r).
\end{equation*}
что приводит нас к неравенству
\begin{equation*}
    c^2 (2 m_\mu^2 - m_e^2) \leq \frac{1}{c^2} \mathscr{E}_1 \mathscr{E}_2 - \vc{p}_1 \cdot \vc{p}_2.
\end{equation*}
При равных энергия $\mathscr{E}_1 = \mathscr{E}_2 = \mathscr{E}$ и $\vc{p}_1 = - \vc{p}_2$ верно, что
\begin{equation*}
    \vc{p}_1^2 = \frac{\mathscr{E}^2}{c^2} - m^2 c^2,
\end{equation*}
тогда
\begin{equation*}
    c^2 (2 m_\mu^2 - m_e^2) \leq \frac{1}{c^2} \mathscr{E}^2 + \vc{p}_1^2 = \frac{2}{c^2} \mathscr{E}^2 - m_e^2 c^2,
\end{equation*}
таким образом 
\begin{equation*}
    \mathscr{E} \geq m_\mu c^2,
    \hspace{5 mm}
    T_{\textnormal{порог}} = (m_\mu - m_e) c^2,
\end{equation*}
а эффективной массе системы соответствует ...

При налете на неподвижную частицу $\mathscr{E}_2 = m_e c$ и $\vc{p}_2 = 0$, тогда
\begin{equation*}
    (2 m_\mu^2 - m_e^2) c^2 \leq \mathscr{E}_1 m_e,
    \hspace{0.5cm} \Rightarrow \hspace{0.5cm}
    \mathscr{E}_1 \geq \left(
        2 \frac{m_\mu^2}{m_e} - m_e
    \right) c^2.
\end{equation*}
Соответсвенно для пороговой энергии верно
\begin{equation*}
    T_{\textnormal{порог}} = \frac{2 c^2}{m_e} \left(
        m_\mu^2 - m_e^2
    \right),
\end{equation*}
а эффективной массе значение ...