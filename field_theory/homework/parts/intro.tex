Для кинематики полезно было бы ввести следующие величины
\begin{equation*}
    \gamma(v) = \gamma_v = \left(1 - \frac{v^2}{c^2}\right)^{-1/2},
    \hspace{1 cm}
    \beta(v) = \beta_v = \frac{v}{c},
    \hspace{1 cm}
    \Lambda(v, OX) = \begin{pmatrix}
        \gamma_v & - \beta_v \gamma_v & 0 & 0\\
        - \beta_v \gamma_v & \gamma & 0 & 0\\
        0 & 0 & 1 & 0\\
        0 & 0 & 0 & 1\\
    \end{pmatrix},
\end{equation*}
где $\Lambda$ -- преобразование Лоренца, для которого, кстати, верно, что $\Lambda^{-1} (v) = \Lambda(-v)$.

Также преобразование Лоренца можно записать в виде
\begin{equation}
\label{LORENTS}
    \begin{pmatrix}
        c t' \\ \vc{r}'
    \end{pmatrix} = 
    \begin{pmatrix}
        \gamma & - \vc{\beta}_v \gamma 
        \vphantom{\dfrac{1}{2}} \\
        - \vc{\beta}_v \gamma & \mathbb{E} + \frac{\gamma_v - 1}{\beta_v^2} \vc{\beta} \otimes \vc{\beta}  
        \vphantom{\dfrac{1}{2}} \\
    \end{pmatrix}
\end{equation}