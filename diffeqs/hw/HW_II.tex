\documentclass{article}
\newcommand{\rg}{\text{rg}}
% \usepackage[unicode, pdftex]{hyperref}
% \usepackage{amsmath,amsthm,amssymb}
% \usepackage{mathtext}
% \usepackage{xcolor}
% \usepackage{hyperref}

% \usepackage[pdftex]{graphicx}


\usepackage[T2A]{fontenc}                   %!? закрепляет внутреннюю кодировку LaTeX
\usepackage[utf8]{inputenc}                 %!  закрепляет кодировку utf8
\usepackage[english,russian]{babel}         %!  подключает русский и английский
% \usepackage[margin=1.8cm]{geometry}         %!  фиксирует оступ на 2cm
\usepackage[bottom=20mm]{geometry}

\usepackage[unicode, pdftex]{hyperref}      %!  оглавление для панели навигации по PDF-документу + гиперссылки

\usepackage{amsthm}                         %!  newtheorem и их сквозная нумерация
\usepackage{hypcap}                         %?  адресация на картинку, а не на подпись к ней
\usepackage{caption}                        %-  позволяет корректировать caption 
\usepackage{fancyhdr}                       %   добавить верхний и нижний колонтитул
\usepackage{wrapfig}                        %!  обтекание таблиц и рисунков

\usepackage{amsmath}                        %!  |
\usepackage{amssymb,textcomp, esvect,esint} %!  |важно для формул 
\usepackage{amsfonts}                       %!  математические шрифты
\usepackage{mathrsfs}                       %  добавит красивые E, H, L
\usepackage{ulem}                           %!  перечеркивание текста
\usepackage{abraces}                        %?  фигурные скобки сверху или снизу текста
\usepackage{pifont}                         %!  нужен для крестика
\usepackage{cancel}                         %!  аутентичное перечеркивание текста
\usepackage{esvect}                         %  добавит вектора стрелочками

\usepackage{graphicx}                       %?  графическое изменение текста
\usepackage{indentfirst}                    %   добавить indent перед первым параграфом
\usepackage{xcolor}                         %   добавляет цвета
\usepackage{enumitem}                       %!  задание макета перечня.


\graphicspath{{pictures/}}
\DeclareGraphicsExtensions{.png}

\definecolor{linkcolor}{HTML}{00009F} % цвет ссылок
\definecolor{urlcolor}{HTML}{00009F} % цвет гиперссылок

\hypersetup{pdfstartview=FitH,  linkcolor=linkcolor,
urlcolor=urlcolor, colorlinks=true}

\oddsidemargin=-5.4mm
\textwidth=180mm
\headheight=0mm
\headsep=0mm

\newcommand{\diag}{\mathop{\mathrm{diag}}\nolimits}
\newcommand{\grad}{\mathop{\mathrm{grad}}\nolimits}
\renewcommand{\div}{\mathop{\mathrm{div}}\nolimits}
\newcommand{\rot}{\mathop{\mathrm{rot}}\nolimits}
\newcommand{\Ker}{\mathop{\mathrm{Ker}}\nolimits}
\newcommand{\Spec}{\mathop{\mathrm{Spec}}\nolimits}
\newcommand{\sign}{\mathop{\mathrm{sign}}\nolimits}
\newcommand{\tr}{\mathop{\mathrm{tr}}\nolimits}
\newcommand{\rg}{\mathop{\mathrm{rg}}\nolimits}
\usepackage{amssymb}
\usepackage{ mathrsfs }
\newcommand{\Lagr}{\mathcal{L}}
\begin{document}


\setlength{\abovedisplayskip}{3pt}
\setlength{\abovedisplayshortskip}{3pt}
\setlength{\belowdisplayskip}{3pt}
\setlength{\belowdisplayshortskip}{3pt}

% \numberwithin{equation}{section}

\begin{center}
    \LARGE \textsc{Задание по курсу <<Дифференциальные уравнения II>>}
\end{center}

\hrule

\phantom{42}

\begin{flushright}
    \begin{tabular}{rr}
    % written by:
        \textbf{Автор}: 
        & Шишкин П.Е. \\ 
        &\\
    % date:
        \textbf{От}: &
        \textit{\today}\\
    \end{tabular}
\end{flushright}

\thispagestyle{empty}
\tableofcontents 
\newpage


\section{От авторов}
\subsection{Шрифт для личных сообщений}
\textcolor[rgb]{0.480469, 0.566406, 0.480469}{\textit{Меня попросили писать текст не имеющий отношения к решению как-то выделенно, поэтому отныне текст, который я пишу просто от души и сердца, будет написан курсивным шрифтом цвета лягушки в обмороке (я серьёзно, такой цвет есть)}}
\subsection{Благодарности}\label{thxs}                                            
\textcolor[rgb]{0.480469,0.566406,0.480469}{\textit{Я благодарен Илье Викторовичу, Биглову Камилю, Хоружему Кириллу, Примаку Евгению, Анастасии Громовой, Александру Двуреченскому, Мише Зотову, Андрею Привалову, Корогоду Дмитрию, Гаврилову Дмитрию и всей 926 группе за прямую или косвенную помощь в создании этого теханного решения (исправление опечаток, помощь в работе с техом и помощь в сохранении рассудка тоже за помощь считается). Было бы грубо не упомянуть тех кто постоянно говорил мне об опечатках и откровенных ошибках, тех с кем я сверял ответы, если видел полную жесть, тех кто помогал научиться техать быстро и не совсем уж плохо, тех кто помогал мне не бросить это дело, и естественно, моего любимого семинариста Илью Викторовича тоже надо упомянуть. Если я кого забыл, пишите, добавим :)}}
\subsection{Заходите в гости}
 \textcolor[rgb]{0.480469,0.566406,0.480469}{\textit{Заходите в гости, всегда всем рад :)}}                                                                                              

\section{I. Первые интегралы и их использование для решений автономных систем}
\subsection{C. \S14: 12}
Исследовать при всех значениях вещественного параметра $a$ поведение фазовых траекторий на всей фазовой плоскости для системы:
\begin{equation}
    \begin{cases}
       \dot{x} = y + ax(x^2 + y^2 - 2)\\
        \dot{y} = - x + ay(x^2 + y^2 - 2)\\
    \end{cases}
\end{equation}
Уравнение выглядит, как что-то в полярных координатам. Чтож, перейдём к ним 
\begin{align*}
x = r \sin{(\varphi)}&&\dot{x} = r \cos {(\varphi)} \dot{\varphi} + \dot{r} \sin {(\varphi )} \\ 
y = r \cos{(\varphi)}&&\dot{y} = -r \sin {(\varphi)} \dot{\varphi} + \dot{r} \cos {(\varphi )} \\ 
\end{align*}
Откуда:
\begin{equation*}
    \begin{cases}
       r \cos {(\varphi)} \dot{\varphi} + \dot{r} \sin {(\varphi )} = r \cos{(\varphi)} + ar \sin{(\varphi)}(r^2 - 2)\\  
      -r \sin {(\varphi)} \dot{\varphi} + \dot{r} \cos {(\varphi )} = -r \sin{(\varphi)} + ar \cos{(\varphi)}(r^2 - 2)\\
    \end{cases}
\end{equation*}
Откуда можно получить 
\begin{equation*}
\begin{cases}
    \dot{r}=ra(r^2-2)\\
    \dot{\varphi}=1\\
        \end{cases}
\end{equation*}
Здесь уже очевидны 3 случая: 1)$a>0$, 2)$a<0$ 3) $a=0$. 1) неустойчивый предельный цикл радиуса $\sqrt 2$ 2) устойчивый 
предельный цикл радиуса $\sqrt 2$  3) центр. Разница при различных знакоопределённых параметрах будет в скорости "навивания" на предельный цикл, но характер движения будет схожий.\\
 Кстати, $\dot{\varphi}=1$ - это первый интеграл. В целом мы сказали при каких параметрах, что но можно это всё безобразие построить \ref{fig:14.12}.\textcolor[rgb]{0.480469,0.566406,0.480469}{\textit{Чтобы было максимально красиво, построим фазовую диаграмму для $r$ график в декартовых координатах и зависимость угла от радиуса. Кстати такая фигня с производной фи получилась из-за неклассической замены икса с синусом. Т.е. немного контринтуитивно что $\varphi =1$ это цикл по часовой стрелке, но и замена $x=\sin{(\varphi)}$ это что-то безумное (я просто хотел кушать а не думать)}}
\begin{figure}[h!]
\center{\includegraphics[width=1\linewidth]{14.12.png}}
\caption{Фазовые диаграммы 14.12 в различных координатах для различных параметров}
\label{fig:14.12}
\end{figure}                                                                 
\subsection{Ф.: 1149}
Решить систему уравнений:
\begin{equation}\label{f1149}
    \begin{cases}
       \dot{x}=y-x\\
       \dot{y}=x+y+z\\
       \dot{z}=x-y\\
    \end{cases}
\end{equation}
Довольно очевидно, что выделить 2 каких-то уравнения без третьей переменной тут не выйдет  \textcolor[rgb]{0.480469,0.566406,0.480469}{(\textit{ну или я слишком слаб и не могу этого сделать)}} поэтому воспользуемся правилом пропорции (чёрной магией):
\begin{equation*}
A/B=C/D=E/F=k;\forall \alpha, \beta,\gamma \in \mathbb{R}, \alpha^2+\beta^2+\gamma^2 \neq 0 \rightarrow \frac{\alpha A + \beta C + \gamma E}{\alpha B + \beta D + \gamma F}=k                                                 
\end{equation*}
Тогда можно записать учитывая $k=dt$:
\begin{equation*}
    \frac{\alpha dx + \beta dy + \gamma dz}{\alpha (y-x) + \beta (x+y+z) + \gamma (x-y)}=dt                                       
\end{equation*}      \
Возьмём $\alpha=1, \beta=0, \gamma=1$ тогда знаменатель обнулится, а значит и числитель должен быть ноль  \textcolor[rgb]{0.480469,0.566406,0.480469}{\textit{(на самом деле можно было бы сформулировать этот шаг гораздо проще, просто вычитая первое и третье уравнения друг из друга, но 1) правило пропорции весьма полезная штуковина в этих задачах, так что чего бы не сформулировать его, 2) обожаю делить на 0 эхэхэхэхэ}}\\
Итак, мы получаем что: $dx+dz=0$ а значит мы нашли первый интеграл системы \ref{f1149}: $C_1=x+z$. Подставим его во второе и первое уравнения \ref{f1149} и получим 
\begin{gather*}
    \frac{dy}{y+C_1}=\frac{dx}{y-x}\\
    y'(y-x)=(y+C_1) \\
    \text{//} y=x+t \text{//} \\
     \textcolor[rgb]{0.480469,0.566406,0.480469}{\textit{нормальная замена, не обижайте её, не будьте как Дима(}}                                               \\
    tt'=x+C_1  \\
    t^2=x^2+2C_1x+C_2 \\
    C_2=y^2-2xy+2(x+z)x
\end{gather*}
Найдены 2 ПИ. Осталось проверить их на независимость.
\begin{equation*}
    \rg \begin{vmatrix}
        \frac{\partial C_1}{\partial x} && \frac{\partial C_1}{\partial y} && \frac{\partial C_1}{\partial z}\\
        \frac{\partial C_2}{\partial x} && \frac{\partial C_2}{\partial y} && \frac{\partial C_2}{\partial z}
    \end{vmatrix}= \rg \begin{vmatrix}
        1&&0&&1\\
        -2y+4x+2z &&2y-2x && 2x
    \end{vmatrix} = 2
\end{equation*}
 \textcolor[rgb]{0.480469,0.566406,0.480469}{\textit{тут хочется сказать 2 вещи: 1) интегралов целая куча зависимых, и то что мой $C_2$ не совпадает с ответом в задачнике, это ок, потому что они друг через друга выражаются 2) так-то очевидно что они независимы, всё-таки второй зависит от $y$ а перавый - нет; но на письмаках требуют считать ранг, потому проверяю так}}\\
 Ответ: $C_2=y^2-2xy+2(x+z)x$,  $C_1=\frac{11(x+y)}{09.2002}$
\subsection{T1} 
Найти первые интегралы уравнений. Используя их, исследовать поведение траекторий на фазовой плоскости.\\
а) $\ddot x + \sin{(x)}=0$\\
Cделаем замену $\dot x = y$ тогда:
\begin{equation*}
    \begin{cases}
        \dot x=y\\
        \dot y = -\sin{(x)}
    \end{cases}
\end{equation*}
по правилу пропорции и обнуляя знаменатель:
\begin{gather*}
    \frac{\alpha dx + \beta dy}{\alpha y - \beta \sin{(x)}}=dt \\
     //\beta=y,\alpha=\sin{(x)}//\\ 
     -\cos{(x)}+\frac{y^2}{2}=C_1
\end{gather*}
Второго первого интеграла тут не будет, иначе бы задача математического маятника решалась слишком легко. Зато у нас есть интеграл энергии. Из него можно немного подумать и получить различные ситуации: $C_1=0$, $C_1>0$ , $C_1<0$.Подставив этот первый интеграл, можно сделать линеаризацию системы, получить что $(2\pi n,0)$ - центры, $(\pi (2k-1),0)$ - сёдла, и получить такое поведение: \ref{fig:Т1a}
\begin{figure}[h!]
\center{\includegraphics[width=1\linewidth]{T1a.png}}
\caption{Т1(a)}
\label{fig:Т1a}
\end{figure} \\
б) $\ddot x-x+x^2=0$ \\
Cделаем замену $\dot x = y$ тогда:
\begin{equation*}
    \begin{cases}
        \dot x=y\\
        \dot y = x-x^2
    \end{cases}
\end{equation*}
по правилу пропорции и обнуляя знаменатель:
\begin{gather*}
    \frac{\alpha dx + \beta dy}{\alpha y + \beta (x-x^2)}=dt \\
     //\beta=y,\alpha=-(x-x^2)//  \\ 
     -3x^2+2x^3+3y^2=C_1
\end{gather*}
Можно построить эту петельку: \ref{fig:Т1b}
\begin{figure}[h!]
\center{\includegraphics[width=1\linewidth]{T1b.png}}
\caption{Т1(б)}
\label{fig:Т1b}
\end{figure} \\

\subsection{C. \S16: 5}
Найдя первый интеграл, решить систему в указанной области
\begin{equation}\label{16.5}
    \begin{cases}
        \dot x = - \frac{x}{y},\\
        \dot y = \frac{y}{x}, (x>0,y>0).\\
    \end{cases}
\end{equation}
Поскольку $dx \frac{y}{x}+dy \frac{x}{y}=0$:
\begin{gather*}
    \frac{dy}{y^2}=-\frac{dx}{x^2}\\
    \frac{1}{y}=-\frac{1}{x}+C_1\\
    y = \frac{x}{C_1 x - 1}
\end{gather*}

Подставим $y$ в первое уравнение \ref{16.5}:
\begin{gather*}
    \dot x = 1-C_1 x\\
    \frac{dx}{1-C_1x}=dt\\
    x=\frac{C_2}{C_1}e^{-C_1t}+\frac{1}{C_1}
\end{gather*}
Подставим $x$ в $y$ и получаем ответ:\\
Ответ: $x=\frac{C_2}{C_1}e^{-C_1t} + \frac{1}{C_1}$, $y=  \frac{e^{C_1t}}{C_1 C_2} + \frac{1}{C_1} $
 \textcolor[rgb]{0.480469,0.566406,0.480469}{\textit{ответ не сходится с ответом в учебнике в силу разных обозначений $C_2$. Так-то ответ мой правильный (ответ учебника я в вольфраме не проверял)}}                                               

\subsection{C. \S16: 26}   
Найдя два независимых первых интеграла системы, решить систему в указанной области.
\begin{equation}\label{16.26}
    \begin{cases}
        \dot x = x^2,\\
        \dot y = 2x^3-xy-z,\\
        \dot z = xz - 2x^4, (x>0)
    \end{cases}
\end{equation}
Хмм, кажется что первое уравнение системы интегрируется. Ну раз так, проинтегрируем: $x=\frac{1}{C_1-t}$. Далее смотрим на оставшиеся 2 уравнения. Возьмём то в котором кроме $x$ не более 1 другой переменной. Т.е. третье. Подставив $x$ можем получить:
\begin{gather*}
    \dot z = \frac{z}{C_1-t} -2 \frac{1}{(C_1-t)^4}\\
    // \tau = C_1-t, \dot z = -\frac{dz}{d \tau}=-z'  // \\
    z'+\frac{z}{\tau}=2 \frac{1}{\tau^4}\\
    // \text{О, это же уравнение Эйлера, его мы умеем решать заменой } \tau=e^T, z'_\tau=z'_T e^{-T}// \\
    z'+z=2e^{-3T}\\
    //\text{находит общее, угадываем частное, благо тут оно очевидное и получаем ответ}// \\
    z(T)=C_2 e^{-T}-e^{-3T}=z(\tau)=\frac{C_2}{\tau}-\frac{1}{\tau^3}=z(t)=\frac{C_2}{C_1-t}-\frac{1}{(C_1-t)^3}
\end{gather*}
Теперь подставим $x(t)$ и $z(t)$ во второе уравнение системы \ref{16.26}:
\begin{gather*}
    \dot y = \frac{3}{(C_1-t)^3}-\frac{C_2}{(C_1-t)}-\frac{y}{C_1-t}\\
    //C_1-t=e^{T}, \dot y = y'_T e^{-T}//\\
    -y' e^{-T} = 3e^{-3T}-C_2 e^{-T}-y e^{-T}\\
    y'-y=C_2-3e^{-2T}\\
    y(T)=C_3 e^T-C_2+e^{-2 T}=y(t)=C_3(C_1-t)-C_2+\frac{1}{(C_1-t)^3}
\end{gather*}
И чтобы посмотреть на этот ужас скопанованно:\\
Ответ: $x(t)=\frac{1}{C_1-t}$,\\
        $y(t)=C_3(C_1-t)-C_2+\frac{1}{(C_1-t)^3}$,\\
        $z(t)=\frac{C_2}{C_1-t}-\frac{1}{(C_1-t)^3}$.\\
На самом деле надо ещё показать что первые интегралы $C_1,C_2$ независимы. Ну если очень хочется можно их выразить, явно посчитать ранг и т.д. Но я воспользуюсь следующим утверждением: поскольку система разрешима (мы смогли решить 2 уравнение) с использованием этих первых 2 интегралов, то они независимы.
 \textcolor[rgb]{0.480469,0.566406,0.480469}{\textit{Пока эти первые интегралы больше используются, как окнстанты. И мало смысла смотреть на них как-то иначе. В следующей части задания это будет не совсем так.}}


\section{II. Линейные однородные уравнения в частных производных первого порядка}
\subsection{C. \S17: 5}
Найти общее решение уравнения и решить задачу Коши с указанным
начальным условием
\begin{equation}\label{17.5}
x \frac{\partial u}{\partial x}+y \frac{\partial u}{\partial y}+z^{2}(x-3 y) \frac{\partial u}{\partial z}=0, u=\frac{x^{2}}{y} \text { при } 3 y z=1
\end{equation}
Найдём характерестическую систему уравнения \ref{17.5}.
\begin{equation*}
    \begin{cases}
        \dot x=x, (1)\\
        \dot y = y, (2)\\
        \dot z = z^2(x-3y),(3)\\
    \end{cases}
\end{equation*}
Для решения надо найти 2 независимых первых интеграла. В данной задаче можно просто составить 2 линейные комбинации: 
\begin{align*}
    (1) \cdot y-(2) \cdot x=0 && (1)\cdot z^2-(2)\cdot 3z^2 -(3)=0\\
    y \cdot dx  = x \cdot dy &&   z^2 \cdot dx - 3z^2 \cdot dy -dz=0\\
    \frac{dx}{x}=\frac{dy}{y} && dx - 3dy - \frac{dz}{z^2}=0\\
    \ln{\frac{x}{y}}=C_1(x,y,z) && x-3y+\frac{1}{z}=C_2(x,y,z)\\
\end{align*}
Вообще должно быть очевидно, что $C_1$ и $C_2$ независимы, потому что $C_2$ зависит от $z$ а $C_1$ - нет. Но давайте проверим ранг.  \textcolor[rgb]{0.480469,0.566406,0.480469}{\textit{просто это-то тут очевидно, а если наткнуться на задачу где не очевидно, а матаппарат не будет отработан будет печально}}\\
\begin{gather*}
\rg \begin{vmatrix}
        \frac{\partial C_1}{\partial x} && \frac{\partial C_1}{\partial y} && \frac{\partial C_1}{\partial z}\\
        \frac{\partial C_2}{\partial x} && \frac{\partial C_2}{\partial y} && \frac{\partial C_2}{\partial z}
    \end{vmatrix} = 
     \rg \begin{vmatrix}
        \frac{1}{x}&& -\frac{1}{x } && 0\\
        1 &&-3 && -\frac{1}{z^2}
    \end{vmatrix} = 2                                                   
\end{gather*}                                               
Значит решением \ref{17.5} будет $F[U_1(x,y,z),U_2(x,y,z)]$ где $F$ - любая непрерывно дифференцируемая функция, $U_1(x,y,z)=\frac{x}{y}, U_2(x,y,z)=x-3y+\frac{1}{z}$. Здесь я выкинул логарифм, потому что если логарифм отношения первый интеграл, то и без логарифма первый интеграл. Вообще это надо было сделать раньше, но я сделал тут :)\\
Решим ЗК.
\begin{gather*}
\begin{cases}
        u=\frac{x^2}{y}\\
        3yz=1\\
        \frac{x}{y}=U_1\\
        x-3y+\frac{1}{z}=U_2\\
    \end{cases}    \\
    \begin{cases}
    u=\frac{x^2}{y}\\
        3yz=1\\
        \frac{x}{y}=U_1\\
        x=U_2\\
    \end{cases}    \\
    u=U_1 U_2= \frac{x}{y}(x-3y+\frac{1}{z})
\end{gather*}
Ответ:\\
 Oбщее решение \ref{17.5}: $F[U_1(x,y,z),U_2(x,y,z)]$ где $F$ - любая непрерывно дифференцируемая функция, $U_1(x,y,z)=\frac{x}{y}, U_2(x,y,z)=x-3y+\frac{1}{z}$.\\
 Pешение ЗК \ref{17.5}: $u=U_1 U_2= \frac{x}{y}(x-3y+\frac{1}{z})$

\subsection{C. \S17: 16}
Найти общее решение уравнения и решить задачу Коши с указанным
начальным условием
\begin{equation}\label{17.16}
(z-x+3 y) \frac{\partial u}{\partial x}+(z+x-3 y) \frac{\partial u}{\partial y}-2 z \frac{\partial u}{\partial z}=0, u=\frac{4 y}{z} \text { при } x=3 y \text { . }
\end{equation}
\textcolor[rgb]{1,1,1}{\text{белый текст}                                            }
Найдём характерестическую систему уравнения \ref{17.16}.
\begin{equation*}
    \begin{cases}
        \dot x = z-x+3y, (1)\\
        \dot y = z+x-3y, (2)\\
        \dot z = -2z,(3)\\
    \end{cases}
\end{equation*}
Тут линейная комбинация только одна (ну или я слаб) $(1) + (2) + (3)=0 \Rightarrow dx+dy+dz=0 \Rightarrow x+y+z = C_1$. Подставим этот первый интеграл во вторые 2 уравнения:
\begin{gather*}
    \begin{cases}
        \dot y = C_1-4y\\
        \dot z =-2z\\
    \end{cases}\\
//\text{перемножая крест накрест получим}//\\
-2z \dot y = \dot z(C_1-4y)\\
\frac{dz}{2z}=\frac{-dy}{C_1-4y}\\
C_2=\frac{z^2}{C_1-4y}= \frac{z^2}{x+z-3y}
\end{gather*}
Проверим независимость: 
\begin{equation*}
    \rg \begin{vmatrix}
        \frac{\partial C_1}{\partial x} && \frac{\partial C_1}{\partial y} && \frac{\partial C_1}{\partial z}\\
        \frac{\partial C_2}{\partial x} && \frac{\partial C_2}{\partial y} && \frac{\partial C_2}{\partial z}
    \end{vmatrix} = ... = 2
\end{equation*}
Осталось только решить ЗК.
\begin{gather*}
\begin{cases}
        u=\frac{4y}{z}\\
        x=3y\\
        x+y+z=U_1\\
        \frac{z^2}{x+z-3y}=U_2\\
    \end{cases}    \\
    \begin{cases}
    u=\frac{4y}{z}\\
        x=3y\\
        4y+z=U_1\\
        z=U_2\\
    \end{cases}    \\
    u=U_1/ U_2-1 = \frac{x}{y}(x-3y+\frac{1}{z})=\frac{(x+y+z)(x-3 y+z)}{z^{2}}-1
\end{gather*}
Ответ:\\
 Oбщее решение \ref{17.16}: $F[U_1(x,y,z),U_2(x,y,z)]$ где $F$ - любая непрерывно дифференцируемая функция, $U_1(x,y,z)=x+y+z, U_2(x,y,z)=\frac{z^2}{x+z-3y}$.\\
 Pешение ЗК \ref{17.16}: $u=u=U_1/ U_2-1 = \frac{x}{y}(x-3y+\frac{1}{z})=\frac{(x+y+z)(x-3 y+z)}{z^{2}}-1$


\subsection{C. \S17: 22}
Найти общее решение уравнения и решить задачу Коши с указанным
начальным условием
\begin{equation}\label{17.22}
\begin{array}{l}
\left(2 x^{2} z^{2}+x\right) \frac{\partial u}{\partial x}-\left(4 x y z^{2}-y\right) \frac{\partial u}{\partial y}-\left(4 x z^{3}-z\right) \frac{\partial u}{\partial z}=0, u=y z^{2} \text { при } x=z
\end{array}
\end{equation}

найдём характерестическую систему уравнения \ref{17.22}.
\begin{equation*}
    \begin{cases}
        \dot x =\left(2 x^{2} z^{2}+x\right) \\
        \dot y = -\left(4 x y z^{2}-y\right)\\
        \dot z =-\left(4 x z^{3}-z\right) \\
    \end{cases}
\end{equation*}
Найдём первые интегралы:
$z \dot y -  y\dot z=0 \Rightarrow z/y=C_1$\\
Cо вторым интегралом немного больнее. Придётся вспомнить уравнения в полных дифференциалах. Второе и третье уже связаны первым интегралом, так что берём первое и 3 (как раз там нет $y$). Перемножим их крест-накрест.
\begin{gather*}
    dx(4xz^3-z)+dz(2x^2z^2+x)=0\\
    //\text{как мы видим, это пока ещё не полный дифференциал. Попробуем домножить на} \mu(z) \text{оно зависит только от z!} //\\
    dx \underbrace{\mu(4xz^3-z)}_{=R} + dz \underbrace{\mu (2x^2z^2+x)}_{=S} = 0 \stackrel{?}{=} d(Q)\\
    //\text{чтобы левая часть была равна дифференциалу чего-то, должно выполняться} \frac{\partial R}{\partial z} = \frac{\partial S}{\partial x}  //\\
    \mu'_z(4xz^3-z)+\mu(12xz^2-1)=\mu(4xz^2+1)\\
    \mu'_z=-\mu \frac{2}{z}\\    
    \frac{d \mu}{\mu}=-\frac{2 dz}{z}\\
    \mu = \frac{1}{z^2}, \\
    //\text{Теперь можно подставить интегрирующий множитель, найти величину, полный дифференциал который ноль}//\\
    dx \underbrace{(4xz-1/z)}_{=R} + dz \underbrace{(2x^2+\frac{x}{z^2})}_{=S} = d(Q)=0\\
    //\text{Тут уже тривиально угадать функцию}//\\
    Q=2x^2z- \frac{x}{z} = \text{const} = C_2
\end{gather*}

Проверим независимость ($C_1$ зависит от $z,y$, $C_2$ зависит от $z,x$. Ну очевидно они независимы): 
\begin{equation*}
    \rg \begin{vmatrix}
        \frac{\partial C_1}{\partial x} && \frac{\partial C_1}{\partial y} && \frac{\partial C_1}{\partial z}\\
        \frac{\partial C_2}{\partial x} && \frac{\partial C_2}{\partial y} && \frac{\partial C_2}{\partial z}
    \end{vmatrix} = \text{очевидно...} = 2
\end{equation*}
Осталось только решить ЗК.
\begin{gather*}
\begin{cases}
        u=yz^2\\
        x=z\\
        U_1=z/y\\
        U_2=2x^2z- \frac{x}{z}\\
    \end{cases}    \\
    \begin{cases}
        u=\frac{4y}{z}\\
         x=z\\
        U_1=z/y\\
        U_2=2z^3 - 1\\
    \end{cases}    \\
    u=\frac{U_2+1}{2U_1} = \frac{y\left(z-x+2 x^{2} z^{2}\right)}{2 z^{2}}
\end{gather*}
Ответ:\\
 Oбщее решение \ref{17.22}: $F[U_1(x,y,z),U_2(x,y,z)]$ где $F$ - любая непрерывно дифференцируемая функция, $U_1(x,y,z)=z/y, U_2(x,y,z)=2x^2z- \frac{x}{z}$.\\
 Pешение ЗК \ref{17.22}: $u=\frac{y\left(z-x+2 x^{2} z^{2}\right)}{2 z^{2}}$

\subsection{C. \S17: 79}
Найти общее решение уравнения и решить задачу Коши с указанным
начальным условием
\begin{equation}\label{17.79}
2 x y \frac{\partial u}{\partial x}+\left(1-y^{2}-2 x z\right) \frac{\partial u}{\partial y}-\frac{y}{x} \frac{\partial u}{\partial z}=0, u=\frac{1}{2}-y^{2} \text { при } y^{2}+x z=1
\end{equation}

найдём характерестическую систему уравнения \ref{17.79}.
\begin{equation*}
    \begin{cases}
        \dot x = 2xy\\
        \dot y = 1-y^{2}-2 x z\\
        \dot z = -\frac{y}{x} \\
    \end{cases}
\end{equation*}
Найдём первые интегралы:
Перемножим крест накрест 1 и 3 уравнения. 
\begin{gather*}
    -dx \frac{y}{x}=2dz \cdot xy\\
    \frac{dx}{x^2}=-2 dz\\
    \frac{1}{x} - 2z = C_1
\end{gather*}
Второй, как и всегда, искать менее тривиально.
Возьмём 1 и 2 уравнения и перемножим крест-накрест. Получится:
\begin{gather*}
    2xy \cdot dy = (1-y^2-2xz) dx\\
    2xy y'=(C_1x-y^2)dx\\
    y'+y \frac{1}{2x}= \frac{C_1}{2y}\\
    //\text{О, это же уравнение Бернулли. } t=y^2,t'=2y y'  //\\
    \frac{t'}{2 \sqrt t}+\sqrt t \frac{1}{2x} = \frac{C_1}{2 \sqrt t}\\
    t'+t \frac{1}{x}= C_1\\
    t'x+t \cdot x' = C_1 x\\
    (t\cdot x)'=C_1x\\
    t x = \frac{C_1 x^2}{2} + C_2\\
    C_2 = xy^2- \frac{C_1 x^2}{2}= x(y^2+zx)-\frac{x}{2}
\end{gather*}
Проверим независимость: Один ПИ зависит от $x$ другой нет, очевидно они независимы.
\begin{equation*}
    \rg \begin{vmatrix}
        \frac{\partial C_1}{\partial x} && \frac{\partial C_1}{\partial y} && \frac{\partial C_1}{\partial z}\\
        \frac{\partial C_2}{\partial x} && \frac{\partial C_2}{\partial y} && \frac{\partial C_2}{\partial z}
    \end{vmatrix} = ... = 2
\end{equation*}
Осталось только решить ЗК.
\begin{gather*}
\begin{cases}
        u= \frac{1}{2} - y^2\\
        y^2+xz=1\\
        U_1=\frac{1}{x} - 2z\\
        U_2= x(y^2+zx)-\frac{x}{2}\\
    \end{cases} \\
    //\text{я считаю, что это какая-то ужасная бяка, поэтому просто переберём варианты } U_1 \cdot U_2, U_1/U_2 \text{ и т.д.}//\\
    u=U_1 \cdot U_2 = (2 x z-1)\left(y^{2}-\frac{1}{2}+x z\right)
\end{gather*}
Ответ:\\
 Oбщее решение \ref{17.79}: $F[U_1(x,y,z),U_2(x,y,z)]$ где $F$ - любая непрерывно дифференцируемая функция, $U_1(x,y,z)=\frac{1}{x} - 2z, U_2(x,y,z)= x(y^2+zx)-\frac{x}{2}$.\\
 Pешение ЗК \ref{17.79}: $u=U_1 \cdot U_2 = (2 x z-1)\left(y^{2}-\frac{1}{2}+x z\right)$
\subsection{C. \S17: 83}
Найти общее решение уравнения и решить задачу Коши с указанным
начальным условием
\begin{equation}\label{17.83}
2 x y \frac{\partial u}{\partial x}+\left(2 x-y^{2}\right) \frac{\partial u}{\partial y}+y^{3} z \frac{\partial u}{\partial z}=0, u=x^{2} z^{2} \text { при } y^{2}=2 x
\end{equation}
найдём характерестическую систему уравнения \ref{17.83}.
\begin{equation*}
    \begin{cases}
        \dot x = 2xy \\
        \dot y = 2 x-y^{2}\\
        \dot z = y^{3} z \\
    \end{cases}
\end{equation*}
Найдём первые интегралы: первые 2 уравнения крест-накрест
\begin{gather*}
    dx(2x-y^2)=dy(2xy)\\
    dx(2x-y^2)+dy(-2xy)=0\\
    d(x^2-xy^2)=0\\
    x^2-xy^2=C_1
\end{gather*}
 \textcolor[rgb]{0.480469,0.566406,0.480469}{\textit{первые интегралы всё более стрёмные. Даже самый первый становится какой-то жестью}} 
$y^2=x- \frac{C_1}{x}$\\
Перемножим 1 и 3 крест-накрест:
\begin{gather*}
    dx(y^3z)=dz(2xy)\\
    z'=\frac{y^2z}{2x}\\
    \frac{dz}{z}=\frac{1}{2}(1- \frac{C_1}{x^2})\\
    \ln (z^2) = x + \frac{C_1}{x} + C_2\\
    C_2 = \ln (z^2) - 2x + y^2. 
\end{gather*}


Проверим независимость: как всегда очевидно независят, но проверить формально надо
\begin{equation*}
    \rg \begin{vmatrix}
        \frac{\partial C_1}{\partial x} && \frac{\partial C_1}{\partial y} && \frac{\partial C_1}{\partial z}\\
        \frac{\partial C_2}{\partial x} && \frac{\partial C_2}{\partial y} && \frac{\partial C_2}{\partial z}
    \end{vmatrix} = ... = 2
\end{equation*}
Осталось только решить ЗК.
\begin{gather*}
\begin{cases}
        u=x^2z^2\\
        y^2=2x\\
        U_1=x^2-xy^2\\
        U_2=\ln (z^2) - 2x + y^2\\
    \end{cases}    \\
    \begin{cases}
        u=x^2z^2\\
        y^2=2x\\
        U_1=-x^2\\
        e^{U_2}=z^2\\
    \end{cases}    \\
    u=-U_1e^{U_2}=(xy^2 - x^2) e^{\ln (z^2) - 2x + y^2} = z^2(xy^2 - x^2) e^{- 2x + y^2}
\end{gather*}
Ответ:\\
 Oбщее решение \ref{17.83}: $F[U_1(x,y,z),U_2(x,y,z)]$ где $F$ - любая непрерывно дифференцируемая функция, $U_1(x,y,z)=x^2-xy^2, U_2(x,y,z)=\ln (z^2) - 2x + y^2$.\\
 Pешение ЗК \ref{17.83}: $u=z^2(xy^2 - x^2) e^{- 2x + y^2}$



\subsection{T2}
В области $x>0, \quad y>0, \quad z>0$ найти все решения уравнения
\begin{equation}\label{17.T2}
\left(x^{2}+y^{2}\right) \frac{\partial u}{\partial x}+2 x y \frac{\partial u}{\partial y}+\frac{x^{3}-x y^{2}}{z} \frac{\partial u}{\partial z}=0
\end{equation}
и решить задачу Коши $u=z^{2}$ при $y^{2}-x^{2}=1$\\

найдём характерестическую систему уравнения \ref{17.T2}.
\begin{equation*}
    \begin{cases}
        \dot x =x^2+y^2 \\
        \dot y = 2xy\\
        \dot z = \frac{x^3-xy^2}{z}\\
    \end{cases}
\end{equation*}
%Найдём первые интегралы: крест-накрест первые 2
%\begin{gather*}
 %   dx \cdot 2xy = dy \cdot (x^2+y^2)\\
  %  y'=\frac{2xy}{x^2+y^2}\\
   % //y=x \cdot t(x), y'=t(x)+xt'(x)//\\
    %t'x+t=\frac{2t}{1+t^2}\\
    %dt \frac{1+t^2}{(t-1)^2}=-\frac{dx}{x}\\
    %t-\frac{2}{t-1}+2 \ln (t-1) = -\ln(x)+C_1\\
    %C_1 = \frac{2 x}{x-y}+\frac{y}{x}+2 \ln \left(\frac{y}{x}-1\right)+\ln (x)\\
%\end{gather*}
% \textcolor[rgb]{0.480469,0.566406,0.480469}{\textit{Какой хороший интеграл, настолько хороший, что я лучше найду какой-нибудь ещё}}%      
 \textcolor[rgb]{0.480469,0.566406,0.480469}{\textit{Тут есть такой красивый первый интеграл! Он закомментирован но если кто хочет посмотреть на наркоманию, можно посмотреть исходный код}}                                                                                          
 Тут можно заметить линейную комбинацию: $x dx  - y dy - z dz =0 \Rightarrow C_1 =  x^2-y^2-z^2$. Далее выражаем $y^2=x^2-z^2-C_1$ и берём первое и третье уравнения.
 \begin{gather*}
     dz(2x^2-z^2-C_1)-dx\left(xz+C_1 \frac{x}{z} \right)=0\\
     //\text{Это почти уравнение в полных дифференциалах, надо всего-то домножить на } \frac{z}{(C_1+z^2)^3}//\\
     dz \frac{2zx-z^3-C_1z}{(C_1+z^2)^3} - dx \frac{x}{(C_1+z^2)^2}=d(C_2)=0\\
     C_2=\frac{1}{2 \left(C_1+z^2\right)}-\frac{x^2}{2 \left(C_1+z^2\right)^2} = \frac{1}{2x^2-2y^2} - \frac{x^2}{2 \left(x^2-y^2\right)^2}
 \end{gather*}



Проверим независимость: о чудо, снова один интеграл зависит от $z$, а другой нет. Так что снова всё ок, но проверить тем не менее необходимо...
\begin{equation*}
    \rg \begin{vmatrix}
        \frac{\partial C_1}{\partial x} && \frac{\partial C_1}{\partial y} && \frac{\partial C_1}{\partial z}\\
        \frac{\partial C_2}{\partial x} && \frac{\partial C_2}{\partial y} && \frac{\partial C_2}{\partial z}
    \end{vmatrix} = ... = 2
\end{equation*}
Осталось только решить ЗК.
\begin{gather*}
\begin{cases}
        u=z^2\\
        y^2-x^2=1\\
        U_1= x^2-y^2-z^2\\
        U_2=\frac{1}{2 \left(C_1+z^2\right)}-\frac{x^2}{2 \left(C_1+z^2\right)^2} = \frac{1}{2x^2-2y^2} - \frac{x^2}{2 \left(x^2-y^2\right)^2}\\
    \end{cases}    \\
    \begin{cases}
        u=z^2\\
        y^2-x^2=1\\
        U_1=-z^2-1\\
        U_2=- \frac{1}{2} - \frac{x^2}{2}\\
    \end{cases}    \\
    u=-1-U_1=z^2+y^2-x^2-1
\end{gather*}
Ответ:\\
 Oбщее решение \ref{17.T2}: $F[U_1(x,y,z),U_2(x,y,z)]$ где $F$ - любая непрерывно дифференцируемая функция, $U_1(x,y,z)=x^2-y^2-z^2, U_2(x,y,z)= \frac{1}{2x^2-2y^2} - \frac{x^2}{2 \left(x^2-y^2\right)^2}$.\\
 Pешение ЗК \ref{17.T2}: $u= u=-1-U_1=z^2+y^2-x^2-1$ 
  \textcolor[rgb]{0.480469,0.566406,0.480469}{\textit{Вот и всё с первыми интегралами. Сказал бы всё плохое что я про эти задачи думаю, но я думаю о них только положительно. Какие хорошие задачи}}                                               

\section{III. Вариационное исчисление}
\subsection{C. \S19: 21}
Решить простейшую вариационную задачу:
\begin{equation}\label{19.21}
J(y)=\int_{1}^{e}\left[\frac{1}{2} x\left(y^{\prime}\right)^{2}+\frac{2 y y^{\prime}}{x}-\frac{y^{2}}{x^{2}}\right] d x, y(1)=1, y(e)=2
\end{equation}
\textcolor[rgb]{0.480469,0.566406,0.480469}{\textit{какая простая задача, простейшая я бы сказал}}
Найдём экстремали:
\begin{gather*}
\Lagr= \frac{1}{2} x\left(y^{\prime}\right)^{2}+\frac{2 y y^{\prime}}{x}-\frac{y^{2}}{x^{2}}\\
\frac{\partial \Lagr}{\partial y}  - \frac{d }{d x} \frac{\partial \Lagr}{\partial y'} =0   \\
2 \frac{y'}{x} - 2 \frac{y}{x^2} - \frac{d}{dx}\left(xy' +2\frac{y}{x} \right)                                      \\
2 \frac{y'}{x} - 2 \frac{y}{x^2} - y' - xy'' - 2 \frac{y'}{x}+2 \frac{y}{x^2}=0\\
y''+\frac{y'}{x}=0\\
y(x)=C_1 \ln(x)+C_2\\
//y(1)=1, y(e)=2//\\
\begin{cases}
    y(1)=1=C_2 + C_1 \cdot 0\\
    y(e)=2=C_2 + C_1 \cdot 1\\
\end{cases}\\
    \hat y=1+\ln(x)\\
\end{gather*} 
Ура, найдена экстремаль. Надо посмотреть будет ли это экстремумом.
\begin{gather*}
\Delta J = J(\hat y + h) - J(\hat y) = \int_1^e \left(\frac{1}{2} x h'(x)^2+\frac{h'(x) (2 h(x)+x+2 \ln (x)+2)}{x}-\frac{h(x) (h(x)+2 \ln (x))}{x^2}\right)  dx  \\
//\text{звуки испуга Паши Шишкина}//    \\
\int_1^e   \left(\frac{1}{2}x h'^2 - \frac{h^2}{x^2} -h \frac{2\ln(x)}{x^2} \right) dx + 
\int_1^e \frac{d(h^2)}{x}+
\int_1^e \left(\frac{2\ln(x)}{x} + 1 + \frac{2}{x} \right) d(h)  \\
//\text{два последних интеграла по частям, с учётом граничных условий на } h(1)=h(e)=0//\\
\int_1^e   \left(\frac{1}{2}x h'^2 - \frac{h^2}{x^2} -h \frac{2\ln(x)}{x^2} \right) dx -
\int_1^e \left(- \frac{h^2}{x^2} \right)dx -
\int_1^e \left( -\frac{2 \ln(x)}{x^2} \right) h \cdot dx\\
\int_1^e\frac{1}{2}x h'^2 \cdot dx
\end{gather*}      
Очевидно на $(1,e)$ данный интеграл принимает исключительно положительные значения $\Rightarrow$ $\hat y -$ минимум.\\
Ответ: \\
$\hat y = 1+\ln(x)$    - абсолютный минимум с вариацией $\Delta J = J(\hat y + h) - J(\hat y) = \int_1^e\frac{1}{2}x h'^2 \cdot dx $                       \\
 \textcolor[rgb]{0.480469,0.566406,0.480469}{\textit{а можно я не буду каждый раз решать это, а просто сошлюсь на файл математики, который сразу выдаёт ответ на задачу?}}                                                   

\subsection{C. \S19: 45}
Решить простейшую вариационную задачу:

\begin{equation}
J(y)=\int_{0}^{1}\left[\left(1+x^{2}\right)\left(y^{\prime}\right)^{2}-4 x y^{\prime}+y y^{\prime} \sin ^{2} x+\frac{1}{2} y^{2} \sin 2 x\right] d x, y(0)=0, y(1)=\ln 2
\end{equation}
 \textcolor[rgb]{1,1,1}{\textit{Наверное очень интересно, почему я иногда пишу $y$ а иногда не ленюсь и прописываю $y(x)$. Это могут быть 2 последовательные строчки, которые почти ничем не отличаются. Первый вариант, когда я набираю текст сам, а второй когда копирую вормулы из математики (там хорошо указывать явно от чего зависит функция)}}   
Найдём экстремаль:
\begin{gather*}
    \Lagr =\left(x^2+1\right) y'(x)^2-4 x y'(x)+y(x) \sin ^2(x) y'(x)+\frac{1}{2} y(x)^2 \sin (2 x)\\
    \frac{\partial \Lagr}{\partial y}  - \frac{d }{d x} \frac{\partial \Lagr}{\partial y'} =0   \\
    \left(x^2+1\right) y''(x)+2 x y'(x)=2\\
    y(x) = \ln \left(x^2+1\right)+c_1 \arctg(x)+c_2]\\
    //\text{подставляя граничные условия:}//\\
    \hat y=\ln \left(x^2+1\right)\\
\end{gather*}
Проверим на наличие экстремума.
\begin{gather*}
    \Delta J = J(\hat y + h) - J(\hat y) \\
     \int_0^1 \left(h \left(\sin ^2(x) h'+\frac{2 x \sin ^2(x)}{x^2+1}+\ln \left(x^2+1\right) \sin (2 x)\right)+h' \left(\left(x^2+1\right) h'+\ln \left(x^2+1\right) \sin ^2(x)\right)+h^2 \sin (x) \cos (x)\right) dx\\
    //\text{какая жесть...}//\\
    \int_0^1\left( h^2 \sin{(x)}\cos{(x)} 
    + h'^2(x^2+1)
    +    h \left( \frac{2 x \sin ^{2}(x)}{x^{2}+1}+\ln \left(x^{2}+1\right) \sin (2 x) \right)\right) dx + \\
    + \int_0^1 \sin ^{2}(x)  d\left(h^2/2 \right) + \int_0^1 \ln \left(x^{2}+1\right) \sin ^{2}(x) d(h)\\
    //\text{2 последних по частям, на краях они обнулятся, а подынтегральные выражения сократятся}//\\
    \int_0^1 h'^2(x^2+1) dx
\end{gather*}
Итого, вариация определена положительно, значит экстремаль - абсолютный минимум.\\
Ответ: \\
$\hat y = \ln \left(x^2+1\right)$  абсолютный минимум с вариацией $\Delta J = J(\hat y + h) - J(\hat y) =\int_0^1 h'^2(x^2+1) dx $
\subsection{C. \S19: 72}
Решить простейшую вариационную задачу:
\begin{equation}
J(y)=\int_{1}^{4}\left[15 \sqrt{x} y+3 x^{2} y y^{\prime}-x^{3}\left(y^{\prime}\right)^{2}\right] d x, y(1)=1, y(4)=-3 .
\end{equation}
Для начала найдём экстремаль
\begin{gather*}
    \Lagr =x^3 \left(-y'(x)^2\right)+3 x^2 y(x) y'(x)+15 \sqrt{x} y(x)\\
    \frac{\partial \Lagr}{\partial y}  - \frac{d }{d x} \frac{\partial \Lagr}{\partial y'} =0   \\
    6 x^2 y'(x)+2 x^3 y''(x)+15 \sqrt{x}=6 x y(x)\\
    y(x) =\frac{C_1}{x^3}+\frac{2}{\sqrt{x}}+C_2 x\\
    //\text{подставляя граничные условия:}//\\
    \hat y=\frac{2-x^{3/2}}{\sqrt{x}}\\
\end{gather*}

Проверим на наличие экстремума.
\begin{gather*}
    \Delta J = J(\hat y + h) - J(\hat y) = \int_1^4 \left(3 h(x) \left(x^2 h'(x)-x^2+4 \sqrt{x}\right)-x^{3/2} h'(x) \left(x^{3/2} h'(x)+x^{3/2}-8\right)\right) \, dx  \\
    \int_1^4 \left( 3 h \left(x^{2}+4 \sqrt{x} \right) - h'^2 x^3 \right) dx + \int_1^4 3/2 x^2 d(h^2)  + \int_1^4 - x^3 + 8x^{3/2}d(h) \\
    //\text{2 последних по частям, на краях они обнулятся, а подынтегральные выражения сократятся}//\\
    \int_1^4  \left(-x^3 h'^2 -3x h^2 \right)dx
\end{gather*}
Итого, вариация определена отрицательно, значит экстремаль - абсолютный максимум.\\
Ответ: \\
$\hat y = \frac{2-x^{3/2}}{\sqrt{x}}$  абсолютный максимум с вариацией $\Delta J = J(\hat y + h) - J(\hat y) =\int_1^4  \left(-x^3 h'^2 -3x h^2 \right)dx$


\subsection{C. \S19: 105}
Показать, что допустимая экстремаль не дает экс-
тремум функционала:
\begin{equation}
J(y)=\int_{0}^{\pi}\left[\left(y^{\prime}\right)^{2}-\frac{25}{16} y^{2}+50 x y\right] d x, y(0)=0, y(\pi)=16 \pi
\end{equation}

Для начала найдём экстремаль
\begin{gather*}
    \Lagr =y'(x)^2-\frac{1}{16} 25 y(x)^2+50 x y(x)\\
    \frac{\partial \Lagr}{\partial y}  - \frac{d }{d x} \frac{\partial \Lagr}{\partial y'} =0   \\
    400 x=16 y''(x)+25 y(x)\\
    y(x) = 16 x+C_1 \cos \left(\frac{5 x}{4}\right)+C_2 \sin \left(\frac{5 x}{4}\right)\\
    //\text{подставляя граничные условия:}//\\
    \hat y=16 x\\
\end{gather*}
Покажем теперь что экстремум отсутствует
\begin{gather*}
    \Delta J = J(\hat y + h) - J(\hat y) = \int_0^{\pi } \left(h'(x) \left(h'(x)+32\right)-\frac{25 h(x)^2}{16}\right) \, dx\\
    //\text{Здесь даже упрощать не надо ничего, просто возьмём 2 функции которые разный знак дадут}//\\
    h_1=\sin{(x)} \Rightarrow  \Delta J = \int_0^\pi \cos (x) (\cos (x)+32)-\frac{25 \sin ^2(x)}{16} dx = -\frac{1}{32} (9 \pi )\\
    h_2 = \sin{(2x)} \Rightarrow  \Delta J = \int_0^\pi 64 \cos (2 x)+\frac{1}{32} (89 \cos (4 x)+39) dx = \frac{39 \pi }{32}\\
\end{gather*}
Таким образом знаки приращения могут принимать как положительные так и отрицательные значения, а значит экстремаль ни минимум ни максимум.

\subsection{T3}
Исследовать на экстремум функционал, определив знаки приращения
\begin{equation}
    \int_{1}^{2}\left(\frac{2 y y^{\prime}}{x}-7 \frac{y^{2}}{x^{2}}-\left(y^{\prime}\right)^{2}-12 \frac{y}{x}\right) d x, \quad y(1)=3, \quad y(2)=1
\end{equation}
Как и ранее для начала найдём экстремаль
\begin{gather*}
    \Lagr =-\frac{7 y(x)^2}{x^2}+\frac{2 y(x) y'(x)}{x}-y'(x)^2-\frac{12 y(x)}{x}\\
    \frac{\partial \Lagr}{\partial y}  - \frac{d }{d x} \frac{\partial \Lagr}{\partial y'} =0   \\
    x y''(x)=\frac{6 y(x)}{x}+6\\
    y(x) = C_2 x^3+\frac{C_1}{x^2}-x\\
    //\text{подставляя граничные условия:}//\\
    \hat y=\frac{8 x^5-31 x^3+116}{31 x^2}\\
\end{gather*}
Уже экстремаль тут как-то не очень выглядит, но дальше лучше. Посчитаем вариацию
\begin{gather*}
    \Delta J = J(\hat y + h) - J(\hat y) = \\
    \int_1^2 -\frac{31 x^4 h'(x)^2+2 x \left(-31 x^2 h(x)+16 x^5-348\right) h'(x)+h(x) \left(217 x^2 h(x)+64 x^5+2088\right)}{31 x^4} \, dx\\
    //\text{оно кажется страшным, но это просто из-за цифорок, так-то функция вполе}//\\
    \int_1^2 \left( -h'^2 -\frac{7}{x^2}h^2 \right)dx -
    \int_1^2 \left(\frac{64}{31}x + \frac{2088}{31x^4} \right)h \cdot dx +
    \int_1^2 \frac{1}{x} d(h^2) -
    \int_1^2 \left( \frac{32}{31} x^2 -\frac{696}{31x^3}\right)d(h)\\
    //\text{2 последних по частям, на краях они обнулятся, а подынтегральные выражения сократятся}//\\
    \int_1^2 \left(-h'^2  - \frac{6}{x^2}h^2 \right)dx
\end{gather*}
Итого, вариация определена отрицательно, значит экстремаль - абсолютный максимум.\\
Ответ: \\
$\hat y = \frac{8 x^5-31 x^3+116}{31 x^2}$  абсолютный максимум с вариацией $\Delta J = J(\hat y + h) - J(\hat y) =\int_1^2 \left(-h'^2  - \frac{6}{x^2}h^2\right)dx$
 \textcolor[rgb]{0.480469,0.566406,0.480469}{\textit{На самом деле здесь проще и оптимальнее не подставлять сразу значение экстремали, а работать с $y$, в некоторый момент вылезать будет появляться уравнение Эйлера-Лагранжа, которое можно приравнять к 0. А вот так сразу подставлять $\hat y$ довольно неприятно. Но всё реально, просто чиселки не оч красивые}}                                               

\subsection{C. \S20.1: 9}
Решить задачу со свободным концом
\begin{equation}
J(y)=\int_{1}^{3}\left[8 y y^{\prime} \ln x-x\left(y^{\prime}\right)^{2}+6 x y^{\prime}\right] d x, y(3)=15 .
\end{equation}\\

Вообще всё то же самое, но теория говорит нам что $\frac{\partial \Lagr}{\partial y'} = 0 |_{x=1}$ на свободном конце. И решение Задачи Коши при подставлении гран условий чутка дольше.  \textcolor[rgb]{0.480469,0.566406,0.480469}{\textit{Тех сошёл с ума, несите новый. Там нет белого текста}}    

\begin{gather*}
    \Lagr = -x y'(x)^2+6 x y'(x)+8 y(x) \ln (x) y'(x)\\
    \frac{\partial \Lagr}{\partial y}  - \frac{d }{d x} \frac{\partial \Lagr}{\partial y'} =0   \\
    2 y'(x)+2 x y''(x)-\frac{8 y(x)}{x}-6 = 0\\
    y = C_2 x^2+\frac{C_1}{x^2}-x\\
    //\text{Найдём граничные условия}//\\
    \frac{\partial \Lagr}{\partial y'} = 0 |_{x=1}\\
    -2 x y'(x)+8 y(x) \ln (x)+6 x = 0 |_{x=1}\\
    0 = \left(6-2 y'(1)\right)\\
    y'(1)=3\\
    //\text{Решая ЗК находим константы } C_1, C_2//\\
    \hat y = 2 x^2-x\\
\end{gather*}
Осталось проверить знак приращений.
\begin{gather*}
    \Delta J = J(\hat y + h) - J(\hat y) = \int_1^3 \left(x h'(x) \left(-h'(x)-8 x+8 (2 x-1) \ln (x)+8\right)+8 h(x) \ln (x) \left(h'(x)+4 x-1\right)\right) \, dx\\
    \int_1^3 -x h'^2 x dx + 
    \int_1^3 8 h \ln(x) \left(4x-1\right) dx + 
    \int_1^3 x \left(8+ 8 \ln(x)(2x-1)-8x \right)  d(h)+
    \int_1^3 4\ln(x) d(h^2)\\
     //\text{все с h под дифференциалом по частям, на краях как всегда обнуляется, а под интегралами вся жесть уходит}//\\
    \int_1^3 \left(-x h'^2 - \frac{4}{x}h^2\right)dx
\end{gather*}
Приращение строго отрицательно, значит экстремаль - абсолютный максимум. \\
Отвееееет: \\
$\hat y = 2 x^2-x + \frac{21890375612389670586790}{8976345801976345} \cdot 0$  абсолютный максимум с вариацией $\Delta J = J(\hat y + h) - J(\hat y) =\int_1^3 \left(-x h'^2 - \frac{4}{x}h^2\right)dx$
 \textcolor[rgb]{0.480469,0.566406,0.480469}{\textit{раз тех сошёл с ума, то и мне можно}}                                               
\subsection{C. \S20.1: 12}
Решить задачу без ограничений
\begin{equation}
J(y)=\int_{1}^{e}\left[x\left(y^{\prime}\right)^{2}+\frac{y^{2}}{x}+\frac{2 y \ln x}{x}\right] d x
\end{equation}\\
Тут то же самое, просто производная ноль на 2 краях.
 \textcolor[rgb]{0.480469,0.566406,0.480469}{\textit{Не, а серьёзно, что происходит, что это за пустота? Я в недоумении. Неужели gather настолько слаб, что не умеет переходить на новую страницу? Или слаб я и не понимаю ничего?}}\\                                               
\begin{gather*}
    \Lagr = x y'(x)^2+\frac{y(x)^2}{x}+\frac{2 y(x) \ln (x)}{x}\\
    \frac{\partial \Lagr}{\partial y}  - \frac{d }{d x} \frac{\partial \Lagr}{\partial y'} =0   \\
    -2 y'(x)-2 x y''(x)+\frac{2 y(x)}{x}+\frac{2 \ln (x)}{x}=0\\
    y = -\ln (x)+\frac{C_1}{x}+C_2 x\\
    //\text{Найдём граничные условия}//\\
    \frac{\partial \Lagr}{\partial y'} = 0 |_{x=1;e}\\
    -2 x y'(x)+8 y(x) \ln (x)+6 x = 0 |_{x=1;e}\\
    2 x y'(x) = 0\\
    y'(1)=0\\
    y'(e)=0\\
    //\text{Решая ЗК находим константы } C_1, C_2//\\
    \hat y = -\frac{e-x^2}{e x+x}-\ln (x)\\
\end{gather*}
Это не самая приятная Экстремаль. Поэтому не будем её подставлять и попадать в ловушку Т3. Запишем так:
\begin{gather*}
    \Delta J = J(\hat y + h) - J(\hat y) = \int_1^e \frac{x^2 h'(x) \left(h'(x)+2 y'(x)\right)+2 h(x) (y(x)+\ln (x))+h(x)^2}{x} \, d\\
    \int_1^e \left(h'^2 x + 2 h \frac{y+\ln(x)}{x} + \frac{h^2}{x} \right) dx+
    \int_1^e \left( 2x y' \right) d(h)\\
    //\text{берём по частям второй интеграл}//\\
    \int_1^e \left(h'^2 x + 2 h \frac{y+\ln(x)}{x} + \frac{h^2}{x} \right) dx -
    \int_1^e \left( 2 y' + 2xy'' \right)h \cdot dx \\
    \int_1^e  \left(h'^2 x +\frac{h^2}{x} \right) dx + 
    \int_1^e \underbrace{\left(-2 y'(x)-2 x y''(x)+\frac{2 y(x)}{x}+\frac{2 \ln (x)}{x} \right)}_{
    \textcolor[rgb]{1,1,1}{\frac{\partial \Lagr}{\partial y}  - \frac{d }{d x} \frac{\partial \Lagr}{\partial y'}} =0 \text{ просто потому что мне так захотелось :)}
    } h \cdot dx \\
    \Delta J =\int_1^e  \left(h'^2 x +\frac{h^2}{x} \right) dx
\end{gather*}
Минимум, получается.\\
Ответ: \\
$\hat y = -\frac{e-x^2}{e x+x}-\ln (x)$  абсолютный минимум с вариацией $\Delta J = J(\hat y + h) - J(\hat y) =\int_1^e  \left(h'^2 x +\frac{h^2}{x} \right) dx$ 
 \textcolor[rgb]{0.480469,0.566406,0.480469}{\textit{Там кстати есть белый текст в задаче, можно найти}}                                               

\subsection{T4}
Исследовать на экстремум функционал, определить знаки приращения
\begin{equation}
    \int_{1}^{2}\left(2 y+y y^{\prime}+x\left(y'\right)^2\right) d x, \quad y(1)=1 .
\end{equation}
 \textcolor[rgb]{0.480469,0.566406,0.480469}{\textit{Я уже боюсь Т-шек, вот уверен сейчас что-то настолько же пугающее, как стол моего соседа по общаге будет}}  
 Найдём экстремали:
 \begin{gather*}
    \Lagr = x y'(x)^2+y(x) y'(x)+2 y(x)\\
    \frac{\partial \Lagr}{\partial y}  - \frac{d }{d x} \frac{\partial \Lagr}{\partial y'} =0   \\
    -2 y'(x)-2 x y''(x)+2=0\\
    x+C_1 \ln (x)+C_2\\
    //\text{Найдём граничные условия}//\\
    4 y'(2)+y(2) = 0\\
    //\text{Решая ЗК находим константы } C_1, C_2//\\
    \hat y=x-\frac{6 \ln (x)}{2+\ln (2)}
 \end{gather*}  
 \textcolor[rgb]{0.480469,0.566406,0.480469}{\textit{Ну оно конечно не совсем пугающее, но деление на логарифм 2. Зачем? За что? Кто придумывал эти числа? И да, я скорее всего не ошибся, потому что проверяю себя в Вольфраме :) Но всё равно никто же сюда списывать не приходит, все будут проверять сами и проверят написанное}}
 Определим знаки приращений:
 \begin{gather*}
    \Delta J = J(\hat y + h) - J(\hat y) = 
    \int_1^2 \left(h(x) \left(h'(x)+y'(x)+2\right)+h'(x) \left(x \left(h'(x)+2 y'(x)\right)+y(x)\right)\right) \, dx\\
    \int_1^2 \left( h (y'+2) + h'^2 x \right) dx + 
    \int_1^2 1/2 d(h^2)+
    \int_1^2 (2xy' +xy)d(h)\\
    //\text{берём по частям всё что не радует сердце и душу}//\\
    \int_1^2 \left(x h'^2  \right)dx +
    \int_1^2 \underbrace{\left( -2 y'(x)-2 x y''(x)+2 \right)}_{=0} h \cdot dx\\
    \Delta J = \int_1^2 \left(x h'^2 \right)dx
 % \end{gather*}                                               
 % Кажется что экстремума нет, но он есть на самом деле. Поркажем что это строгий минимум (интеграл всегда положителен)
 % \begin{gather*}
 %    //\text{оценим второе слагаемое}//\\
 %     |h(x)| = \bigg|\int_1^x h' dy \bigg| = \bigg|\int_1^x h' \cdot 1 \cdot dy \bigg| \underbrace{\leq}_{\text{КБШ*}} \sqrt{\int_1^x 1^2 \cdot dy \int_1^xh'^2 dy  } = \sqrt{(x-1) \int_1^xh'^2 dy }\\
 %     \int_1^3 h^2 dx \leq \int_1^3 (x-1) dx \int_1^x h'^2 dy  \leq \int_1^3 (x-1)dx \cdot \int_1^3 h'^2 dx = 2 \int_1^3 h'^2 dx\\
 %     \int_1^3 \frac{1}{2}h^2 dx \leq \int_1^3 h'^2 dx\\ 
 %     //\text{Итак мы оценили второе слагаемое, оценим первое}//\\
 %     \int_1^3 x h'^2 dx \geq \int_1^3 1\cdot h'^2 dx\\
 %     //\text{собирая всё вместе}//\\
 %     \Delta J = \int_1^2 \left(x h'^2 - \frac{1}{2} h^2 \right)dx \geq 0.
  \end{gather*}
%  Доказано.\\
% *Где КБШ  - неравенство Коши-Буняковского-Шварца. Я бы мог написать полностью, но нижняя фигурная скоюочка растянулась бы и было бы не красиво\\
Ответ: \\
$\hat y = x-\frac{6 \ln (x)}{2+\ln (2)}$  абсолютный минимум с вариацией $\Delta J = J(\hat y + h) - J(\hat y) =\int_1^2 \left(x h'^2 \right)dx$ 

\subsection{C. \S20.2: 5}
Найти допустимые экстремали
\begin{equation}
J\left(y_{1}, y_{2}\right)=\int_{0}^{\pi}\left[\left(y_{1}^{\prime}\right)^{2}+\left(y_{2}^{\prime}\right)^{2}-2 y_{1} y_{2}\right] d x, y_{1}(0)=1, y_{2}(0)=-1, y_{1}\left(\frac{\pi}{2}\right)=e^{\frac{\pi}{2}}, y_{2}\left(\frac{\pi}{2}\right)=-e^{\frac{\pi}{2}} .
\end{equation}
Практически то же самое, только теперь у нас система 2 уравнений Лагранжа второго рода.
\begin{gather*}
    \Lagr = y_1'^2 + y_2'^2 -2 y_1 y_2\\
    \begin{cases}
         \frac{\partial \Lagr}{\partial y_1}  - \frac{d }{d x} \frac{\partial \Lagr}{\partial y'_1} = 0  \\
         \frac{\partial \Lagr}{\partial y_1}  - \frac{d }{d x} \frac{\partial \Lagr}{\partial y'_2} = 0  
    \end{cases}\\
    \begin{cases}
        y_1''+y_2=0\\
        y_2''+y_1=0\\
    \end{cases}\\
    \begin{cases}
        y_1(x) = \frac{1}{2} ((C_1+C_3) \cos (x)+(C_1-C_3) \cosh (x)+(C_2+C_4) \sin (x)+(C_2-C_4) \sinh (x))\\
        y_2(x) = \frac{1}{2} ((C_1+C_3) \cos (x)+(C_3-C_1) \cosh (x)+(C_2+C_4) \sin (x)+(C_4-C_2) \sinh (x))
    \end{cases}\\
    //\text{И подставляя граничные условия получаем:}//\\
    \begin{cases}
        \hat y_1=e^x\\
        \hat y_2 = - e^{x}\\
    \end{cases}
\end{gather*}
Ну да, можно было бы угадать сразу. Ну экстремали мы нашли. Проверять дают они экстремум или нет нас не просили, мы и не будем.\\
Ответ:\\
Допустимые экстремали: $\hat y_1=e^x, \hat y_2 = - e^{x}$

\subsection{C. \S20.3: 2}
Исследовать функционал на экстремум, если:
\begin{equation}
J(y)=\int_{0}^{1}\left[2 e^{x} y-\left(y^{\prime \prime}\right)^{2}\right] d x, y(0)=y^{\prime}(0)=1, y(1)=e, y^{\prime}(1)=2 e
\end{equation}
Всё то же самое, только уравнение Лагранжа 2 рода становится уравнением Эйлера-Пуассона.
\begin{gather*}
    \Lagr = 2e^xy - (y'')^2\\
    \frac{\partial \Lagr}{\partial y}-\frac{d}{d x} \frac{\partial \Lagr}{\partial y^{\prime}}+\frac{d^{2}}{d x^{2}} \frac{\partial \Lagr}{\partial y^{\prime \prime}}=0\\
    2e^x - \frac{d ^2}{d x^2} (2y'')=0\\
    y^{(4)}=e^x\\
    y=C_4 x^3+C_3 x^2+C_2 x+C_1+e^x\\
    //\text{из условия} y(0)=y'(0)=1 \text{получаем} //\\
    C_1=C_2=0\\
     y=C_4 x^3+C_3 x^2+e^x\\
     //\text{из условия} y(1)=e, y'(1)=2e \text{ получаем} //\\
     C_4=e\\
     C_3=-e\\
     \hat y = ex^3-ex^2+e^x
\end{gather*}
Это печально, но Экстремум просят проверить\\
\begin{gather*}
    \Delta J = J(\hat y + h) - J(\hat y) = \int_0^1(2e^xh-2 \hat y'' h'' - h''^2)dx\\
    //\text{в задачах со второй производной не только h на концах должен обнуляться, но и } h'//\\
    // \text{Интегрируем вторую производную h дважды по частям}//\\
    \int_0^1 \left(-h''^2+2e^xh  - 2h y^{(4)} \right)dx = \int_0^1 \left(-h''^2 \right)dx \leq 0
\end{gather*}\\
Итого - экстремаль - абсолютный максимум.\\
Ответ:\\
$\hat y = ex^3-ex^2+e^x$  абсолютный максимум с вариацией $\Delta J = J(\hat y + h) - J(\hat y) =\int_0^1 \left(-h''^2 \right)dx \leq 0$ 

\subsection{C. \S21: 1}
Решить изопериметрическую задачу
\begin{equation}
J(y)=\int_{0}^{\pi}\left(y^{\prime}\right)^{2} d x, y(0)=0, y(\pi)=\pi, \int_{0}^{\pi} y \sin x d x=0
\end{equation}
 \textcolor[rgb]{0.480469,0.566406,0.480469}{\textit{В электронной версии учебника тут опечатка и в ограничении интеграл единичка. Но там ответ с условием не сходится, так что берём нолик из конспектов Ильи Викторовича}}                                               
Найдём допустимые экстремали:
\begin{gather*}
    \Lagr = (y')^2 + \lambda \cdot y \sin{(x)}\\
    \frac{\partial \Lagr}{\partial y}  - \frac{d }{d x} \frac{\partial \Lagr}{\partial y'} =0   \\
    y''=\lambda \sin{(x)}/2\\
    y=y(x)=C_2 x+C_1-\frac{1}{2} \lambda  \sin (x)\\
    \begin{cases}
        y(0)=0 \Rightarrow C_1=0\\
        y(\pi)=\pi \Rightarrow C_2 = 1\\
        \int_{0}^{\pi} y \sin x d x=0 \Rightarrow \lambda = 4
    \end{cases}\\
    \hat y = x-2 \sin{(x)}
\end{gather*}
Проверим знаки приращений.
\begin{gather*}
    \Delta J = J(\hat y + h) - J(\hat y) = \int_0^{\pi} (2\hat{y}'h' + h'^2) dx= \int_0^{\pi} (h'^2) dx - \lambda \int_0^\pi h \sin{(x)}dx = \int_0^{\pi} (h'^2) dx
\end{gather*}
Итого экстремаль - абсолютный минимум.\\
Ответ:\\
$\hat y = x-2 \sin{(x)}$  абсолютный минимум с вариацией $\Delta J = J(\hat y + h) - J(\hat y) =\int_0^{\pi} (h'^2) dx \geq 0$ 

\subsection{T5*}
Среди всех кривых на цилиндре $x^{2}+y^{2}=1$, соединяющих точки $(1,0,0)$ и $(0,1,1)$ найти кривую наименьшей длины (геодезическую кривую).

 \textcolor[rgb]{0.480469,0.566406,0.480469}{\textit{Как же хочется записать метрический тензор для полярных координать, приравнять $w_{\varphi}=0,w_{z}=0$ и сразу получить ответ, но мы тут аналмехом не испорчены, не будем}}
 Если вы знаете, как получается метрический тензор в полярных координатах, сюда не надо смотреть, будет больно. Идите сразу к \ref{T5} Это только если вдруг кто-то не знает как это делают в приличном обществе.
 \begin{gather*}
 \begin{cases}
     x=\cos{(\varphi)} ,
     dx = -\sin{(\varphi)} d(\varphi)\\
     y=\sin{(\varphi)} ,
     dy = \cos{(\varphi)} d(\varphi)\\
     ds^2=dx^2+dy^2+dz^2
 \end{cases}\\
    ds^2=d(\varphi)^2+dz^2\\
 \end{gather*} 
    Теперь можно переформулировать задачу, как:\\
    Найти максимум 
\begin{equation}\label{T5}
    \int_0^{\pi / 2} \left( \sqrt{\left(\frac{d z}{d \varphi} \right)^2+1} \right)  d(\varphi) ,z(0)=0,z(\pi/2)=1.
\end{equation}                                              
Такое уже мы нарешались.
Найдём экстремали.
\begin{gather*}
    \Lagr = \sqrt{z'^2+1}\\
     \frac{\partial \Lagr}{\partial z}  - \frac{d }{d x} \frac{\partial \Lagr}{\partial z'} =0 \\
     \frac{z''}{\left((z')^2+1\right)^{3/2}}=0\\
     z''=0\\
     z=C_1+C_2 \varphi\\
     \hat z = \varphi \frac{2}{\pi}
\end{gather*}
Проверять что это минимум нет необходимости, ведь существует Теорема Хопфа — Ринова, уже доказанная в прошлом семестре курса \href{http://rkarasev.ru/common/upload/an_explanations.pdf}{Математического Анализа}. \\
Ответ:\\
в циллиндрических координатах уравнением геодезической будет $ \hat z = \varphi \frac{2}{\pi}$

 \textcolor[rgb]{0.480469,0.566406,0.480469}{\textit{Прочитайте, пожалуйста, \hyperref[thxs]{\text{благодарности}}. Люди про которых я писал там заслуживают минутки внимания}}                                               
\end{document}
