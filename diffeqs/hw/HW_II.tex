\documentclass{article}

% \usepackage[unicode, pdftex]{hyperref}
% \usepackage{amsmath,amsthm,amssymb}
% \usepackage{mathtext}
% \usepackage{xcolor}
% \usepackage{hyperref}

% \usepackage[pdftex]{graphicx}


\usepackage[T2A]{fontenc}                   %!? закрепляет внутреннюю кодировку LaTeX
\usepackage[utf8]{inputenc}                 %!  закрепляет кодировку utf8
\usepackage[english,russian]{babel}         %!  подключает русский и английский
% \usepackage[margin=1.8cm]{geometry}         %!  фиксирует оступ на 2cm
\usepackage[bottom=20mm]{geometry}

\usepackage[unicode, pdftex]{hyperref}      %!  оглавление для панели навигации по PDF-документу + гиперссылки

\usepackage{amsthm}                         %!  newtheorem и их сквозная нумерация
\usepackage{hypcap}                         %?  адресация на картинку, а не на подпись к ней
\usepackage{caption}                        %-  позволяет корректировать caption 
\usepackage{fancyhdr}                       %   добавить верхний и нижний колонтитул
\usepackage{wrapfig}                        %!  обтекание таблиц и рисунков

\usepackage{amsmath}                        %!  |
\usepackage{amssymb,textcomp, esvect,esint} %!  |важно для формул 
\usepackage{amsfonts}                       %!  математические шрифты
\usepackage{mathrsfs}                       %  добавит красивые E, H, L
\usepackage{ulem}                           %!  перечеркивание текста
\usepackage{abraces}                        %?  фигурные скобки сверху или снизу текста
\usepackage{pifont}                         %!  нужен для крестика
\usepackage{cancel}                         %!  аутентичное перечеркивание текста
\usepackage{esvect}                         %  добавит вектора стрелочками

\usepackage{graphicx}                       %?  графическое изменение текста
\usepackage{indentfirst}                    %   добавить indent перед первым параграфом
\usepackage{xcolor}                         %   добавляет цвета
\usepackage{enumitem}                       %!  задание макета перечня.


\graphicspath{{pictures/}}
\DeclareGraphicsExtensions{.png}

\definecolor{linkcolor}{HTML}{00009F} % цвет ссылок
\definecolor{urlcolor}{HTML}{00009F} % цвет гиперссылок

\hypersetup{pdfstartview=FitH,  linkcolor=linkcolor,
urlcolor=urlcolor, colorlinks=true}

\oddsidemargin=-5.4mm
\textwidth=180mm
\headheight=0mm
\headsep=0mm

\newcommand{\diag}{\mathop{\mathrm{diag}}\nolimits}
\newcommand{\grad}{\mathop{\mathrm{grad}}\nolimits}
\renewcommand{\div}{\mathop{\mathrm{div}}\nolimits}
\newcommand{\rot}{\mathop{\mathrm{rot}}\nolimits}
\newcommand{\Ker}{\mathop{\mathrm{Ker}}\nolimits}
\newcommand{\Spec}{\mathop{\mathrm{Spec}}\nolimits}
\newcommand{\sign}{\mathop{\mathrm{sign}}\nolimits}
\newcommand{\tr}{\mathop{\mathrm{tr}}\nolimits}
\newcommand{\rg}{\mathop{\mathrm{rg}}\nolimits}
\begin{document}

\setlength{\abovedisplayskip}{3pt}
\setlength{\abovedisplayshortskip}{3pt}
\setlength{\belowdisplayskip}{3pt}
\setlength{\belowdisplayshortskip}{3pt}

% \numberwithin{equation}{section}

\begin{center}
    \LARGE \textsc{Задание по курсу <<Дифференциальные уравнения II>>}
\end{center}

\hrule

\phantom{42}

\begin{flushright}
    \begin{tabular}{rr}
    % written by:
        \textbf{Автор}: 
        & Шишкин П.Е. \\ 
        &\\
    % date:
        \textbf{От}: &
        \textit{\today}\\
    \end{tabular}
\end{flushright}

\thispagestyle{empty}
\tableofcontents 
\newpage


\section{От авторов}
\subsection{Шрифт для личных сообщений}
\textcolor[rgb]{0.480469, 0.566406, 0.480469}{\textit{Меня попросили писать текст не имеющий отношения к решению как-то выделенно, поэтому отныне текст, который я пишу просто от души и сердца, будет написан курсивным шрифтом цвета лягушки в обмороке (я серьёзно, такой цвет есть)}}
\subsection{Благодарности}                                            
\textcolor[rgb]{0.480469,0.566406,0.480469}{\textit{Я благодарен *список людей* за *причины* FIXME}}
\subsection{Заходите в гости}
 \textcolor[rgb]{0.480469,0.566406,0.480469}{\textit{Заходите в гости, всегда всем рад :)}}                                                                                              

\section{I. Первые интегралы и их использование для решений автономных систем}
\subsection{C. \S14: 12}
Исследовать при всех значениях вещественного параметра $a$ поведение фазовых траекторий на всей фазовой плоскости для системы:
\begin{equation}
    \begin{cases}
       \dot{x} = y + ax(x^2 + y^2 - 2)\\
        \dot{y} = - x + ay(x^2 + y^2 - 2)\\
    \end{cases}
\end{equation}
Уравнение выглядит, как что-то в полярных координатам. Чтож, перейдём к ним 
\begin{align*}
x = r \sin{(\varphi)}&&\dot{x} = r \cos {(\varphi)} \dot{\varphi} + \dot{r} \sin {(\varphi )} \\ 
y = r \cos{(\varphi)}&&\dot{y} = -r \sin {(\varphi)} \dot{\varphi} + \dot{r} \cos {(\varphi )} \\ 
\end{align*}
Откуда:
\begin{equation*}
    \begin{cases}
       r \cos {(\varphi)} \dot{\varphi} + \dot{r} \sin {(\varphi )} = r \cos{(\varphi)} + ar \sin{(\varphi)}(r^2 - 2)\\  
      -r \sin {(\varphi)} \dot{\varphi} + \dot{r} \cos {(\varphi )} = -r \sin{(\varphi)} + ar \cos{(\varphi)}(r^2 - 2)\\
    \end{cases}
\end{equation*}
Откуда можно получить 
\begin{equation*}
\begin{cases}
    \dot{r}=ra(r^2-2)\\
    \dot{\varphi}=1\\
        \end{cases}
\end{equation*}
Здесь уже очевидны 3 случая: 1)$a>0$, 2)$a<0$ 3) $a=0$. 1) неустойчивый предельный цикл радиуса $\sqrt 2$ 2) устойчивый 
предельный цикл радиуса $\sqrt 2$  3) центр. Разница при различных знакоопределённых параметрах будет в скорости "навивания" на предельный цикл, но характер движения будет схожий.\\
 Кстати, $\dot{\varphi}=1$ - это первый интеграл. В целом мы сказали при каких параметрах, что но можно это всё безобразие построить \ref{fig:14.12}.\textcolor[rgb]{0.480469,0.566406,0.480469}{\textit{Чтобы было максимально красиво, построим фазовую диаграмму для $r$ график в декартовых координатах и зависимость угла от радиуса}}
\begin{figure}[ht]
\center{\includegraphics[width=1\linewidth]{14.12.png}}
\caption{Фазовые диаграммы 14.12 в различных координатах для различных параметров}
\label{fig:14.12}
\end{figure}      
 \textcolor[rgb]{0.480469,0.566406,0.480469}{\textit{Я только начал техать, завтра постараюсь залить не 1 задачу, а побольше. Но на сегодня я сделал скелет и всякое там...}}                                                           
\subsection{Ф.: 1149}
\subsection{T1}
\subsection{C. \S16: 5}
\subsection{C. \S16: 26}   


\section{II. Линейные однородные уравнения в частных производных первого порядка}
\subsection{C. \S17: 5}
\subsection{C. \S17: 16}
\subsection{C. \S17: 22}
\subsection{C. \S17: 79}
\subsection{C. \S17: 83}
\subsection{T2}

\section{III. Вариационное исчисление}
\subsection{C. \S19: 21}
\subsection{C. \S19: 45}
\subsection{C. \S19: 72}
\subsection{C. \S19: 105}
\subsection{T3}
\subsection{C. \S20.1: 9}
\subsection{C. \S20:}
\subsection{T4}
\subsection{C. \S20.2: 5}
\subsection{C. \S20.3: 2}
\subsection{C. \S21: 1}
\subsection{T5*}

\end{document}
