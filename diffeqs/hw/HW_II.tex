\documentclass{article}
\newcommand{\rg}{\text{rg}}
% \usepackage[unicode, pdftex]{hyperref}
% \usepackage{amsmath,amsthm,amssymb}
% \usepackage{mathtext}
% \usepackage{xcolor}
% \usepackage{hyperref}

% \usepackage[pdftex]{graphicx}


\usepackage[T2A]{fontenc}                   %!? закрепляет внутреннюю кодировку LaTeX
\usepackage[utf8]{inputenc}                 %!  закрепляет кодировку utf8
\usepackage[english,russian]{babel}         %!  подключает русский и английский
% \usepackage[margin=1.8cm]{geometry}         %!  фиксирует оступ на 2cm
\usepackage[bottom=20mm]{geometry}

\usepackage[unicode, pdftex]{hyperref}      %!  оглавление для панели навигации по PDF-документу + гиперссылки

\usepackage{amsthm}                         %!  newtheorem и их сквозная нумерация
\usepackage{hypcap}                         %?  адресация на картинку, а не на подпись к ней
\usepackage{caption}                        %-  позволяет корректировать caption 
\usepackage{fancyhdr}                       %   добавить верхний и нижний колонтитул
\usepackage{wrapfig}                        %!  обтекание таблиц и рисунков

\usepackage{amsmath}                        %!  |
\usepackage{amssymb,textcomp, esvect,esint} %!  |важно для формул 
\usepackage{amsfonts}                       %!  математические шрифты
\usepackage{mathrsfs}                       %  добавит красивые E, H, L
\usepackage{ulem}                           %!  перечеркивание текста
\usepackage{abraces}                        %?  фигурные скобки сверху или снизу текста
\usepackage{pifont}                         %!  нужен для крестика
\usepackage{cancel}                         %!  аутентичное перечеркивание текста
\usepackage{esvect}                         %  добавит вектора стрелочками

\usepackage{graphicx}                       %?  графическое изменение текста
\usepackage{indentfirst}                    %   добавить indent перед первым параграфом
\usepackage{xcolor}                         %   добавляет цвета
\usepackage{enumitem}                       %!  задание макета перечня.


\graphicspath{{pictures/}}
\DeclareGraphicsExtensions{.png}

\definecolor{linkcolor}{HTML}{00009F} % цвет ссылок
\definecolor{urlcolor}{HTML}{00009F} % цвет гиперссылок

\hypersetup{pdfstartview=FitH,  linkcolor=linkcolor,
urlcolor=urlcolor, colorlinks=true}

\oddsidemargin=-5.4mm
\textwidth=180mm
\headheight=0mm
\headsep=0mm

\newcommand{\diag}{\mathop{\mathrm{diag}}\nolimits}
\newcommand{\grad}{\mathop{\mathrm{grad}}\nolimits}
\renewcommand{\div}{\mathop{\mathrm{div}}\nolimits}
\newcommand{\rot}{\mathop{\mathrm{rot}}\nolimits}
\newcommand{\Ker}{\mathop{\mathrm{Ker}}\nolimits}
\newcommand{\Spec}{\mathop{\mathrm{Spec}}\nolimits}
\newcommand{\sign}{\mathop{\mathrm{sign}}\nolimits}
\newcommand{\tr}{\mathop{\mathrm{tr}}\nolimits}
\newcommand{\rg}{\mathop{\mathrm{rg}}\nolimits}
\begin{document}


\setlength{\abovedisplayskip}{3pt}
\setlength{\abovedisplayshortskip}{3pt}
\setlength{\belowdisplayskip}{3pt}
\setlength{\belowdisplayshortskip}{3pt}

% \numberwithin{equation}{section}

\begin{center}
    \LARGE \textsc{Задание по курсу <<Дифференциальные уравнения II>>}
\end{center}

\hrule

\phantom{42}

\begin{flushright}
    \begin{tabular}{rr}
    % written by:
        \textbf{Автор}: 
        & Шишкин П.Е. \\ 
        &\\
    % date:
        \textbf{От}: &
        \textit{\today}\\
    \end{tabular}
\end{flushright}

\thispagestyle{empty}
\tableofcontents 
\newpage


\section{От авторов}
\subsection{Шрифт для личных сообщений}
\textcolor[rgb]{0.480469, 0.566406, 0.480469}{\textit{Меня попросили писать текст не имеющий отношения к решению как-то выделенно, поэтому отныне текст, который я пишу просто от души и сердца, будет написан курсивным шрифтом цвета лягушки в обмороке (я серьёзно, такой цвет есть)}}
\subsection{Благодарности}                                            
\textcolor[rgb]{0.480469,0.566406,0.480469}{\textit{Я благодарен *список людей* за *причины* FIXME}}
\subsection{Заходите в гости}
 \textcolor[rgb]{0.480469,0.566406,0.480469}{\textit{Заходите в гости, всегда всем рад :)}}                                                                                              

\section{I. Первые интегралы и их использование для решений автономных систем}
\subsection{C. \S14: 12}
Исследовать при всех значениях вещественного параметра $a$ поведение фазовых траекторий на всей фазовой плоскости для системы:
\begin{equation}
    \begin{cases}
       \dot{x} = y + ax(x^2 + y^2 - 2)\\
        \dot{y} = - x + ay(x^2 + y^2 - 2)\\
    \end{cases}
\end{equation}
Уравнение выглядит, как что-то в полярных координатам. Чтож, перейдём к ним 
\begin{align*}
x = r \sin{(\varphi)}&&\dot{x} = r \cos {(\varphi)} \dot{\varphi} + \dot{r} \sin {(\varphi )} \\ 
y = r \cos{(\varphi)}&&\dot{y} = -r \sin {(\varphi)} \dot{\varphi} + \dot{r} \cos {(\varphi )} \\ 
\end{align*}
Откуда:
\begin{equation*}
    \begin{cases}
       r \cos {(\varphi)} \dot{\varphi} + \dot{r} \sin {(\varphi )} = r \cos{(\varphi)} + ar \sin{(\varphi)}(r^2 - 2)\\  
      -r \sin {(\varphi)} \dot{\varphi} + \dot{r} \cos {(\varphi )} = -r \sin{(\varphi)} + ar \cos{(\varphi)}(r^2 - 2)\\
    \end{cases}
\end{equation*}
Откуда можно получить 
\begin{equation*}
\begin{cases}
    \dot{r}=ra(r^2-2)\\
    \dot{\varphi}=1\\
        \end{cases}
\end{equation*}
Здесь уже очевидны 3 случая: 1)$a>0$, 2)$a<0$ 3) $a=0$. 1) неустойчивый предельный цикл радиуса $\sqrt 2$ 2) устойчивый 
предельный цикл радиуса $\sqrt 2$  3) центр. Разница при различных знакоопределённых параметрах будет в скорости "навивания" на предельный цикл, но характер движения будет схожий.\\
 Кстати, $\dot{\varphi}=1$ - это первый интеграл. В целом мы сказали при каких параметрах, что но можно это всё безобразие построить \ref{fig:14.12}.\textcolor[rgb]{0.480469,0.566406,0.480469}{\textit{Чтобы было максимально красиво, построим фазовую диаграмму для $r$ график в декартовых координатах и зависимость угла от радиуса. Кстати такая фигня с производной фи получилась из-за неклассической замены икса с синусом. Т.е. немного контринтуитивно что $\varphi =1$ это цикл по часовой стрелке, но и замена $x=\sin{(\varphi)}$ это что-то безумное (я просто хотел кушать а не думать)}}
\begin{figure}[h!]
\center{\includegraphics[width=1\linewidth]{14.12.png}}
\caption{Фазовые диаграммы 14.12 в различных координатах для различных параметров}
\label{fig:14.12}
\end{figure}                                                                 
\subsection{Ф.: 1149}
Решить систему уравнений:
\begin{equation}\label{f1149}
    \begin{cases}
       \dot{x}=y-x\\
       \dot{y}=x+y+z\\
       \dot{z}=x-y\\
    \end{cases}
\end{equation}
Довольно очевидно, что выделить 2 каких-то уравнения без третьей переменной тут не выйдет  \textcolor[rgb]{0.480469,0.566406,0.480469}{(\textit{ну или я слишком слаб и не могу этого сделать)}} поэтому воспользуемся правилом пропорции (чёрной магией):
\begin{equation*}
A/B=C/D=E/F=k;\forall \alpha, \beta,\gamma \in \mathbb{R}, \alpha^2+\beta^2+\gamma^2 \neq 0 \rightarrow \frac{\alpha A + \beta C + \gamma E}{\alpha B + \beta D + \gamma F}=k                                                 
\end{equation*}
Тогда можно записать учитывая $k=dt$:
\begin{equation*}
    \frac{\alpha dx + \beta dy + \gamma dz}{\alpha (y-x) + \beta (x+y+z) + \gamma (x-y)}=dt                                       
\end{equation*}      \
Возьмём $\alpha=1, \beta=0, \gamma=1$ тогда знаменатель обнулится, а значит и числитель должен быть ноль  \textcolor[rgb]{0.480469,0.566406,0.480469}{\textit{(на самом деле можно было бы сформулировать этот шаг гораздо проще, просто вычитая первое и третье уравнения друг из друга, но 1) правило пропорции весьма полезная штуковина в этих задачах, так что чего бы не сформулировать его, 2) обожаю делить на 0 эхэхэхэхэ}}\\
Итак, мы получаем что: $dx+dz=0$ а значит мы нашли первый интеграл системы \ref{f1149}: $C_1=x+z$. Подставим его во второе и первое уравнения \ref{f1149} и получим 
\begin{gather*}
    \frac{dy}{y+C_1}=\frac{dx}{y-x}\\
    y'(y-x)=(y+C_1) \\
    \text{//} y=x+t \text{//} \\
     \textcolor[rgb]{0.480469,0.566406,0.480469}{\textit{нормальная замена, не обижайте её, не будьте как Дима(}}                                               \\
    tt'=x+C_1  \\
    t^2=x^2+2C_1x+C_2 \\
    C_2=y^2-2xy+2(x+z)x
\end{gather*}
Найдены 2 ПИ. Осталось проверить их на независимость.
\begin{equation*}
    \rg \begin{vmatrix}
        \frac{\partial C_1}{\partial x} && \frac{\partial C_1}{\partial y} && \frac{\partial C_1}{\partial z}\\
        \frac{\partial C_2}{\partial x} && \frac{\partial C_2}{\partial y} && \frac{\partial C_2}{\partial z}
    \end{vmatrix}= \rg \begin{vmatrix}
        1&&0&&1\\
        -2y+4x+2z &&2y-2x && 2x
    \end{vmatrix} = 2
\end{equation*}
 \textcolor[rgb]{0.480469,0.566406,0.480469}{\textit{тут хочется сказать 2 вещи: 1) интегралов целая куча зависимых, и то что мой $C_2$ не совпадает с ответом в задачнике, это ок, потому что они друг через друга выражаются 2) так-то очевидно что они независимы, всё-таки второй зависит от $y$ а перавый - нет; но на письмаках требуют считать ранг, потому проверяю так}}\\
 Ответ: $C_2=y^2-2xy+2(x+z)x$,  $C_1=\frac{11(x+y)}{09.2002}$
\subsection{T1} 
Найти первые интегралы уравнений. Используя их, исследовать поведение траекторий на фазовой плоскости.\\
а) $\ddot x + \sin{(x)}=0$\\
Cделаем замену $\dot x = y$ тогда:
\begin{equation*}
    \begin{cases}
        \dot x=y\\
        \dot y = -\sin{(x)}
    \end{cases}
\end{equation*}
по правилу пропорции и обнуляя знаменатель:
\begin{gather*}
    \frac{\alpha dx + \beta dy}{\alpha y - \beta \sin{(x)}}=dt \\
     //\beta=y,\alpha=\sin{(x)}//\\ 
     -\cos{(x)}+\frac{y^2}{2}=C_1
\end{gather*}
Второго первого интеграла тут не будет, иначе бы задача математического маятника решалась слишком легко. Зато у нас есть интеграл энергии. Из него можно немного подумать и получить различные ситуации: $C_1=0$, $C_1>0$ , $C_1<0$.Подставив этот первый интеграл, можно сделать линеаризацию системы, получить что $(2\pi n,0)$ - центры, $(\pi (2k-1),0)$ - сёдла, и получить такое поведение: \ref{fig:Т1a}
\begin{figure}[h!]
\center{\includegraphics[width=1\linewidth]{T1a.png}}
\caption{Т1(a)}
\label{fig:Т1a}
\end{figure} \\
б) $\ddot x-x+x^2=0$ \\
Cделаем замену $\dot x = y$ тогда:
\begin{equation*}
    \begin{cases}
        \dot x=y\\
        \dot y = x-x^2
    \end{cases}
\end{equation*}
по правилу пропорции и обнуляя знаменатель:
\begin{gather*}
    \frac{\alpha dx + \beta dy}{\alpha y + \beta (x-x^2)}=dt \\
     //\beta=y,\alpha=-(x-x^2)//  \\ 
     -3x^2+2x^3+3y^2=C_1
\end{gather*}
Можно построить эту петельку: \ref{fig:Т1b}
\begin{figure}[h!]
\center{\includegraphics[width=1\linewidth]{T1b.png}}
\caption{Т1(б)}
\label{fig:Т1b}
\end{figure} \\

\subsection{C. \S16: 5}
Найдя первый интеграл, решить систему в указанной области
\begin{equation}\label{16.5}
    \begin{cases}
        \dot x = - \frac{x}{y},\\
        \dot y = \frac{y}{x}, (x>0,y>0).\\
    \end{cases}
\end{equation}
Поскольку $dx \frac{y}{x}+dy \frac{x}{y}=0$:
\begin{gather*}
    \frac{dy}{y^2}=-\frac{dx}{x^2}\\
    \frac{1}{y}=-\frac{1}{x}+C_1\\
    y = \frac{x}{C_1 x - 1}
\end{gather*}

Подставим $y$ в первое уравнение \ref{16.5}:
\begin{gather*}
    \dot x = 1-C_1 x\\
    \frac{dx}{1-C_1x}=dt\\
    x=\frac{C_2}{C_1}e^{-C_1t}+\frac{1}{C_1}
\end{gather*}
Подставим $x$ в $y$ и получаем ответ:\\
Ответ: $x=\frac{C_2}{C_1}e^{-C_1t} + \frac{1}{C_1}$, $y=  \frac{e^{C_1t}}{C_1 C_2} + \frac{1}{C_1} $
 \textcolor[rgb]{0.480469,0.566406,0.480469}{\textit{ответ не сходится с ответом в учебнике в силу разных обозначений $C_2$. Так-то ответ мой правильный (ответ учебника я в вольфраме не проверял)}}                                               

\subsection{C. \S16: 26}   
Найдя два независимых первых интеграла системы, решить систему в указанной области.
\begin{equation}\label{16.26}
    \begin{cases}
        \dot x = x^2,\\
        \dot y = 2x^3-xy-z,\\
        \dot z = xz - 2x^4, (x>0)
    \end{cases}
\end{equation}
Хмм, кажется что первое уравнение системы интегрируется. Ну раз так, проинтегрируем: $x=\frac{1}{C_1-t}$. Далее смотрим на оставшиеся 2 уравнения. Возьмём то в котором кроме $x$ не более 1 другой переменной. Т.е. третье. Подставив $x$ можем получить:
\begin{gather*}
    \dot z = \frac{z}{C_1-t} -2 \frac{1}{(C_1-t)^4}\\
    // \tau = C_1-t, \dot z = -\frac{dz}{d \tau}=-z'  // \\
    z'+\frac{z}{\tau}=2 \frac{1}{\tau^4}\\
    // \text{О, это же уравнение Эйлера, его мы умеем решать заменой } \tau=e^T, z'_\tau=z'_T e^{-T}// \\
    z'+z=2e^{-3T}\\
    //\text{находит общее, угадываем частное, благо тут оно очевидное и получаем ответ}// \\
    z(T)=C_2 e^{-T}-e^{-3T}=z(\tau)=\frac{C_2}{\tau}-\frac{1}{\tau^3}=z(t)=\frac{C_2}{C_1-t}-\frac{1}{(C_1-t)^3}
\end{gather*}
Теперь подставим $x(t)$ и $z(t)$ во второе уравнение системы \ref{16.26}:
\begin{gather*}
    \dot y = \frac{3}{(C_1-t)^3}-\frac{C_2}{(C_1-t)}-\frac{y}{C_1-t}\\
    //C_1-t=e^{T}, \dot y = y'_T e^{-T}//\\
    -y' e^{-T} = 3e^{-3T}-C_2 e^{-T}-y e^{-T}\\
    y'-y=C_2-3e^{-2T}\\
    y(T)=C_3 e^T-C_2+e^{-2 T}=y(t)=C_3(C_1-t)-C_2+\frac{1}{(C_1-t)^3}
\end{gather*}
И чтобы посмотреть на этот ужас скопанованно:\\
Ответ: $x(t)=\frac{1}{C_1-t}$,\\
        $y(t)=C_3(C_1-t)-C_2+\frac{1}{(C_1-t)^3}$,\\
        $z(t)=\frac{C_2}{C_1-t}-\frac{1}{(C_1-t)^3}$.\\
На самом деле надо ещё показать что первые интегралы $C_1,C_2$ независимы. Ну если очень хочется можно их выразить, явно посчитать ранг и т.д. Но я воспользуюсь следующим утверждением: поскольку система разрешима (мы смогли решить 2 уравнение) с использованием этих первых 2 интегралов, то они независимы.
 \textcolor[rgb]{0.480469,0.566406,0.480469}{\textit{Пока эти первые интегралы больше используются, как окнстанты. И мало смысла смотреть на них как-то иначе. В следующей части задания это будет не совсем так.}}


\section{II. Линейные однородные уравнения в частных производных первого порядка}
\subsection{C. \S17: 5}
Найти общее решение уравнения и решить задачу Коши с указанным
начальным условием
\begin{equation}\label{17.5}
x \frac{\partial u}{\partial x}+y \frac{\partial u}{\partial y}+z^{2}(x-3 y) \frac{\partial u}{\partial z}=0, u=\frac{x^{2}}{y} \text { при } 3 y z=1
\end{equation}
Найдём характерестическую систему уравнения \ref{17.5}.
\begin{equation*}
    \begin{cases}
        \dot x=x, (1)\\
        \dot y = y, (2)\\
        \dot z = z^2(x-3y),(3)\\
    \end{cases}
\end{equation*}
Для решения надо найти 2 независимых первых интеграла. В данной задаче можно просто составить 2 линейные комбинации: 
\begin{align*}
    (1) \cdot y-(2) \cdot x=0 && (1)\cdot z^2-(2)\cdot 3z^2 -(3)=0\\
    y \cdot dx  = x \cdot dy &&   z^2 \cdot dx - 3z^2 \cdot dy -dz=0\\
    \frac{dx}{x}=\frac{dy}{y} && dx - 3dy - \frac{dz}{z^2}=0\\
    \ln{\frac{x}{y}}=C_1(x,y,z) && x-3y+\frac{1}{z}=C_2(x,y,z)\\
\end{align*}
Вообще должно быть очевидно, что $C_1$ и $C_2$ независимы, потому что $C_2$ зависит от $z$ а $C_1$ - нет. Но давайте проверим ранг.  \textcolor[rgb]{0.480469,0.566406,0.480469}{\textit{просто это-то тут очевидно, а если наткнуться на задачу где не очевидно, а матаппарат не будет отработан будет печально}}\\
\begin{gather*}
\rg \begin{vmatrix}
        \frac{\partial C_1}{\partial x} && \frac{\partial C_1}{\partial y} && \frac{\partial C_1}{\partial z}\\
        \frac{\partial C_2}{\partial x} && \frac{\partial C_2}{\partial y} && \frac{\partial C_2}{\partial z}
    \end{vmatrix} = 
     \rg \begin{vmatrix}
        \frac{1}{x}&& -\frac{1}{x } && 0\\
        1 &&-3 && -\frac{1}{z^2}
    \end{vmatrix} = 2                                                   
\end{gather*}                                               
Значит решением \ref{17.5} будет $F[U_1(x,y,z),U_2(x,y,z)]$ где $F$ - любая непрерывно дифференцируемая функция, $U_1(x,y,z)=\frac{x}{y}, U_2(x,y,z)=x-3y+\frac{1}{z}$. Здесь я выкинул логарифм, потому что если логарифм отношения первый интеграл, то и без логарифма первый интеграл. Вообще это надо было сделать раньше, но я сделал тут :)\\
Решим ЗК.
\begin{gather*}
\begin{cases}
        u=\frac{x^2}{y}\\
        3yz=1\\
        \frac{x}{y}=U_1\\
        x-3y+\frac{1}{z}=U_2\\
    \end{cases}    \\
    \begin{cases}
    u=\frac{x^2}{y}\\
        3yz=1\\
        \frac{x}{y}=U_1\\
        x=U_2\\
    \end{cases}    \\
    u=U_1 U_2= \frac{x}{y}(x-3y+\frac{1}{z})
\end{gather*}
Ответ:\\
 Oбщее решение \ref{17.5}: $F[U_1(x,y,z),U_2(x,y,z)]$ где $F$ - любая непрерывно дифференцируемая функция, $U_1(x,y,z)=\frac{x}{y}, U_2(x,y,z)=x-3y+\frac{1}{z}$.\\
 Pешение ЗК \ref{17.5}: $u=U_1 U_2= \frac{x}{y}(x-3y+\frac{1}{z})$

\subsection{C. \S17: 16}
Найти общее решение уравнения и решить задачу Коши с указанным
начальным условием
\begin{equation}\label{17.16}
(z-x+3 y) \frac{\partial u}{\partial x}+(z+x-3 y) \frac{\partial u}{\partial y}-2 z \frac{\partial u}{\partial z}=0, u=\frac{4 y}{z} \text { при } x=3 y \text { . }
\end{equation}
\textcolor[rgb]{1,1,1}{\text{белый текст}                                            }
Найдём характерестическую систему уравнения \ref{17.16}.
\begin{equation*}
    \begin{cases}
        \dot x = z-x+3y, (1)\\
        \dot y = z+x-3y, (2)\\
        \dot z = -2z,(3)\\
    \end{cases}
\end{equation*}
Тут линейная комбинация только одна (ну или я слаб) $(1) + (2) + (3)=0 \Rightarrow dx+dy+dz=0 \Rightarrow x+y+z = C_1$. Подставим этот первый интеграл во вторые 2 уравнения:
\begin{gather*}
    \begin{cases}
        \dot y = C_1-4y\\
        \dot z =-2z\\
    \end{cases}\\
//\text{перемножая крест накрест получим}//\\
-2z \dot y = \dot z(C_1-4y)\\
\frac{dz}{2z}=\frac{dy}{C_1-4y}\\
C_2=\frac{z^2}{C_1-4y}= \frac{z^2}{x+z-3y}
\end{gather*}
Проверим независимость: 
\begin{equation*}
    \rg \begin{vmatrix}
        \frac{\partial C_1}{\partial x} && \frac{\partial C_1}{\partial y} && \frac{\partial C_1}{\partial z}\\
        \frac{\partial C_2}{\partial x} && \frac{\partial C_2}{\partial y} && \frac{\partial C_2}{\partial z}
    \end{vmatrix} = ... = 2
\end{equation*}
Осталось только решить ЗК.
\begin{gather*}
\begin{cases}
        u=\frac{4y}{z}\\
        x=3y\\
        x+y+z=U_1\\
        \frac{z^2}{x+z-3y}=U_2\\
    \end{cases}    \\
    \begin{cases}
    u=\frac{4y}{z}\\
        x=3y\\
        4y+z=U_1\\
        z=U_2\\
    \end{cases}    \\
    u=U_1/ U_2-1 = \frac{x}{y}(x-3y+\frac{1}{z})=\frac{(x+y+z)(x-3 y+z)}{z^{2}}-1
\end{gather*}
Ответ:\\
 Oбщее решение \ref{17.16}: $F[U_1(x,y,z),U_2(x,y,z)]$ где $F$ - любая непрерывно дифференцируемая функция, $U_1(x,y,z)=x+y+z, U_2(x,y,z)=\frac{z^2}{x+z-3y}$.\\
 Pешение ЗК \ref{17.16}: $u=u=U_1/ U_2-1 = \frac{x}{y}(x-3y+\frac{1}{z})=\frac{(x+y+z)(x-3 y+z)}{z^{2}}-1$


\subsection{C. \S17: 22}
Найти общее решение уравнения и решить задачу Коши с указанным
начальным условием
\begin{equation}\label{17.22}
\begin{array}{l}
\left(2 x^{2} z^{2}+x\right) \frac{\partial u}{\partial x}-\left(4 x y z^{2}-y\right) \frac{\partial u}{\partial y}-\left(4 x z^{3}-z\right) \frac{\partial u}{\partial z}=0, u=y z^{2} \text { при } x=z
\end{array}
\end{equation}

найдём характерестическую систему уравнения \ref{17.22}.
\begin{equation*}
    \begin{cases}
        \dot x =\left(2 x^{2} z^{2}+x\right) \\
        \dot y = -\left(4 x y z^{2}-y\right)\\
        \dot z =-\left(4 x z^{3}-z\right) \\
    \end{cases}
\end{equation*}
Найдём первые интегралы:
$z \dot y -  y\dot z=0 \Rightarrow z/y=C_1$\\
Cо вторым интегралом немного больнее. Придётся вспомнить уравнения в полных дифференциалах. Второе и третье уже связаны первым интегралом, так что берём первое и 3 (как раз там нет $y$). Перемножим их крест-накрест.
\begin{gather*}
    dx(4xz^3-z)+dz(2x^2z^2+x)=0\\
    //\text{как мы видим, это пока ещё не полный дифференциал. Попробуем домножить на} \mu(z) \text{оно зависит только от z!} //\\
    dx \underbrace{\mu(4xz^3-z)}_{=R} + dz \underbrace{\mu (2x^2z^2+x)}_{=S} = 0 \stackrel{?}{=} d(Q)\\
    //\text{чтобы левая часть была равна дифференциалу чего-то, должно выполняться} \frac{\partial R}{\partial z} = \frac{\partial S}{\partial x}  //\\
    \mu'_z(4xz^3-z)+\mu(12xz^2-1)=\mu(4xz^2+1)\\
    \mu'_z=-\mu \frac{2}{z}\\    
    \frac{d \mu}{\mu}=-\frac{2 dz}{z}\\
    \mu = \frac{1}{z^2}, \\
    //\text{Теперь можно подставить интегрирующий множитель, найти величину, полный дифференциал который ноль}//\\
    dx \underbrace{(4xz-1/z)}_{=R} + dz \underbrace{(2x^2+\frac{x}{z^2})}_{=S} = d(Q)=0\\
    //\text{Тут уже тривиально угадать функцию}//\\
    Q=2x^2z- \frac{x}{z} = \text{const} = C_2
\end{gather*}

Проверим независимость ($C_1$ зависит от $z,y$, $C_2$ зависит от $z,x$. Ну очевидно они независимы): 
\begin{equation*}
    \rg \begin{vmatrix}
        \frac{\partial C_1}{\partial x} && \frac{\partial C_1}{\partial y} && \frac{\partial C_1}{\partial z}\\
        \frac{\partial C_2}{\partial x} && \frac{\partial C_2}{\partial y} && \frac{\partial C_2}{\partial z}
    \end{vmatrix} = \text{очевидно...} = 2
\end{equation*}
Осталось только решить ЗК.
\begin{gather*}
\begin{cases}
        u=yz^2\\
        x=z\\
        U_1=z/y\\
        U_2=2x^2z- \frac{x}{z}\\
    \end{cases}    \\
    \begin{cases}
        u=\frac{4y}{z}\\
         x=z\\
        U_1=z/y\\
        U_2=2z^3 - 1\\
    \end{cases}    \\
    u=\frac{U_2+1}{2U_1} = \frac{y\left(z-x+2 x^{2} z^{2}\right)}{2 z^{2}}
\end{gather*}
Ответ:\\
 Oбщее решение \ref{17.22}: $F[U_1(x,y,z),U_2(x,y,z)]$ где $F$ - любая непрерывно дифференцируемая функция, $U_1(x,y,z)=z/y, U_2(x,y,z)=2x^2z- \frac{x}{z}$.\\
 Pешение ЗК \ref{17.22}: $u=\frac{y\left(z-x+2 x^{2} z^{2}\right)}{2 z^{2}}$

\subsection{C. \S17: 79}
Найти общее решение уравнения и решить задачу Коши с указанным
начальным условием
\begin{equation}\label{17.79}
2 x y \frac{\partial u}{\partial x}+\left(1-y^{2}-2 x z\right) \frac{\partial u}{\partial y}-\frac{y}{x} \frac{\partial u}{\partial z}=0, u=\frac{1}{2}-y^{2} \text { при } y^{2}+x z=1
\end{equation}

найдём характерестическую систему уравнения \ref{17.79}.
\begin{equation*}
    \begin{cases}
        \dot x = 2xy\\
        \dot y = 1-y^{2}-2 x z\\
        \dot z = -\frac{y}{x} \\
    \end{cases}
\end{equation*}
Найдём первые интегралы:
Перемножим крест накрест 1 и 3 уравнения. 
\begin{gather*}
    -dx \frac{y}{x}=2dz \cdot xy\\
    \frac{dx}{x^2}=-2 dz\\
    \frac{1}{x} - 2z = C_1
\end{gather*}
Второй, как и всегда, искать менее тривиально.
Возьмём 1 и 2 уравнения и перемножим крест-накрест. Получится:
\begin{gather*}
    2xy \cdot dy = (1-y^2-2xz) dx\\
    2xy y'=(C_1x-y^2)dx\\
    y'+y \frac{1}{2x}= \frac{C_1}{2y}\\
    //\text{О, это же уравнение Бернулли. } t=y^2,t'=2y y'  //\\
    \frac{t'}{2 \sqrt t}+\sqrt t \frac{1}{2x} = \frac{C_1}{2 \sqrt t}\\
    t'+t \frac{1}{x}= C_1\\
    t'x+t \cdot x' = C_1 x\\
    (t\cdot x)'=C_1x\\
    t x = \frac{C_1 x^2}{2} + C_2\\
    C_2 = xy^2- \frac{C_1 x^2}{2}= x(y^2+z^2)-\frac{x}{2}
\end{gather*}
Проверим независимость: Один ПИ зависит от $x$ другой нет, очевидно они независимы.
\begin{equation*}
    \rg \begin{vmatrix}
        \frac{\partial C_1}{\partial x} && \frac{\partial C_1}{\partial y} && \frac{\partial C_1}{\partial z}\\
        \frac{\partial C_2}{\partial x} && \frac{\partial C_2}{\partial y} && \frac{\partial C_2}{\partial z}
    \end{vmatrix} = ... = 2
\end{equation*}
Осталось только решить ЗК.
\begin{gather*}
\begin{cases}
        u= \frac{1}{2} - y^2\\
        y^2+xz=1\\
        U_1=\frac{1}{x} - 2z\\
        U_2= x(y^2+z^2)-\frac{x}{2}\\
    \end{cases} \\
    //\text{я считаю, что это какая-то ужасная бяка, поэтому просто переберём варианты } U_1 \cdot U_2, U_1/U_2 \text{ и т.д.}//\\
    u=U_1 \cdot U_2 = (2 x z-1)\left(y^{2}-\frac{1}{2}+x z\right)
\end{gather*}
Ответ:\\
 Oбщее решение \ref{17.79}: $F[U_1(x,y,z),U_2(x,y,z)]$ где $F$ - любая непрерывно дифференцируемая функция, $U_1(x,y,z)=\frac{1}{x} - 2z, U_2(x,y,z)= x(y^2+z^2)-\frac{x}{2}$.\\
 Pешение ЗК \ref{17.79}: $u=U_1 \cdot U_2 = (2 x z-1)\left(y^{2}-\frac{1}{2}+x z\right)$
\subsection{C. \S17: 83}
Найти общее решение уравнения и решить задачу Коши с указанным
начальным условием
\begin{equation}\label{17.83}
2 x y \frac{\partial u}{\partial x}+\left(2 x-y^{2}\right) \frac{\partial u}{\partial y}+y^{3} z \frac{\partial u}{\partial z}=0, u=x^{2} z^{2} \text { при } y^{2}=2 x
\end{equation}
найдём характерестическую систему уравнения \ref{17.83}.
\begin{equation*}
    \begin{cases}
        \dot x = 2xy \\
        \dot y = 2 x-y^{2}\\
        \dot z = y^{3} z \\
    \end{cases}
\end{equation*}
Найдём первые интегралы: первые 2 уравнения крест-накрест
\begin{gather*}
    dx(2x-y^2)=dy(2xy)\\
    dx(2x-y^2)+dy(-2xy)=0\\
    d(x^2-xy^2)=0\\
    x^2-xy^2=C_1
\end{gather*}
 \textcolor[rgb]{0.480469,0.566406,0.480469}{\textit{первые интегралы всё более стрёмные. Даже самый первый становится какой-то жестью}} 
$y^2=x- \frac{C_1}{x}$\\
Перемножим 1 и 3 крест-накрест:
\begin{gather*}
    dx(y^3z)=dz(2xy)\\
    z'=\frac{y^2z}{2x}\\
    \frac{dz}{z}=\frac{1}{2}(1- \frac{C_1}{x^2})\\
    \ln (z^2) = x + \frac{C_1}{x} + C_2\\
    C_2 = \ln (z^2) - 2x + y^2. 
\end{gather*}


Проверим независимость: как всегда очевидно независят, но проверить формально надо
\begin{equation*}
    \rg \begin{vmatrix}
        \frac{\partial C_1}{\partial x} && \frac{\partial C_1}{\partial y} && \frac{\partial C_1}{\partial z}\\
        \frac{\partial C_2}{\partial x} && \frac{\partial C_2}{\partial y} && \frac{\partial C_2}{\partial z}
    \end{vmatrix} = ... = 2
\end{equation*}
Осталось только решить ЗК.
\begin{gather*}
\begin{cases}
        u=x^2z^2\\
        y^2=2x\\
        U_1=x^2-xy^2\\
        U_2=\ln (z^2) - 2x + y^2\\
    \end{cases}    \\
    \begin{cases}
        u=x^2z^2\\
        y^2=2x\\
        U_1=-x^2\\
        e^{U_2}=z^2\\
    \end{cases}    \\
    u=-U_1e^{U_2}=(xy^2 - x^2) e^{\ln (z^2) - 2x + y^2} = z^2(xy^2 - x^2) e^{- 2x + y^2}
\end{gather*}
Ответ:\\
 Oбщее решение \ref{17.83}: $F[U_1(x,y,z),U_2(x,y,z)]$ где $F$ - любая непрерывно дифференцируемая функция, $U_1(x,y,z)=x^2-xy^2, U_2(x,y,z)=\ln (z^2) - 2x + y^2$.\\
 Pешение ЗК \ref{17.83}: $u=z^2(xy^2 - x^2) e^{- 2x + y^2}$



\subsection{T2}
В области $x>0, \quad y>0, \quad z>0$ найти все решения уравнения
\begin{equation}\label{17.T2}
\left(x^{2}+y^{2}\right) \frac{\partial u}{\partial x}+2 x y \frac{\partial u}{\partial y}+\frac{x^{3}-x y^{2}}{z} \frac{\partial u}{\partial z}=0
\end{equation}
и решить задачу Коши $u=z^{2}$ при $y^{2}-x^{2}=1$\\

найдём характерестическую систему уравнения \ref{17.T2}.
\begin{equation*}
    \begin{cases}
        \dot x =x^2+y^2 \\
        \dot y = 2xy\\
        \dot z = \frac{x^3-xy^2}{z}\\
    \end{cases}
\end{equation*}
%Найдём первые интегралы: крест-накрест первые 2
%\begin{gather*}
 %   dx \cdot 2xy = dy \cdot (x^2+y^2)\\
  %  y'=\frac{2xy}{x^2+y^2}\\
   % //y=x \cdot t(x), y'=t(x)+xt'(x)//\\
    %t'x+t=\frac{2t}{1+t^2}\\
    %dt \frac{1+t^2}{(t-1)^2}=-\frac{dx}{x}\\
    %t-\frac{2}{t-1}+2 \ln (t-1) = -\ln(x)+C_1\\
    %C_1 = \frac{2 x}{x-y}+\frac{y}{x}+2 \log \left(\frac{y}{x}-1\right)+\log (x)\\
%\end{gather*}
% \textcolor[rgb]{0.480469,0.566406,0.480469}{\textit{Какой хороший интеграл, настолько хороший, что я лучше найду какой-нибудь ещё}}%      
 \textcolor[rgb]{0.480469,0.566406,0.480469}{\textit{Тут есть такой красивый первый интеграл! Он закомментирован но если кто хочет посмотреть на наркоманию, можно посмотреть исходный код}}                                                                                          
 Тут можно заметить линейную комбинацию: $x dx  - y dy - z dz =0 \Rightarrow C_1 =  x^2-y^2-z^2$. Далее выражаем $y=x^2-z^2-C_1$ и берём первые 2 уравнения.
 \begin{gather*}
     dz(2x^2-z^2-C_1)-dx\left(xz+C_1 \frac{x}{z} \right)=0\\
     //\text{Это почти уравнение в полных дифференциалах, надо всего-то домножить на } \frac{z}{(C_1+z^2)^3}//\\
     dz \frac{2zx-z^3-C_1z}{(C_1+z^2)^3} - dx \frac{x}{(C_1+z^2)^2}=d(C_2)=0\\
     C_2=\frac{1}{2 \left(C_1+z^2\right)}-\frac{x^2}{2 \left(C_1+z^2\right)^2} = \frac{1}{2x^2-2y^2} - \frac{x^2}{2 \left(x^2-y^2\right)^2}
 \end{gather*}



Проверим независимость: о чудо, снова один интеграл зависит от $z$, а другой нет. Так что снова всё ок, но проверить тем не менее необходимо...
\begin{equation*}
    \rg \begin{vmatrix}
        \frac{\partial C_1}{\partial x} && \frac{\partial C_1}{\partial y} && \frac{\partial C_1}{\partial z}\\
        \frac{\partial C_2}{\partial x} && \frac{\partial C_2}{\partial y} && \frac{\partial C_2}{\partial z}
    \end{vmatrix} = ... = 2
\end{equation*}
Осталось только решить ЗК.
\begin{gather*}
\begin{cases}
        u=z^2\\
        y^2-x^2=1\\
        U_1= x^2-y^2-z^2\\
        U_2=\frac{1}{2 \left(C_1+z^2\right)}-\frac{x^2}{2 \left(C_1+z^2\right)^2} = \frac{1}{2x^2-2y^2} - \frac{x^2}{2 \left(x^2-y^2\right)^2}\\
    \end{cases}    \\
    \begin{cases}
        u=z^2\\
        y^2-x^2=1\\
        U_1=-z^2-1\\
        U_2=- \frac{1}{2} - \frac{x^2}{2}\\
    \end{cases}    \\
    u=-1-U_1=z^2+y^2-x^2-1
\end{gather*}
Ответ:\\
 Oбщее решение \ref{17.T2}: $F[U_1(x,y,z),U_2(x,y,z)]$ где $F$ - любая непрерывно дифференцируемая функция, $U_1(x,y,z)=x^2-y^2-z^2, U_2(x,y,z)= \frac{1}{2x^2-2y^2} - \frac{x^2}{2 \left(x^2-y^2\right)^2}$.\\
 Pешение ЗК \ref{17.T2}: $u= u=-1-U_1=z^2+y^2-x^2-1$ 
  \textcolor[rgb]{0.480469,0.566406,0.480469}{\textit{Вот и всё с первыми интегралами. Сказал бы всё плохое что я про эти задачи думаю, но я думаю о них только положительно. Какие хорошие задачи}}                                               

\section{III. Вариационное исчисление}
\subsection{C. \S19: 21}
\subsection{C. \S19: 45}
\subsection{C. \S19: 72}
\subsection{C. \S19: 105}
\subsection{T3}
\subsection{C. \S20.1: 9}
\subsection{C. \S20:}
\subsection{T4}
\subsection{C. \S20.2: 5}
\subsection{C. \S20.3: 2}
\subsection{C. \S21: 1}
\subsection{T5*}

\end{document}
