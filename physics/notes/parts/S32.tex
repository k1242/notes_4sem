Пространственную когерентность $\gamma_{1, 2}$ для точек $Q_1$ и $Q_2$ экрана, освещаемого протяженным квазимонохроматическим самосветящимся источником света. Если рассматриваемая точка $P$ равноудалена от $Q_1$ и $Q_2$ то можем рассматривать просто волны в $Q_1$ и $Q_2$. В качетсве источника рассматривается площадка $\sigma$ $\parallel$ экрану. 


В точках $Q_1$ и $Q_2$ может быть определена интенсивность
\begin{equation*}
    I_1 \equiv I(Q_1) = \int_\sigma \frac{I(S) \d S}{r_1^2}, \hspace{5 mm} 
    I_2 \equiv I(Q_2) = \int_\sigma \frac{I(S) \d S}{r_2^2}.
\end{equation*}
Введя нормирующий множитель, можем найти $\gamma_{1,2} (\theta \ll 1)$:
\begin{equation*}
    \gamma_{1,2} (0) = \frac{1}{\sqrt{I_1 I_2}} \int \frac{I(S)}{r_1 r_2} e^{ik (r_2-r_1)} \d S.
\end{equation*}
Таким образом мы говорим, что:

\begin{to_thr}[теорема Ван-Циттера-Цернике]
    Комплексная степень взаимной когерентности в точках $Q_1$ и $Q_2$ равна комплексной амплитуде в точке $Q_1$ соответстствующей дифрагированной волны.
\end{to_thr}