\subsection{Линейный электрооптический эффект Поккельса}


Рассмотрим \textit{ангармонический осциллятор}  при наличии внешнего постоянного электрического поля $E_0$ 
\begin{equation*}
    \ddot{r} + 2 \gamma \dot{r} + \omega_0^2 r + \beta r^2 = -\frac{e}{m}E_0,
\end{equation*}
где $\beta$ -- постоянная. Считая $r = r_0 + q$ можем перейти к уравнению с новой частотой
\begin{equation*}
    \ddot{q} + 2 \gamma \dot{q} + (\omega_0^2+ 2\beta r_0) q =0,
\end{equation*}
откужа видно изменение частоты колебания на 
\begin{equation*}
    \Delta \omega_0^2 = -\frac{2 e \beta}{m \omega_0^2} E_0^2.
\end{equation*}
Смещение собственных частот меняет кривую дисперсии, т.е. показатель преломления $n$ среды. В простейшем случае, когда $\omega_0$ одна (см. \S 84), изменение $n$ определяется выражением
\begin{equation*}
    \Delta n = \frac{\partial n}{\partial \omega_0^2} \Delta \omega_0^2 = - \frac{\partial n}{\partial \omega_0^2} 
    - \frac{\partial n}{\partial \omega_0^2} \frac{2e\beta}{m \omega_0^2} E_0 = 
    \frac{\partial n}{\partial \omega} \frac{e\beta}{m \omega \omega_0^2} E_0.
\end{equation*}
При фиксированном внешнем $\vc{E}_0$ величина $\Delta n$ зависит от направления распространения света.
Это сказывается на двойном преломлении среды. \textit{Изменеие двойного преломления вещества из-за смещения собственной частоты во внешнем электрическом поле называется электрооптическим эффектом Поккельса}.

В этом эффекте изменения пропорциональны первой степени $E_0$. \textit{Эффект Поккельса может наблюдаться только в 
кристаллах, не обладающих центром симметрии.} Устройство, основанное на эффекте Поккельса, называют \textit{ячейкой Поккельса}. 

Она представляет собой кристалл, помещаемый между двумя скрещенными николями. 
Такое устройство действует так же, как и ячейка Керра. Николи
не пропускают свет, когда нет внешнего электрического поля,
но при наложении такого поля пропускание появляется. 
Необходимо, чтобы кристалл до наложения внешнего электрического
поля не давал двойного преломления. Этого можно достигнуть,
если взять оптически одноосный кристалл, вырезанный 
перпендикулярно к оптической оси, а свет направить вдоль этой оси.
Внешнее поле Eq может быть направлено либо перпендикулярно
(поперечный модулятор света), либо параллельно 
распространению света (продольный модулятор).



