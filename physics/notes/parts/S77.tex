

\textbf{Поляризационные устройства}. Комбинация кристаллов -- поляризационная призма\footnote{
    Самая первая призма -- \textit{николь}, 1828 г.
} . Существуют \textit{однолучевые} (на полном внутренне отражении) и \textit{двулучевые}. 

\begin{to_def}
    Допустимая разность углов наклона между крайними лучами падающего на призму пучка называется \textit{апертурой полной поляризации призмы}.
\end{to_def}

\begin{to_def}
    \textit{Дихроизм} -- свойство кристаллов, состоящее в различном поглощении веществом света в зависимости от его поляризации. Всего различают: \textit{линейный дихроизм} (при $\bot$ направлениях линейной поляризации); \textit{эллиптический дихроизм} (различное поглощение для правой и левой эллиптической поляризации); \textit{круговой дихроизм} (различные направления круговой поляризации, иначе -- \textit{эффект Коттона}). 
\end{to_def}

