
Если линейно поляризованный свет проходит через плоскопараллельный слой вещества, то в некоторых случаях плоскость поляризации света оказывается повернутой относительно своего исходного положения. Это явление называется \textit{вращением плоскости поляризации} или оптической активностью. Если вещество не находится во внешнем магнитном поле, то оптическая активность и вращение плоскости поляризации называются \textit{естестыенными}. В противоположнос случае говорят о \textit{магнитном вращении плоскости поляризации}, или \textit{эффекте Фарадея}. 


Вращение против часовов -- \textit{положительное}, по часовой -- \textit{отрицательное}. Это свойство, как и в случе с шурупом, не зависит от того, в каком из двух прямо противоположных напралний распространяетя свет\footnote{
    Если свет заставить пройти туда и обратно через естественно-активное вещество, отразив его от
    зеркала, то плоскость поляризации возвратится к своему исходисходному направлению.
} . 


В области прозрачности и малого поглощения эта история хорошо согласуется с опытом формула Друде
\begin{equation*}
    \xi = \alpha L,
    \hspace{5 mm} 
    \alpha = \sum_i \frac{B_i}{\lambda^2-\lambda_i^2},
\end{equation*}
где $B_i$ -- постоянные, $\lambda_i$ -- длины волн, соответсвующие собтсвенным чатсота рассматриваемого вещества. 



По Френелю вращение плоскости поляризации -- проявление \textit{кругового двойного лучепрпеломления}. Две волны, которые могут распространятся в оптически активной среде с разными скоростями, поляризованы \textit{по кругу}: по левому и по правому.

Покажем достаточность такого предположения:
\begin{equation*}
    \left.\begin{aligned}
        E_x &= A \cos \xi \cos (\omega t - k z), \\
        E_y &= A \sin \xi \cos (\omega y - k z),
    \end{aligned}\right.
    \hspace{5 mm} 
    \xi = - \alpha z,
    \hspace{0.5cm} \Rightarrow \hspace{0.5cm}
    \left.\begin{aligned}
        E_x &= \textstyle \frac{A}{2} \cos(\omega t - k z + \alpha z) + \textstyle \frac{A}{2} \cos(\omega t - k z - \alpha z), \\
        E_y &= \textstyle \frac{A}{2} \cos(\omega t - k z + \alpha z + \pi/2) + \textstyle \frac{A}{2} \cos(\omega t - kz - \alpha z - \pi/2).
    \end{aligned}\right.
\end{equation*}
Разложим полученную волну на две: $\vc{E} = \sub{\vc{E}}{п} + \sub{\vc{E}}{л}$, где для  $ \sub{\vc{E}}{п}$ и $\sub{\vc{E}}{л}$ имеет смысл ввеси $\sub{k}{п} = k-\alpha$  и $\sub{k}{л} = k + \alpha$. Полученные волны соответствуют правой и левой круговой поляризации. Скорости этих волн определяются выражениями
\begin{equation*}
    \sub{v}{п} = \frac{\omega}{k-\alpha}, \hspace{5 mm} \sub{v}{л} = \frac{\omega}{k+\alpha},
\end{equation*}
и соответсвующие покзатели преломления $n = c/v$. Подробнее,
\begin{equation*}
    \sub{n}{r} = \frac{c}{\sub{v}{r}} = \frac{c}{\omega}(k-\alpha),
    \hspace{5 mm} 
    \sub{n}{l} = \frac{c}{\sub{v}{l}} = \frac{c}{\omega}(k+\alpha),
    \hspace{0.5cm} \Rightarrow \hspace{0.5cm}
    \alpha = \frac{\omega}{2c}(\sub{n}{l}-\sub{n}{r}).
\end{equation*}




Френель выдвинул гипотезу, что возможно независимое распространения поляризованных по кругу волн, с сохранением поляризации, которую подтвердил эксперементально. Тем самым задача объяснения вращения плоскости поляризации была сведена к задаче объяснения кругового двойного лучепреломления.

Поляризованные по кругу в противоположных направлениях
волны в окрестности полос или линий поглощения могут 
отличаться не только скоростями распространения, но и 
коэффициентами поглощения. Тогда они выйдут с различными 
амплитудами. Если падающий свет был поляризован линейно, то 
выходящий будет поляризован эллиптически. Это явление 
называется круговым дихроизмом. 