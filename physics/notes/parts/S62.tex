
\textbf{Общие слова}. 
Если взять немного йода с хинином, то получится герапатит, из которого часто делают \textit{поляроиды}. 
Рассмотрим систему из двух поляроидов: через оба поляроида пройдёт свет с электрическим вектором 
$\vc{E} \equiv \sub{\vc{E}}{$\parallel$}$, тогда интенсивность света после двух поляроидов:
\begin{equation*}
    I = I_0 \cos^2 \alpha,
\end{equation*}
что называют \textit{законом Малюса}. 

Рассмотрим два взаимноперпендикулярных колебания:
\begin{equation*}
    \left.\begin{aligned}
        x &= a \cos \omega t \\
        y &= b \cos (\omega t + \delta)
    \end{aligned}\right.    
    \hspace{0.5cm} \Rightarrow \hspace{0.5cm}
    \frac{x^2}{a^2} - 2 \frac{xy}{ab} \cos \delta + \frac{y^2}{b^2} = \sin^2 \delta.
\end{equation*}
Получили эллипс. Колебания с большей фазой -- \textit{опережающие},  с меньшей -- \textit{запаздывающие}. Соответсвенно, рассматривая скорости
\begin{equation*}
    \dot{x} = - \omega a \sin \omega t, \hspace{5 mm} 
    \dot{y} = - \omega b \sin(\omega t + \delta).
\end{equation*}
Тогда при $\delta < 0$ частица в $t=0$ движется вверх, а тогда описывает эллипс против часовой стрелки, а при $\delta > 0$ описывает эллипс по часовой стрелке. 

Для круговой поляризации можем заметить, что необходимо $\delta = \pm \pi/2$ и $a = b$, в частности можно различать правую и левую поляризацию. 

В частности, для монохроматического поля в векторной поле можем записать
\begin{align*}
    \vc{E} = \vc{A}_1(\vc{r}) \cos \omega t + \vc{A}_2 (\vc{r}) \sin \omega t,
\end{align*}
если $\vc{A}_1$ и $\vc{A}_2$ коллинеарны, то $\vc{E}$ поляризован линейной. Иначе же скажем, что $OX \parallel \vc{A}_1$ и $OY \parallel \vc{A}_2$, тогда получим
\begin{equation*}
     E_x = a_x (\vc{r}) \cos (\omega t + \delta_x), \hspace{5 mm} 
     E_y = a_y (\vc{r}) \cos (\omega t + \delta_y),
 \end{equation*} 
 что приводит как раз к сумме двух колебаний со сдвигом по фазе. 

\textit{Степень поляризации} света может быть определена, как
\begin{equation*}
    \Delta = \frac{I_{\bot}-I_{\parallel}}{I_\bot + I_\parallel}.
\end{equation*}
аналогично видности и не только. 


\textbf{Частично поляризованный свет}. Всякая монохроматическая волна по определению поляризована, обычно имеем дело с \textit{почти} монохроматическими, содержашимеся в некотором интервале $\Delta \omega$: рассмотрим такую волну с некоторой средней частотой $\omega$. Тогда поле в заданной точке пространства
\begin{equation*}
    \vc{E} = \vc{E}_0 (t) e^{-i \omega t},
\end{equation*}
где $\vc{E}_0 (t)$ -- медленно меняющаяся функция. Таким образом волна \textit{частично поляризована}. 

Наблюдаем почти всегда усредненное значение интенсивности, так что можем рассмотреть некоторый тензор
\begin{equation*}
    J_{\alpha \beta} = \overline{E_{0 \alpha} E_{0 \beta}^*},
\end{equation*}
имеющий всего четыре значения (здесь и далее считаем, что $\alpha,\, \beta = 1,\, 2$). Свёртка $J_{\alpha \beta}$  есть
\begin{equation*}
    J = J_{\alpha \alpha} = \overline{\vc{E}_0 \vc{E}_0^*},
    \hspace{5 mm} 
    \rho_{\alpha \beta} = \frac{J_{\alpha\beta}}{J},
\end{equation*}
таким образом приходим к \textit{поляризационному тензору}. Из определения понятно, что оба тензора эрмитовы, откду $\rho_{11}$ и $\rho_{22}$ вещественны, при чём $\rho_{11} + \rho_{22} = 1$, и $\rho_{21} = \rho_{12}^*$. Итого осталось три вещественных параметра. 

Пусть свет очень похож на поляризованный, тогда $J_{\alpha \beta} = J \rho_{\alpha \beta} E_{0 \alpha} E_{0 \beta}^*$, то есть компоненты некоторого постоянного вектора. Необходимо и достаточно $\det \rho_{\alpha \beta} = 0$. 

Аьтернатива: \textit{естественный} свет, тогда направления в плоскости $\bot$ распространению эквивалентны, тогда $\rho_{\alpha \beta} = \frac{1}{2} \delta_{\alpha \beta}$, тогда $\det \rho_{\alpha \beta} = 1/4$.

Итого возможны значения для $\det \rho_{\alpha \beta} \in [0, 1/4]$, соответсвенно \textit{степенью поляризации} назовём величину $P$ равную
\begin{equation*}
    P \overset{\mathrm{def}}{\colon} \ \ \det \rho_{\alpha \beta} = \frac{1}{4} \left(1-P^2\right), \hspace{5 mm} 
    P \in [0, 1]. 
\end{equation*}
Да, можно и так. 


Произвольный тензор $\rho_{\alpha \beta}$ может быть разложен на две части: симметричную и антисимметричную, из них первая $S_{\alpha \beta} = \frac{1}{2} \left(\rho_{\alpha \beta} + \rho_{\beta \alpha}\right)$ -- вещественная в силу эрмитовости $\rho_{\alpha \beta}$, антисимметричная же части чисто мнима, которая сводится к псевдоскаляру
\begin{equation*}
    \frac{1}{2}\left(\rho_{\alpha\beta} - \rho_{\beta \alpha}\right) = - \frac{i}{2} e_{\alpha \beta} A,
\end{equation*}
где $A$ -- вещественный псевдоскаляр. Итого, поляризационный тензор представим в виде
\begin{equation*}
    \rho_{\alpha \beta} = S_{\alpha \beta} - \frac{i}{2} e_{\alpha \beta} A.
\end{equation*}
Для поляризованной по кругу волны $\vc{E}_0 = \const $, тогда $A = \pm 1$, а $S_{\alpha \beta} = \frac{1}{2} \delta_{\alpha \beta}$, а вот для линейно поляризованной волны $A = 0$. Вообще $A$ -- степень круговой поляризации, $A \in [-1, + 1]$, где предельные значения -- правая и левая поляризации. 

Вектор $S_{\alpha \beta}$ может быть сведен к гланым осяи с главными значениями $\lambda_1$ и $\lambda_2$, направления которых нормальны, при чём $\lambda_1 + \lambda_2 = 1$. Тогда
\begin{equation*}
    S_{\alpha \beta} = \lambda_1 n_\alpha^{(1)} n_{\beta}^{(1)} + \lambda_2 n_\alpha^{(2)} n_\beta^{(2)}.
\end{equation*}
При $A = 0$ каждое из слагаемы -- произведения постоянного вещественного веткора $\sqrt{ \lambda_1} \vc{n}^{(1)}$ и $\sqrt{ \lambda_2} \vc{n}^{(2)}$, получается соответсвет линейно поляризованному свету, независимых \textit{некогерентных} волн. В общем же случае свет может быть представлен как наложение двух некогернтных эллиптически поляризованных волн, эллипсы которых нормальны. 


Пусть $\varphi$ -- угол между $Oy$ и $\vc{n}^{(1)}$, тогда
\begin{equation*}
    \vc{n}^{(1)} = (\cos \varphi,\, \sin \varphi)\T, \hspace{5 mm} 
    \vc{n}^{(2)} = (-\sin \varphi,\,  \cos \varphi).
\end{equation*}
Вводя величину $l = \lambda_1 - \lambda_2$ можем представить компаненты тензора в виде
\begin{equation*}
    S_{\alpha \beta} = \frac{1}{2} \begin{pmatrix}
        1 + l \cos 2 \varphi & l \sin 2 \varphi  \\
        l \sin 2 \varphi & 1 - l \cos 2 \varphi  \\
    \end{pmatrix}.
\end{equation*}
Итого можем в качетсве параметров выбрать $A$ -- степень круговой поляризации, $l$ -- степень максимальной линейной поляризации, $\varphi$ -- угол межу $\vc{n}^{(1)}$ и осью $Oy$. 

Но можно выбрать параметры удобнее:
\begin{equation*}
    \xi_1 = l \sin 2 \varphi,
    \hspace{5 mm} 
    \xi_2 = A,
    \hspace{5 mm} 
    \xi_3 = l \cos 2 \varphi,
    \hspace{5 mm} 
    \text{---\  \textit{параметры Стокса}}.
\end{equation*}
Тогда поляризационный тензор примет вид
\begin{equation*}
    \rho_{\alpha \beta} = \frac{1}{2} \begin{pmatrix}
        1+\xi_3 & \xi_1 - i \xi_2  \\
        \xi_1 + i \xi_2 & 1-\xi_3  \\
    \end{pmatrix},
    \hspace{5 mm} 
    \xi_1, \xi_2, \xi_3 \in [-1, +1].
\end{equation*}
где $\xi_3$ -- линейная поляризации вдоль $y$ и $z$, $\xi_1$ -- поляризации вдоль направлений под углом $\pi/4$, $\xi_2 = A$. Определитель 
\begin{equation*}
    \det \rho_{\alpha \beta} = \frac{1}{4}\left(1 - \xi_1^2 - \xi_2^2 - \xi_3^2\right),
    \hspace{5 mm} 
    P = \sqrt{\xi_1^2 + \xi_2^2 + \xi_3^2},
\end{equation*}
что очень удобно. Кстати, $\xi_2 = A$ и $\sqrt{\xi_1^2 + \xi_3^2} = l= \lambda_1 - \lambda_2$ -- инваринантны относительно преобразований Лоренца, что также очень логично. 

