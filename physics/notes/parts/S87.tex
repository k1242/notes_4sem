\subsection{Дисперсия плазмы}


\textbf{Плазменная частота}. Можно написать штуку из \S 84, но ненужно. 
Если атомы колеблются на $\omega_0$, частоты волны $\omega$, концентрация атомов в единице объема $N$, масса атомов $m$, $\gamma$ -- ну, какая-то гамма для затухания, то можем прийти к формуле
\begin{equation*}
    \varepsilon = 1 + \frac{4 \pi N e^2/m}{\omega_0^2 - \omega^2 + 2 i w \gamma},
\end{equation*}
где последнее слагаемое как раз характеризует поляризуемость атома. 
На самом деле достаточно рассмотреть а-ля конденсатор в плазме, как в прошлом семестре. 

Диэлектрическая проницаемость плазмы определяется в основном \textit{свободными электронами}. Полагая $\omega_0 = 0$, пренебрегая затуханием ($\gamma=0$), получаем для плазмы
\begin{equation*}
    \varepsilon = 1 - \left(\frac{\omega_p}{\omega}\right)^2,
    \hspace{5 mm} 
    \omega_p^2 = 4 \pi N\frac{e^2}{m},
\end{equation*}
где $N$ -- концентрация свободных электронов. Величина $\omega_p$ -- \textit{плазменная}, или \textit{ленгмюровская} частота, играющая для плазмы роль \textit{собственной частоты}. 



При $\omega < \omega_p$ $\varepsilon < 0$, так что длинные эм волны отражаются от плазмы. Эксплуатируя этот эффект можем реализовать дальнюю радиосвязь: на земле $N$ меняется с высотой неравномерно, есть несколько максимумов, и область с одним из таких максимумов и называется ионосферным слоем. Характерные значения для $N \in [10^4; 10^6]$ электронов на $1$ см$^3$. 


\textbf{Фазовая и групповая скорость}. Для волнового числа можем записать
\begin{equation*}
    c^2 k^2 = \omega^2 \varepsilon = \omega^2 - \omega^2_p,
    \hspace{0.5cm} \Rightarrow \hspace{0.5cm}   
    c^2 k \d k = \omega \d \omega,
    \hspace{0.5cm} \Rightarrow \hspace{0.5cm}
    \frac{\omega}{k} \frac{d \omega}{d k}  = c^2,
\end{equation*}
иначе можем записать, как
\begin{equation*}
    v u = c^2, \hspace{5 mm} 
    v = \frac{c}{\sqrt{\varepsilon}} = \frac{c}{\sqrt{1-\omega_p^2/\omega^2}} > c,
    \hspace{5 mm} 
    u = \frac{c^2}{v} = c \sqrt{1- \frac{\omega_p^2}{\omega^2}} < c.
\end{equation*}
Соответственно, некоторый важный итог:
\begin{equation*}
    n = \frac{c}{v} = \sqrt{1- \frac{\omega_p^2}{\omega}}, \hspace{5 mm} 
    \omega_p^2 = 4 \pi N \frac{e^2}{m}.
\end{equation*}