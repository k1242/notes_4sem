\subsection{Комбинационное рассеяние света (эффект Рамана)}


\begin{to_def}
    \textit{Сателлит} -- система линий измененной частоты в рассеянном свете, сопровождающих падающий свет.
\end{to_def}

Изменение длины волны оказывается значительно больше, чем при рассеянии Мандельштама.  Это явление называется \textit{комбинационным рассеянием света} или \textit{эффектом Рамана}. 

\textbf{Экспериментальные ниблюдения}.
Далее будем считать, что частоты сетеллитов отличаются от возбуждающей линии на $\{\Delta \sub{\omega}{комб}^j\}$. При переходе от одной спектральной линии первичного пучка к другой $\{\Delta \sub{\omega}{комб}^j\}$ \textit{сохраняется}, -- она характерна для рассматриваемого вещества. 

Каждому сателлиту с частотой $\omega - \Delta \sub{\omega}{комб}$, смещенной в красную сторону спектра, соответствует сателлит с частотой $\omega + \Delta \sub{\omega}{комб}$, смещенный в фиолетовую сторону. Первые называют \textit{красными} или \textit{стоксовыми}, вторые -- \textit{фиолетовыми} или \textit{антистоксовыми}. 

Интенсивности фиолетовых сателлитов значительно меньше интенсивностей, соответствующих им красных. Постоянные $\Delta \sub{\omega}{комб}$ совпадают с собственными частотами $\Omega_{\text{инфр}}$ инфракрасных колебаний. 

Линии комбинационного рассеяния света более или менее \textit{поляризованы}. Характер поляризации красных и фиолетовых сателлитов, соотвествующих $\Delta \sub{\omega}{комб}$, всегда одинаков и не зависит от частоты основной линии. 



\textbf{Теоретическое объяснение}. В поле световой волны $\vc{E}$ электроны внутри молекулы приходят в колебания, и молекула приобретает индуцированный диполный момент $\vc{p} = \beta \vc{E}$. Вообще $\beta$ -- тензор, определяемый мгновенным положением атомынх ядер, но сами ядра совершают беспорядочное тепловое движение, $\Rightarrow \beta \neq \const $, в частности представима наложением гармноических колебаний, частоты которых определяются собственными частотами инфракрасных колебаний молекулы, возникает \textit{модуляция колебаний} индуцированных моментов $\vc{p}$. 

Если внешнее поле $\vc{E}$ меняется с частотой $\omega$, то в колебаниях дипольного момента $\vc{p}$ появляются частоты $\omega \pm \sub{\Omega}{инфр}$, такие же частоты появятся ив излучении дипольных моментов. 


Математически, скажем что у молекула $f = 3 s - 6$ степеней свободы на \textit{внутреннее движение ядер молекул}. Выберем нормальные обобщенные координаты для описания $q_j$, и пусть $q_j = a_j \cos(\Omega_j t + \delta_j)$ с инфракрасной частотой $\Omega_j$ и хаотически меняющейся фазой $\delta_j$. Пусть $\beta$ для простоты скаляр:
\begin{equation*}
    \beta = \beta_0 + (\partial_{q_m} \beta) q^m = 
    \beta_0 +  \textstyle \frac{1}{2} a_m (\partial_{q_m} \beta) \cdot \left(
        e^{i(\Omega_m t + \delta_m} + e^{-i(\Omega_m t + \delta_m}
    \right).
\end{equation*}
Наконец, подставляя $\vc{E} = \vc{E}_0 e^{i \omega t}$, находим
\begin{equation*}
    \vc{p} = \beta_0 \vc{E}_0 e^{i\omega t} + \textstyle \frac{\smallvc{E}_0}{2} a_m (\partial_{q_m} \beta) \left(
        e^{i(\omega + \Omega_m)t + \delta_m} + e^{i(\omega - \Omega_m)t - \delta_m}
    \right),
\end{equation*}
откуда видно возникновение дуплетов в излучении, а также ясно, что волны, рассеиваемые отдельными молекулами, \textit{некогерентны}. 




\textbf{Проявление квантмеха}.
До тех пора, пока атомы тяжелые, классическая теоря $\pm$ справляется. но только через кванты получается показать, что интенсивность красных сателлитов всегда больше интенсивности соответсвенных фиолетовых сателлитов. 

\red{Пропуская выкладки, можем получить, что} для отношени интенсивностей верно, что
\begin{equation*}
    \frac{\sub{I}{кр}}{\sub{I}{фиол}} = \frac{N_n}{N_m},
    \hspace{10 mm}  
    \frac{N_n}{N_m} = \exp\left(- \frac{\mathcal E_n - \mathcal E_m}{k T}\right) = \exp\left(\frac{\hbar |\Omega_{nm}|}{kT}\right),
    \hspace{0.5cm} \Rightarrow \hspace{0.5cm}
    \frac{\sub{I}{кр}}{\sub{I}{фиол}} =\exp\left(\frac{\hbar |\Omega_{nm}|}{kT}\right),
\end{equation*}
что вполне объясняет наблюдаемое соотношение.




\textbf{Вынужденное комбинационное рассеяние}.
В мощных импульсах лазерного излучения наблюдается явление, называемое \textit{вынужденным комбинационным рассеянием света}, которое возникает из-за обратного воздействия световой волны на молекулы среды. Точнее на молекулу действует сила $\vc{F} = \left(\vc{p} \cdot \nabla\right) \vc{E}$, индуцированные $\vc{p} \sim \vc{E}$, так что $F \sim E^2$. Поле $\vc{E}$ складывается из $\vc{E}_0$ и $\vc{E}'$, где $\vc{E}'$ слабое, но усиливается.

Среди слагающих сил $\left([\vc{E}_0 + \vc{E}']\cdot \nabla\right) (\vc{E}_0 + \vc{E}')$ присутствуют члены с произведением полей $\vc{E}_0$ и $\vc{E}'$, частоты которых совпадают с соответствующими часотами инфракрасных колебаний молекулы, они вызывают \textit{резонансное усиление} соответствующих линий комбинационного рассеяния. Вынужденные колебания ядер молекул происходят в фазе с падащей волной, а потому вынужденноме комбинационное рассеяние когерентно с падающей волной.