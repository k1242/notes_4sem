\subsection{Многолучевая интерференция}


Обозначим через $R$ коэффицент отражение света от границы раздела пластинки с воздухом. При отсутсвии поглощения $(1-R)$ проходит через границу, если среды по обе стороны одинаковы, то и $R$ будут одинаковы. Пусть свет монохроматичен
Пусть интенсивность света $I_0$, тогда интенсивности прошедших пучков будут
\begin{equation*}
    I_{1'} = (1-R)^2, \hspace{5 mm} 
    I_{2'} = R^2(1-R)^2 I_0, \hspace{5 mm} 
    I_{3'} = R^4 (1-R)^2 I_0, \hspace{5 mm}  \ldots
\end{equation*}
а соответсвеющие вещественные амплитуды
\begin{equation*}
    a_{1'} = (1-R) a_0, \hspace{5 mm} 
    a_{2'} = R(1-R) a_0, \hspace{5 mm} 
    a_{3'} = R^2(1-R) a_0, \ldots .
\end{equation*}