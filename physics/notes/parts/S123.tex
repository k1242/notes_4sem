\subsection{Нелинейная поляризацим среды}

% § 91, 99, 100).

При распространении света в среде нелинейные явления в оптике связаны прежде всего с \textit{нелинейной зависимостью} вектора поляризации среды $\vc{P}$ от напряженности электрического поля $\vc{E}$ световой волны. Если поле $\vc{E}$ ещё не <<очень сильное>>, то вектор $\vc{P}$ можно разложить во степеням $\vc{E}$:
\begin{equation*}
    P_j = \alpha_{jk} E_k + \alpha_{jkl} E_k E_l + \alpha_{jklm} E_k E_l E_m + \ldots, 
\end{equation*}
где $\alpha_{jk}$ -- \textit{линейная поляризуемость среды}, а тензоры высших порядков называют соответственно квадратичной, кубичной, и т.д. \textit{поляризуемостями}. Поле $\vc{E}$ предполагаем монохроматичным, среду однороднойЮ немагнитной, без дисперсии, а $\alpha$ -- функции частот $\omega$. Для изотропной среды все тензоры $\alpha$ вырождаются в скаляры. 



% Про нелинейные эффекты: познакомимся с выпрямление света, генерациоей второй гармоники и самофокусировка света. 
В средах, в которых все точки явяются центрами симметрии, квадратичный член равен нулю. Однако, можем рассмотреть \textit{качественно} процессы, полагая
\begin{equation*}
    \vc{P} = \alpha \vc{E} + \alpha_2 E \vc{E} + \alpha_3 E^2 \vc{E} + \ldots,
\end{equation*}
где мы принимаем ущербность такого приближения, но зато можем сделать несколько правильных шагов. Разобьем поляризацию, а также индукцию, на линейную и нелиненую: $\vc{P} = \sub{\vc{P}}{l}+\vc{P}_{\textnormal{nl}}$, где нелинейная часть $\sub{\vc{P}}{nl} = \alpha_2 E \vc{E} + \alpha_3 E^2 \vc{E} + \ldots$, а линейная $\sub{\vc{P}}{l} = \alpha \vc{E}$. Тогда и $\vc{D} = \vc{E} + 4 \pi \vc{P}$ предсавится, как $\sub{\vc{D}}{l}=E+4 \pi \sub{\vc{P}}{l}$ и нелинейная $\sub{\vc{D}}{nl}=4 \pi \vc{P}_{\textnormal{nl}}$. Линейная часть $\sub{\vc{D}}{l}=\varepsilon \vc{E}$, где $\varepsilon$ -- диэлектрическая проницаемость. Теперь можем записать уравнения Максвелла в виде
\begin{equation*}
\left.\begin{aligned}
    \rot \vc{H} &= \frac{1}{c} \frac{\partial \vc{D}}{\partial t}, \\
    \rot \vc{E} &= - \frac{1}{c} \frac{\partial \vc{H}}{\partial c}, \\
    \div \vc{D} &= 0, \\
    \div \vc{H} &= 0, \\
\end{aligned}\right.
\hspace{0.75cm} \Rightarrow \hspace{0.75cm}
\left.\begin{aligned}
            \rot \vc{H}  &=  \frac{\varepsilon}{c} \frac{\partial \vc{E}}{\partial t} + \frac{4 \pi}{c} \frac{\partial \sub{\vc{P}}{nl}}{\partial t} , \\
    \rot \vc{E}  &=  \frac{1}{c} \frac{\partial \vc{H}}{\partial t}, \\
    \div(\varepsilon \vc{E}) &= - 4\pi \div \sub{\vc{P}}{nl}, \\
    \div \vc{H} &= 0.
    \end{aligned}\right.    
\end{equation*}
Система решается \textit{методом последовательных приближений}. В нулевом приближение $\sub{\vc{P}}{nl}=0$, получаются уравнения \textit{линейной электродинамики}. В качестве нулевого приближения рассмотрим
\begin{equation*}
    \vc{E} = \vc{E}_0 = \vc{A} \cos(\omega t - \vc{k} \cdot \vc{r}),
\end{equation*}
где $\vc{k}^2 = \varepsilon \omega^2/c^2$. Для нахождения первого приближения вместо $\vc{E}$ подставим $\vc{E}_0$, после чего снова получим линейные уравнения, но неоднородные. Правые части могут восприниматься как если бы каждый $\d V$ переизлучал волны аки \textit{диполь Герца} с моментом $\sub{\vc{P}}{nl} \d V$. Такими итерациями может найти сколь угодно приближений. 


Вообще среда диспергирует. Формально всё будет работать если взять эту охапку диффуров и решать её оидельно для слагаемых с частотой $\omega$, частотой $2\omega$, и т.д., подставляя везде свои $\varepsilon$. По идее это работает. 