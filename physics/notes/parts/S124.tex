В нулевом приближении можем найти нелинейную добавку
\begin{equation*}
    \sub{P}{nl} = \alpha_2 E_0^2 = \frac{\alpha_2 A^2}{2} + \frac{\alpha_2 A^2}{2} \cos\left[2(\omega t - \vc{k} \cdot \vc{r})\right].
\end{equation*}
Как ни странно -- это вполне адекватный результат, первое слагаемое называют \textit{оптическим детектированием}, илиоптическим выпрямлением, -- возникновением в нелинейной среде постоянной электрической поляризации при прохождении мощной световой волны. 

Второе слагаемое гармонически меняется во времени. Оно вызывает \textit{генерацию второй гармоники в нелинейной среде}, т.е. волны с частотой $\omega_2 = 2 \omega$. Найдём поле этой гармоники:
\begin{equation*}
    \left.\begin{aligned}
        \rot \vc{H} &= \frac{\varepsilon[2\omega]}{c} \frac{\partial \vc{E}}{\partial t} + i \omega \frac{4 \pi \alpha_2}{c} A \vc{A} e^{2 (i \omega t - \smallvc{k} \smallvc{r})}, \\
        \rot \vc{E} &= \frac{1}{c}\frac{\partial \vc{H}}{\partial t}, \\
        \div \vc{E} &= \div \vc{H} = 0,
    \end{aligned}\right.
    \hspace{0.5cm} \Rightarrow \hspace{0.5cm}
    \vc{E} = A_1 e^{2 i (\omega t - \smallvc{k} \smallvc{r})},
    \hspace{5 mm} 
    \vc{H} = B_1 e^{2 i (\omega t - \smallvc{k} \smallvc{r})},
\end{equation*}
что соответсвует частному решению от вынужденных колебаний. Из второго уравнения следует, что $\vc{E} \bot \vc{H}$, также верно, что $(\vc{k} \cdot \vc{A}_1) = (\vc{k} \cdot \bar{B}_1)=0$, т.е плоская волна поперечна относительно $\vc{E}$ и $\vc{H}$. Учитывая, что $k^2 c^2 = \omega^2 \varepsilon[\omega]$ можем получить:
\begin{equation*}
    \vc{A}_1 = \frac{2 \pi \alpha_2}{\varepsilon[\omega]-\varepsilon[2\omega]} A \vc{A}.
\end{equation*}
Если же к частном решению, добавим общее, то увидем, что можем подобрать такую его амплитуду, чтобы интенсивность второй гармоники в начале координат обращалась в нуль:
\begin{equation*}
    \vc{E}_1 = \frac{2 \pi \alpha_2}{\varepsilon[\omega]-\varepsilon[2\omega]} A \vc{A} \left(
        \cos[2(\omega t - \vc{k} \cdot \vc{r})] - \cos[2 \omega t - \vc{k}_2 \cdot \vc{r}]
    \right),
\end{equation*}
где $k_2^2 = \omega_2^2 \varepsilon[2\omega]/c^2$. Возводя в квадрат и усредняя можем найти интенсивность
\begin{equation*}
    I_1 \sim \frac{\alpha_2^2 \omega^2 x^2 I^2}{n^2 c^2} \left(\frac{\sin \beta}{\beta}\right)^2,
    \hspace{5 mm} 
    \beta = \frac{(2\vc{k} - \vc{k}_2)\cdot \vc{r} }{2} = \frac{(2k-k_2)x}{2},
\end{equation*}
где $x$ -- пройденное расстояние. Тут принебрегли различием $n[\omega]$ и $n[2\omega]$. 


Таким образом с возрастанием $x$ возрастает интенсивность второй гармоники, когда $\beta \in [0, \pi/2]\cup[\pi, 3\pi/2]$, и т.д. В этих сдучаях \textit{энергия переходит от исходной волны ко второй гармоники}. На других интервалах энергия возвращается от второй, к первой. Условие $\beta=\pi/2$ определяет расстояние, до которого происходит перекачка энергии. Это расстояние называется \textit{когерентной длиной}, для которого верно, что
\begin{equation*}
    \sub{L}{coh} = \frac{\lambda}{4|n[\omega]-n[2\omega]|},
\end{equation*}
где $\lambda$ -- длина исходной волны. 

Когда $n[\omega]=n[2\omega]$ верно, что $2 \vc{k} = \vc{k}_2$, тогда и $\sub{L}{coh}$ обращается в бесконечность. Это условие -- \textit{фазовый синхронизм}. 


Ещё в 1962 году было эксперментально продемонстрирована возможность осущиствить фазовый синхронизм на частотах $\omega$ и $2 \omega$ между обыкновенной и необыкновенной волной в некоторых кристаллах. 


Аналогичное явление -- \textit{генерация волн с суммарной и разностной частотами}. Если на нелинейную среду направить два можных пучка света с различными частотами $\omega_1$ и $\omega_2$, то из неё будет выходить свет с частотами $\{\omega_1,\, \omega_2,\, 2 \omega_1,\, 2 \omega_2,\, \omega_1+\omega_2,\, \omega_1-\omega_2\}$. Так можно получить излучение в инфракрасной и ультрафиолетовой области, например, $\approx 80$ нм. 