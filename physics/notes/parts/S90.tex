

\subsection{Двойное преломление в электрическом и магнитном полях (эффект Керра)}

\textit{Электрический эффект Керра состоит в том}, \textit{что многие изотропные тела при введении в постоянное электрическое поле становится оптически анизотропным}. В частности, ведут себч как одноосные двупреломляющие кристаллы, оптическая ось которых параллельна приложенному электрическому полю. 


Пусть внешнее поле $\vc{E}_0$ \textit{однородно}. Понятно, что $\sub{n}{e}-\sub{n}{o}$ зависит от $\vc{E}_0$ в виде
\begin{equation*}
    \sub{n}{e} - \sub{n}{o} = q E_0^2,
\end{equation*}
для малых полей, где $q$ зависит только от вещества и от $\lambda$. В таком случае разность фаз между обыкновенной и необыкновенными лучами будет
\begin{equation*}
    \varphi = \frac{2\pi}{\lambda}(\sub{n}{e} - \sub{n}{o}) l = 2 \pi B l E^2,
\end{equation*}
где $l$ -- толщина образца, а $B \equiv q/\lambda$ -- \textit{постоянная Керра}. \textbf{Явление Керра объясняется анизотропией самих молекул.} 

Для эффекта Керра в газах, в случае полностью анизотропных молекул, можно показать, что при $\vc{E} \parallel \vc{E}_0$ показатель преломления будет \textit{необыкновенным}, тогда
\begin{equation*}
    n = 1 + \frac{2\pi}{3} N \beta,
\end{equation*}
где $\beta$ -- поляризуемость молекулы вдоль оси молекулы. Если же $\vc{E} \bot \vc{E}_0$, то  показатель преломления будет обыкновенным, и
\begin{equation*}
    \sub{n}{o} = 1 + 2 \pi N \beta \langle \sin^ \vartheta \rangle,
\end{equation*}
где $\vartheta$ -- угол\footnote{
    \red{Дописать}.
}  между $\vc{E}$ и $\vc{s}$.

Забавный факт: из полученных соотноешний можем получить
\begin{equation*}
    \frac{\sub{n}{e}-n}{\sub{n}{o}-n} = -2,
\end{equation*}
что выполняется для большинства веществ. 


Проводя некоторый аккуратны расчёт можем получить выражение для постоянной Керра:
\begin{equation*}
    \sub{n}{e} - \sub{n}{o} = \frac{n-1}{5} \frac{\beta}{kT} E_0^2.
\end{equation*}
