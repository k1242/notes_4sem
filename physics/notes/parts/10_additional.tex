Отвлечемся теперь от изложения материала из Сивухина, и обратимся к специализированной литературе в поисках чего-нибудь большего.
Вообще, можно дочитать этот 54 параграф, где Сивухин текстом описывает свойства и объемных голограмм тоже, можно открыт Кириченко, где это ещё снабжено и формулами. 
Здесь же я обращусь к дополнительной литературе из задавальника:\textit{Р. Кольер, Оптическая голография. – М. : Мир, 1973}.

Какие главы и параграфы взять для нашей программы.