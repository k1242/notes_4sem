\subsection*{Разрешающая способность при когерентном и некогерентном освещении}
Рассматриваем идеальные оптические системы, а конечный рассматриваемый объект как совокупность точечных источников, каждый из которых изображается кружком Эйри(с окружающими его дифракционными кольцами).
Наша задача сводится к рассмотрению двух случаев точечных
\begin{enumerate}
	\item некогерентных источников --- складываются их интенсивности --- самосветящиеся --- телескоп;
	\item когерентых источнико --- складываются их напряженности --- освещаемые --- микроскоп.
\end{enumerate}

Разрешающие способности соответственно:
\begin{equation*}
	\text{телескоп: } \vartheta_\text{мин} = 1,22 \frac{\lambda}{D}
	\hspace{2 cm}
	\text{микроскоп: } l_\text{мин} = 0.61 \frac{\lambda}{n \sin \alpha} 
\end{equation*}
где $\alpha$ -- апертурный угол, $l$ -- расстояние между кружками Эйри, $\vartheta$ -- угловой размер наблюдаемого объекта.