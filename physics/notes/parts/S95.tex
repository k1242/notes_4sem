\subsection{Магнитное вращение плоскости поляризации (эффект Фарадея)}

Опыты Фарадея показали, что при наличии внешнего магнитного поля вдоль оптической оси системы, угол поворота зависит от длины пути $l$ и напряженноести внешнего поля $B$, как
\begin{equation*}
    \xi = R\, l B,
\end{equation*}
де $R$ -- \textit{постоянная Верде}, или \textit{магнитная вращательная способность}. 

При внесении в магнитное поле $\vc{B}$ у осцилляторов вещества появляются две новые резонансные частоты $\omega_0 + \Omega$ и $\omega_0 - \Omega$, где $\Omega$ -- ларморовская частота. Эти собственны частоты проявляеются не только в испускании (\textit{прямой эффект Зеемана}), но и в поглощении света (\textit{обратный эффект Зеемана}). 

Нормальные волны, которые могут распространятся вдоль магнитного поля, поляризованы по кругу. Когда направления распространения света и магнитного поля совпадают, большей частоте $\omega_+ = \omega_0 + \Omega$ соответсвует вращение по, а меньшей $\omega_-$ -- против часовой стрелки, если смотреть в направлении магнитного поля. Так как $\omega_+$ и $\omega_-$ различны, то происходит сдвиг фаз волн, а соответсвенно, и повород плоскости поляризации на гол
\begin{equation*}
    \xi = \frac{\omega l}{2c} (n_- - n_+) = \frac{\pi l}{\lambda} (n_- - n_+).
\end{equation*}

Если построить $n_- - n_+$, то можно увидеть, что, как и в случае ларморовского вращения $\Omega$, вращение плоскости поляризации определяется только направлением магнитного поля $\vc{B}$ и не зависят от направления распространения света.  При изменение на противоположное направления распространеняи света не изменятся, в противоположность естественного вращения. 

Вообще, в эффекте Фарадея, воспользовавшись формулой Зеемана можно получить \textit{формулу Беккереля} для постоянной Верде:
\begin{equation*}
    R = - \frac{e}{2 mc^2} \lambda \frac{d n}{d \lambda},
\end{equation*}
где $m$ -- масса электрона, $e > 0$ -- его абсолютный заряд.  


Ещё можно было бы поговорить про \textit{эффект Макалюзо и Корбино}, объясненный Фохтом, но оставим это на светлое будущее. 


