Для нахождения \textit{второго приближения} воспользуемся 
\begin{equation*}
    \sub{\vc{P}}{нл} = \alpha_2 (E_0 + E_1)(\vc{E}_0 + \vc{E}_1) + \alpha_3 E_0^2 \vc{E}_0,
\end{equation*}
однако учитывая только изотропные среды переходим к $\alpha_2 = 0$, а тогда
\begin{equation*}
    \sub{\vc{P}}{nl} = \frac{3 \alpha_3 A^2}{4} \vc{A} \cos\left[\omega t - \vc{k} \vc{r}\right] + 
    \frac{\alpha_3 A^2}{4} \vc{A} \cos[3\left(\omega t - \vc{k} \vc{r}\right)],
\end{equation*}
где второе слагаемое соответствует генерации тритьей гармоники.


Интересно  взглянуть на первое слагаемое: множитель $\vc{A} \cos[\omega t - \vc{k} \vc{r}]$ -- исходная падающая волна $\vc{E}_0$, которую можно заменить на $\vc{E}$, тогда
\begin{equation*}
    \rot \vc{H} - \frac{1}{c}\left[
        \varepsilon(\omega) + 3 \pi \alpha_3 (\omega) A^2
    \right] \partial_t \vc{E} = 0,
    \hspace{0.5cm} \Rightarrow \hspace{0.5cm}
    n = n_0 + n_2 A^2,
\end{equation*}
где учет рассматриваемого слагаемого эквивалентен изменению $\varepsilon(\omega)$ среды.



Вообще есть другие причины такого поведения: свет вообще давит на среду, греет среду, что приводит к изменению плотности и показателя преломления среды. В жидкостях это может быть высокочастотный эффект Керра, но во всех этих случаях $\Delta n \sim A^2$. К слову, $n_2$ бывает $>0$ и $<0$.


Так приходим к прохождению пучка через оптических неоднородную среду, в которой луч загибается в сторону большего показателя преломления. С этим связано явление \textit{самофокусировки} ($n_2 > 0$) и дефокусировки $(n_2 < 0)$.



Рассмотрим плоскопараллельный пучок лучей кругового сечения, диаметра $D$. Показатель преломления в пространстве с пучком $n = n_0 + n_2 A^2$, пусть $n_2 > 0$. Из-за дифракции пучок расширяется, однако все направления луче сосредоточатся в пределах конса с углом при вершине $2 \sub{\vartheta}{диф}$, где $\sub{\vartheta}{диф} = 1.22 \lambda/(D n_0)$. Предельный угол скольжения $\vartheta_0$ определяется соотношением
\begin{equation*}
    \cos \vartheta_0 = \frac{n_0}{n_0 + n_2 A^2},
    \hspace{0.5cm} \Rightarrow \hspace{0.5cm}
    \vartheta_0^2 \approx 2 A^2 \frac{n_2}{n_0}.
\end{equation*}
Еслм $\sub{\vartheta}{диф} > \vartheta_0$ то пучок будет расширяться. При $\sub{\vartheta}{диф} > \vartheta_0$ пучок начнём сжиматься в тонкий шнур, -- \textit{самофокусировка}. 


При $\sub{\vartheta}{диф} = \vartheta_0$ имеет место \textit{самоканализация}, для которой можем найти необходимую мощность пучка
\begin{equation*}
    P = \frac{c n_0 A^2}{8 \pi} \frac{\pi D^2}{4} = \frac{c n_0 D^2}{32} A^2,
    \hspace{0.5cm} \Rightarrow \hspace{0.5cm}
    \sub{P}{порог} \approx c \frac{(0.61\, \lambda)^2}{16 n_2}.
\end{equation*}
Расстояние от края среды, на которой фокусируются крайние лучи пучка, легко оценить:
\begin{equation*}
    \sub{f}{эф} = \frac{D}{2 \sub{\vartheta}{диф}} \approx \frac{n_0 D^2}{2.44\, \lambda},
\end{equation*}
что называют \textit{эффективным фокусным расстоянием для крайних лучей пучка}. 


