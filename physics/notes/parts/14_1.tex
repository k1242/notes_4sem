\subsection{Направляемые лучи в планарных волноводах}

Классический многомодовый волновод характеризуется значениями $\sub{n}{cl}$ и $\sub{n}{co}$, а также толщиной сердцевины $2 \rho$, что вместе с длиной волны может быть собрано в один безразмерный параметр
\begin{equation*}
    V = \frac{2\pi}{\lambda} \rho \sqrt{\sub{n}{co}^2 - \sub{n}{cl}^2},
\end{equation*}
собственно лучевой подход применим только  при $V \gg 1$. 


В случае ступенчатого профиля всегда можем описать $n(x)$ как
\begin{equation*}
    n(x) = \left\{\begin{aligned}
        &\sub{n}{co}, &-\rho < x < \rho; \\
        &\sub{n}{cl}, &|x| \geq \rho,
    \end{aligned}\right.
\end{equation*}
что может упроситить работу с построением лучей. Одна и наиболее важных задач -- определение условий, при которых луч является \textit{направляемым}, т.е. распространяется вдоль непоглащающего волновода без потерь мощности. 





