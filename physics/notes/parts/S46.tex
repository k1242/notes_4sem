
\begin{minipage}{0.45\textwidth}
    \includegraphics[width=1\textwidth]{figures/s36_1.png}
\end{minipage}
\hfill
\begin{minipage}{0.45\textwidth}
	Имеем простейший случай: лучи падают перпендикулярно, $d$ -- \textit{период решетки}, $\vartheta$ -- угол дифракции.
	Разность хода между волнами, исходящими из соседних щелей 
	\begin{equation*}
		\Delta = d \sin \vartheta, 	
	\end{equation*}
	а разность фаз 
	\begin{equation*}
		\delta = k d \sin \vartheta = 2 \pi d \sin \vartheta / \lambda.
	\end{equation*}  
	Поле, наблюдаемое от первой щели определяется формулой
	\begin{equation*}
		E_1 = \frac{b \sin\alpha}{\alpha}.
	\end{equation*}
\end{minipage}

Поля же излучаемые остальными щелями:
\begin{equation*}
	E_2 = E_1 e^{-i \delta}, 
	\hspace{0.5 cm}
	E_3 = E_1 e^{-2i \delta},
	\hspace{0.5 cm}
	\ldots
	\hspace{0.5 cm}
	E_N = E_1 e^{- i (N-1)\delta}.
\end{equation*}
Полное поле представится как сумма всех:
\begin{equation*}
	E = E_1 \frac{1- e^{- i N \delta}}{1 - e^{-i\delta}} = E_1 \frac{\sin \left(\frac{N \delta}{2}\right)}{\sin \left(\frac{\delta}{2}\right)}e^{- i (N-1)\delta/2}
	\hspace{1 cm}
	\Rightarrow
	\hspace{1 cm}
	I = I_1 \left[\frac{\sin \left(\frac{N \delta}{2}\right)}{\sin \left(\frac{\delta}{2}\right)}\right]^2.
\end{equation*}
Заметим, что при  
\begin{equation*}
	\delta/2 = m \pi
	\hspace{1 cm}
	\Leftrightarrow
	\hspace{1 cm}
	d \sin \vartheta = m \lambda.
\end{equation*}
получаем $I = N^2 I_1$ -- \textit{главные максимумы}, где $m$ -- целое число, \textit{порядок главного максимума}. Это соотношение так же определяет направление $\vartheta$ на главные максимумы.

Если у решетки $a = b$, то все главные максимумы четных порядков вообще не появятся, так как условие максимума решетки перейдёт в условие минимума дифракции на одной щели $I_1=0$:
\begin{equation*}
	d \sin \vartheta = 2 n \lambda
	\hspace{1 cm}
	\Rightarrow
	\hspace{1 cm}
	b \sin \vartheta = n \lambda.
\end{equation*}
Таким образом в рассматриваемом направлении ни одна щель, а потому и решетка в целом не излучают.

Дифракционные минимумы получаются из условия:
\begin{equation*}
	d \sin \vartheta = \left(m \frac{p}{N}\right)\lambda.
\end{equation*}
Максимумы, получающиеся между двумя соседними минимумами, называются \textit{второстепенными максимумами}. 
Таким образом между двумя соседними максимумами располагается $(N-1)$ минимум и $(N-2)$ добавочный максимум. И на эту всю красоту накладывается минимумы дифракции на одной щели.

Второстепенные максимумы находятся примерно между минимумами давайте определим направление $\delta$ на них:
\begin{equation*}
	\frac{\delta}{2} = \left(m + \frac{2 p +1}{2 N}\right)\pi.
\end{equation*}

Найдём теперь интенсивность второстепенных максимумов в окрестности главного, то есть при $N \gg 1$ и малых номерах этих максимумов $p$, а $\delta/2$ -- мал:
\begin{equation*}
	\sin \frac{\delta}{2} = \pm \sin \frac{2 p +1}{2 N} \approx \pm \frac{2 p +1}{2 N} \pi
	\hspace{0.5 cm}
	\Rightarrow
	\hspace{0.5 cm}
	I = \frac{I_1}{\pi^2}\left(\frac{2N}{2p+1}\right)^2 = \frac{4}{(2p +1)^2 \pi^2}I_\text{гл}.
\end{equation*}
Таким образом интенсивности максимумов к главному относятся как очень малые величины. И при большом числе щелей они вообще не играют роли.

\begin{minipage}{0.35\textwidth}
    \includegraphics[width=1\textwidth]{figures/s46_2.png}
\end{minipage}
\hfill
\begin{minipage}{0.55\textwidth}
	Если же волна падает на решетку под углом, то разность хода между соседними щелями становится равной:
	\begin{equation*}
		d (\sin \vartheta - \sin \vartheta_0).
	\end{equation*}
	Мы имеем новое условие на главные максимумы:
	\begin{equation*}
		d (\sin \vartheta - \sin \vartheta_0) = m \lambda,
	\end{equation*}
	а минимумы:
	\begin{equation*}
		d(\sin \vartheta - \sin \vartheta_0) = \left(m + \frac{p}{N}\right) \lambda.
	\end{equation*}
\end{minipage}

И наконец если решетка -- грубая, то при углах падения $\vartheta_0$ близких к $90^\circ$ можно написать:
\begin{equation*}
	d \cos \vartheta_0 \cdot (\vartheta - \vartheta_0) = m \lambda.
\end{equation*}
Где $d \cos \vartheta_0$ -- типа новый период, который мы уже вправе от угла падения уменьшать, делая решетку менее грубой.