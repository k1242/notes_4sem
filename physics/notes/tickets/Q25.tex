\Ssec{54}{Про голографию из Сивухина}

\begin{wrapfigure}{r}{0.5\textwidth}
  \begin{center}
    \includegraphics[width=0.28\textwidth]{figures/s54_1.png}
  \end{center}
\end{wrapfigure}
Идея голографии Габора заключается в том, что свет, попадая на объект рассеивается, и рассеянные лучи несут все информацию о форме и рельефе объекта. Однако для восстановления из таких лучей изображения нужна ещё информация о фазе изначально падавшего пучка, которую можно получить, если предварительно разделить этот пучок и часть его пустить на зеркало, чтобы переотразить в область, где мы хотим получать голограмму.
Пучок, рассеивающийся на предмете называется, называется \textit{предметным}, а несущий информацию о фазе -- \textit{опорным}.
Полученная картина на $\Gamma$ записывается, например на чувтсвительную пластину,  и называется голограммой, от греческого <<голог>> -- полный, <<графе>> -- пишу. Позже осветив её той же опорной волной можно получить восстановленное изображение объекта.

Запись голограммы -- очень тонкое дело. Необходимая степень монохроматичности света:
	$\boxed{\frac{\lambda}{\delta l} \gtrsim m}$,
	где $m$ -- максимальный порядок интерференции наблюдающийся при голографировнии.

Для хорошей установки и объекта с линейными размерами $L $ этот порядок можно оценить $\boxed{m \sim \frac{L}{\lambda}}$.
Таким образом должно быть: $\boxed{\delta\lambda < \frac{\lambda^2}{L}}$.

Требования к размеру источника тоже достаточно жестки: $\Delta x = \lambda/\alpha$, где $\alpha$ -- угол схождения крайних интерферирующих лучей. По порядку это что-то вроде $\alpha = h/l$, где $h$ -- ширина опорного пучка, а $l$ -- расстояния между предметом и голограммой. Таким образом $\boxed{\Delta x < \frac{\lambda l}{h}}$.

И так, о чем же сама голография. Представим поле рассеянной волны и отраженной:
\begin{equation*}
	u = a (\vc{r}) e^{i [\omega t - \Phi(\smallvc{r})]},
	\hspace{0.5 cm}
	v = b e^{i (\omega t - \smallvc{k} \smallvc{r})}.
\end{equation*}
Тут мы сделали упрощение, что поля не векторные, а скалярные, что для нас не сильно критично, и при таком упрощении $a(\vc{r})$, $\Phi(\vc{r})$ и $b$ будем считать вещественными. На пластинке $\Gamma$ интенсивность:
\begin{equation*}
	I  = v^* u + v u^* + v^* v + u^* u,
	\hspace{0.5 cm}
	I_0 = b a (x,y,0) e^{i[k_x x - \Phi(x,y,0)]} + b a (x,y,0) e^{-i[k_x x - \Phi(x,y,0)]} + b^2 + a^2(x,y,0).
\end{equation*}
где мы направили ось $Z$ перпендикулярно плоскости $\Gamma = XY$.

Допустим теперь, что нашу пластину, покрытую фотоэмульсией мы проявили и \href{https://youtu.be/y-MyaUcMkhs?t=252}{скопировали}, получив позитив голограммы. Пусть у позитива пропускаемость $D = I_0$, такую позитивную голограмму можно использовать для восстановления $u(\vc{r},t)$.
Для этого голограмму просвечивают таким же $v(\vc{r},t)$, он испытает дифракцию на голограмме типа как на диф-решетке. По методу Рэлея получим поле на выходе и будем искать решение волнового уравнения:
\begin{equation*}
	E_{\text{вых}} = D v (x,y,0) = I_{0} b e^{i (\omega t - k_x x)},
	\hspace{1 cm}
	\frac{\partial^2 E}{\partial x^2} + \frac{\partial^2 E}{\partial y^2} + \frac{\partial^2 E}{\partial z^2} + k^2 E = 0.
\end{equation*}
Будем искать решение для $E$ в виде $E = E_1 + E_2 + E_3 + E_4$, с граничными условиями:
\begin{equation*}
	\begin{aligned}
		&E_{1 \text{ вых}} = b^2 a(x,y,0) e^{i [\omega t - \Phi(x,y,0)]},\\
		&E_{2 \text{ вых}} = b^2 a(x,y,0) e^{i [\omega t + \Phi(x,y,0) - 2 k_x x]},	
	\end{aligned}
	\hspace{1 cm}
	\begin{aligned}
		&E_{3 \text{ вых}} = b^3 e^{i (\omega t - k_x x)},\\
		&E_{4 \text{ вых}} = b a^2(x,y,0) e^{i (\omega t - k_x x)}.
	\end{aligned}
\end{equation*}
Будем решать. Проще всего найти функцию $E_3 = b^3 e^{i (\omega t - \smallvc{k} \smallvc{r})} = b^2 v(\vc{r},t)$, это есть ни что иное, как опорная волна, распространяющаяся за голограмму.

Основной же интерес для голографии представляет собой поле $E_1$, $b$ -- постоянно, тогда:
\begin{equation*}
	E_1 = b^2 a (x,y,z) e^{i [\omega t - \Phi(x,y,z)]} = b^2 u (\vc{r},t).
\end{equation*}
Действительно, видим, что это волна, уходящая от голограммы. Она даст мнимое изображение объекта, в том же самом месте, в котором он находился до получения голограммы. 

Для нахождения $E_2$ сначала посмотрим на случай, когда опорный луч падает нормально плоскости голограммы, тогда $k_x = 0$, и немного другое граничное условие дадут решение
\begin{equation*}
	E_{2 \text{ вых}} = b^2 a(x,y,0) e^{i[\omega t + \Phi(x,y,0)]}
	\hspace{1 cm}
	\Rightarrow
	\hspace{1 cm}
	\tilde{E}_2 (x,y,z) = b^2 a (x,y,z) e^{i [\omega t + \Phi(x,y,z)]}.
\end{equation*}
Такая волна снова создаёт мнимое изображение, как и только что рассмотренная выше $E_1$, но она распространяется к голограмме, а не от нее, значит не может служить решением рассматриваемой нами задачи.
Чтобы починиться в уравнении Гельмгольца заменим $z \mapsto -z$
\begin{equation*}
	\frac{\partial^2 E}{\partial x^2} + \frac{\partial^2 E}{\partial y^2} + \frac{\partial^2 E}{\partial (-z)^2} + k^2 E = 0
	\hspace{1 cm}
	\Rightarrow
	\hspace{1 cm}
	E_2 = b^2 a(x,y,-z) e^{i[\omega t + \Phi(x,y,-z)]}.
\end{equation*}
Теперь волна идёт от голограммы, является решением нашей краевой задачи, и формирует таким образом действительное изображение.

\Esec{Голограмма Габора}
Рассмотрим простейшую голограмму -- голограмму точечного источника (рассеивателя), которая также называется \textit{голограммой Габора}. От него идут монохроматические сферические волны. А запись идёт тонкослойную пластину.
\begin{wrapfigure}{r}{0.65\textwidth}
     \centering
     \includegraphics[width=0.5\textwidth]{figures/gabor_point.png}
     \caption{Голограмма Габора. Точечный источник $S$, находящийся на расстоянии $v$ от плоскости голограммы $H$, освещает рассеивающий центр $P$, находящийся на расстоянии $u$ от $H$. 
	}
     %\label{fig:}
 \end{wrapfigure}
Интерференция волн от $P$ и $S$ задается выражением: $I = I_1 + I_2 + 2 a_1 a_2 \cos (\varphi_2 - \varphi_1)$. Предполагая, что пропускание голограммы пропорционально $I$ получаем на голограмме интерференционную картину, зависящую от разности фаз $\Delta \varphi = \varphi_2 - \varphi_1$.

Пусть источник непрерывно излучает волну длинной $\lambda$ и чистотой $\nu$, фаза источника $\varphi = 2 \pi \nu t$.
Для скорости распространения света $c$ разность фаз в точке $Q$ в момент времени $t_Q$ на голограмме для лучей от источника и объекта будет:
\begin{equation*}
	\hspace{-5mm} \varphi_r - \varphi_s = \frac{2 \pi \nu}{c} (SPQ - SQ) = \Delta \varphi = \frac{2 \pi \Delta l}{\lambda}.
\end{equation*}
В тех случаях, когда $\Delta \varphi = 1$, то есть $\Delta l = n \lambda$ получаем максимумы по обозначениям рисунка и имея $x \ll u,v$ для опыта Габора:
\begin{equation*}
	\Delta l = (r+s) - t = (v - u + s) - t = v - u + \sqrt{u^2 + x^2} - \sqrt{v^2 + x^2} \approx v - u + u + \frac{x^2}{2 u} - v - \frac{x^2}{2 v} = \frac{x^2}{2}\left(\frac{1}{u} - \frac{1}{v}\right).
\end{equation*}
Вводя фокусное расстояние так называемой \textit{зонной пластинки} $f$ получаем:
\begin{equation*}
 	\frac{1}{f} = \frac{1}{u} - \frac{1}{v},
 	\hspace{1 cm}
 	\Rightarrow
 	\hspace{1 cm}
 	\Delta l = \frac{x_n^2}{2 f} = n \lambda.
 \end{equation*} 
 В силу осевой симметрии получаем таким образом радиусы колец $x_n$ максимальной яркости.

 Выражая $x_n$ получаем очень похожее на френелевские зоны выражение:
 \begin{equation*}
 	x_n = \sqrt{f \lambda} \cdot \sqrt{2 n}.
 \end{equation*}

 \begin{wrapfigure}{r}{0.5\textwidth}
     \centering
     \includegraphics[width=0.5\textwidth]{figures/vosstan_gabor.png}
     \caption{Фокусирующие свойства зонной пластинки.}
     %\label{fig:}
 \end{wrapfigure}
 Изготовляя далее позитив голограммы получаем обнаруживаем, что он обладает соответственно теми же свойствами что и зонная пластинка Френеля, с той оговоркой, мы получили синусоидальную кривую пропускания, а не прямоугольную. При дифракции на нашей голограмме будут возникать только волны $\pm 1$ порядков, в то время как у Френеля и более высших порядков, но если последнего ограничить только по первым гармоникам, то картина получится идентичная.

 Мы говорили про $f$ -- фокус получившейся решетки. И для него получили выражение соответствующее формуле тонкой линзы.
 Таким образом имеем аналогию, что голограмма точечного объекта ведет себя подобно диф-решетке и подобно отрицательной линзе, создающей мнимое и действительное изображение. 

В более общем случае, как и в выводе из Сивухина, если представить протяженный объект как совокупность точек и пренебречь интерференцией волной от этах точек по сравнению с интерференцией с опорной волной, получим изображение объекта как суперпозицию зонных пластинок. И когда такая голограмма осветится опорной волной, каждая индивидуальная голограмма создаст мнимое изображения соответствующей точки объекта, а в процессе восстановления изображения точек создадут образ всего протяженного объекта.

И так, отвлекаясь от точки, все ещё можем записать разность хода между опорной и рассеянной волной:
\begin{equation*}
	\Delta l \frac{x^2 + y^2}{2} \left(\frac{1}{v} - \frac{1}{u}\right) = \frac{\rho^2}{2 f} = n \lambda.
\end{equation*}
Можно определить локальную пространственную частоту полос $\nu(\rho)$ следующим образом:
\begin{equation*}
	\nu(\rho) = \frac{\partial (\Delta \varphi)}{\partial\rho} \cdot \frac{1}{2 \pi} = \frac{\partial}{\partial \rho} \left(\frac{\Delta l}{\lambda_1}\right)
	\hspace{1 cm}
	\Rightarrow
	\hspace{1 cm}
	\nu(\rho) = \frac{\rho}{f \lambda}.
\end{equation*}
То есть по мере удаления от центра частота полос увеличивается. И при некотором $\rho$ частота $\nu$ может привысить разрешающую способность светочувствительной среды $\nu_m$.


\Esec{Голограмма с наклонным опорным пучком}
\begin{wrapfigure}{r}{0.5\textwidth}
     \centering
     \includegraphics[width=0.5\textwidth]{figures/leyt_holo.png}
     \caption{Первоначальная схема получения внеосевых голограмм. (По Лейту и Упатниексу)}
     %\label{fig:}
 \end{wrapfigure}
Для получения осевых голограмм, как например, для точки из прошлого параграфа, требуется соблюдение ряда условий:
\begin{enumerate}
	\item объект должен состоять из малых непрозрачных участков на большом прозрачном фоне;
	\item с исходной голограммы приходится делать позитивный отпечаток;
	\item для устранения помех нужно в области действительного изображения голограммы загасить нежелательную дифракцию  от мнимого изображения объекта.
\end{enumerate}
Ученые, помучавшись с осевой геометрией системы с различными фильтрациями в итоге отказались от идеи осевой симметрии и перешли к наклонным опорным пучкам в голограммах, а именно это сделали \textit{Лейт и Упатниекс}. 

\begin{figure}[htb]
     \centering
     \includegraphics[width=0.5\textwidth]{figures/naclon_holo.png}
     \caption{Голограмма, образованная, точечным объектом $P$, расположенным не на оси, и аксиальной плоской опорной волной. }
     \label{fig:naclon}
 \end{figure}

Хочется сказать, что для внеосевой диаграмме, в соответствии с обозначениями на рисунке получаем разность хода, как и для осевого случая:
\begin{equation*}
	\Delta l = \frac{x_2^2 + y_2^2}{2} \left(\frac{1}{z_1} - \frac{1}{z_1}\right) - x_2 \left(\frac{x_r}{z_r} - \frac{x_1}{z_1}\right)
	 - y_2 \left(\frac{y_r}{z_r} - \frac{y_1}{z_1}\right) = n \lambda.
\end{equation*}
То есть уравнение окружности с координатами центра:
\begin{equation*}
	x_2 = \frac{z_1 x_r - z_r x_1}{z_1 - z_r}
	\hspace{0.5 cm}
	y_2 = \frac{z_1 y_r - z_r y_1}{z_1 - z_r}
	\hspace{1 cm}
	\Rightarrow
	\hspace{1 cm}
	\rho = x_2^2 + y_2^2 + \frac{2 n \lambda z_1 z_r}{z_1 - z_r}.
\end{equation*}
Частоту полос в направлении $x_2$ получим дифференцируя $\Delta l/\lambda$ по этому направлению при условии $x_r = y_r = y_1 = 0$ (см. рисунок \ref{fig:naclon}) и $z = \infty$. Получаем:
\begin{equation*}
	\frac{\partial}{\partial x_2} \frac{\Delta l}{\lambda} = \xi = \frac{x_2}{z_1 \lambda} + \frac{x_1}{z_1 \lambda}.
\end{equation*}
Сравнивая с частотой для осевой диаграммы наблюдаем такое же учащение полос при отхождении от $x_2 = x_\text{осевой} = 0$. А разность у них сохраняется равной $x_1/(z_1\lambda)$

Для того, чтобы на внеосевой голограмме была зарегистрирована интерференционная картина, нужно чтобы разрешающая способность светочувствительной среды была на $x_1/z_1 \lambda$ больше чем для осевой диаграммы. И для маленьких углов падения $x_r/z_r \approx \theta_r$ разность дастся как:
\begin{equation*}
	\left(\frac{x_1}{z_1} - \frac{x_r}{z_r}\right)\frac{1}{\lambda} \approx \frac{\theta_1 - \theta_r}{\lambda},
\end{equation*}
где $\theta_1$ -- средний угол между осью $z$и предметной волной.
% \newpage