\begin{frame}
\maketitle
\end{frame}

\begin{frame}
\frametitle{\rmfamily Цель работы}
В предыдущем задании нашей группе требовалось переписать текст из книги Ландау Лившица, том 4 \cite{LL}, с чем мы успешно справились. В этот раз была поставлена задача освоить пакет beamer, создав в нём презентацию по заданному материалу.

\end{frame}
\begin{frame}
\frametitle{Реализация команды itemize (задачи к \S41)}
\begin{itemize}
\item { Определить изменение направления поляризации частицы при ее 
движении в плоскости, перпендикулярной однородному магнитному полю
(${\bf v \bot H}$).

Р е ш е н и е. В правой стороне уравнения (\ref{Eq4109}) остается лишь первый член, т.~е. вектор $\vc{\zeta}$ прецессирует вокруг направления {\bf H} (ось $z$) с угловой скоростью
$$ -\frac{2\mu m + 2\mu '(\varepsilon - m)}{\varepsilon}{\bf H} = 
-\left(\frac{e}{\varepsilon}+2\mu'\right){\bf H}.$$
С этой угловой скоростью вращается в плоскости $xy$ проекция $\vc{\zeta}$
на эту плоскость (обозначим ее $\vc{\zeta}_1$). Вектор же {\bf v} вращается в той же плоскости с угловой скоростью $-e{\bf H}/{\varepsilon}$.
 Отсюда видно, что 
$\vc{\zeta}_1$ поворачивается относительно направления {\bf v} с угловой скоростью -- $2\mu'{\bf H}$ .
\item То же при движении вдоль направления магнитного поля.

Р е ш е н и е. При совпадающих направлениях {\bf v} и {\bf H} уравнение
 (\ref{Eq4109}) приводится к виду
 $$\frac{d\zeta}{dt} = \frac{2\mu m}{\varepsilon}[\zeta {\bf H}],$$
 т.~е. \vc {\zeta} прецессирует вокруг общего направления {\bf v}
  и {\bf H} с угловой скоростью $-2\mu m {\bf H}/{\varepsilon}$.
\item То же при движении в однородном электрическом поле.

Р е ш е н и е. Пусть поле {\bf E} направлено вдоль оси $x$, а движение происходит в плоскости $xy$. Из (\ref{Eq4109})
  видно, что вектор \vc{\zeta} прецессирует вокруг оси $z$ с мгновенной угловой скоростью
  $$-\left(\frac{e}{e+m}+2\mu'\right)E \:\frac{p_y}{\varepsilon}.$$

  Снова разложим $\vc{\zeta}$ на составляющие $\vc{\zeta}_z$ и 
  $\vc{\zeta}_1$
  (в плоскости $xy$). Тогда 
  $$\vc{\zeta}_{\|} = \vc{\zeta}_1 \cos \varphi, \qquad
  \vc{\zeta}_{\bot}{\bf E}= -\vc{\zeta}_1 
  \sin \varphi \cdot \frac{v_y}{v}.
  $$
  Из (\ref{Eq4111}) находим, что $\vc{\zeta}_1$ вращается относительно направления {\bf v} с мгновенной угловой скоростью 
  $$\dot \varphi = \frac{2v_y}{v^2}\left(\frac{\mu m^2}{\varepsilon^2 
  }- \mu'\right) = \frac{p_y}{\varepsilon}\left(\frac{em}{p^2}
  -2\mu'\right).$$}
\end{itemize}
\end{frame}


\begin{frame}
\frametitle{\rmfamily Использование формул}
Энергия и импульс частицы с ${m = 0}$ связаны соотношением
${\xi = |{\bf p}|}$.  Поэтому для плоской волны (${\eta _p \:\underline{\infty}\: e^{-ipx}}$) уравнение
 (\ref{Eq3002}) дает    
 \begin{equation}
    (\vc{n\sigma})\eta_p = -\eta_p, \label{Eq3003}
 \end{equation}
где {\bf n} — орт вектора {\bf р}. Такое же уравнение
\begin{equation}
    (\vc{n\sigma})\eta_{-p} = -\eta_{-p}, \label{Eq3004}
\end{equation}
имеет место и для волны с <<отрицательной частотой>> 
(${\eta _{-p} \:\underline{\infty}\: e^{-ipx}}$).

Вторично квантованный \text{$\Psi$-оператор:}
\begin{equation}
    \hat \eta = \sum_{\bf p} (\eta_p \hat a_p + \eta_{-p}\hat b_p^+),       \qquad
    \hat \eta^+ = \sum_{\bf p} (\eta_p^* \hat a_p^+ + \eta_{-p}^*\hat b_p). \label{Eq3005}
\end{equation}
Отсюда, как обычно, следует, что $\eta_{-p}^*$— волновые функции античастицы.
\end{frame}
\begin{frame}
\frametitle{\rmfamily Использование рисунка}
В качестве рисунка вставил скриншот страницы книги Ландау Лившица:
\begin{figure}[h]
\centering\includegraphics[width=1cm]{Landau.png}
\caption{страница 141 книги Ландау Лившица}
\label{fig:image}
\end{figure}
\end{frame}

\begin{frame}
\frametitle{\rmfamily Список литературы}
\begin{thebibliography}{9}
\bibitem{LL}
Л.Д.Ландау и Е.М.Лифшиц. Том 4. Квантовая электродинамика. Издание третье, исправленное.
\end{thebibliography}
\end{frame}