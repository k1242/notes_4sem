% Observation of one slit Fraunhofer diffraction ($\delta x \approx \const$).
% % \begin{figure}[h]
%     \begin{minipage}{0.48\textwidth}
%             % \hspace{-0.5cm}
%             \begin{figure}[h]
%                 \centering
%                 \includegraphics[width=1\textwidth]{../figures/fraun_plot_1.pdf}
%                 \caption{Dependency $N_1(x_m)$}
%                 %\label{fig:}
%             \end{figure}
%     \end{minipage}
%     \hfill
%     \begin{minipage}{0.4\textwidth}
        
%     \end{minipage}
% % \end{figure}




\begin{minipage}{0.48\textwidth}
        \hspace{-1cm}
        \includegraphics[width=1\textwidth]{../figures/fraun_plot_1.pdf}
\end{minipage}
\hfill
\begin{minipage}{0.48\textwidth}
    Measured slit thickness:
    % \vspace{-5mm}
    \begin{align*}
        b_{\text{micro}} &= (298 \pm 10) \ \mu\text{m} \\
        b_{\text{scr}} &= (307 \pm 20) \ \mu\text{m} \\
        \vspace{3mm}
        b_{\text{plot}} &= (301 \pm 5) \ \mu\text{m} \\
    \end{align*}
    \begin{align*}
            \delta x &= (411 \pm 14) \ \mu\text{m} \\
            &\boxed{b = (301 \pm 6) \ \mu\text{m}}.
    \end{align*}
    

    All 4 measurement methods give results that coincide within the margin of error.
\end{minipage}
