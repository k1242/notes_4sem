\documentclass[]{beamer}

\input{settings/special_beamer.tex}
\usepackage[T2A]{fontenc}
\usepackage[utf8]{inputenc}
\usepackage[english]{babel}
\usepackage{hyperref}     % ТАК_НУЖНО
\hypersetup{unicode=true} % ТАК_НУЖНО
\usepackage{amsmath}
\usepackage{amssymb,textcomp, esvect,esint}
\usepackage{amsfonts}
\usepackage{amsthm}
\usepackage{graphicx}
\usepackage{indentfirst}
\usepackage{xcolor}
% \usepackage{enumitem} %--- ломал нумерацию!?

\usepackage{graphicx}
\usepackage{booktabs}
\usepackage{caption}
\usepackage{listings}
\usepackage{tikz}
\usepackage{xcolor}



\usepackage{media9}
\usepackage{animate}
\usepackage{threeparttable}
\usepackage{pifont}


\usepackage{import}
\usepackage{xifthen}
\usepackage{pdfpages}
\usepackage{transparent}

\usepackage[skip=1pt]{caption}
% add (renew) commands

\renewcommand{\Im}{\mathop{\mathrm{Im}}\nolimits}
\renewcommand{\Re}{\mathop{\mathrm{Re}}\nolimits}
\renewcommand{\d}{\, d}
\renewcommand{\leq}{\leqslant}
\renewcommand{\geq}{\geqslant}

\renewcommand{\P}{\mathop{\mathrm{P}}\nolimits}
\newcommand{\E}{\mathop{\mathrm{E}}\nolimits}
\newcommand{\D}{\mathop{\mathrm{D}}\nolimits}
\newcommand{\cov}{\mathop{\mathrm{cov}}\nolimits}

\newcommand{\vc}[1]{\mbox{\boldmath $#1$}}
\newcommand{\T}{^{\text{T}}}

\newcommand{\incfig}[1]{%
    \def\svgwidth{\columnwidth}
    \import{./figures/}{#1.pdf_tex}
}

\newcommand{\diag}{\mathop{\mathrm{diag}}\nolimits}
\newcommand{\card}{\mathop{\mathrm{card}}\nolimits}
\newcommand{\grad}{\mathop{\mathrm{grad}}\nolimits}
\renewcommand{\div}{\mathop{\mathrm{div}}\nolimits}
\newcommand{\rot}{\mathop{\mathrm{rot}}\nolimits}
\newcommand{\Ker}{\mathop{\mathrm{Ker}}\nolimits}
\newcommand{\Spec}{\mathop{\mathrm{Spec}}\nolimits}
\newcommand{\sign}{\mathop{\mathrm{sign}}\nolimits}
\newcommand{\tr}{\mathop{\mathrm{tr}}\nolimits}
\newcommand{\rg}{\mathop{\mathrm{rg}}\nolimits}

\newcommand{\const}{\text{const}}
\newcommand{\red}[1]{\textcolor{red}{#1}}
\newcommand{\green}[1]{\textcolor{urlcolor}{#1}}
\newcommand{\blue}[1]{\textcolor{ublue}{#1}}

\newcommand{\qs}[1]{\left| #1 \right\rangle}
\newcommand{\cqs}[2]{\left| #1 \right\rangle_{\hspace{-1pt} #2}}
\newcommand{\cmark}{\text{\ding{51}}}
\newcommand{\xmark}{\text{\ding{55}}}

\newcommand{\smallvc}[1]{\scalebox{0.65}{\mbox{\boldmath $#1$}}}
% \newcommand{\progressbar}{%
	\pgfmathsetmacro{\slidewidth}{\paperwidth}
	\pgfmathsetmacro{\progressstep}{\paperwidth/\inserttotalframenumber}
	\pgfmathsetmacro{\progresspos}{(\insertframenumber - 0.5) * \progressstep}
	\begin{tikzpicture}[scale = 0.035, line width = 1ex]
		\node[inner sep=0pt] (cat) at (\progresspos,0)	{\includegraphics[width=30pt]{settings/cat.png}};
		\path[red] (0,0) -- (\slidewidth,0);
	\end{tikzpicture}
}

\makeatletter
\setbeamertemplate{footline}
{
\hfill \progressbar%
  \leavevmode%
  \hbox{%
  \begin{beamercolorbox}[wd=.433333\paperwidth,ht=2.25ex,dp=1ex,center]{section in head/foot}%
    \usebeamerfont{author in head/foot}\insertshortauthor~~\beamer@ifempty{\insertshortinstitute}{}{(\insertshortinstitute)}
  \end{beamercolorbox}%
  \begin{beamercolorbox}[wd=.333333\paperwidth,ht=2.25ex,dp=1ex,center]{section in head/foot}%
    \usebeamerfont{title in head/foot}\insertshorttitle
  \end{beamercolorbox}%
  \begin{beamercolorbox}[wd=.333333\paperwidth,ht=2.25ex,dp=1ex,right]{section in head/foot}%
    % \usebeamerfont{date in head/foot}\insertshortdate{}\hspace*{2em}
    % \insertframenumber{} / \inserttotalframenumber\hspace*{2ex} 
  \end{beamercolorbox}}%
  \vskip0pt%
}

\setbeamertemplate{caption}[numbered]
\title[Choice Question]{Implementations of dynamic chaos \\
in different optical systems}

\author{
Khoruzhii K., Primak E.}
\institute[MIPT]

\begin{document}
\date{1.06.2021}
\maketitle




\section{Chaos}

\frame{
\textit{Globally} we are to observe different realization of dynamic chaos and implement some on our own.

\phantom{42}

Here we will consider the following steps:
\begin{itemize}
    \iitem{study the dynamical chaos on its own;}
    \iitem{implement laser theory to create chaos;}
    \iitem{model laser behavior and attempt to build a circuit;}
    \iitem{study light branches chaoticly penetrating thin layer;}
    \iitem{model and implement this model.}
\end{itemize}
 \frametitle{Goals}}

\frame{

  Map\footnote{
    W. Hirsch, S. Smale, Introduction to Chaos.
} $f$ is \textbf{chaotic}, if \\
\begin{itemize}
    \iitem{$f$ \textbf{sensitive} to the initial conditions.}
    \iitem{periodic orbits are dense everywhere;}
    \iitem{orbits are mixed;}
\end{itemize}

% \vspace{-5mm}

\begin{figure}[h]
    \begin{minipage}[h]{0.49\linewidth}
        \center{
        \includegraphics[width=0.7\linewidth]{figures/attractor.pdf}
        \\ 
        \vspace{-5mm} 
        Dense mixed orbits example}
    \end{minipage}
    \hfill
    \begin{minipage}[h]{0.49\linewidth}
        \center{
        \includegraphics[width=0.98\linewidth]{figures/ics.pdf}
        \\ 
        \vspace{-5mm} 
        Sensitivity example}
    \end{minipage}
\end{figure}

% \begin{figure}[h]
%     \centering
%     \includegraphics[width=0.35\textwidth]{figures/attractor.pdf}
%     \hspace{5 mm}
%     \includegraphics[width=0.49\textwidth]{figures/ics.pdf}
%     %\caption{}
%     %\label{fig:}
% \end{figure}


% подписать графики \frametitle{Definition of dynamic chaos and applications}}

\frame{



Possible\footnote{
    M. Pecora, L. Carroll, Synchronization in Chaotic Systems, 1990.
}  synchronization of chaotic systems:

% \hspace{-5mm}
\begin{minipage}{0.65\textwidth}
        \textit{enough} to transmit\\
        \phantom{python} part of the signal;\\
        \phantom{python} configure system parameters. 
\end{minipage}
\hfill
\begin{minipage}{0.3\textwidth}
    \begin{center}
        \incfig{scheme}
    \end{center}
\end{minipage}


\phantom{42}

The use of optics to transmit the signal allows to achieve a greater bandwidth of the channel. 

\phantom{42}

UHFO (ultrahight frequency oscillations) is a characteristic to optic systems.

 \frametitle{Synchronization}}

% \section{Laser}

\frame{
To start the laser idea we need to obtain:
\begin{itemize}
	\iitem{Solution of the Schr\"odinger equation in a semiconductor medium for the wavefunction of an electron;}
	\iitem{Induced polarization for distribution of holes and electrons in a semiconductor;}
	\iitem{Interaction of electrons in a semiconductor with an wave equation and outer electric field.}
\end{itemize}
 \frametitle{The concept of a semiconductor laser}}

\frame{
We will need the Schr\"odinger equation:
\begin{equation*}
	H_{\text{crystal}} \Psi_n(\vc{r}) = \left[\frac{\vc{p}^2}{2 m_0} + U_p(\vc{r})\right] \Psi_n(\vc{r}),
\end{equation*}
where $\vc{p} = - i \hbar \nabla$ is the momentum operator, $m_0$ is the free electron mass, $U_p(\vc{r})$ is th periodic potential of the bulk semiconductor.

The solution is the Bloch function:
\begin{equation*}
	\Psi_{n, \vc{k}}(\vc{r}) = u_{n, \vc{k}}(\vc{r}) \frac{1}{\sqrt{V}} e^{i \vc{k} \cdot \vc{r}}.
\end{equation*} \frametitle{Electronic states in a semiconductor}}


\frame{
\begin{figure}[h]
    \centering
    \includegraphics[width=1.1\textwidth]{images/lit_obzor_scary.pdf}
    %\caption{}
    %\label{fig:}
\end{figure}
\footnotetext{Physics and Applications of Laser Diode Chaos by
M. Sciamanna, and K. A. Shore
}
 \frametitle{From the root article a tree has grown}}

\frame{
\begin{minipage}{0.45\textwidth}
	\textbf{Pure Optical Feedback}
	\begin{itemize}
		\iitem{Frequency depends on laser}
		\iitem{Coupler and fibre is needed}
		\iitem{Precise instrumentations}
	\end{itemize}
	\begin{figure}[h]
    \centering
    \includegraphics[width=1\textwidth]{images/pure_optic.jpg}
    %\caption{}
    %\label{fig:}
\end{figure}
\end{minipage}
\hfill
\begin{minipage}{0.45\textwidth}
    \textbf{Electro-Optical Feedback}
    \begin{itemize}
		\iitem{All relaxations must correlate}
		\iitem{Electrical amplifier and fibre is needed}
		\iitem{Well stabilization is desired}
	\end{itemize}
	$\longleftarrow$\textit{"Optomechanically induced stochastic resonance
and chaos transfer between optical fields" by
Faraz Monifi}
\end{minipage}
 \frametitle{Problems of implementations}}

\frame{
\begin{minipage}{0.55\textwidth}
    \begin{figure}[h]
    \centering
    \includegraphics[width=0.9\textwidth]{images/tang_exp.png}
    % \caption{}
    %\label{fig:}
\end{figure}
\end{minipage}
\hfill
\begin{minipage}{0.35\textwidth}
	\begin{figure}
		\includegraphics[width=1\textwidth]{images/tang_circr.png}
    \caption{Experimental results of the time series and power spectra of different pulsing states at different delay times.
    On a circuit dashed line represents optical path.}
	\end{figure}
\end{minipage}
\footnotetext{Chaotic Pulsing and Quasi-Periodic Route to
Chaos in a Semiconductor Laser with Delayed
Opto-Electronic Feedback
S. Tang and J. M. Liu (2001)} \frametitle{The decided variant}}

% \section{Theory}

\frame{
\begin{minipage}{0.45\textwidth}
    \begin{figure}[h]
    \centering
    \includegraphics[width=1.1\textwidth]{images/semi_medium.png}
    %\label{fig:}
\end{figure}
\end{minipage}
\hfill
\begin{minipage}{0.49\textwidth}
	Photon density
    \begin{equation}
	\frac{d S}{d t} = v_g \Gamma_{MD} G(E) S.
	\end{equation}
	And also carrier density
	\begin{equation}
	\frac{d N_{MD}}{d t} = - v_g G(E) S.
\end{equation}
\end{minipage}

\phantom{239}

Y-directed wave in active layer obeys the equation:
\begin{equation}
	\triangle E(\vc{r},t) - \mu_0 \varepsilon(\vc{r}) \frac{\partial^2}{\partial t^2} E(\vc{r}, t) = \mu_0 \frac{\partial^2}{\partial t^2} P_{in}(\vc{r},t)
	\label{1.46}
\end{equation}

 \frametitle{The equations and ideas behind}}

\frame{
\begin{minipage}{0.45\textwidth}
    \begin{figure}[h]
    \centering
    \includegraphics[width=1\textwidth]{images/semi_medium.png}
    \caption{}
    %\label{fig:}
\end{figure}
\end{minipage}
\hfill
\begin{minipage}{0.49\textwidth}
    In the active layer we obtain the Schr\"odinger equation:
    \begin{equation*}
	H_{\text{crystal}} \Psi_n(\vc{r}) = \left[\frac{\vc{p}^2}{2 m_0} + U_p(\vc{r})\right] \Psi_n(\vc{r})
\end{equation*}
\end{minipage}
 \frametitle{Laser device geometry and induced polarization}}
% \frame{
% The \textit{demensional gain coefficient} is approximately:
\begin{equation}
	G \approx G_0 = \frac{1}{V_{MD}} \sum_{k_q, x, z} \frac{E}{E_0} \frac{|\mu(k_q, x, z)|^2}{\hbar c \varepsilon_0 n_r}[\rho_{ee} + \rho_{hh} - 1] (E - E_0)
\end{equation}

And the \textit{dimensional coupling factor}.
\begin{equation}
	\Gamma_{MD} = \frac{\varepsilon(x,z) |E_0(x,z)|^2}{\iint \varepsilon(x,z) |E_0(x,z)|^2 dx d z}\bigg|_{x,z=0,0}
\end{equation}
And goptical gain coefficient:
\begin{equation}
	g = g_0 + g_n(N-N_0) + g_p(S - S_0).
\end{equation} \frametitle{Coefficients}}
\frame{
And from the previous equation we can describe the quantum well laser behavior\footnote{
    Semiconductor
Lasers I
Fundamentals Ch.I by Bin Zhao, Amnon Yariv.
}
\begin{equation}
	\frac{d S}{d t} = \Gamma_{MD} G_0 v_g S - \frac{S}{\tau_p}
    \hspace{0.3 cm}
    \Rightarrow
    \hspace{0.3 cm}
    \frac{d S}{d t} = v_g g S(t) - \gamma_p S. 
\end{equation}
\begin{equation}
	\frac{d N}{d t} = \frac{J}{e} - \frac{N}{\tau_n} - \Gamma_{MD} G_0 v_g S 
    \hspace{0.3 cm}
    \Rightarrow
    \hspace{0.3 cm}
    \frac{d N}{d t}
    = \frac{J}{e d} - \gamma_s - g S.
\end{equation}
for previously mentioned optical gain coefficient:
\begin{equation}
	g = g_0 + g_n(N-N_0) + g_p(S - S_0).
\end{equation} \frametitle{The concluding formulas}}

\frame{
\textbf{Thr.} If there is no stationary points on the enclosed 2D region $G$ and some trajectory exists $\gamma \subset G$, then $\gamma$ is a closed loop path or tends to the closed one.

\phantom{239}

But there is still a hope for a chaos!

\phantom{239}

We add positive optoelectronic feedback in order to raise to 3D our equation
\begin{align}
	&\frac{d S}{d t} = v_g S g(S,N) - \gamma_p S,\\
	&\frac{d N}{d t} = \frac{J}{e d}[1 + \frac{\xi S(t-\tau)}{S_0}] - \gamma_n N - S g(S,N).
\end{align} \frametitle{Poincar\'e-Bendixson theorem}} Женя хочет в своем рассказе вопроса по выбору это раскоментить
% \section{Theory}

\frame{
\begin{minipage}{0.45\textwidth}
    \begin{figure}[h]
    \centering
    \includegraphics[width=1.1\textwidth]{images/semi_medium.png}
    %\label{fig:}
\end{figure}
\end{minipage}
\hfill
\begin{minipage}{0.49\textwidth}
	Photon density
    \begin{equation}
	\frac{d S}{d t} = v_g \Gamma_{MD} G(E) S.
	\end{equation}
	And also carrier density
	\begin{equation}
	\frac{d N_{MD}}{d t} = - v_g G(E) S.
\end{equation}
\end{minipage}

\phantom{239}

Y-directed wave in active layer obeys the equation:
\begin{equation}
	\triangle E(\vc{r},t) - \mu_0 \varepsilon(\vc{r}) \frac{\partial^2}{\partial t^2} E(\vc{r}, t) = \mu_0 \frac{\partial^2}{\partial t^2} P_{in}(\vc{r},t)
	\label{1.46}
\end{equation}

 \frametitle{The equations and ideas behind}}

\frame{
\begin{minipage}{0.45\textwidth}
    \begin{figure}[h]
    \centering
    \includegraphics[width=1\textwidth]{images/semi_medium.png}
    \caption{}
    %\label{fig:}
\end{figure}
\end{minipage}
\hfill
\begin{minipage}{0.49\textwidth}
    In the active layer we obtain the Schr\"odinger equation:
    \begin{equation*}
	H_{\text{crystal}} \Psi_n(\vc{r}) = \left[\frac{\vc{p}^2}{2 m_0} + U_p(\vc{r})\right] \Psi_n(\vc{r})
\end{equation*}
\end{minipage}
 \frametitle{Laser device geometry and induced polarization}}
% \frame{
% The \textit{demensional gain coefficient} is approximately:
\begin{equation}
	G \approx G_0 = \frac{1}{V_{MD}} \sum_{k_q, x, z} \frac{E}{E_0} \frac{|\mu(k_q, x, z)|^2}{\hbar c \varepsilon_0 n_r}[\rho_{ee} + \rho_{hh} - 1] (E - E_0)
\end{equation}

And the \textit{dimensional coupling factor}.
\begin{equation}
	\Gamma_{MD} = \frac{\varepsilon(x,z) |E_0(x,z)|^2}{\iint \varepsilon(x,z) |E_0(x,z)|^2 dx d z}\bigg|_{x,z=0,0}
\end{equation}
And goptical gain coefficient:
\begin{equation}
	g = g_0 + g_n(N-N_0) + g_p(S - S_0).
\end{equation} \frametitle{Coefficients}}
\frame{
And from the previous equation we can describe the quantum well laser behavior\footnote{
    Semiconductor
Lasers I
Fundamentals Ch.I by Bin Zhao, Amnon Yariv.
}
\begin{equation}
	\frac{d S}{d t} = \Gamma_{MD} G_0 v_g S - \frac{S}{\tau_p}
    \hspace{0.3 cm}
    \Rightarrow
    \hspace{0.3 cm}
    \frac{d S}{d t} = v_g g S(t) - \gamma_p S. 
\end{equation}
\begin{equation}
	\frac{d N}{d t} = \frac{J}{e} - \frac{N}{\tau_n} - \Gamma_{MD} G_0 v_g S 
    \hspace{0.3 cm}
    \Rightarrow
    \hspace{0.3 cm}
    \frac{d N}{d t}
    = \frac{J}{e d} - \gamma_s - g S.
\end{equation}
for previously mentioned optical gain coefficient:
\begin{equation}
	g = g_0 + g_n(N-N_0) + g_p(S - S_0).
\end{equation} \frametitle{The concluding formulas}}

\frame{
\textbf{Thr.} If there is no stationary points on the enclosed 2D region $G$ and some trajectory exists $\gamma \subset G$, then $\gamma$ is a closed loop path or tends to the closed one.

\phantom{239}

But there is still a hope for a chaos!

\phantom{239}

We add positive optoelectronic feedback in order to raise to 3D our equation
\begin{align}
	&\frac{d S}{d t} = v_g S g(S,N) - \gamma_p S,\\
	&\frac{d N}{d t} = \frac{J}{e d}[1 + \frac{\xi S(t-\tau)}{S_0}] - \gamma_n N - S g(S,N).
\end{align} \frametitle{Poincar\'e-Bendixson theorem}} можно и раскоментить, но зачем теория, если она и так валяется в конце?

\section{Feedback implementation}

\frame{
\begin{minipage}{0.51\textwidth}
   \begin{figure}[h]
       \centering
       \includegraphics[width=0.8\textwidth]{figures/plot_addition.pdf}
       %\caption{}
       %\label{fig:}
   \end{figure}
   
\end{minipage}
\hfill
\begin{minipage}{0.45\textwidth}
    \begin{center}
        \incfig{scheme2}
    \end{center}
\end{minipage}

\phantom{42}

It is interesting the burning of the laser, allowing an increase after feedback, therefore interests the value of order $0.85$ V. \frametitle{Concept}}

\frame{

After several experiments came to this scheme with the summing amplifier:
\begin{center}
    \incfig{scheme3}
\end{center}

Photodiode power is enough to not use an additional amplifier. \frametitle{Scheme}}

\frame{
For testing, the assembly was carried out on the dumping board.
\begin{center}
    \incfig{scheme6}
\end{center}

\hspace{-3mm}
\textbf{Thus, a scheme with positive feedback was implemented.} \\

\hspace{-3mm}
However, no desired oscillations were observed.

 \frametitle{Realization}}

\section{Modeling}


\begin{frame}
	\frametitle{Chaos modelling. Different regimes.}
	
%	Unimodular, multimodular and chaotic regimes of laser work:	
	\begin{figure}
		\centering
		\includegraphics[width=\linewidth]{figures/chaos_and_spectra.pdf}
	\end{figure}
	
	\begin{center}
		\begin{tabular}{c|c|c}
			Delay time $\tau$:\ \   $2 \ T_r$ & $7.5 \ T_r$ & $12 \ T_r$
		\end{tabular}
	\end{center}

%	Here dimensionless variable $s = \dfrac{S}{S_0}$.
	
	
\end{frame}	

\begin{frame}
	\frametitle{Chaos modelling.}
	Lyapunov exponents calculation :
	
\begin{figure}[h]
    \includegraphics[width=1.0\textwidth]{figures/lyapunovs.pdf}
    %\caption{}
    %\label{fig:}
\end{figure}



\begin{columns}
	\begin{column}{0.4\linewidth}
		\begin{tabular}{c|c|c|c}
		$\tau$ & $2 \ T_r$ & $7.5 \ T_r$ & $12 \ T_r$	\\ \hline
		$\lambda$ & $0.0$ & $1.62 \ f_r$ & $1.84 \ f_r$
		\end{tabular}	
	\end{column}
	\begin{column}{0.6\linewidth}
		$\Delta S \sim \exp(\lambda t) \ \Rightarrow \ \log \Delta S = \lambda t + \const$\\[5pt]
	\end{column}
\end{columns}	

\phantom{42}


\end{frame}	



\section{Branches of Light}
\frame{
\frametitle{The idea in nature}

\begin{figure}[h]
    \centering
    \includegraphics[width=0.8\textwidth]{images/nature_article.png}
    %\label{fig:}
\end{figure}
\textit{Patsyk, A., Sivan, U., Segev, M. et al. "Observation of branched flow of light". Nature 583, 60–65 (2020).}}

\frame{
\frametitle{Theory of refraction index}
\begin{minipage}{0.35\textwidth}
    \begin{figure}[h]
    \centering
    \includegraphics[width=1\textwidth]{images/nature_scheme.png}
    %\label{fig:}
\end{figure}
\end{minipage}
\hfill
\begin{minipage}{0.55\textwidth}
	The Helmholtz equation again
	\begin{equation*}
		\Delta E + k_0^2 n^2(y) E = 0
	\end{equation*}    
	while solving like $E = \psi(x,z) G(y)$ gives a solutions:
\end{minipage}
	\begin{equation*}
		\partial_{y y}G + k_0^2 n^2(y) G = k_0^2 n^2_{\text{eff}} G,
	\hspace{1 cm}
		\nabla_{\perp}^2 \psi + k_0^2 n_\text{eff}^2 \psi = 0.
	\end{equation*}
	And solving the left one we obtain
	\begin{equation*}
		k_0^2 y \sqrt{n_\text{soap}^2 - n_\text{eff}^2} + 2 \arctan\left(\frac{\sqrt{n_\text{soap}^2 - n_\text{eff}^2}}{\sqrt{n_\text{eff}^2 - n_\text{air}^2}}\right) - \pi(m+1) = 0.
	\end{equation*}
}


% \section{Slow adder}



\frame{
In real system\footnote{
    S. Tang, J. M. Liu, <<Chaotic pulsing and quasi-periodic route to chaos in a semiconductor laser with delayed opto-electronic feedback>>, 2001.
}  $S_0^{-1} \int_{0}^{\infty} f(\eta) S(t-\eta)d\eta$ instead of $S(t-\tau)$. \\
 It reduces oscillations.
 \frametitle{Real system}}



% % \section{Problems}

% \frame{
% With used amplifiers, the following oscillations at the amplifier output with DC power can be observed:

\begin{center}
    \incfig{scheme5}
\end{center}

This is due to the instability of the amplifier. 
% This instability can be eliminated by the adjustment of the scheme. 

\phantom{42}

The main problem is that desired oscillations $\sim 10$ ns. \\

\phantom{42}

We proceeded to experiments with faster amplifiers, but it is usefull to understnd results of such delays.




 \frametitle{Problems}}

% % \frame{
% % However, it was not possible to move to the chaotic regime in the laser. Possible cause of the problem may be
\begin{itemize}
    \iitem{parasitic capacity and inductance};
\end{itemize}

Capacitator capacity \frametitle{Problems}}


% \frame{
% In real system\footnote{
    S. Tang, J. M. Liu, <<Chaotic pulsing and quasi-periodic route to chaos in a semiconductor laser with delayed opto-electronic feedback>>, 2001.
}  $S_0^{-1} \int_{0}^{\infty} f(\eta) S(t-\eta)d\eta$ instead of $S(t-\tau)$. \\
 It reduces oscillations.
 \frametitle{Real system}}



\section{Results}

% добаввить 2 слайда:
% успехи численного моедлирования
    % показали, что хаос есть
    % численно нашли ограничения
    % про это не говорилось в статье
% успехи реализация
    % разработали схему
    % пос
    % стабильные оу
% ключевые ограничения:
    % 

% 



% \frame{
% Modeled good numerical solution. The existence of chaos is shown. The maximum <<slowness>> of the system is estimated.

\begin{minipage}[b]{0.49\textwidth}
      \begin{figure}[h]
          \centering
          \includegraphics[width=0.85\textwidth]{figures/modeling_vertical.pdf}
          \caption{Modeling}
      \end{figure}
\end{minipage}
\hfill
\begin{minipage}[b]{0.49\textwidth}
    \begin{figure}[h]
    \centering
    \includegraphics[width=0.85\textwidth]{images/tang_exp.png}
    \caption{Article experiment}
    %\label{fig:}
    \end{figure}
\end{minipage}

 \frametitle{Results in the modeling}}

% \frame{
% Principal restrictions in implementation:
\begin{itemize}
    \iitem{High-frequency oscillations,
        so, as we found, equipment for working at a frequency $\geq  (10 \tau_\text{r})^{-1} \approx 200$ MGz is necessary.
    }
    \iitem{No possibility for a specific laser change the frequency of oscillations.}
\end{itemize}


Successes on the way to implementation:
\begin{itemize}
    \iitem{Optimal values for $\tau_\text{delay}$ and laser range were found selected; stabilized operational amplifiers;}
    \iitem{A scheme has been developed, a slow positive feedback is implemented.}
    \iitem{A video signal transmission system is adjusted, also ovserved the constant behavior
    with $\tau_{\text{i}} \approx 50 \tau_{\text{r}}$
    .}
\end{itemize}

 \frametitle{Results in the practice work}}


\frame{
As a result of the project:
\begin{itemize}
    \iitem{According to the equations of the laser's evolution with a feedback, the \textbf{numerical model was built},
    compliant observations from the article
    , in the ideal and non ideal case.}
   
    \iitem{The another system was observed to show that chaotic systems is all around us. The branches of light were observed and modelled. Also we found an analogy with billiard theory.}
     \iitem{It is shown that \textbf{chaos is possible in the optical systems}. Chaos parameters are estimated. The frequency limitation were evaluated for the system. }
\end{itemize}



 \frametitle{Conclusion}}
\frame{
\begin{center}
	\includegraphics[width=0.3\textwidth]{figures/photo3.jpg}
	\hfill
	\includegraphics[width=0.35\textwidth]{figures/modeling_vertical.pdf}
	\hfill
	\includegraphics[width=0.3\textwidth]{images/equipment.png}	
\end{center}



% \centering
% \big{\text{Chaos is everywhere!} \frametitle{Chaos is everywhere!}}





\appendix
\section{Photos, thoughts and ideas}
\frame[noframenumbering]{
\begin{itemize}
	\iitem{W. Hirsch, S. Smale, \textit{"Introduction to Chaos"};}
	\iitem{"Chaos Through-Wall Imaging Radar" by Hang Xu et al. (2009)}
	\iitem{\textit{"Chaos-based communications"} by Apostolos Argyris et al. (2005);}
	\iitem{M. Pecora, L. Carroll, \textit{"Synchronization in Chaotic Systems"} (1990);}
	\iitem{\textit{"Physics and Applications of Laser Diode Chaos"} by
M. Sciamanna, and K. A. Shore (2015);}
	\iitem{\textit{"Chaotic Pulsing and Quasi-Periodic Route to
Chaos in a Semiconductor Laser with Delayed
Opto-Electronic Feedback"} by
S. Tang and J. M. Liu (2001)}
	\iitem{\textit{"Semiconductor Lasers I
Fundamentals"} Chapter I by Ch.I by Bin Zhao, Amnon Yariv (1998)}
\end{itemize} \frametitle{Literature}}
\frame[noframenumbering]{
\begin{figure}[h]
    \centering
    \includegraphics[width=0.8\textwidth]{images/newhope.jpg}
    \caption{Exactly from the article that we were inspired by...}
    %\label{fig:}
\end{figure}
 \frametitle{A new experimental set up came}}

\section{Theory expanded}

\frame[noframenumbering]{
To start the laser idea we need to obtain:
\begin{itemize}
	\iitem{Solution of the Schr\"odinger equation in a semiconductor medium for the wavefunction of an electron;}
	\iitem{Induced polarization for distribution of holes and electrons in a semiconductor;}
	\iitem{Interaction of electrons in a semiconductor with an wave equation and outer electric field.}
\end{itemize}
 \frametitle{The concept of a semiconductor laser}}

\frame[noframenumbering]{
We will need the Schr\"odinger equation:
\begin{equation*}
	H_{\text{crystal}} \Psi_n(\vc{r}) = \left[\frac{\vc{p}^2}{2 m_0} + U_p(\vc{r})\right] \Psi_n(\vc{r}),
\end{equation*}
where $\vc{p} = - i \hbar \nabla$ is the momentum operator, $m_0$ is the free electron mass, $U_p(\vc{r})$ is th periodic potential of the bulk semiconductor.

The solution is the Bloch function:
\begin{equation*}
	\Psi_{n, \vc{k}}(\vc{r}) = u_{n, \vc{k}}(\vc{r}) \frac{1}{\sqrt{V}} e^{i \vc{k} \cdot \vc{r}}.
\end{equation*} \frametitle{Electronic states in a semiconductor}}

\frame[noframenumbering]{
In case of an optical field the Hamiltonian changes to
\begin{equation*}
	H = \frac{[\vc{p} + e \vc{A}(\vc{r}, t)]^2}{2 m_0} + U_p(\vc{r}) = H_\text{crystal} + H'
\end{equation*}
$\vc{A}(\vc{r},t)$ is the vector potential of the optical field. So the interaction Hamiltonian 
\begin{equation*}
	H' \approx \frac{e}{m_0} \vc{A}(\vc{r},t) \cdot \vc{p}.
\end{equation*} \frametitle{Hamilton in an outer field}}

\frame[noframenumbering]{
For y-propagating field: $E(\vc{r}, t) = \frac{1}{2} E_0 e^{i (\beta y - \omega t)} + \const$ the Hamiltonian is
\begin{equation*}
	H' = \frac{1}{2} \mu(k_q,x,z) [E_0 e^{i (\beta y - \omega t)} + \const]
\end{equation*}
where $\mu$ is the transition matrix that describes the semiconductor, $k_q$ -- quantized wavevector of the electron. \frametitle{Hamilton in an outer field}}

\frame[noframenumbering]{
As one can obtain after rewritten density matrix in terms of carrier distributions, and in order avoid irrelevantly enormous formulas we get polarization as:
\begin{equation*}
	\mathcal{P}_{in}(\vc{r}, t) = \frac{1}{2} \mathcal{P}_{in,0} e^{i (\beta y - \omega t)} + \const =
\end{equation*}
\begin{equation*}
	= - \sum \frac{\xi(\vc{r},k_q, x, z)}{V(k_q, x, z)}[\rho_{eh}(k_q, x, z)\mu(k_q, x, z) + \const]	
\end{equation*}
where $V(k_q, x, z)$ is the confinement volume of electrons and holes and
\begin{equation*}
	\xi(\vc{r}, k_q, x, z) = 
	\left\{ \begin{aligned}
		&1, \ \vc{r} \text{ inside V}\\
		&0, \ \vc{r} \text{ outside V}
	\end{aligned}
	\right.
\end{equation*}

 \frametitle{Induced polarization in an outer field}}

\frame[noframenumbering]{
For the wave porpagating along the y direction in active layer the equation is:
\begin{equation}
	\triangle E(\vc{r},t) - \mu_0 \varepsilon(\vc{r}) \frac{\partial^2}{\partial t^2} E(\vc{r}, t) = \mu_0 \frac{\partial^2}{\partial t^2} P_{in}(\vc{r},t)
	\label{1.46}
\end{equation}
The partial solution for this equation we will be searching in a form of
\begin{equation*}
	E(\vc{r}, t) = A_0(t) E_{\text{eig}}(x,z) = \frac{1}{2} E_0 e^{i (\beta y - \omega t)} + \const.
\end{equation*} \frametitle{Injection the light}}

\frame[noframenumbering]{
Substituting in wave equation\eqref{1.46} the obtained polarization and solution for $E(\vc{r}, t)$ and assuming that $A(t)$ changes slowly we get
\begin{equation}
	\frac{d A_0}{d t} = \frac{i \omega}{2 \varepsilon_0 n_r^2} A_0 \sum_{k_q, x, z}\Gamma_{MD} \frac{1}{V_{MD}} |\mu(k_q, x, z)|^2[\rho_{ee} + \rho_{hh} - 1]
\end{equation}
where 
\begin{equation*}
	\Gamma_{MD} = \frac{\varepsilon(x,z) |E_0(x,z)|^2}{\iint \varepsilon(x,z) |E_0(x,z)|^2 dx d z}\bigg|_{x,z=0,0}
\end{equation*}
is a \textit{dimensional coupling factor}. \frametitle{Partial solution part}}

\frame[noframenumbering]{
Now we can rewrite
\begin{equation}
	\frac{d A_0}{d t} = \frac{1}{2} v_g [\Gamma(0) G - i \Gamma(0) N_r]A_0,
\end{equation}
where $v_g = c/n_r$ for $c$ -- the speed of light in vacuum, and $n_{r}$ -- refraction coefficient. We will call the \textit{gain coefficient} $g = \Gamma(0)G$.
From Schr\"odinger equation we obtain the photon density as:
\begin{equation*}
	S = \frac{1}{2} \frac{\varepsilon_0 n_r^2 |A_0 E(0)|^2}{E_0}.
\end{equation*} \frametitle{Getting the solution}}

\frame[noframenumbering]{
Using the equation for $A_0$ we obtain for the photon density (real part):
\begin{equation}
	\frac{d S}{d t} = v_g \Gamma(0) G S.
\end{equation}
and for electrons (holes) density (complex part):
\begin{equation}
	\frac{d N}{d t} = - v_g G S.
\end{equation}

And the demensional gain coefficient is approximately:
\begin{equation}
	G = G_0 - G_1 \frac{S}{S_s}.
\end{equation} \frametitle{The laser equations}}

\frame[noframenumbering]{
Irrelevantly enormous
formulas if someone really need it:

\begin{equation*}
	\frac{d}{d t}[\rho_{ee}(k_q, x, z)] = \frac{i}{\hbar}[H' \rho_{eh}(k_q, x, z) - \const] - \frac{\rho_{ee}(k_q, x, z) - f_e}{\tau_e},
\end{equation*}
\begin{equation*}
	\frac{d}{d t}[\rho_{hh}(k_q, x, z)] = \frac{i}{\hbar}[H' \rho_{eh}(k_q, x, z) - \const] - \frac{\rho_{hh}(k_q, x, z) - f_e}{\tau_e},
\end{equation*}
\begin{equation*}
	\frac{d}{d t}[\rho_{eh}(k_q, x, z)] = \frac{i}{\hbar} H'[\rho_{ee} + \rho_{hh} - 1] - \frac{i}{\hbar} E_{tr} \rho_{e h} - \frac{\rho_{eh}}{T_{deph}}.
\end{equation*} \frametitle{Carrier density in an outer field}}

\frame[noframenumbering]{
And the demensional gain coefficient is approximately:
\begin{equation}
	G \approx G_0 = \frac{1}{V_{MD}} \sum_{k_q, x, z} \frac{E}{E_0} \frac{|\mu(k_q, x, z)|^2}{\hbar c \varepsilon_0 n_r}[\rho_{ee} + \rho_{hh} - 1] (E - E_0)
\end{equation}

And from the previous equation we can describe the quantum well laser behavior\footnote{
    Semiconductor
Lasers I
Fundamentals Ch.I by Bin Zhao, Amnon Yariv.
}
\begin{equation*}
	\frac{d S}{d t} = \Gamma_{MD} G_0 v_g S - \frac{S}{\tau_p} = v_g g S(t) - \gamma_p S. 
\end{equation*}
\begin{equation*}
	\frac{d N}{d t} = \frac{J}{e} - \frac{N}{\tau_n} - \Gamma_{MD} G_0 v_g S = \frac{J}{e d} - \gamma_s - g S.
\end{equation*}
for previously mentioned optical gain coefficient:
\begin{equation*}
	g = g_0 + g_n(N-N_0) + g_p(S - S_0).
\end{equation*} \frametitle{The laser equations}}

\frame[noframenumbering]{
\textbf{Thr.} If there is no stationary points on the enclosed 2D region $G$ and some trajectory exists $\gamma \subset G$, then $\gamma$ is a closed loop path or tends to the closed one.

\phantom{239}

But there is still a hope for a chaos!

\phantom{239}

We add positive optoelectronic feedback in order to raise to 3D our equation
\begin{align}
	&\frac{d S}{d t} = v_g S g(S,N) - \gamma_p S,\\
	&\frac{d N}{d t} = \frac{J}{e d}[1 + \frac{\xi S(t-\tau)}{S_0}] - \gamma_n N - S g(S,N).
\end{align} \frametitle{Poincar\'e-Bendixson theorem}}

\frame[noframenumbering]{
\begin{minipage}{0.51\textwidth}
   \begin{figure}[h]
       \centering
       \includegraphics[width=0.8\textwidth]{figures/plot_addition.pdf}
       %\caption{}
       %\label{fig:}
   \end{figure}
   
\end{minipage}
\hfill
\begin{minipage}{0.45\textwidth}
    \begin{center}
        \incfig{scheme2}
    \end{center}
\end{minipage}

\phantom{42}

It is interesting the burning of the laser, allowing an increase after feedback, therefore interests the value of order $0.85$ V. \frametitle{Range}}


\end{document}



