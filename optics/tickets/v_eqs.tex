\section{Скоростные уравнения}

\subsection{Межуровневые переходы фотонов}


Что умеет электрон? Когда в него влетает фотон электрон умеет поглощать фотон, с некоторой вероятностью, характеризовать этот процесс будем некоторым сечением поглощения $\sigma$, -- поглощение, \textit{индуцированное} фотоном. 


Аналогично для \textit{вынужденного излучения} есть некоторая $\sigma$, которая такая же как и для поглощения. А ещё, что здорово, излученные фотоны практически идентичны. 

Также есть  спонтанное излучение куда-то  с $h \nu$. А ещё есть безизлучательное излучение, когда энергия уходит в тепло. Всё происходит с какой-то вероятностью, за некоторое свойственное характерное время $\tau$. 


Сечение перехода $\sigma (\nu)$ характеризует взаимодействие атома с излучением. Площадь под графиком
\begin{equation*}
    S = \int_0^\infty \sigma(\nu) \d \nu,
\end{equation*}
назывется \textit{силой перехода} или \textit{силой осциллятора}. Функция $g(\nu) = \sigma(\nu)/S$ характеризует относительную величину взаимодействия с фотонами различных частот. 

Функция $g(\nu)$ центрирована на резонансе $\nu_0$, где $\sigma(\nu_0)$ -- максимум, таким образом ширина $g(\nu)$ -- ширина линии перехода. 

\textbf{Коэффициенты Эйнштейна}. Для начала заметим, что в монохроматическом случае
\begin{equation*}
    W_i = \varphi \sigma(\nu),
\end{equation*}
где $\varphi$ -- поток фотонов через единичную площадь, $\sigma(\nu)$ -- эффективная площадь поперечного сечения атома, а $W_i = \varphi \sigma(\nu)$ --  часть фотонного потока <<захваченного>> атомом с целью поглощения или вынужденного излучения. 


В немонохроматическом случае можем получить аналогичную формулу, вида
\begin{equation*}
    W_i = \frac{\lambda^3}{8 \pi h \sub{t}{сп}} \rho(\nu_0),
\end{equation*}
где $\lambda = c/\nu_0$ и $\rho(\nu_0)$ -- спектральная плотность света на резонансной частоте. 

Удобно ввести \textit{коэффициенты Эйнштейна}, как
\begin{equation*}
    \sub{P}{сп} = \mathbb{A}; \hspace{5 mm} W_i = \mathbb{B} \rho(\nu_0);
\end{equation*}
где, выражая через время спонтанного перехода, 
\begin{equation*}
    \mathbb{A} = \frac{1}{\sub{t}{сп}};  \hspace{5 mm} \mathbb{B} = \frac{\lambda^3}{8 \pi h} \frac{1}{\sub{t}{сп}},
\end{equation*}
где отношение $\mathbb{A}$ к $\mathbb{B}$  -- результат микроскопических вероятностных законов. 

Итого, атомный переход описывается резонансной частотой $\nu_0 = (E_2-E_1)/h$, спонтанным временем жизни $\sub{t}{сп}$ и функцией формы линии $g(\nu)$ с шириной $\Delta \nu$, также удобно ввести среднее сечение перехода:
\begin{equation*}
    \bar{\sigma}(\nu) = \frac{\lambda^2}{8 \pi} \frac{1}{\sub{t}{сп}} g(\nu).
\end{equation*}

\subsection{Скоростные уравнения}


Скоростные (кинематические) уравнения:
\begin{align*}
    \dot{n}_2 = - \frac{n_2}{\tau_{\text{Б}}} - \frac{n_2}{\tau_{\text{С}}} - \ldots,
\end{align*}
что соответсвует $n_2 = n_{20} \exp(-t/\langle \tau\rangle_{\textnormal{harm}})$. 

Пусть $F = I/(h \nu)$ -- плотность потока фотонов, тогда 
\begin{equation*}
    \dot{n}_1 = - F \sigma n_1,
    \hspace{0.5cm} \Rightarrow \hspace{0.5cm}
    n_1 (t) = n_1(0) \exp(-F \sigma t),
\end{equation*}
и, соответсвенно, при различных случаях
\begin{equation*}
    \dot{n}_1 = - F \sigma n_1 + F \sigma n_2 + \frac{n_2}{\tau_{\text{Б}}}.
\end{equation*}



\textbf{В отсутствие усиливаемого излучения}. Пусть $R_2$ -- скорость возрастания плотностей зачеленности уровня 2,
$R_1$ -- скорость с которой с первого уровня уходят электроны, тогда под действием накачки:
\begin{equation*}
    \left\{\begin{aligned}
        \dot{n}_2 &= R_2 - n_2/\tau_2
        \dot{n} &= -R_1 - n_1/\tau_1 + n_2/t_{21},
    \end{aligned}\right.
    \hspace{0.5cm} \Rightarrow \hspace{0.5cm}
    n_0 = \lim n_2-n_1 = R_2 \tau_2 \left(1 - \frac{\tau_1}{\tau_{21}}\right) + R_1 \tau_1.
\end{equation*}


Например, при $R_1 =0$ и $R_2$ реализуется возбуждением из основного состояния $E=0$ на уровень 2 фотонами частоты $E_2/h$ поглощаемых с вероятностью $W$. Пусть также $\tau_2 \approx \sub{t}{сп}$  и $\tau_1 \ll \sub{t}{сп}$, так что в стационаном состоянии $n_1 \approx 0$ и $n_0 \approx R_2 \sub{t}{сп}$. Тогда разность заселенности 
\begin{equation*}
    n_0 \approx (n_1 + n_2 + n_3) \frac{\sub{t}{сп} W}{1 + 2 \sub{t}{сп} W}.
\end{equation*}

\textbf{В случае присутствия усиливаемого излучения}.  Здесь скоростные уравнения примут вид
\begin{align*}
    \frac{d n_2}{d t} &= R_2 - \frac{n_2}{\tau_2} - n_2 W_i + n_1 W_i, \\
    \frac{d n_1}{d t} &= - R_1 - \frac{N_1}{\tau_1} + \frac{N_2}{\tau_{21}} + N_2 W_i - N_1 W_i.
\end{align*}
где $W_i = \varphi \sigma(\nu)$. В стационарном режиме можем получить
\begin{equation*}
    N = \frac{N_0}{1 + \tau_s W_i}, \hspace{5 mm} 
    \tau_s = t_2 + \tau_1 \left(1 - \frac{\tau_2}{\tau_{21}}\right).
\end{equation*}
В случае $\tau_s W_i \ll 1$ (\textit{приближение слабого сигнала}) можем считать $N \approx N_0$, но при росте $W_i$ можно заметить, что $N \to 0$ независимо от знака $N_0$. Величина $\tau_s$ -- время насыщения. 



\subsection{Внутридоплеровская спектроскопия}

\textbf{Постановка задачи}: есть куча бегающих атомов, несколько уровней энергии. Вообще можем следить за излучением, можем за поглощением. Из-за уширения разные линии уровня могут слиться в одну. Хотелось бы этого избежать.

Пусть $n_1 + n_2 = N$. Можем записать некоторое уравнение баланса $n_2 = N - n_1$
\begin{equation*}
    \left\{\begin{aligned}
        \dot{n}_1 &= - n_1 \sigma F + n_2 \sigma F + {n_2}/{\tau_{\text{Б}}}, \\
        \dot{n}_2 &= + n_1 \sigma F - n_2 \sigma F - {n_2}/{\tau_{\text{Б}}},        
    \end{aligned}\right.
    \hspace{0.1cm} \Rightarrow \hspace{0.1cm}
    \dot{n}_1 + n_1 (2 \sigma F + \tau_{\text{Б}}^{-1}) = N \sigma F + \frac{N}{\tau_{\text{Б}}},
    \hspace{0.1cm} \Rightarrow \hspace{0.1cm}
    {n}_1 = N \frac{\sigma F + \tau_{\text{Б}}^{-1}}{2 \sigma F + \tau_{\text{Б}}^{-1}} = 
    N \frac{F + \frac{1}{\sigma \tau_{\text{Б}}}}{2F + \frac{1}{\sigma \tau_{\text{Б}}}},
\end{equation*}
что соответсвует $\dot{n}_1 = \dot{n}_2 = 0$.


Если посмотреть на $n_1(F)$ и $n_2(F)$, то они стремятся к $N/2$, при $F \to + \infty$. А вообще, считая $F_s^{-1} = \sigma \tau_{\text{Б}}$, можем записать
\begin{equation*}
    n_1 (F) = N \frac{F + F_s}{2 F + F_s}.
\end{equation*}
Введя $\alpha = n_1 \sigma - n_2 \sigma$ -- коэффициент поглощения. В термодинамике была $n \sigma \lambda$ обратно пропорционально длине свободного пробега
\begin{equation*}
    d F =  - (n_1 \sigma - n_2 \sigma) F \d z,
    \hspace{0.5cm} \Rightarrow \hspace{0.5cm}
    d F = - n_1 \sigma F \d z + n_2 \sigma F \d z.
\end{equation*}
Получается, что $\alpha$ стремится к $0$ во времени. 


Таким образом, если ооочень сильно светить на вещество, то можно его пробить при большой интенсивности. 

При равенстве частоты лазера и частоты атома возникнет провал Лэмба. В какой-то момент научились менять частоту лазера, и научились так измерять $\omega_0$. Если частот будет две, то также можем зафиксировать два провала Лэмба.


\subsection{Затухание интенсивности}

Вспомним кинематическое уравнение:
\begin{equation*}
    \dot{n}_2 = - A n_2 = -\frac{n_2}{\tau_{\text{Б}}},
    \hspace{10 mm}
    \dot{n}_2 =  - \frac{\tau_{\text{Б}}}{\tau_{\text{сп}}}
\end{equation*}
где $\tau_{\text{Б}}$ -- характерное время безизлучательного перехода, $\tau_{\text{сп}}$ -- спонтанного перехода.
Также есть поглощение:
\begin{equation*}
    \dot{n}_1 = - \sigma F n_1,
\end{equation*}
и вынужденное излучение
\begin{equation*}
    \dot{n}_2 = - \sigma F n_2.
\end{equation*}
Вообще $A, B, \sigma F$ -- коэффициенты Эйнштейна. 


Вспомним про <<просветление>>, запишем набор кинематических уравнений
\begin{equation*}
    \left\{\begin{aligned}
        \dot{n}_1 &=  - n_1 \sigma F + n_2 \sigma F + n_2/\tau  \\
         N &= n_1 + n_2
    \end{aligned}\right.
    \hspace{0.5cm} \Rightarrow \hspace{0.5cm}
    \dot{n}_1 = \dot{n}_2 = 0,
    \hspace{5 mm} \text{--- \ \ стационарное приближение}.
\end{equation*}
решая эту систему находим, что
\begin{equation*}
    n_1 (F) = N \frac{F \sigma \tau + 1}{1 + 2 F \sigma \tau},
    \hspace{5 mm}
    n_2 (F) = N \frac{F \sigma \tau}{1 + 2 F \sigma \tau}.
\end{equation*}
Также, вспоминая про поглощение, понимая как происходит поглощение фотонов, 
\begin{equation*}
    \d F = D (\sigma n_2 - \sigma n_1) \d z,
\end{equation*}
где $\alpha \overset{\mathrm{def}}{=}  \sigma n_1 - \sigma n_2 = \sigma N$, так приходим к стандартному закону
\begin{equation*}
    d F = - F \alpha \d z,
    \hspace{0.5cm} \Rightarrow \hspace{0.5cm}
    F = F_0 \exp \left(- \alpha h\right),
\end{equation*}
где $h$ -- толщина образца. И вообще можно найти $\alpha (F)$, который просто монотонно убывает.





\subsection{Затемнение}

Добавим к системе третий уровень. Договоримся, что электроны умеют переходить с 1 на 2 уровень, то на самом деле он запрыгивает чуть выше второго, и сваливается, но не с частотой $h \nu$, поэтому вынужденного излучения тут не будет. 

Также за $\tau_2$ происходит излучение с 3 на 2 не с $h \nu$. Итого остаются процессы с $\sigma_2$, $\sigma_1$, $\tau_1$ и $\tau_2$. На практике $\tau_2 \ll \tau_1$, примерно как 1 пс $\ll$ 1 нс.

Запишем кинематические уравнения:
\begin{align*}
    \dot{n}_1 &= - n_1 \varphi_1 F +  \frac{n_2}{\tau_1}, \\
    \dot{n}_2 &= n_1 \sigma_1 F - n_2 \sigma_2 F - \frac{n_2}{\tau_1} + \frac{n_3}{\tau_2}, \\
    n_1 + n_2 + n_3 = N,
\end{align*}
где мы сразу перейдём к стационарному приближнию $\dot{n}_1 = \dot{n}_2 = \dot{n}_3 = 0$. 
Считая
\begin{equation*}
    \frac{1}{\sigma_1 \tau_1} = F_1, \hspace{5 mm} 
    \frac{1}{\sigma_2 \tau_2} = F_2, \hspace{5 mm}
    n_2 = n_1 \frac{F}{F_1}, \hspace{5 mm}
    n_3 = n_1 \frac{F^2}{F_1 F_2}
\end{equation*}
и, наконец, получаем
\begin{equation}
    n_1 = N \left(
        1 + \frac{F}{F_1} + \frac{F^2}{F_1 F_2}
    \right)^{-1}.
\end{equation}
Что с $n_1$ -- она затухает, также $n_2$ слегка растёт, а потом затухает, а $n_3$ выходит на $N$. 
Если внимательно посмотреть на $\alpha (F) = \sigma_1 n_1 \red{\pm} \sigma_2 n_2$,
\begin{equation*}
    \frac{d F}{d z} = - \alpha F.
\end{equation*}
Так как $\tau_2$ мааленький, то $n_3$ растёт оочень медленно. Для $\alpha$ получается немонотонная зависимость. 
На деле $n_3$ всегда ноль, т.к. $F \ll \sqrt{F_1 F_2}$. В итоге $n_2$ идёт к $N$, $n_1$ к 0. 

Так вот, $\alpha = \sigma (n_1 - n_2)$. Также $\d F = - \alpha F \d z$. Тогда
\begin{align*}
    \textnormal{if} \ \ n_1 > n_2,
    \hspace{0.5cm} \Rightarrow \hspace{0.5cm}
    \alpha > 0, \hspace{5 mm} \frac{d F}{d z} < 0, \\
    \textnormal{if} \ \ n_1 < n_2,
    \hspace{0.5cm} \Rightarrow \hspace{0.5cm}
    \alpha < 0, \hspace{5 mm} \frac{d F}{d z} > 0.
\end{align*}
Если $n_2 > n_1$, то переходим к ситуации с инверсией населенности. 

Теперь устроим её. Пусть со второго на третий уровень происходит очень быстрое сваливание (третий ниже второго). 
Снова запищем кинематические уравнения
\begin{gather*}
    n_1 + n_3 = N \\
    \dot{n}_1 = - n_1 W + \frac{n_3}{\tau} + n_3 \sigma F = 0.
\end{gather*}
Но такие схемы уже редки, чаще сейчас используют четырех уровневые системы.

