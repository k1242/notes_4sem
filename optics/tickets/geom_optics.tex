\section{Геометрическая оптика}

\subsection{Принцип Ферма}

Вспомним уравнения Максвелла
\begin{equation*}
    \left\{\begin{aligned}
            \div \vc{D} &= 4 \pi \rho \\
            \div \vc{B} &= 0 \\
            \rot \vc{E} &= - \frac{1}{c}\frac{\partial \vc{B}}{\partial t} \\
            \rot \vc{H} &= \frac{4\pi}{c} \vc{j} + \frac{1}{c} \frac{\partial \vc{D}}{\partial t}.
    \end{aligned}\right.
    \hspace{0.5cm} \Rightarrow \hspace{0.5cm}
    \rot \rot \vc{E} = - \frac{1}{c} \frac{\partial }{\partial t} \rot \vc{B}
    \hspace{0.5cm} \Rightarrow \hspace{0.5cm}
    \nabla^2 \vc{E} - \frac{\varepsilon \mu}{c^2} \frac{\partial^2 \vc{E}}{\partial t^2} = 0
\end{equation*}

\textbf{Путь напрямик}.
Подставляя $E = a(\vc{r}) e^{i(\omega t - \smallvc{k}_0 \Phi(\smallvc{r}))}$ в уравнение Гельмгольца можем получить следующую систему:
\begin{align*}
    (\grad \Phi)^2 = n^2 + \frac{\Delta a}{k_0^2 a} \\
    a \Delta \Phi + 2 (\grad a) (\grad \Phi) = 0
\end{align*}
В котором хотим увидеть
\begin{equation*}
    \bigg| \frac{\Delta a}{k_0^2 a} \bigg| = \bigg| \frac{\Delta a}{a} \bigg| \frac{\lambda^2}{4 \pi^2} \ll n^2 
    \hspace{2.5 mm} \Leftrightarrow \hspace{2.5 mm} 
    \left\{\begin{aligned}
        \lambda |\partial_x a| \ll a, \\
        \lambda |\partial_{xx}^2 a| \ll |\partial_x a|,
    \end{aligned}\right.
\hspace{2.5 mm} \Rightarrow \hspace{2.5 mm}
(\grad \Phi)^2 = n^2,
\hspace{2.5 mm} \Leftrightarrow \hspace{2.5 mm} 
\grad \Phi = n \vc{s}.
\end{equation*}
Из второго уравнения можем получить независимость поведения лучей друг относительно друга. 


\textbf{В терминах 4-векторов}. Давайте скажем, что наше скалярное поле это
$$f = a e^{i(\smallvc{k} \smallvc{r} - \omega t + \alpha)} = a e^{i \Phi}$$, где $\Phi$ -- Эйконал. Тогда $\grad \Phi = \vc{k}$, $\partial_t \Phi = - \omega$, и вполне естественно говорить про 4-вектор $k^i = \left(\frac{\omega}{c},\, \vc{k}\right)\T$, тогда $k_i x^i = \omega t - \vc{k} \cdot \vc{r}$ и 
\begin{equation*}
    k_i k^i = \frac{\omega^2}{c^2} - k^2 = 0,
    \hspace{0.5cm} \Rightarrow \hspace{0.5cm}
    \partial_i \Phi \partial_i \Phi = 0.
\end{equation*}
Тогда пусть $\Phi = - \omega t + \Phi_0 (x, y, z)$, соответственно $\partial^i \Phi = \left(\frac{\omega}{c},\, \frac{\partial \Phi_0}{\partial \smallvc{r}} \right)$, так приходим к
\begin{equation*}
    \left(\grad \Phi_0\right)^2 = \frac{\omega^2}{c^2} =  k_0^2,
    \hspace{0.5cm} \overset{c \to c/n}{\Rightarrow}  \hspace{0.5cm}
    \left(\grad \Phi_0 / k_0\right)^2 = n^2.
\end{equation*}


\textbf{Принцип Ферма}. Рассмотрим интеграл по замкнотому контору
\begin{equation*}
    \oint \grad \Phi \cdot \d \vc{l} = \oint n (\vc{s} \cdot \d \vc{l}) = 0
\end{equation*}
Но это соответсвует экстремальности $\int n \d l$, то есть
\begin{equation*}
    \delta \int n \d l  = 0,
\end{equation*}
для прямого пути. 

Чуть более явно можем выписать $\grad \Phi = n \vc{s}$, тогда $d_l \Phi = n$, и $\vc{s} = d_l \vc{r}$, тогда
\begin{equation*}
    \frac{d }{d l} \left(n \frac{d \vc{r}}{d l} \right) = \frac{d }{d l}  \grad \Phi = \grad n.
\end{equation*}

\textbf{Координатная запись}. Заметим, что $\d l = \sqrt{1 + (y'_x)^2} \d x$, тогда
\begin{equation*}
    \delta \int n \sqrt{1 + (y'_x)^2} \d x = 0,
\end{equation*}
что соответсвует уравнению Эйлера-Лагранжа:
\begin{equation*}
    \frac{d }{d x}  \frac{\partial L}{\partial y'_x} - \frac{\partial L}{\partial y} = 0,
    \hspace{5 mm}   
    L = n \sqrt{1 + (y'_x)^2}, \hspace{5 mm} 
    n \equiv n(x, y).
\end{equation*}


\textbf{Оптический элемент с переменным $n$}. Рассмотрим, например $n^2 = n_0^2 (1-\alpha^2 y^2)$, рассмотрим параксиальное приближение и увидим
\begin{equation*}
    y''_{xx} = \frac{1}{n} \frac{d n}{d y},
    \hspace{0.5cm} \Rightarrow \hspace{0.5cm}
    y''_{xx} = - \alpha^2 y,
\end{equation*}
что показывает, что мы снова свели всё к гармоническому осциллятору. 

% Если отойти от геометрической оптики, то получим дифф


\subsection{Матричная оптика}

% 𝐴𝐵𝐶𝐷 матрицы. Матрицы простых оптических элементов. Матрицы периодических оптических систем. Условие устойчивости траектории оптической системы. Устойчивые и неустойчивые резонаторы. Условие гармонической траектории.
 Жизнь в параксиальном приближение позволяет жить в линейном мире, где особенно удобен аппарат линейных преобразований, который и предлагается построить.

 \textbf{Частные случаи}. Пусть луч распространяется в однородной среде, под $\theta_1$ распространяется, тогда $\theta_2 = \theta_1$. Что произошло с $y$? Ну, $y_2 = y_1 + l \theta_1$, тогда
\begin{equation}
    \begin{pmatrix}
        y_2 \\ n_2 \theta_2
    \end{pmatrix} = 
    \begin{pmatrix}
        1 & l/n \\
        0 & 1 \\
    \end{pmatrix} 
    \begin{pmatrix}
        y_1 \\ n_1 \theta_1 
    \end{pmatrix},
    \hspace{1 cm}
    \begin{pmatrix}
        y_2 \\ \theta_2
    \end{pmatrix} = 
    \begin{pmatrix}
        1 & l \\
        0 & 1 \\
    \end{pmatrix} 
    \begin{pmatrix}
        y_1 \\ \theta_1 
    \end{pmatrix}, 
\end{equation}
где соответствующую матрицу обозначим за $T$. 

Матрица преломления на сферической поверхности $P$:
\begin{equation}
    P(y, V) = \begin{pmatrix}
        1 & 0 \\
        -\frac{n_2-n_1}{R} & 1 \\
    \end{pmatrix},
    \hspace{1 cm}
    P(y, \theta) = 
    \begin{pmatrix}
        1 & 0 \\
        \frac{n_2-n_1}{n_2 R} & \frac{n_1}{n_2} \\
    \end{pmatrix},
\end{equation}
где $(n_2-n_1)/R$ называют \textit{оптической силой}.


Была некоторая матрица преломления
\begin{equation*}
     P(y, V) = \begin{pmatrix}
        1 & 0 \\
        -\frac{n_2-n_1}{R} & 1 \\
    \end{pmatrix},
\end{equation*}
и матрица распространения
\begin{equation*}
    T = \begin{pmatrix}
        1 & l/n \\
        0 & 1 \\
    \end{pmatrix}.
\end{equation*}
В случае отражения видимо хочется заменить $n \to -n$. Знак $\theta$ определяется, как вниз, или вверх.
Пусть теперь $n_1=n,$ $n_2=-n$, тогда матрица отражения
\begin{equation*}
    R = \begin{pmatrix}
        1 & 0 \\
        -(-n-n)/r & 1 \\
    \end{pmatrix} = \begin{pmatrix}
        1 & 0 \\
        2n/r & 1 \\
    \end{pmatrix}.
\end{equation*}
Так фокусное расстояние для сферического зеркала $R/2$, что логично. В случае же, если мы захотим следить за направлениями осей, то можно вернуться к переменным $\{z, \theta\}$.


\textbf{Общий подход}. 
Обратим внимание на порядок применения матриц: 
\begin{equation*}
    \underbrace{M_3 M_2 M_1}_{M} \vc{a} = \vc{b},
    \hspace{1 cm}
    \Leftrightarrow
    \hspace{1 cm}
    \vc{a} = M_1^{-1} M_2^{-1} M_3^{-1} \vc{b}.
\end{equation*}
Посмотрим на коэффициенты, приравнивая их к 0. Пусть 
\begin{equation*}
    \begin{pmatrix}
        y_2 \\ \theta_2
    \end{pmatrix} = 
    \begin{pmatrix}
        A & B \\
        C & 0 \\
    \end{pmatrix}
    \begin{pmatrix}
        y_1 \\ \theta_1
    \end{pmatrix},
\end{equation*}
тогда $\theta_2 = c y_1$, тогда при $D=0$,  получается, что $O\Pi_1$ -- фокальная плоскость (слева).

Пусть $B=0$, тогда $y_2 = A y_1$, тогда это изображение, и говорим, что эти \textit{плоскости сопряженные}, а коэффициент $A$ -- \textit{коэффициент поперечного увеличения}. 

Пусть $C=0$, тогда $\theta_2 = \theta_1 D$, что соответствует телескопической системой, а коэффициент $D$ -- \textit{коэффициент углового увеличения}.

Теперь рассмотрим $A = 0$, получается, что это фокальная плоскость справа. 



\textbf{Периодические системы}. 
Пусть есть некоторая периодическая система с матрицей $ABCD$, действующая на луч $\{y_0, V_0\}$
\begin{equation*}
    \begin{pmatrix}
        y_m \\ V_m
    \end{pmatrix} = 
    \begin{pmatrix}
        A & B \\
        C & D \\
    \end{pmatrix}^m
    \begin{pmatrix}
        y_0 \\ V_0
    \end{pmatrix}.
\end{equation*}
Хотелось бы понять на устойчивость такого действия системы на луч, для этого посмотрим на 
\begin{equation*}
    \begin{pmatrix}
        y_{m+1} \\ V_{m+1}
    \end{pmatrix} = 
    \begin{pmatrix}
        A & B \\
        C & D \\
    \end{pmatrix} \begin{pmatrix}
        y_m \\ V_m
    \end{pmatrix},
    \hspace{0.5cm} \Rightarrow \hspace{0.5cm}
    \left\{\begin{aligned}
        y_{m+1} = A y_m + B V_m \\
        V_{m+1} = C y_m + D V_m
    \end{aligned}\right.
    \hspace{0.5cm} \Rightarrow \hspace{0.5cm}
    V_m = \frac{y_{m+1}-Ay_m}{B},
\end{equation*}
теперь можем, забыв про $V$, говорить про $y_m$
\begin{equation*}
    \frac{y_{m+1}-A y_{m+1}}{B} = C y_m + \frac{D (y_{m+1}-Ay_m)}{B},
    \hspace{0.5cm} \Rightarrow \hspace{0.5cm}
    y_{m+2} - A y_{m+1} = BC y_m + D (y_{m+1} - A y_m)
\end{equation*}
и, наконец,
\begin{equation*}
    y_{m+2} - (A+D) y_{m+1} + 
    \cancelto{1}{(A D - B C)}
     y_m = 0,
\end{equation*}
которое решается также, как и диффур, подстановкой $y_m = y_0 h^m$, тогда
\begin{equation*}
    y_m = y_0 h^m, \hspace{0.5cm} \Rightarrow \hspace{0.5cm}
    h^2 - (A+D) h + 1 = 0,
\end{equation*}
считая $\tr M = A+D \overset{\mathrm{def}}{=} 2 b$, находим, что
\begin{equation*}
    h^2 - 2 b h  +1 = 0, 
    \hspace{0.5cm} \Rightarrow \hspace{0.5cm}
    h_{1, 2} = b \pm \sqrt{b^2-1}
\end{equation*}
что приводит нас к некоторому к следующей классификации

$|b|>1$ -- неустойчивый режим

$|b|=1$ -- граница

$|b|<1$ -- устойчивость

\noindent
Рассмотрим $b < 1$ и для удобства положим $b = \cos \varphi$, тогда
\begin{equation*}
    h_{1, 2} = \cos \varphi \pm i \sin \varphi = e^{i\varphi},
    \hspace{0.5cm} \Rightarrow \hspace{0.5cm}
    y_m = \alpha_1 e^{im \varphi} + \alpha_2 e^{-im \varphi} = 
    \underline{y_{\textnormal{max}}} \sin (m \varphi + \underline{\varphi_0}),
\end{equation*}
где подчеркнутые параемтры определяются начальными условиями. 


Стоит заметить, что хотелось бы $\theta_m$ периодической. 




\subsubsection*{Пример (Оптический резонатор)}

Рассмотрим распространение света на расстояние $L$ между двумя зеркалами кривизы $R_1,\, R_2$, тогда $b$ 
\begin{equation*}
    b = 2 \bigg(\underbrace{1+\frac{L}{R_1}}_{g_1}\bigg)\bigg(\underbrace{1+\frac{L}{R_2}}_{g_2}\bigg) - 1,
\end{equation*}
и рассмотрим $0 \leq g_1 g_2 \leq 1$.


Первый случай (1) плоского резонатора находится на гранце. Другой случай (2) это симметричный кофоканый резонатор, тогда $R_1 = -L$, и $R_2 = - L$.  Возможен симметричный концентрический $R_1 = R_2 = -L/2$. 


Пусть есть некоторая активная среда, процесс накачки. 

