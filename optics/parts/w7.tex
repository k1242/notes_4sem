Вспомним кинематическое уравнение:
\begin{equation*}
    \dot{n}_2 = - A n_2 = -\frac{n_2}{\tau_{\text{Б}}},
    \hspace{10 mm}
    \dot{n}_2 =  - \frac{\tau_{\text{Б}}}{\tau_{\text{сп}}}
\end{equation*}
где $\tau_{\text{Б}}$ -- характерное время безизлучательного перехода, $\tau_{\text{сп}}$ -- спонтанного перехода.
Также есть поглощение:
\begin{equation*}
    \dot{n}_1 = - \sigma F n_1,
\end{equation*}
и вынужденное излучение
\begin{equation*}
    \dot{n}_2 = - \sigma F n_2.
\end{equation*}
Вообще $A, B, \sigma F$ -- коэффициенты Эйнштейна. 


Вспомним про <<просветление>>, запишем набор кинематических уравнений
\begin{equation*}
    \left\{\begin{aligned}
        \dot{n}_1 &=  - n_1 \sigma F + n_2 \sigma F + n_2/\tau  \\
         N &= n_1 + n_2
    \end{aligned}\right.
    \hspace{0.5cm} \Rightarrow \hspace{0.5cm}
    \dot{n}_1 = \dot{n}_2 = 0,
    \hspace{5 mm} \text{--- \ \ стационарное приближение}.
\end{equation*}
решая эту систему находим, что
\begin{equation*}
    n_1 (F) = N \frac{F \sigma \tau + 1}{1 + 2 F \sigma \tau},
    \hspace{5 mm}
    n_2 (F) = N \frac{F \sigma \tau}{1 + 2 F \sigma \tau}.
\end{equation*}
Также, вспоминая про поглощение, понимая как происходит поглощение фотонов, 
\begin{equation*}
    \d F = D (\sigma n_2 - \sigma n_1) \d z,
\end{equation*}
где $\alpha \overset{\mathrm{def}}{=}  \sigma n_1 - \sigma n_2 = \sigma N$, так приходим к стандартному закону
\begin{equation*}
    d F = - F \alpha \d z,
    \hspace{0.5cm} \Rightarrow \hspace{0.5cm}
    F = F_0 \exp \left(- \alpha h\right),
\end{equation*}
где $h$ -- толщина образца. И вообще можно найти $\alpha (F)$, который просто монотонно убывает.





\subsection{Затемнение}

Добавим к системе третий уровень. Договоримся, что электроны умеют переходить с 1 на 2 уровень, то на самом деле он запрыгивает чуть выше второго, и сваливается, но не с частотой $h \nu$, поэтому вынужденного излучения тут не будет. 

Также за $\tau_2$ происходит излучение с 3 на 2 не с $h \nu$. Итого остаются процессы с $\sigma_2$, $\sigma_1$, $\tau_1$ и $\tau_2$. На практике $\tau_2 \ll \tau_1$, примерно как 1 пс $\ll$ 1 нс.

Запишем кинематические уравнения:
\begin{align*}
    \dot{n}_1 &= - n_1 \varphi_1 F +  \frac{n_2}{\tau_1}, \\
    \dot{n}_2 &= n_1 \sigma_1 F - n_2 \sigma_2 F - \frac{n_2}{\tau_1} + \frac{n_3}{\tau_2}, \\
    n_1 + n_2 + n_3 = N,
\end{align*}
где мы сразу перейдём к стационарному приближнию $\dot{n}_1 = \dot{n}_2 = \dot{n}_3 = 0$. 
Считая
\begin{equation*}
    \frac{1}{\sigma_1 \tau_1} = F_1, \hspace{5 mm} 
    \frac{1}{\sigma_2 \tau_2} = F_2, \hspace{5 mm}
    n_2 = n_1 \frac{F}{F_1}, \hspace{5 mm}
    n_3 = n_1 \frac{F^2}{F_1 F_2}
\end{equation*}
и, наконец, получаем
\begin{equation}
    n_1 = N \left(
        1 + \frac{F}{F_1} + \frac{F^2}{F_1 F_2}
    \right)^{-1}.
\end{equation}
Что с $n_1$ -- она затухает, также $n_2$ слегка растёт, а потом затухает, а $n_3$ выходит на $N$. 
Если внимательно посмотреть на $\alpha (F) = \sigma_1 n_1 \red{\pm} \sigma_2 n_2$,
\begin{equation*}
    \frac{d F}{d z} = - \alpha F.
\end{equation*}
Так как $\tau_2$ мааленький, то $n_3$ растёт оочень медленно. Для $\alpha$ получается немонотонная зависимость. 
На деле $n_3$ всегда ноль, т.к. $F \ll \sqrt{F_1 F_2}$. В итоге $n_2$ идёт к $N$, $n_1$ к 0. 

Так вот, $\alpha = \sigma (n_1 - n_2)$. Также $\d F = - \alpha F \d z$. Тогда
\begin{align*}
    \textnormal{if} \ \ n_1 > n_2,
    \hspace{0.5cm} \Rightarrow \hspace{0.5cm}
    \alpha > 0, \hspace{5 mm} \frac{d F}{d z} < 0, \\
    \textnormal{if} \ \ n_1 < n_2,
    \hspace{0.5cm} \Rightarrow \hspace{0.5cm}
    \alpha < 0, \hspace{5 mm} \frac{d F}{d z} > 0.
\end{align*}
Если $n_2 > n_1$, то переходим к ситуации с инверсией населенности. 

Теперь устроим её. Пусть со второго на третий уровень происходит очень быстрое сваливание (третий ниже второго). 
Снова запищем кинематические уравнения
\begin{gather*}
    n_1 + n_3 = N \\
    \dot{n}_1 = - n_1 W + \frac{n_3}{\tau} + n_3 \sigma F = 0.
\end{gather*}
Но такие схемы уже редки, чаще сейчас используют четырех уровневые системы.

Есть уровни $1,4,3,2$ -- $E_{14} \gg k T$. Также $\tau_{23} = 10$пс. Также с $\tau_{41} = 10$пс. Итого очень легко добиться инверсии населенности между третьим и четвертым уровнем. Накачка происходит с первого на второй уровень. 


\subsection*{Лазер}

Есть некоторая среда, есть накачка, с помощью которой происходит переброс с $1$ на $4$ уровень. Пока что это только усилитель. Теперь добавим зеркало 100\% слева и 50\% справа. 

И тут на сцену выходит произвольное излучение. Как только один полетит в удачном направлении, захватит с собой остальных -- лавина, успех. 


