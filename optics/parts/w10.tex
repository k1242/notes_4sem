\subsection{Решйтка Брэгга}


Есть некоторая решётка с периодом $a$, на которую светят волной с угом падения $a$. 
Найдём разность хода $\Delta = CA+DC-BD = 2 a \cos \theta = m \lambda$, тогда у коэффициента $R$ и $T$ будет зависимость от $\theta$ и $\lambda$.

Суммируя интенсивности можем получить
\begin{equation*}
    I_{\text{out}} = I_0 \frac{\sin^2 (N \varphi/2)}{\varphi/2},
    \hspace{5 mm} 
    2 k a \cos \theta = \pi m,
    \hspace{0.5cm} \Rightarrow \hspace{0.5cm}
    \cos \theta =  \frac{\pi m }{2 k a}
    = \frac{m \lambda}{2 a},
    \hspace{0.25cm} \Rightarrow \hspace{0.25cm}
    \theta = \arccos\left(\frac{m \lambda}{2a}\right).
\end{equation*}


Аналогично можем переписать в терминах матриц, и, по индукции, получить выражение вида
\begin{equation*}
    M_0 = \begin{pmatrix}
        1/t^* & r/t  \\
        r^*/t^* & 1/t  \\
    \end{pmatrix},
    \hspace{5 mm} 
    M = M_0^N, \hspace{5 mm} 
    \det M_0 = 1,
    \hspace{5 mm} 
    M_0^N = \Psi_N M_0 - \Psi_{N-1} \hat{E},
    \hspace{5 mm} 
    \Psi_N = \frac{\sin N \Phi}{\sin \Phi},
    \hspace{5 mm} 
    \cos \Phi = \Re\left(\frac{1}{t}\right).
\end{equation*}
Так мы приходим к 
\begin{equation*}
    M = \begin{pmatrix}
        1/t^*_N & r_N/t_N  \\
        r^*_N / t^*_N & 1/t^N  \\
    \end{pmatrix},
    \hspace{5 mm} 
    \frac{\sin N \Phi}{\sin \Phi} \begin{pmatrix}
        1/t^* & r/t  \\
        r^*/t^* & 1/t  \\
    \end{pmatrix} - \frac{\sin (N-1)\Phi}{\sin \Phi} \begin{pmatrix}
        1 & 0  \\
        0 & 1  \\
    \end{pmatrix}.
\end{equation*}
Итого, из предельно простых рассуждений, находим
\begin{equation*}
    \left\{\begin{aligned}
        1/t_N^* &= \Psi_N / t^* - \Psi_{N-1} \\
        r_N / t_N &= \Psi_N r / t
    \end{aligned}\right.
\end{equation*}
вводя $T_N = t_N^* t_N$ переходим к 
\begin{equation*}
    \frac{R_N}{T_N} = |\Psi_N|^2 \cdot \frac{R}{T}, \hspace{5 mm} 
    \left.\begin{aligned}
        R_N &= 1 - T_N \\
        R = 1 - T \\
    \end{aligned}\right.
    \hspace{0.5cm} \Rightarrow \hspace{0.5cm}
    \frac{1-T_N}{T_N} = |\Psi_N|^2 \cdot \frac{1-T}{T},
\end{equation*}
итого, финальная формула (почти)
\begin{equation*}
    \frac{1}{T_N} = |\Psi_N|^2 \frac{1-T}{T} + 1,
    \hspace{0.5cm} \Rightarrow \hspace{0.5cm}
    T_N = T \frac{1}{|\Psi_N|^2 (1-T)+T},
    \hspace{5 mm} 
    R_N = \frac{|\Psi_N|^2 (1-T)}{|\Psi_N|^2 (1-T)+T}.
\end{equation*}





\subsubsection*{Предельные случаи.}


\textbf{ Случай 1}. Считая $R \ll 1$, и $\Psi_N^2 R << 1$, тогда и $R_N \approx \Psi_N^2 R$.


\textbf{ Случай 2, режим частичного отражения}. Рассматривается случай вида $|\Re 1/t| < 1$, тогда  $\Phi = \arccos\left(\Re 1/t\right)$. Можем тогда получить, что $R_{N\, max}$ будет вида
\begin{equation*}
    R_{N\, max} = \frac{N^2 (1-T)}{N^2 (1-T) + T}, \hspace{5 mm} \Psi_N = N.
\end{equation*}
И может быть минимум 
\begin{equation*}
    R_N = 0, \hspace{5 mm} \left\{\begin{aligned}
        \sin N \Phi &= 0, \\
        \sin \Phi &\neq 0
    \end{aligned}\right.
\end{equation*}



\textbf{ Случай 3, режим полного отражения}. Рассмотрим $\Re 1/t > 1$, тогда $\cos \Phi > 1$, тогда
\begin{equation*}
    \cos \left(\Phi_R + i \Phi_I\right) = \cos \Phi_R \cos (i \Phi_I) - \sin \Phi_R \sin (i \Phi_I) = 
    \cos \Phi_R \ch \Phi_I - i \sin \Phi_R \sh \Phi_I = \Re \left(\frac{1}{t}\right)
\end{equation*}
тогда $\sin \Phi_R \sh \Phi_I = 0$, иначе $\Phi_r = \pi m$, итого находим
\begin{equation*}
    \ch \Phi_I = |\Re 1/t|.
\end{equation*}
Приходим к значению
\begin{equation*}
    \Psi_N = (-1)^N \frac{\sin i \Phi_I N}{\sin i \Phi_I},
    \hspace{0.5cm} \Rightarrow \hspace{0.5cm}
    R_N = \bigg|
        \frac{\sh^2 \Phi_I N}{\sh^2 \Phi_I}
    \bigg| (1-T).
\end{equation*}
