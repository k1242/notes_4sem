\subsection{Волновое уравнение}

В общем оптика устроена как-то так: $\text{ГО} \subset \text{ВО} \subset \text{ЭО} \subset \text{КО}$.
Вспомним уравнения Максвелла
\begin{equation*}
    \left\{\begin{aligned}
            \div \vc{D} &= 4 \pi \rho \\
            \div \vc{B} &= 0 \\
            \rot \vc{E} &= - \frac{1}{c}\frac{\partial \vc{B}}{\partial t} \\
            \rot \vc{H} &= \frac{4\pi}{c} \vc{j} + \frac{1}{c} \frac{\partial \vc{D}}{\partial t}.
    \end{aligned}\right.
    \hspace{0.5cm} \Rightarrow \hspace{0.5cm}
    \rot \rot \vc{E} = - \frac{1}{c} \frac{\partial }{\partial t} \rot \vc{B}
    \hspace{0.5cm} \Rightarrow \hspace{0.5cm}
    \nabla^2 \vc{E} - \frac{\varepsilon \mu}{c^2} \frac{\partial^2 \vc{E}}{\partial t^2} = 0
\end{equation*}
Будем считать, что нет свободных токов и зарядов. Как вариант, можно найти решение в виде
\begin{equation*}
    \vc{E} = \vc{E}_0 \exp\left(
        i \omega t - \vc{k} \cdot \vc{r}
    \right).
\end{equation*}
Важно, что верны формально замены
\begin{equation*}
    \frac{\partial }{\partial t} \to i \omega,
    \hspace{1 cm}
    \left\{\begin{aligned}
         \partial_{x} &\to - i k_x, \\
         \partial_{y} &\to - i k_y, \\
         \partial_{z} &\to - i k_z,  \\   
    \end{aligned}\right.
    \hspace{0.5cm} \Rightarrow \hspace{0.5cm}
    \nabla \to - i \vc{k},
    \hspace{1 cm}
    \nabla^2 \to - k^2.
\end{equation*}
Приходим к уравнению вида
\begin{equation*}
    - k^2 \vc{E} + \frac{\varepsilon \mu}{c^2} \omega^2 \vc{E} = 0
    \hspace{0.5cm} \Rightarrow \hspace{0.5cm}
    \frac{\omega^2}{k^2} = \frac{c^2}{\varepsilon \mu},
    \hspace{0.5cm} \to \hspace{0.5cm}
    \frac{\omega}{k} = \frac{c}{\sqrt{\varepsilon \mu}} = \frac{c}{\sqrt{\varepsilon}} = \frac{c}{n}.
\end{equation*}
Можем посмотреть на $\omega t - \vc{k} \cdot \vc{r} = \const$. Тогда
\begin{equation*}
    \omega \d t - k \d z = 0,
    \hspace{0.5cm} \Rightarrow \hspace{0.5cm}
    \frac{dz}{dt} = \frac{\omega}{k} = \frac{c}{n}.
\end{equation*}


\subsection{Уравнения эйконала}

\begin{enumerate}
    \item Свет распространяется в виде лучей.
    \item Среда характеризуется показателем преломления $n$, более того\footnote{
        Будем считать, что лучу нужно проходить больший оптический путь.
    }  $c_{\text{ср}} = c / n$.
    \item $\int n \d l \to \min$ (принцип Ферма).
\end{enumerate}

\begin{to_def}
    \textit{Оптический путь} можем определить, как
    \begin{equation*}
        S = \int_A^B n(\vc{r}) \d l.
    \end{equation*}
\end{to_def}


Посмотрим на уравнение
\begin{equation*}
    \nabla^2 \vc{E} - \frac{n^2}{c^2} \frac{\partial^2 \vc{E}}{\partial t^2} = 0,
    \hspace{0.5cm} \Rightarrow \hspace{0.5cm}
    {E} (\vc{r}, t) = a(\vc{r}) \exp\left(
        i k_0 \Phi(\vc{r}) - i \omega t
    \right),
\end{equation*}
где $\Phi(\vc{r})$ называем \textit{эйконалом}, а $a$ - амплитуда.

\textbf{Вычисления}. Формально получаем следующее:
\begin{equation*}
    \frac{\partial }{\partial t} \to -i \omega,
    \ 
    \frac{\partial^2 }{\partial t^2} \to - \omega^2,
    \hspace{1 cm}
    \frac{\partial }{\partial x} E = 
    a_x' \exp(\ldots) + a(\vc{r}) i k_0 \, \Phi_x' \exp(\ldots),
\end{equation*}
И для второй производной
\begin{equation*}
    \frac{\partial^2 E}{\partial x^2} = a''_{xx} \exp(\ldots) + 
    2 i k_0 a'_x \Phi'_x \exp(\ldots) + i k_0 a \Phi_{xx}'' \exp(\ldots) - a(\vc{r}) k_0^2 |\Phi'_x|^2 \exp(\ldots).
\end{equation*}
Таким образом нашли $\Delta E$
\begin{equation*}
    \nabla^2 E = \nabla^2 a \exp(\ldots) - a(\vc{r}) k_0^2 |\grad \Phi|^2 \exp(\ldots) + 
    i \left(
        2 k_0 (\grad a, \grad \Phi) + k_0 a \nabla^2 \Phi
    \right) \exp(\ldots).
\end{equation*}
Внимательно посмотрели на волновое уравнение, решили сгруппировать вещественную часть и мнимую
\begin{equation*}
    \nabla^2 a \exp(\ldots) - a(\vc{r}) k_0^2 |\grad \Phi|^2 \exp(\ldots) + \frac{\omega^2}{c^2} n^2 a \exp(\ldots) = 0,
    \hspace{0.5cm} \Rightarrow \hspace{0.5cm}
    |\grad \Phi|^2 = 
    \underbrace{
        \frac{1}{a l_0^2} \nabla^2 a
    }_{
        \text{изм. ампл.}
    }
     + n^2.
\end{equation*}
Ну, будем считать, что (настоящая область применимости волновой оптики)
\begin{equation*}
    \bigg|
        \lambda \frac{\partial^2 a}{\partial x^2} 
    \bigg| \ll 
    \bigg|
        \frac{\partial a}{\partial x} 
    \bigg|,
    \hspace{0.5cm} \Leftrightarrow \hspace{0.5cm}
    \bigg|
            \lambda \frac{\partial a}{\partial x} 
    \bigg| \ll a,
    \hspace{1 cm}
    \lambda \to 0.
\end{equation*}
И приходим к \textit{уравнению Эйконала} 
\begin{equation}
    \boxed{
        |\grad \Phi| = n.
    }
\end{equation}
Ещё раз вспомним, что волновой фронт имеет вид
\begin{equation*}
    \omega t - k_0 \Phi = \const.
\end{equation*}
Запишем, что (живём вдоль $\vc{S}$)
\begin{equation*}
    \grad \Phi = n \vc{S},
    \hspace{1 cm}
    \|\vc{S}\| = 1,
    \hspace{1 cm}
    \frac{\partial \Phi}{\partial S} = n.
\end{equation*}
Тогда
\begin{equation*}
    \omega \d t - k_0 \d \Phi = 0,
    \hspace{0.5cm} \Rightarrow \hspace{0.5cm}
    \omega \d t = k_0 \d \Phi = k_0 \frac{\partial \Phi}{\partial S} \d S = k_0 n \d S.
\end{equation*}


\subsection{Принцип Ферма}

% поврехность постоянной фазы.
Пусть $\Phi$ -- однозначно задан, тогда 
\begin{equation*}
    \grad \Phi = n \vc{S},
    \hspace{0.5cm} \Rightarrow \hspace{0.5cm}
    \oint n \vc{S} \cdot \d \vc{l} = 0,
    \hspace{0.5cm} \Rightarrow \hspace{0.5cm}
    \int_{ACB} n \vc{S} \cdot \d \vc{l} = 
    \int_{ADB} n \vc{S} \cdot \d \vc{l}.
\end{equation*}
Но $\vc{S} \cdot \d \vc{l} = S \d l = \d l$ на $ACB$. Тогда
\begin{equation*}
    \int_{ACB} n \d l = \int_{ADB} n \vc{S} \cdot \d \vc{l} \leq \int_{ADB} n \d l.
\end{equation*}
Что доказывает принцип Ферма.


\subsection{Траектория луча (?)}

Для луча верно, что
\begin{equation*}
    n \vc{S} = \grad \Phi,
    \hspace{1 cm}
    | d \vc{r} | = \d l, 
    \hspace{1 cm}
    \vc{S} = \frac{d \vc{r}}{d l}.
\end{equation*}
В таком случае верно, что
\begin{equation*}
    n \frac{d \vc{r}}{d l} = \grad \Phi,
    \hspace{0.5cm} \Rightarrow \hspace{0.5cm}
    \frac{d }{d l} (n \frac{d \vc{r}}{d l} ) = \frac{d }{d l} \grad \Phi = \grad \frac{d \Phi}{d l} = \grad n.
\end{equation*}
Получили  \textit{уравнение траектории луча} 
\begin{equation}
    \boxed{
        \frac{d }{d l} \left(
            n \frac{d \vc{r}}{d l} 
        \right) = \grad n
        }.
\end{equation}
Например, в однородной среде
\begin{equation*}
    n = \const,
    \hspace{0.5cm} \Rightarrow \hspace{0.5cm}
    \frac{d^2 \vc{r}}{d l^2} = 0,
    \hspace{0.5cm} \Rightarrow \hspace{0.5cm}
    \vc{r} = \vc{a} l + \vc{b}.
\end{equation*}


Можно сделать ещё так (найти кривизну траектории?)
\begin{equation*}
    \vc{S} \frac{d n}{d l} + n \frac{d \vc{S}}{d l} = \nabla n,
    \hspace{0.5cm} \Rightarrow \hspace{0.5cm}  
    \frac{d \vc{S}}{d l} = \frac{1}{n}\left(
        \nabla n - \vc{S} \frac{d n}{d l}     
    \right).
\end{equation*}
Получаем (вспомнив трёхгранник Френе)
\begin{equation*}
    \frac{\vc{N}}{R} = \frac{1}{n} \left(
        \nabla n - \vc{S} \frac{d n}{d l} 
    \right),
    \hspace{0.5cm} \Rightarrow \hspace{0.5cm}
    0 < \frac{N^2}{R} = \frac{\left(\vc{N} \cdot \nabla n\right)}{n},
\end{equation*}
или
\begin{equation}
    \left(
        \vc{N} \cdot \nabla n
    \right) > 0,
    \hspace{0.5cm} \Rightarrow \hspace{0.5cm}
    \text{луч поворачивает в $\uparrow n$}.
\end{equation}




\subsection{Уравнение луча в параксиальном приближение}

Пусть есть некоторая $n(y)$. Пусть луч движется $\theta (y)$, рассмотрим ситуацию преломления, тогда
\begin{equation*}
    n(y) \cos \theta(y) = n (y + \d y) \cos \theta(y + d y),
    \hspace{0.5cm} \Rightarrow \hspace{0.5cm}
    \left(
        n(y) + \frac{d n}{d y} \Delta y
    \right)  \left(
        \cos \theta(y) - \sin \theta(y)
    \right).
\end{equation*}
Раскрыв скобки, получим
\begin{equation*}
    n(y) \cos \theta(y) = n(y) \cos \theta(y) + \frac{d n}{d y} \cos \theta(y) \Delta y - n(y) \sin \theta(y) \frac{d \theta}{d y} \Delta y.
\end{equation*}
Запишем чуть аккуратнее:
\begin{equation*}
    \frac{d n}{d y} \cos \theta(y) = n(y) \sin \theta(y) \frac{d \theta}{d y},
\end{equation*}
Считая, что $\sin \theta(y) \approx \theta(y) = d y /d x$, тогда
\begin{equation}
    \frac{1}{n} \frac{d n}{d y} = \tg \theta \frac{d \theta}{d y} = \frac{d \theta}{d x} = \frac{d^2 y}{d x^2},
    \hspace{0.5cm} \Rightarrow \hspace{0.5cm}
    y_{xx}'' = \frac{1}{n}\frac{d n}{d y}.
\end{equation}




\subsection{Пример слоистой среды}


Рассмотрим вещество с коммерческим названием SELFOC и переменным показателем преломления вида
\begin{equation*}
    n^2 = n_0^2 (1 - \alpha^2 y^2) 
\end{equation*}
Считая $\alpha y \ll 1$, подставляя в уравнение луча находим, что
\begin{equation*}
    y_{xx}'' = \frac{1}{n_0(1-\alpha^2 y^2)^{1/2}} \frac{d n}{d y} =
    \frac{-n_0 \alpha^2 y}{n_0} = - \alpha^2 y,
\end{equation*}
и мы снова всё свели к гармоническому осциллятору.

\phantom{42}

\noindent
\red{Нужно ещё разобрать мнимую часть, в которой сидит факт об отсутствии взаимодействия лучей.}