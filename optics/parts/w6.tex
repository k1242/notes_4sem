\subsection{Естественная ширина линии}

Это скорее чисто квантовое явление, поэтому давайте сейчас немного помашем руками, пусть атом это осциллятор
\begin{equation*}
    \ddot{x} + 2 \gamma \dot{x}  +\omega_0^2 x = 0,
    \hspace{5 mm}
    x(0) = x_0,
    \hspace{5 mm}
    \dot{x}(0) = 0,
\end{equation*}
что приводит к известному решению
\begin{equation*}
    x(t) = x_0 e^{- \gamma t} \left(
        \cos (\omega t) + \frac{\gamma}{2\omega} \sin (\omega t)
    \right) \approx 
    x_0 e^{-\gamma t} \cos (\omega_0 t)
    ,
    \hspace{5 mm}
    \omega = \sqrt{\omega_0^2 - \gamma^2}.
\end{equation*}
Далее, пусть есть преобразование Фурье
\begin{equation*}
    x(t) = \frac{1}{\sqrt{2\pi}} \int_{-\infty}^{+\infty} A(\omega) e^{i \omega t} \d \omega,
    \hspace{5 mm}
    A(\omega) = \frac{1}{\sqrt{2\pi}} \int_{-\infty}^{+\infty} x(t) e^{-i \omega t} \d t.
\end{equation*}
Тогда
\begin{align*}
    A(\omega) &= \frac{1}{\sqrt{2\pi}} \int_{0}^{+\infty} x_0 e^{- \gamma t} \cos (\omega_0 t)
    e^{- i \omega t} \d t = \bigg/
        \cos (\omega_0 t) = \frac{1}{2} \left( e^{i \omega_0 t} + e^{- i \omega_0 t}\right)
    \bigg/ = \\
    &=  \frac{1}{\sqrt{2\pi}} \frac{1}{2} \int_{0}^{\infty} x_0 e^{-\gamma t} e^{i (\omega_0 - \omega)t} \d t + 
    \frac{1}{\sqrt{2\pi}} \frac{1}{2} \int_{0}^{+\infty} x_0 e^{-\gamma t} e^{i(\omega_0+\omega)t} \d t = \\
    &= \frac{x_0}{2 \sqrt{2\pi}}  \left[
        \frac{e^{-\gamma t + i(\omega_0 - \omega) t}}{i (\omega_0 - \omega) - \gamma} + 
        \frac{e^{-\gamma t + i (\omega_0 + \omega)t}}{- (\omega_0 + \omega) - \gamma}
    \right] \bigg|_0^\infty \approx
    - \frac{x_0}{2 \sqrt{2\pi}} \frac{1}{i (\omega_0 - \omega) - \gamma}
    ,
\end{align*}
где вторым слагаемым можно пренеречь, в силу ...
\begin{equation*}
    I = A A^* = \frac{x_0^2}{8 \pi} \frac{1}{(\omega_0-\omega)^2 + \gamma^2},
\end{equation*}
что дает кривую Лоренцова профиля.

Если говорить про жизнь, то ширина на полувысоте $\in [10^7, \, 10^9]$ Гц.






\subsection{Доплеровское уширение}

В зависимости от взаимного движения излучения от атома и приемника будет фиксироваться частота
\begin{equation*}
    \omega = \omega_0 (1 + \vc{k} \cdot \vc{v})
\end{equation*}
Если аккуратно взять $f(v_x)$, то само собой максимум в $v_x = 0$, и максимум $J(\omega)$ будет в $\omega_0$, а сама кривая -- перекочевавший из $f(v_x)$ гауссов колокольчик.

Если говорить точнее, то
\begin{equation*}
    v_0^2 = \frac{2kT}{m}, \hspace{5 mm}
    d N = N \exp\left(
        - \frac{(\omega-\omega_0)^2c^2}{\omega_0^2 v_0^2}
    \right) \frac{c}{\omega_0} \d \omega,
\end{equation*}
где, переходя к числам будем наблюдать толщину $\in [10^0, 10^9]$ Гц.





\subsection{Другие примеры}

Есть столкновительное уширение, глобально потенциальная энергия зависит от расстояния между атомами, начинает фаза, или частота зависеть от времени. 

Есть ещё времяпролетное уширение, частица пролетает конечный пучок, тогда интегрировать нужно не от $0$ до $\infty$, а от $0$ до $T$, с чем тоже как-то борются.

Глобально можно их резделить на два типа: однородное и неоднородное. Например, естественная ширина линии однородна, а доплеровское неоднородно. 




\subsection{Внутридоплеровская спектроскопия}

\textbf{Постановка задачи}: есть куча бегающих атомов, несколько уровней энергии. Вообще можем следить за излучением, можем за поглощением. Из уширения разные линии уровня могут слиться в одну. Хотелось бы этого избежать.



\textbf{Квантмех для самых маленьких}: что умеет электрон? Электрон умеет поглощать фотон, с некоторой вероятностью, характеризовать этот процесс будем некоторым сечением поглощения $\sigma$. 

Аналогично для вынужденного излучения есть некоторая $\sigma$, которая такая же как и для поглощения. А ещё, что здорово, излученные фотоны практически идентичны. 

Также спонтанное излучение куда-то  с $h \nu$. А ещё есть безизлучательное излучение, когда энергия уходит в тепло. Всё происходит с какой-то вероятностью, за некоторое свойственное характерное время $\tau$. 

Скоростные (кинематические) уравнения:
\begin{align*}
    \dot{n}_2 = - \frac{n_2}{\tau_{\text{Б}}} - \frac{n_2}{\tau_{\text{С}}} - \ldots,
\end{align*}
что соответсвует $n_2 = n_{20} \exp(-t/\langle \tau\rangle_{\textnormal{harm}})$. 

Пусть $F = I/(h \nu)$ -- плотность потока фотонов, тогда 
\begin{equation*}
    \dot{n}_1 = - F \sigma n_1,
    \hspace{0.5cm} \Rightarrow \hspace{0.5cm}
    n_1 (t) = n_1(0) \exp(-F \sigma t),
\end{equation*}
и, соответсвенно, при различных случаях
\begin{equation*}
    \dot{n}_1 = - F \sigma n_1 + F \sigma n_2 + \frac{n_2}{\tau_{\text{Б}}}.
\end{equation*}
И теперь вернемся к доплеровской спектроскопии. 

Пусть $n_1 + n_2 = N$. Можем записать некоторое уравнение баланса $n_2 = N - n_1$
\begin{equation*}
    \left\{\begin{aligned}
        \dot{n}_1 &= - n_1 \sigma F + n_2 \sigma F + {n_2}/{\tau_{\text{Б}}}, \\
        \dot{n}_2 &= + n_1 \sigma F - n_2 \sigma F - {n_2}/{\tau_{\text{Б}}},        
    \end{aligned}\right.
    \hspace{0.1cm} \Rightarrow \hspace{0.1cm}
    \dot{n}_1 + n_1 (2 \sigma F + \tau_{\text{Б}}^{-1}) = N \sigma F + \frac{N}{\tau_{\text{Б}}},
    \hspace{0.1cm} \Rightarrow \hspace{0.1cm}
    {n}_1 = N \frac{\sigma F + \tau_{\text{Б}}^{-1}}{2 \sigma F + \tau_{\text{Б}}^{-1}} = 
    N \frac{F + \frac{1}{\sigma \tau_{\text{Б}}}}{2F + \frac{1}{\sigma \tau_{\text{Б}}}},
\end{equation*}
что соответсвует $\dot{n}_1 = \dot{n}_2 = 0$.


Если посмотреть на $n_1(F)$ и $n_2(F)$, то они стремятся к $N/2$, при $F \to + \infty$. А вообще, считая $F_s^{-1} = \sigma \tau_{\text{Б}}$, можем записать
\begin{equation*}
    n_1 (F) = N \frac{F + F_s}{2 F + F_s}.
\end{equation*}
Введя $\alpha = n_1 \sigma - n_2 \sigma$ -- коэффициент поглощения. В термодинамике была $n \sigma \lambda$ обратно пропорционально длине свободного пробега
\begin{equation*}
    d F =  - (n_1 \sigma - n_2 \sigma) F \d z,
    \hspace{0.5cm} \Rightarrow \hspace{0.5cm}
    d F = - n_1 \sigma F \d z + n_2 \sigma F \d z.
\end{equation*}
Получается, что $\alpha$ стремится к $0$ во времени. 


Таким образом, если ооочень сильно светить на вещество, то можно его пробить при большой интенсивности. 

При равенстве частоты лазера и частоты атома возникнет провал Лэмба. В какой-то момент научились менять частоту лазера, и научились так измерять $\omega_0$. Если частот будет две, то также можем зафиксировать два провала Лэмба.


