\subsubsection*{Т7}


Возьмём функцию, которая лежит в $L_2$, но не лежит в $
\overset{\vspace{-1pt}\scalebox{0.5}{$\circ$}}{C} [-\pi,\pi]$, например, ограничение $\sign x$. И рассмотрим
подпространство $V\subset \overset{\vspace{-1pt}\scalebox{0.5}{$\circ$}}{C} [-\pi,\pi]$, заданное ортогональностью к
ней, то есть заданное формулой
\[
\int_{-\pi}^0 f(x)\; dx = \int_0^\pi f(x)\; dx.
\]

Это $V$ есть замкнутое подпространство в $\overset{\vspace{-1pt}\scalebox{0.5}{$\circ$}}{C}[-\pi,\pi]$ и в нём
можно выбрать какую-то полную систему, и даже её ортогонализовать. Если
начать с тригонометрической системы, то косинусы и чётные синусы и так
лежат в $V$, нечётные синусы надо будет подправить, скомбинировав их с
$\sin x$, а потом ещё ортогонализовать (что может быть неприятно).

В итоге, система не может быть полна в $\overset{\vspace{-1pt}\scalebox{0.5}{$\circ$}}{C}[-\pi,\pi]$, так как её
линейные комбинации не выходят за пределы $V$. А что касается
замкнутости, то переходя в гильбертово $L_2$ видно, что ортогональное
дополнение к замыканию образа $V$ в гильбертовом пространстве одномерно
и натянуто на этот вот $\sign x$, который разрывен и не лежит в образе
$\overset{\vspace{-1pt}\scalebox{0.5}{$\circ$}}{C}[-\pi,\pi]$. Так что замкнутость в терминах $\overset{\vspace{-1pt}\scalebox{0.5}{$\circ$}}{C}[-\pi,\pi]$ есть.