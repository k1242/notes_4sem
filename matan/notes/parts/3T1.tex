\subsubsection*{Т1}

Построим табличку сходимостей. Для начала вспомним, что если $\mu(A) < + \infty$ и 
$1 \leq p_1 < p_2 \leq + \infty$, то
\begin{equation*}
    \|\cdot\|_{p_1} \leq C(\mu(A), p_1, p_2) \|\cdot\|_{p_2},
    \hspace{5 mm} 
    C(\ldots) = \left(\mu(A)\right)^{\frac{p_2-p_1}{p_1 p_2}}.
\end{equation*}
В частности, можно перейти к пределу, и обнаружить, что
\begin{equation*}
    \|\cdot\|_1 \leq C(\ldots) \|\cdot\|_{\infty},
    \hspace{5 mm} 
    \|\cdot\|_{\infty} = \lim_{p\to \infty} \|\cdot\|_p \equiv \|\cdot\|_C.
\end{equation*}
Таким образом из сходимости $L_2$ следует сходимость в $L_1$. 

Ещё раз напишем, что значит сходимость по норме:
\begin{equation*}
    f_n \underset{L_p}{\to} f
    \hspace{5 mm} \Leftrightarrow \hspace{5 mm} 
    \forall \varepsilon > 0 \ 
    \exists N_\varepsilon \in \mathbb{N} \ 
    \forall n \geq N_\varepsilon \ 
    \|f_n - f\|_p < \varepsilon.
\end{equation*}
Тогда рассмотрим
\begin{equation*}
    \|f_n - f\|_1 \leq \sqrt{b-a}  \|f_n -f\| < \varepsilon,  \ \qed.
\end{equation*}
Теперь докажем $f_n \underset{C}{\to} f \ \Rightarrow \ f_n \underset{L_2}{\to} f$, где сходимость по $C$-норме:
\begin{equation*}
    \forall \varepsilon > 0 \ 
    \exists N_\varepsilon \in \mathbb{N} \ 
    \forall n \geq N_\varepsilon \ 
    \forall x \in A \ 
    |f_n(x)-f(x)| < \varepsilon.
\end{equation*}
Ну, действительно,
\begin{equation*}
    \|f_n -f\|_2^2 = \int_A |f_n-f|^2(x) \mu(\d x) \leq \int_A \left\{
        \sup_{x \in A} |f_n - f|(x)
    \right\}^2 \mu(\d x) = 
    \left\{
        \sup_{x \in A} |f_n - f|(x)
    \right\}^2 \mu(A),
\end{equation*}
где множитель перед $\mu(A)$ стремится к 0 при $n \to \infty$, $\qed$. 

Также стоит вспомнить, что из равномерной сходимости следует поточечная сходимость. По определению, поточечная сходимость:
\begin{equation*}
    \forall x \in A \ 
    \forall \varepsilon > 0 \ 
    \exists N(\xi, \varepsilon) \in \mathbb{N} \ 
    \forall n \geq N \ 
    |f_n-f|(x) < \varepsilon.
\end{equation*}
Получается, что достаточно взять $N(x, \varepsilon) = N(\varepsilon)$ и получить искомое утверждение. 

В качетсве контрпримера рассмотрим $f_n(x) = n \arcctg(n/x^2)$ с $A = [1, _\infty)$. По отрицанию условия Коши, если
\begin{equation*}
    \exists \varepsilon_0 \ 
    \forall k \in \mathbb{N} \ 
    \exists n \geq k \ 
    \exists p \in \mathbb{N} \
    \exists \tilde{x} \in A \colon \ 
    |f_{n+p}-f_n|(\tilde{x}) \geq \varepsilon_0,
\end{equation*}
то последовательность $f_n$ не явдяется равномерно сходящейся. Действительно,
при $n=k$, $p=2k-2n$, $\tilde{x} = \sqrt{k} = \sqrt{n}$, верно, что
\begin{equation*}
    |f_{n+p}-f_n|(\tilde{x}) = n|2\arctg 2 - \arctg 1| \geq |2 \arctg 2 - \pi/4| = \varepsilon_0 > 0,
\end{equation*}
что говорит об отсутсвие равномерной сходимости. При этом $f_n \to x^2$ поточечно на $x \in E$. 

\textbf{Контрпримеры}. Покажем, что $f_n \underset{L_1}{\to} f \not \Rightarrow f_n \underset{L_2}{\to} f$. Прямую мы умеем строить по двум точкам
\begin{equation*}
    \frac{f-f_0}{f_1-f_0} = \frac{x-x_0}{x_1-x_0},
\end{equation*}
Построим последовательность функций вида
\begin{equation*}
    \frac{f_n - c_n}{0-c_n} = \frac{x-0}{x_n-0},
    \hspace{5 mm} 
    f_n(x) = \left\{\begin{aligned}
        &c_n(1-\textstyle \frac{x}{x_n}), &x\in[0, x_n), \\
        &0, &x\in[x_n, 1].
    \end{aligned}\right.
\end{equation*}
Контропримеры строим на отрезке $[0, 1]$. Выберем последоательность сходящуюся к $0$ в $L_1$ норме:
\begin{equation*}
    \|f_n - 0\|_1 = \int_0^{x_n} \left| 
        c_n\left(1- \textstyle\frac{x}{x_n}\right)
    \right| \mu(\d x) = \frac{1}{2} c_n x_n,
    \hspace{5 mm} 
    \|f_n - 0\|_2^2 = \frac{1}{3} c_n^2 x_n,
    \hspace{0.5cm} \Rightarrow \hspace{0.5cm}
    \|f_n\|_2 = \frac{1}{\sqrt{3}} c_n \sqrt{x_n}.
\end{equation*}
Пусть $c_n x_n = \alpha_n$ -- бесконечно малая последовательность. Выберем $x_n = 1/n$, тогда $c_n = n \alpha_n$. 

Для $\|f_n\|_2 = \frac{1}{\sqrt{3}} \frac{\alpha_n}{\sqrt{x_n}} = \frac{1}{\sqrt{3}} \sqrt{n} \alpha_n$, что устремим к $\infty$, выбрав
\begin{equation*}
    \alpha_n = \frac{1}{n^{1/2-\xi}},
    \hspace{5 mm} 
    \xi \in\left(0, \frac{1}{2}\right),
    \hspace{5 mm} 
    \qed.
\end{equation*}
Эту историю можно обобщить до отсутсвия следствия в $\|f_n\|_p = c_n x_n^{1/p} (1+p)^{-1/p}$. Тогда можем взять $\alpha_n = \left(n^{1-1/p-\xi}\right)^{-1}$, для $\xi \in (0, 1-1/p)$.

Теперь покажем, что $f_n \underset{L_2}{\to}  f \not \Rightarrow f_n \underset{C}{\to} f$. Пусть $f_n \to 0$ в $L_2$ норме. Пусть
\begin{equation*}
    \|f_n - 0\|_2 = \|f_n\| = \frac{c_n}{\sqrt{3}} x_n = \alpha_n,
    \hspace{5 mm}  x_n = \frac{1}{n}.
\end{equation*}
Пусть $f_n$ вида
\begin{equation*}
    f_n(x) = \left\{\begin{aligned}
        &\sqrt{3} \alpha_n \sqrt{n} (1-nx), &x\in[0, 1/n)
        &0, &x\in(1/n, 1].
    \end{aligned}\right. 
\end{equation*}
В таком случае
\begin{equation*}
    \sup_{x\in[0, 1]} |f_n - 0|(x) = \sup_{x\in[0, 1/n]} |\sqrt{3} \alpha_n \sqrt{n} (1-nx)| =
    \sqrt{3}  \alpha_n \sqrt{n} \not \to 0,
    \hspace{5 mm}   
    \alpha_n = \frac{1}{n^{1/2-\xi}},
    \hspace{5 mm} \xi \in[0, 1/2),
\end{equation*}
что и требовалось доказать. 

Пусть теперь есть поточечная сходимость но нет сходимости в $L_1$. Построим пилу, вида
\begin{equation*}
    f_n (x) = \left\{\begin{aligned}
        &c_n \textstyle \frac{x_{1,n}-x}{x_{1,n}-x_n}, 
        &x \in (x_{1,n}, x_n], \\
        &c_n \textstyle \frac{x_{2,n}-x}{x_{2,n}-x_n}, 
        &x \in[x_n, x_{2,n}), \\
        &0, 
        & x\in[0, x_{1,n}] \cup [x_{2,n}, 1].
    \end{aligned}\right.
\end{equation*}
В этой задаче достаточно считать $x_{1,n} = 1/(n+1)$, а $x_{2,n} = 1/n$, тогда
\begin{equation*}
    \|f_n\|_1 = c_n \frac{x_{2,n}-x_{1,n}}{2} = \frac{c_n}{2} \frac{1}{(n+1)n} \to \infty.
\end{equation*}
Чтобы это сделать, достаточно выбрать $c_n = n^{2+\xi}$. Однак поточечно такой зуб пилы сходится к $0$. Действительно, при $x=0$  $f_n(0) = 0$. Для остальных $x$ можно показать, что по определению $\lim_{n\to \infty} f_n (x) = 0$. 