\subsubsection*{Т19}

Выведем из теоремы Хана-Банаха, что всякое конечномерное подпространство $V$ в банаховом постранстве $E$ имеет замкнутое дополнение $W \subseteq E$, такое что $E = V \oplus W$. 

\begin{to_thr}[]
    Для всякого ненулевого элемента $x$ нормированного пространства $X$ найдётся такой функционал $l$, что $\|l\|=1$ и $l[f]=\|f\|$. 
\end{to_thr}

\begin{proof}[$\triangle$]
На одномерном пространствеЮ порожденном $x$Ю положим $l_0(tx) = t \|x\|$. Тогда $l_0 (x) = \|x\|$ и $\|l_0\|$ =1. Остается продолжить $l$ на $x$ с сохранением нормы. 
\end{proof}

Из этой теоремы можно получить, что в случае бесконечномерного пространства $X$ для всякого $n$ найдутся такие векторы $x_1, \ldots, x_n \in X$ и функционалы $l_1, \ldots, l_n \in X^*$, что $l_i (x_j) = \delta_{ij}$. В частности поэтому, сопряженное пространство тоже бесконечномерно. 

\begin{to_con}
    Пусть $X_0$ -- конечномерное подпростанство нормированного пространства $X$. Тогда $X_0$ топологически дополняемо в $X$, т.е. существует такое замкнутое линейное подпространство $X_1$, что $X$ является прямой алгебраической суммой $X_0$ и $X_1$, а естественные алгебраические проекции $P_0$ и $P_1$ на $X_0$ и $X_1$ непрерыны. 
\end{to_con}

\begin{proof}[$\triangle$]
    \textit{Можно} найти базис $x_1, \ldots, x_n$ пространства $X_0$ и элементы $l_i \in X^*$ с $l_i (x_j) = \delta_{ij}$. Положим
    \begin{equation*}
        X_1 \overset{\mathrm{def}}{=}  \bigcap_{i=1}^{n} \Ker l_i,
        \hspace{5 mm} 
        P_0 [x] \overset{\mathrm{def}}{=} \sum_{i=1}^{n} l_i (x) x_i,
        \hspace{5 mm} 
        P_1[x] \overset{\mathrm{def}}{=} x - P_0 x.
    \end{equation*}
    Для всякого $j$ имеем $P_0 [x_j] - l_j (x_j) x_j = x_j$. В таком контексте становится понятно, что $P_0 |_{X_1} = 0$, и $X_0 \cap X_1 = \{0\}$, $X = X_) \oplus X_1$, ибо $x - P_0 x \in X_1$ ввиду равенств $l_j (x-P_0 x) = l_j(x) - l_j (x) l_j(x_j) = 0$. Непрерывность $P_0$ и $P_1$ понятна из опредления, более того сопадают с алгебраическими проектированиями на $X_0$ и $X_1$.
\end{proof}


