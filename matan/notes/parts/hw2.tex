\subsection*{Т17}

Для последовательностей 
\begin{equation*}
    x = (x(1), \ldots, x(k), \ldots),
\end{equation*}
рассмотрим пространство вида
\begin{equation*}
    l_p = \{x \mid \|x\|_p \in \mathbb{R} \},
\end{equation*}
где 
\begin{equation*}
    \|x\|_p = \l(
        \sum_{k=1}^{\infty} |x(k)|^p
    \r)^{1/p}.
\end{equation*}
Возьмём пространство $l_p$ как множество, но добавим норму из пространства $l_q$, где $\infty > q > p$. 

Рассмотрим шар $A_n$ вида
\begin{equation*}
    A_n = \{x \in l_p \mid \|x\|_p \leq n\},
    \hspace{5 mm}
    l_p = \bigcap_{n=1}^{\infty} A_n.
\end{equation*}
Докажем от противного, что $A_n$ нигде не плотно. 

Пусть существует такой $R > 0$ и $x_0 \in A_n \colon B_R (x_0) \subset \cl A_n = A_n$. 
\begin{equation*}
    \forall x \in l_p \colon  \ \ \ 
    \rho_q (x, x_0) < R,
    \hspace{0.5cm} \Rightarrow \hspace{0.5cm}
    x \in A_n \hspace{0.5cm} \Rightarrow \hspace{0.5cm}
    \|x\|_p \leq n.
\end{equation*}
Рассмотрим некоторую последовательность
\begin{equation*}
    z(k) = \frac{R}{2} \frac{1}{\sum_{\kappa=1}^{\infty} \frac{1}{\kappa^{q/p}}} \frac{1}{k^{1/p}}.
\end{equation*}
Для начала,
\begin{equation*}
    \left(
        \sum_{k=1}^{\infty} (z(k))^{q}
    \right)^{1/q} = \|z\|_q = \frac{R}{2} < + \infty.
\end{equation*}
Далее, видим гармонический ряд
\begin{equation*}
    \sum_{k=1}^{\infty} (z(k))^p = + \infty,
    \hspace{0.5cm} \Rightarrow \hspace{0.5cm}
    \exists N  \colon  \sum_{k=1}^{N} (z(k))^p > (2n)^p.
\end{equation*}
Теперь рассмотрим набор <<частниных последовательностей>>
\begin{equation*}
    y(k) = \left\{\begin{aligned}
        &z(k), \ &k \leq N,
        &0, &k > N.
    \end{aligned}\right.
\end{equation*}
Теперь рассмотрим последовательность $h(k) = (x_0 + y) (k)$, для которой верно, что
\begin{enumerate}
    \item $\rho_q (h, x_0) = \|y\|_q \leq R/2$, откуда следует $\|h\|_p \leq n$.
    \item $\|h\|_p \geq \|y\|_p - \|x_0\|_p > 2n -n=n$, а тогда $\|h\|_p >n$, таким образом пришли к противоречию. 
\end{enumerate}

Полное пространство нельзя представить, как объединение нигде не плотных множеств, получается $l_p$ не полно. Осталось доказать, что $A_n$ замкнуто.

Пусть $t$ -- точка прикосновения. Тогда $\forall \varepsilon > 0$ найдётся 
\begin{equation*}
    \forall\varepsilon > 0 \ \ \ 
    \exists x_\varepsilon \in A_n \colon 
    \rho_q (t, x_\varepsilon) < \varepsilon,
    \hspace{0.5cm} \encircled{\Rightarrow} \hspace{0.5cm}
    \sum_{k=1}^{N} |t(k) - x_\varepsilon(k)|^q < \varepsilon
    \hspace{0.5cm} \Rightarrow \hspace{0.5cm}
    |t(k) - x_\varepsilon (k)|< \varepsilon^{1/q},
\end{equation*}
получается это правда и для
\begin{equation*}
    \encircled{\Rightarrow} \hspace{5 mm}
    \forall N \in \mathbb{N} \ 
    \left(
        \sum_{k=1}^{N} |t(k)|^p
    \right)^{1/p} \leq t(k) - x_\varepsilon (k) + x_\varepsilon(k) 
    \leq
    \left(
        \sum_{k=1}^{N} |t(k) - x_\varepsilon (k)|^p
    \right)^{1/p}  + 
    \left(
        \sum_{k=1}^{N} |x_\varepsilon (k)|^p
    \right)^{1/p} \leq 
    \left(
        N \varepsilon^{p/q}
    \right)^{1/p} + n,
\end{equation*}
что стремится к $n$ при $\varepsilon \to 0$. Таким образом $\|t\|_p \leq n$. 


И, наконец, докажем, что не выполняеся принцип равномерной ограниченности. Рассмотрим функционалы
\begin{equation*}
    F_n [x] = \sum_{k=1}^{n} x(k).
\end{equation*}
Верно, что
\begin{equation*}
    \forall x \in l_1 \ \ 
    |F_n[x]| \leq \|x\|_1.
\end{equation*}
По норме $\|\circ\|_2$ верно, что $(x, e_n)$, где $e_n = (1, \ldots, 1, 0,\ldots,0,\ldots)$.



\subsection*{Т18}

\begin{to_lem}[Лемма Рисса или лемма о перпендикуляре]
    Если $X_0$ -- замкнутое линейное подпространство в нормированом пространстве $X$, $X_0 \neq X$, тогда
    \begin{equation*}
        \forall \varepsilon > 0, \ \exists x_\varepsilon \in X \colon \|x_\varepsilon\| = 1,
        \hspace{5 mm} 
        \|x_\varepsilon - y\| \geq 1 - \varepsilon \ \ \forall y \in X_0.
    \end{equation*}
\end{to_lem}

\begin{proof}[$\triangle$]
    Найдётся $z \in X \backslash X_0$, положим $\delta = \inf\{
        \|z - u\| \mid y \in X_0
    \} > 0$.
    Тогда выберем
    \begin{equation*}
        \varepsilon_0 > 0 \colon  \frac{\delta}{\delta+\varepsilon_0} > 1 - \varepsilon,
    \end{equation*}
    выберем $y_0 \in X_0$ такой, что $\|z - y_0\| < \delta + \varepsilon_0$.

    Далее, считая
    \begin{equation*}
        x_\varepsilon = \frac{z-y_0}{\|z-y_0\|}, \  \ \forall y \in X_0.
    \end{equation*}
    Теперь оценим
    \begin{equation*}
        \|x_\varepsilon - y\| = \frac{1}{\|z-y_0\|} \|z-y_0-\|z-y_0\|y\| \geq \frac{\delta}{\delta+\varepsilon_0} > 1 - \varepsilon.
    \end{equation*}
    Заметим, что
    \begin{equation*}
        v = y_0 + \|z-y_0\| y \in X_0,
        \hspace{0.5cm} \Rightarrow \hspace{0.5cm}
        \|z-v\| \geq \delta.
    \end{equation*}
\end{proof}



\begin{to_con}
    В $\forall X$  (бесконеномерном, нормированном пространстве) $\exists (x_n) \colon  \|x_n\| = 1$ и
    $\|x_n - x_k\| \geq 1$, $n \neq k$.
\end{to_con}

Как следставие все шары $R > 0$ в $X$ некомпактны. 

Всякое бесконечное подмножество компакта имеет предельную точку. 





\subsection*{Т19}

% Колмогоров Фомин
% Богачев Смолянов


\begin{to_thr}
    Пусть $E$ -- банахово пространство, $F \subset E$ -- его линейное подпространство. Тогда всякий ограниченный линейный функционал $\lambda \in F'$ продолжается до линейного функционала на всём $E$ без увеличения его нормы.
\end{to_thr}

% трансфинитная индукция
% дашков -- фивты

