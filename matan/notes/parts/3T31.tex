\subsubsection*{Т31}
 
\begin{to_thr}[]
    Для $\forall$ СОФ $g$ $\in \mathcal D'$, с носителем в открытом шаре, существует такая РОФ $f$ и $k \in \mathbb{N}$, что $f^{(k)} = g$.
\end{to_thr}

\begin{to_def}
    \textit{Носитель обобщенной функции} $\supp f$ -- дополнение к объединению всех открытых множеств $U$, на которых $f$ равна нулю.  Обобщённая функция $f$ равна нулю на $U$, если $\langle f \,|\, \varphi \rangle =0$ для всех $\varphi$ таких, что $\supp \varphi$ содержится в $U$. 
\end{to_def}

Примером такой функции (которая не является $m$-й производной РОФ), носитель которой не помещается в открытый шар, может служить распределение вида
\begin{equation*}
    f = \sum_{k=1}^{\infty} \delta^{(k)} (x-k).
\end{equation*}
Докажем от противного, пусть $g^{(m)} = f$ и $g$ -- РОФ. 