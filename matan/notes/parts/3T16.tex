% см. Конспект Карасева сразу после Банаха-Щтейнгауза

\subsubsection*{Т16}


\begin{to_thr}[Расходимость ряда Фурье в точке]
    Существует непрерывная $2\pi$-периодическая функция, ряд Фурье которой расходится в точке 0.
\end{to_thr}

\begin{proof}[$\triangle$]
На пространстве $\dot{C}[-\pi, \pi]$ непрерывных $2\pi$ -периодических функций с нормой $\|\cdot\|_C$ определим линейный функционал 
\begin{equation*}
    \lambda_n (f) = \int_{-\pi}^{\pi} f(t) D_n(t) \d t,
\end{equation*}
это значение $n$-й частичной суммы ряда Фурье в точке $0,\, T_n(f, 0)$. Можно заметить по определению нормы, что его норма равна
\begin{equation*}
    \|\lambda_n\| = \int_{-\pi}^{\pi} |D_n(t)| \d t.
\end{equation*}
Оценим интеграл модуля ядра Дирихле стандартным способом:
\begin{align*}
    I &= \int_{-\pi}^{\pi} \frac{|\sin(n+1/2)x|}{2 \pi |\sin x/2|} \d x \geq 
    \int_{-\pi}^{\pi} \frac{|\sin(n+1/2)x|}{\pi |x|} \d x =  \int_{-\pi(1+1/2)}^{\pi(1+1/2)} \frac{|\sin u|}{\pi |u|} \d u 
    \geq \\ &\geq 
    \int_{-\pi(1+1/2)}^{\pi(1+1/2)} \frac{\sin^2 u}{\pi |u|} \d u = \int_{-\pi(1+1/2)}^{\pi(1+1/2)} \frac{1-\cos 2u}{2\pi |u|}\d u \to 
    \int_{-\infty}^{\infty} \frac{1 - \cos 2 u}{2\pi |u|} \d u = + \infty,
    \hspace{5 mm} 
    n \to \infty.
\end{align*}
Получается, то нормы функционалов $\lambda_n$ при $n \to \infty$ не являются ограниченными. Следовательно, по теореме Банаха-Штейгауза, примененной в обратную сторону, для некоторой функции $f \in \dot{C}[-\pi, \pi]$ значения $\lambda_n (f) = T_n(f, 0)$ не будут ограничены, и, следоватеьно, расходятся при $n \to \infty$. 


\end{proof}