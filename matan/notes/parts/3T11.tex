
\subsubsection*{Т11. Поиск функционала}

Далее будем обозначать за $\mathcal D(A)$ область определения оператора $A$, и $\mathcal R (A)$ -- область значений. Оператор действует $A \colon  X \mapsto Y$, где $X$ и $Y$ -- линейные нормированные пространства. 

\begin{to_def}
    Говорится, что линейный оператор $A \colon  X \mapsto Y$ \textit{непрерывен} а точка $x \in \mathcal D(A)$, если $\forall \, \{x_n\}_{n=1}^{\infty} \subset \mathcal D(A)$, сходящейся к $x$ в $X$, $A x_n \to A x$ в $Y$. Оператор \textit{глобально непрерывен}, если он непрерывен $\forall x \in \mathcal D(A)$. 
\end{to_def}

\begin{to_lem}
    Для того, чтобы линейный оператор $A$ был непрерывен на всей $\mathcal D(A)$, необходимо и достаточно, чтобы он был непрерывен в нуле.
\end{to_lem}

\begin{to_def}
    Линейный оператор $A \colon  X \mapsto Y$ называется \textit{ограниченным}, если $\exists C >0 \colon  \|A x\|_Y \leq C \cdot \|x\|_X$ $\forall x \in \mathcal D(A)$. Наименьшее из чисел $C$ называется \textit{нормой} оператора $A$ и обозначается $\|A\|$. 
\end{to_def}

\begin{to_lem}
    Для того, чтобы линейный оператор был ограниченным, необходимо и достаточно, чтобы он переводил всякое ограниченное в $X$ множество, в ограниченное в $Y$. 
\end{to_lem}

\begin{to_thr}[]
    Оператор $A$ непрерывен тогда, и только тогда, когда он ограничен.
\end{to_thr}

\begin{to_thr}[о норме линейного оператора]
    Верно, что 
    \begin{equation*}
        \|A\| = \sup_{\|x\|=1} \|Ax\| = \sup_{\|x\|\neq 0} \frac{\|Ax\|}{\|x\|}.
    \end{equation*}
\end{to_thr}


Найдём норму функционала
\begin{equation*}
    A \colon  f \mapsto \sum_{k=0}^N (-1)^k f\left(\frac{k}{N}\right),
\end{equation*}
на пространстве $C[0,1]$. 


Вообще нормированным пространством мы называем пару вида $(X, \|\circ\|_X)$. И пусть есть некоторый непрерывный ограниченный оператор из $X$ в $Y$. Если $Y = \mathbb{C}(\mathbb{R})$, 
\begin{equation*}
    A = F \colon  X \to \mathbb{C}(\mathbb{R}),
\end{equation*}
то  $A$ называют \textit{функционалом}. Выберем в качетсве $X = C[0, 1]$, а в качетсве $F \colon C[0, 1] \mapsto \mathbb{C}(\mathbb{R})$.
Функционал вида
\begin{equation*}
    F[f] = \sum_{k=0}^{n}(-1)^k f\left(\frac{k}{n}\right).
\end{equation*}
Что есть норма функционала? Норма функционала есть
\begin{align*}
    \|F\| 
    &= \sup_{\|f\|_{\infty} \leq 1} |F[f]|  
    = \sup_{\|f\|_{\infty} = 1} |F[f]| 
    = \inf \{
        L > 0 \mid |F[f]| \leq L\|f\|_\infty
    \}, 
    \hspace{5 mm} \forall f \in C[0, 1].
\end{align*}
\texttt{Глобально, это доказывается, например, в Константинове очень подробно.} 

\texttt{Всегда легко сверху ограничить.} Тривиальный шаг:
\begin{equation*}
    |F[f]| = \bigg|
        \sum_{k=0}^{n} (-1)^k f\left(
            \frac{k}{n}
        \right)
    \bigg| \leq \sum_{k=0}^{n} \bigg|
        f\left(\frac{k}{n}\right)
    \bigg| \leq \sum_{k=0}^{n} \sup_{x \in [0,1 ]} |f(x)| = (n+1) \cdot \|f\|_{\infty}.
\end{equation*}
Продолжаем, 
\begin{equation*}
    \frac{|F[f]|}{\|f\|_{\infty}} \leq n + 1,
    \hspace{0.5cm} \Rightarrow \hspace{0.5cm}
    \|F\| = \sup_{\|f\|_{\infty} = 1} |F[f]| \leq n+1.
\end{equation*}
Теперь выберем функцию $f_s(x) = f(k/n) = (-1)^k$. На ней мы действительно достигаем супремум, тогда
\begin{equation*}
    \|F\| = |F[f_s]| = n+1.
\end{equation*}
Таким образом нашли норму оператора. 

В более общем случае можем показать, что
\begin{equation*}
F[f] = \sum_{k=1}^{n}  c_k x(t_k),
\hspace{5 mm} 
    |F[f]| \leq \sum_{k=1}^{n} |c_k| \cdot \|f\|_{\infty},
    \hspace{0.5cm} \Rightarrow \hspace{0.5cm}
    \|F\| \leq \sum_{k=1}^{n} |c_k|.
\end{equation*}
Далее, определив схожим образом непрерывную функцию $\tilde{f}$, равную $\sign c_k$ в $t=t_k$ увидим, что $\|\tilde{f}\|=1$, 
\begin{equation*}
    \|F[\tilde{f}]\| \geq |F[\tilde{f}]| = \sum_{k=1}^{n}  |c_k|,
\end{equation*}
таким образом решили чуть более общую задачу. 






