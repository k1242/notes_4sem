% пропускается начало семинара 
% https://www.youtube.com/watch?v=bXpmoUzjm3k&t=361s&ab_channel=%D0%A1%D1%82%D0%B0%D0%BD%D0%B8%D1%81%D0%BB%D0%B0%D0%B2%D0%9B%D0%B5%D0%BE%D0%BD%D0%B8%D0%B4%D0%BE%D0%B2%D0%B8%D1%87%D0%9E%D0%B3%D0%B0%D1%80%D0%BA%D0%BE%D0%B2


\subsection{Распредления (обобщенные функции)}

Работать будем с $\mathcal D (X) \overset{\mathrm{def}}{=} C_0^{\infty} (X)$, $X \subseteq \mathbb{R}$.  Функция называется \textit{финитной}, если $\supp \varphi = K \subset X$, 
\begin{equation*}
    \supp \varphi \overset{\mathrm{def}}{=} \bar{Y},
    \hspace{5 mm} 
    Y = \{x \in X \mid \varphi(x) \neq 0\}.
\end{equation*}
Далее будем считать $\DS \equiv \mathcal D$. 

Вспомним, что $\varphi_n \dto \varphi$ означает $\exists [a, b] \supset \supp \varphi_n$ и $\supp \varphi$, а также $\varphi_n^{(k)} \overset{[a, b]}{\rightrightarrows}  \varphi^{(k)}$, и тогда пишут, что $\lim_{n\to\infty} \varphi_n \overset{\mathcal D}{=}  \varphi$. 

Хочется определить пространство линейный непрерывных функционалов. Далее, договоримся обозначать $f(\varphi) \equiv f[\varphi] \overset{\mathrm{def}}{=} \langle f \,|\, \varphi \rangle$. 

\begin{to_def}
    Функционал $f \colon \mathcal D \mapsto \mathbb{C}(\mathbb{R})$ \textit{непрерывен} в $\mathcal D'$, если
    \begin{equation*}
        \lim_{n\to\infty} \varphi_n \overset{\mathcal D}{=}  \varphi,
        \hspace{0.5cm} \Rightarrow \hspace{0.5cm}
        \lim_{n\to\infty} \langle f \,|\, \varphi_n \rangle  = \langle f \,|\, \varphi \rangle.
    \end{equation*}
\end{to_def}

\begin{to_def}
    Всякий линейный функционал из $\mathcal D'$ называют \textit{обобщенной функцией} на $\mathcal D$. 
\end{to_def}

Каждая локально-интегрируемая функция порождает некоторую обобщенную, их назовём \textit{регулярными}. Если не существует такой локально-интегрируемой функции в $D$ для функционала из $\mathcal D'$, то это \textit{сингулярная обобщенная функция}. Стоит заметить, что регулярные обобщенные функции плотны в $\mathcal D'$, а их пополнением являются сингулярные. 

Например, $\delta(x)$ можно представить как предел РОФ, где под пределом имеется ввиду
\begin{equation*}
    f_n \dto f \hspace{5 mm} 
    (f_n)_{n\in \mathbb{N}} \subset \mathcal D', \ f \sin \mathcal D',
    \hspace{5 mm} 
    \forall \varphi \in \mathcal D \ \ \lim_{n \to \infty} \langle f_n \,|\, \varphi \rangle  = \langle f \,|\, \varphi \rangle,
\end{equation*}
в частности тогда пишут
\begin{equation*}
    \lim_{n \to \infty} f_n \dseq f 
    \hspace{5 mm} \Leftrightarrow \hspace{5 mm} 
    *w. \lim_{n \to \infty} f_n = f. 
\end{equation*}


% зорич, глава 17, параграф 4, свёртка функци, п.4 -- начальные сведения. 


\subsubsection*{Т23}

Найдём пределы последовательностей регулярныъ элементов пространства $\mathcal D'$, при
\begin{equation*}
    \lim_{n \to \infty} \langle \cos(nx) \,|\, \varphi \rangle  = \lim_{n\to \infty} 
    \int_{-\infty}^{+\infty} \cos(nx) \varphi(x) \d x = \lim_{n\to \infty} 
    \Re \int_{-\infty}^{+\infty} \d x e^{i nx} \varphi(x) = 
    \lim_{n\to \infty} \Re \hat{\varphi}(n) =  \langle 0 \,|\, \varphi \rangle \dseq 0.
\end{equation*}
По той же причине 
\begin{equation*}
    * w. \lim_{n\to \infty} n \sin (nx) - 0.
\end{equation*}
Найдём некоторые пределы в терминах обобщенных функций. В частности,
\begin{equation*}
    * w \lim_{a\to+0} \frac{a}{\pi (a^2+x^2)} = * w \lim_{\mathcal B_a} \frac{a}{\pi (a^2+x^2)},
\end{equation*}
где $\mathcal B_a$ -- база, состоящяя из всех последовательностей, стремящихся к $0$. 
В частности, при $a=1/n$, перейдём к Т24(а). Прямым вычислением, находим
\begin{equation*}
    \bigg\langle \frac{a}{\pi(a^2+x^2)} \,\bigg|\, \varphi \bigg\rangle  = 
    \int_{-\infty}^{+\infty} \frac{a \varphi(x)}{\pi (a^2+x^2)} \d x = 
    \left(
    \lim_{\Lambda_+ \to +\infty} \int_{0}^{\Lambda_+} + 
    \lim_{\Lambda_+ \to -\infty} \int_{\Lambda_-}^{0}  
    \right) \frac{a \varphi(x)}{\pi (a^2+x^2)} \d x,
\end{equation*}
что интегрируя по частям можем свести  к $\arctg x$:
\begin{align*}
    \frac{1}{\pi} \lim_{\Lambda_+ \to +\infty} \left\{
        \arctg \frac{x}{a} \varphi(x) \bigg|_{0}^{\Lambda_+} - 
        \int_{0}^{\Lambda_+} \left(\arctg \frac{x}{a}\right) \varphi' (x) \d x
    \right\} + \frac{1}{\pi} \lim_{\Lambda_- \to -\infty} \left\{
        \arctg \frac{x}{a}\varphi(x) \bigg|_{\Lambda_-}^{0} - \int_{\Lambda_-}^{0}  
        \left(\arctg \frac{x}{a}\right) \varphi' (x) \d x
    \right\}
     = \\ = 
     - \frac{1}{2} \varphi(x) \bigg|_{0}^{\infty}  + \frac{1}{2} \varphi(x) \bigg|_{-\infty}^{0} = \varphi(0) = \langle \delta(x) \,|\, \varphi \rangle,
\end{align*}
таким образом мы нашли, что
\begin{equation*}
    \bigg\langle \frac{a}{\pi(a^2+x^2)} \,\bigg|\, \varphi \bigg\rangle  =  
     \langle \delta(x) \,|\, \varphi \rangle.
\end{equation*}

Второй пункт сводится к интегрированию
\begin{equation*}
    \frac{1}{\pi} \int_{0}^{x}  \frac{1}{t} \sin \frac{t}{a} \d t = \frac{1}{\pi} \int_{0}^{x/a} \frac{d \sin y}{d y}  \d y = \frac{1}{\pi} \si\left(\frac{x}{a}\right).
\end{equation*}
Вспоминая, что
\begin{equation*}
    \frac{1}{\pi}\si(+\infty) = \frac{1}{\pi} \frac{\pi}{2} = \frac{1}{2},
    \hspace{5 mm}   
    \frac{1}{\pi} \si(-\infty) = \frac{1}{\pi} \left(-\frac{\pi}{2}\right),
    \hspace{0.5cm} \Rightarrow \hspace{0.5cm}
    \lim_{n\to \infty} \frac{1}{\pi} \frac{\sin nx}{x} \dseq \delta(x). 
\end{equation*}




\subsubsection*{Т25}

Теперь найдём предел вида
\begin{equation*}
    * w. \lim_{n\to \infty} \frac{n^3 x}{(1+n^2 x^2)^2} = 
    * w. \lim_{a \to +0} \frac{x a}{(x^2+a^2)^2} = F,
\end{equation*}
для этого 
\begin{equation*}
    \bigg\langle \frac{xa}{(x^2+a^2)^2} \,\bigg|\, \varphi \bigg\rangle  = 
    \int_{-\infty}^{+\infty} \frac{xa}{(x^2+a^2)^2} \varphi(x) \d x = \int_{-\infty}^{+\infty} 
    \left(-\frac{1}{2}\right)\left(
        \frac{\partial }{\partial x} \frac{a}{x^2 + a^2}
    \right) \varphi(x) \d x = 
    \frac{1}{2} \int_{-\infty}^{+\infty}  \frac{a \varphi'(x)}{x^2 + a^2} \d x \underset{a\to+0}{\to} \frac{\pi}{2} \varphi'(0),
\end{equation*}
что, учитывая предыдущую задачу, позволяет записать
\begin{equation*}
     \frac{\pi}{2} \langle \delta(x) \,|\, \varphi' \rangle = \bigg\langle \left(-\frac{\pi}{2}\right)\delta'(x)   \,\bigg|\, \varphi \bigg\rangle,
     \hspace{0.5cm} \Rightarrow \hspace{0.5cm}
     w. \lim_{a \to +0} \frac{x a}{(x^2+a^2)^2}  = -\frac{\pi}{2}\delta'(x), 
\end{equation*}
\subsubsection*{Т26}

Алгоритмично, обработаем выражение
\begin{equation*}
    \langle d \,|\, \varphi \rangle = \langle g \cdot \delta \,|\, \varphi  \rangle  = 
    \langle \delta \,|\, g \cdot \varphi \rangle  = g(0) \varphi(0) = \langle g(0_ \delta) \,|\, \varphi \rangle ,
\end{equation*}
так приходим к упрощенному выражению вида
\begin{equation*}
    g(x) \delta(x) \dseq g(0) \delta(x).
\end{equation*}

Во втором пункте $f = g \delta'$, упростим выражение
\begin{align*}
    \langle f \,|\, \varphi \rangle  &= \langle g \delta' \,|\, \varphi \rangle = 
    \langle \delta' \,|\, g \varphi \rangle  = - \langle \delta \,|\, (g\varphi)' \rangle = -
    \langle \delta \,|\, g' \varphi + g \varphi' \rangle  
    = \\ &=
    -g'(0) \varphi(0) - g(0) \varphi'(0) = - g'(0) \langle \delta \,|\, \varphi \rangle - g(0) \langle \delta \,|\, \varphi' \rangle = \langle g(0) \delta' - g'(0) \delta \,|\,  \rangle,
\end{align*}
таким образом приходим к равенству в $\mathcal D'$:
\begin{equation*}
    g(x) \delta'(x) \dseq - g'(0) \delta(x) + g(0) \delta'(x).
\end{equation*}


\subsubsection*{Т27}

\begin{to_lem}
    В $\mathcal D'$ верно, что
    \begin{equation*}
        (g \cdot f)^{(m)} = \sum_{k=0}^{m} C_m^k g^{(k)} f^{(m-k)}.
    \end{equation*}
\end{to_lem}

Найдём производные отдельных <<строительных блоков>>:
\begin{equation*}
    H(x) = \left\{\begin{aligned}
        &1, &x\geq 0, \\
        &0, &x<0,
    \end{aligned}\right.
    \hspace{5 mm}
    \tilde{H} (x) = \left\{\begin{aligned}
        &1,   &x>0, \\
        &1/2, &x=0, \\
        &0,   &x<0. \\
    \end{aligned}\right. 
\end{equation*}
Докажем, что
\begin{equation*}
    \sign x = 2 \tilde{H}(x) - 1,
    \hspace{0.5cm} \Rightarrow \hspace{0.5cm}
    \sign'(x) = 2 \tilde{H}'(x) = 2 \delta(x).
\end{equation*}
Первый шаг, по определению,
\begin{align*}
    \big\langle \sign'(x) \,\big|\, \varphi  \big\rangle  = -\langle \sign x \,|\, \varphi' \rangle = - \int_{-\infty}^{+\infty} \sign x \varphi'(x) \d x =   
    \int_{-\infty}^{0} \varphi'(x) \d x - \int_{0}^{+\infty} \varphi'(x) \d x = 2 \varphi(0) = 
    \langle 2 \delta(x) \,|\, \varphi \rangle.
\end{align*}


Теперь покажем, что
\begin{equation*}
    |x|' = (x \sign x)' = \sign x + x \sign' x = \sign x + x 2 \delta(x) = \sign x.
\end{equation*}
Также можем найти вторую производную
\begin{equation*}
    |x|'' = \sign'(x) = 2 \delta(x). 
\end{equation*}


\textbf{Пункт а}. Теперь легко посчитать, что
\begin{equation*}
    \left(g(x) \sign x\right)' = g'(x) \sign x + g(x) \sign' (x) = g'(x) \sign x + 2 g(0) \delta(x),
\end{equation*}
где равенства подразумеваются в пространстве $\mathcal D'$. Для второй производной, находим
\begin{align*}
    \left(g(x) \sign x\right)'' &= g'' \sign x + 2 g'(x) \sign'x + g(x) \sign''(x) = 
    g''(x) \sign x + 4 g'(0) \delta(x) + 2 g(x) \delta'(x) 
    = \\ &=
    g''(x) \sign x + 4 g'(0) \delta(x) + 2\left(
        -g'(0) \delta(x) + g(0) \delta'(x)
    \right) = g''(x) \sign x + 2 g'(0) \delta(x) + 2 g(0) \delta'(x).
\end{align*}


\textbf{Пункт б}. Сразу подставим значение $g(x) = (x+1)e^{|x|}$:
\begin{align*}
    g' &= e^{|x|}\left(1+(x+1)\sign x \right), \\
    g'' &= 
    e^{|x|} \left(
        1 + \sign x + 2 \delta(x) (x+1) + \sign x + x + 1
    \right) = 2 e^{|x|} \left(
        1 + x/2 + \sign x + \delta(x)
    \right).
\end{align*}
\subsubsection*{Т28}

Докажем, что слабая сходимость $\delta_{x_n} \to \delta_{x_0}$ эквивалентна обычной сходимости $x_n \to x_0$. Другими словами есть набор $f_n (x) = \delta(x-x_n)$ которые в пределе сходится к $f(x) = \delta(x-x_0)$. 

По определению,
\begin{equation*}
    \forall \varphi \in \mathcal D \ \ 
    \lim_{n\to \infty} \langle \delta(x-x_n) \,|\, \varphi \rangle = \langle \delta(x-x_0) \,|\, \varphi \rangle.
\end{equation*}
В силу непрерывности функций в $\mathcal D$:
\begin{equation*}
    \forall \varphi \in D \ \ \lim_{n\to \infty}   \varphi(x_n) \varphi(x_0).
\end{equation*}
Наконец, это можно переписать в виде
\begin{equation*}
    \forall \varphi \in \mathcal D \ 
    \forall \varepsilon > 0 \ 
    \exists N(\varphi, \varepsilon) \in \mathbb{N} \ 
    \forall n \geq N(\varphi. \varepsilon) \ \ 
    |\varphi(x_n) - \varphi(x_0)| < \varepsilon.
\end{equation*}
Это было дано. Хочется показать, что из этого следует $x_n \to x_0$, или
\begin{equation*}
    \lim_{n\to \infty} x_n = x_0 
    \hspace{5 mm} 
    \forall \varepsilon > 0 \ 
    \exists N_\varepsilon \in \mathbb{N} \forall n \geq N_\varepsilon \ \ 
    |x_n - x_0| < \varepsilon.
\end{equation*}
Докажем от противного, пусть $x_n \to x_1 \neq x_0$. Тогда пусть $\varkappa = |x_1-x_0|/3$, выберем функцию $\varphi = \cf{X_0} (x)+ -\cf{X_1} (x)$,  где $X_0 = [x_0-\varkappa, x_0 + \varkappa]$, $X_1 = [x_1-\varkappa, x_1 + \varkappa]$. В таком случае, в пределе, $\langle f_n (x) \,|\, \varphi \rangle = -1$, при этом по условию $\langle f(x) \,|\, \varphi \rangle = 1$, что приводит нас к противоречию. 



% 


\subsubsection*{Т29 и Т30}


Сначала доакажем, что всякое распределение $\lambda \in \mathcal D'(\mathbb{R})$  имеет первообразную, то есть такую $\mu \in \mathcal D' (\mathbb{R})$, что $\mu' = \lambda$ в смысле дифференцирования обобщенных функций. Потом
докажем, что любые две первообразные одного и того же распределения отличаются на константу. 



\begin{to_lem}
    Пусть  $f \in \mathcal D' \left(\mathbb{R}\right)$ и так оказалось, что $f' =0$, тогда $f$ имеет вид $\langle f | \varphi \rangle = c \int_{-\infty}^{\infty} \varphi(x) \d x$.
\end{to_lem}

\begin{proof}[$\triangle$]
    Утверждается, что $c = \langle f \,|\, \varphi_0 \rangle $ годится, где
    \begin{equation*}
        \varphi_0 \in \mathcal D \left(\mathbb{R}\right) \colon  
        \int_{-\infty}^{+\infty}  \varphi_0(x) \d x = 1.
    \end{equation*}
    Итак, любую функцию $\varphi \in \mathcal D$ можно представить в виде
    \begin{equation*}
        \varphi = - \theta \cdot \varphi_0 + \theta \cdot \varphi_0,
        \hspace{10 mm} 
        \theta = \int_{-\infty}^{+\infty}  \varphi(x) \d x.
    \end{equation*}
    Зададим функцию от вида
    \begin{equation*}
        \psi(x) = \int_{\infty}^{x} \left(
            \varphi(t) - \theta \varphi_0 (t)
        \right) \d t \ \ \in \DS.
    \end{equation*}
    Собирая всё вместе находим
    \begin{equation*}
        \psi' = \varphi - \theta \cdot \varphi_0,
        \hspace{0.5cm} \Rightarrow \hspace{0.5cm}
        \langle f \,|\, \varphi \rangle = \langle f \,|\, \psi' + \theta \varphi_0 \rangle = 
        \langle f \,|\, \psi' \rangle + \theta \langle f \,|\, \varphi_0 \rangle,
    \end{equation*}
    где $- \langle f' \,|\, \psi \rangle =0$ по условию. Также $\langle f \,|\, \varphi_0 \rangle  = c$, тогда верно, что
    \begin{equation*}
        \psi' = c \cdot \theta = c \int_{-\infty}^{+\infty}  \varphi(x) \d x, 
        \hspace{5 mm} 
        \QED
    \end{equation*}
\end{proof}


\begin{to_thr}[]
    Для всякой обобщенной функции $f$ из $\DS$ существует $g \in \mathcal D' (\mathbb{R})$ такая, что $g' \overset{D'}{=}  f$. Для всякой другой $h \in \mathcal D'\left(\mathbb{R}\right)$ верно, что  если $h' \overset{\mathcal D'}{=} f$, то $g - h \overset{\mathcal D'}{=} c$.
\end{to_thr}

\begin{proof}[$\triangle$]
    Точно также берем некоторую $\varphi$, $\psi$. Положим, по определению, что 
    \begin{equation*}
        \langle g \,|\, \varphi \rangle \overset{\mathrm{def}}{=} - \langle f \,|\, \Psi \rangle ,
    \end{equation*}
    для которого хотелось бы показать линейность и непрерывность. 

Для этого рассмотрим
\begin{equation*}
    \langle g \,|\, \varphi_1 + \varphi_2 \rangle = - \langle f \,|\, \psi_1 + \psi_2 \rangle =
    - \bigg\langle f \,\bigg|\, \int_{-\infty}^{x} (\varphi_1 + \varphi_2 - (\theta_1 + \theta_2) \varphi_0) \d t \bigg\rangle = - \langle f \,|\, \psi_1 \rangle- \langle f \,|\, \psi_2 \rangle.
\end{equation*}
Осталось показать непрерывность, точнее показать, что линейной отображение $\varphi \to \psi$ непрерывно на $\DS$.

Рассмотрим в частности $\varphi_k \dto 0$, для них $\theta_k \to 0$  при $k \to \infty$. Построим теперь $\varphi_k - \theta_k \varphi_0 \in \mathcal D(\mathbb{R})$ и имеют нулевые интегралы. Более того 
\begin{equation*}
    \hat{l} (\varphi_k) = \psi_k = \int_{-\infty}^{x} \left(
        \varphi_k (t) - \theta_k \varphi_0 (t)
    \right) \d t \in \DS.
\end{equation*}
Итого $\psi_k \to 0$ при $k \to \infty$, что и завершает доказательство непрерывности. 

\end{proof}

% \input{parts/3T30.tex}
\subsubsection*{Т31}
 
\begin{to_thr}[]
    Для $\forall$ СОФ $g$ $\in \mathcal D'$, с носителем в открытом шаре, существует такая РОФ $f$ и $k \in \mathbb{N}$, что $f^{(k)} = g$.
\end{to_thr}

\begin{to_def}
    \textit{Носитель обобщенной функции} $\supp f$ -- дополнение к объединению всех открытых множеств $U$, на которых $f$ равна нулю.  Обобщённая функция $f$ равна нулю на $U$, если $\langle f \,|\, \varphi \rangle =0$ для всех $\varphi$ таких, что $\supp \varphi$ содержится в $U$. 
\end{to_def}

Примером такой функции (которая не является $m$-й производной РОФ), носитель которой не помещается в открытый шар, может служить распределение вида
\begin{equation*}
    f = \sum_{k=1}^{\infty} \delta^{(k)} (x-k).
\end{equation*}
Докажем от противного, пусть $g^{(m)} = f$ и $g$ -- РОФ.  %-

