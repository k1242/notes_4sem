Для формулы Ньютона-Лейбница условие липшицевости можно ослабить до следующего:

\begin{to_def}
    Функция $F$ на промежутке $I$ \textit{абсолютно непрерывна} , если $\forall \varepsilon > 0 \ \exists \delta_\varepsilon > 0$, такое что
    $\forall x_1 \leq y_1 \leq x_2 \leq y_2 \leq \ldots \leq x_N \leq y_N \in I$ из неравенства
    \begin{equation*}
        |x_1 - y_1| + |x_2 - y_2| + \ldots + |x_N - y_N| \leq \delta
    \end{equation*}
    следует, что
    \begin{equation*}
        |F(x_1)-F(y_1)| + 
        |F(x_2)-F(y_2)| + 
        \ldots +
        |F(x_N)-F(y_N)| \leq \varepsilon.
    \end{equation*}
    \texttt{
    Говоря неформально, сумма модулей приращений функции на системе непересекающихся отрезков должна \\ стремиться к нулю при суммарной длине системы, стремящейся к нулю.
    } 
\end{to_def}

\begin{to_lem}
    Абсолютно непрерывная на отрезке функция имеет на нём ограниченную вариацию.
\end{to_lem}


\begin{to_thr}[]
    Всякая обобщенная первообразная 
    \begin{equation*}
        F(x) = \int_a^x f(t) \d t,
    \end{equation*}
    для некоторой $f \in L_1 [a, b]$, является абсолютно непрерывной и её производная всюду существует и совпадает с $f$.
\end{to_thr}


\begin{to_lem}
    Абсолютно непрерывная на отрезке функция раскладывается в сумму двух монотонных абсолютно непрерывных функций.
\end{to_lem}


\begin{to_thr}[]
    Абсолютно непрерывная функция $F \colon [a, b] \mapsto \mathbb{R}$ почти всюду имеет производную и является обобщенной первообразной своей производной с выполнением формулы Ньютона-Лейбница
    \begin{equation*}
        F(b) - F(a) = \int_a^b F' (t) \d t.
    \end{equation*}
\end{to_thr}

\red{Легко показать через\ldots ух, ну по \textbf{лемме Безиковича}.}

\begin{to_con}[обобщенное интегрирование по частям]
    Если $f \in L_1 [a, b]$, а $g$ абсолютно непрерывна, то верна формула интегрирования по частям
    \begin{equation*}
        \int_a^b f g \d x = F(x) g(x) \bigg|_a^b
        - \int_a^b F(x) g'(x) \d x,
    \end{equation*}
    где $F(x) = \int_a^x f(t) \d t$.
\end{to_con}


\begin{to_lem}
    Функция $f \colon [a, b] \mapsto \mathbb{R}$ абсолютно непрерывна тогда и только тогда, когда она может быть сколь угодно близко в $B$-норме приближена кусочно-линейными функциями.
\end{to_lem}

\red{А дальше про борелевские меры на отрезках и интеграл Лебега–Стилтьеса.}