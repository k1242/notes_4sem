Предел по базе -- предел по фильтру. Для любых двух множеств из фильтра

% интеграл макшейна -- сумма римана 
% вычисления на решетке -- 

\subsection{Производные обобщенных функций}



% Богачев Смолянов -- топ

% \subsubsection*{Т29, \ Т30}

% % 8.5.3.

% \red{Следующее утверждение -- страница 472, Богачев-Смолянов.}



\subsection{Преобразование Фурье обобщенных функций}


% колмогоров фомин -- тоже только табличка


Найдём фурье преобразование $n$-й производной дельта-функции
\begin{equation*}
    I = \langle F[\delta^{(n)} (x)] \,|\, \varphi \rangle = \langle \delta^{(n)} (x) \,|\, F[\varphi] \rangle  = (-1)^n F^{(n)} [\varphi][0] = 
    \frac{(-1)^n}{i^n} F[x^n \varphi](0),
\end{equation*}
тогда
\begin{equation*}
    I = = 
    \frac{(-1)^n}{i^n} \langle \delta(x) \,|\, F[x^n \varphi] \rangle = i^n \langle F[\delta(x) \,\mid\, x^n \varphi]\rangle,
    \hspace{0.5cm} \Rightarrow \hspace{0.5cm}
    F\left[\delta^{(n)} (x)\right] \dseq \frac{(ix)^n}{\sqrt{2\pi}}.
\end{equation*}
 
\textbf{Физический подход}:
\begin{equation*}
    \int_{-\infty}^{+\infty} \frac{\d t}{\sqrt{2\pi}} e^{-ixt} \frac{d^n}{d t^n} \delta(t) = 
    (-1)^n \int_{-\infty}^{+\infty} \frac{d t}{\sqrt{2\pi}} \delta(t) \frac{d^n }{d t^n} e^{- i x t} = 
    (i x)^n \int_{-\infty}^{+\infty} \frac{d t}{\sqrt{2\pi}}  \delta(t) e^{-ixt} = (ix)^n \cdot \frac{1}{\sqrt{2\pi}}.
\end{equation*}

\red{Для третьего примера -- Кудрявцев, учебник, том 3, страница 297 по файлу}



\subsubsection*{Т33}

\begin{to_thr}[]
    Пусть $f \in \DS$ и оказалось, что $F[f] \in \DS$. Тогда $f \equiv 0$.
\end{to_thr}





