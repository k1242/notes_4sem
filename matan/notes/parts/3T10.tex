\subsubsection*{Т10. Критерий Йордана-фон Неймана}

% \begin{to_def}
%     Если норма в банаховом пространстве $E$ порождается положительно определенным скалярным произведением 
%     \begin{equation*}
%         \|x\| = \sqrt{(x, x)},
%     \end{equation*}
%     то $E$ называется \textit{гильбертовым пространством}.
% \end{to_def}

Хочется понять, можно ли ввести на пространстве $C[a,b]$ скалярное произведение так, что норма пространства будет получаться из этого скадяного произведения.

\begin{to_thr}[критерий Йордана-фон Неймана]
    Норма $\|\circ\|_X$ порождается скалярным произведением тогда, и тоглько тогда, когда
    $\|\circ\|_X$ удовлетворяет правилу параллелограмма:
    \begin{equation*}
        \forall x, y \in X \ \ \ 
        \|x+y\|^2_X + \|x-y\|^2_X = 2 \|x\|_X^2 + 2 \|y\|_X^2.
    \end{equation*}
\end{to_thr}

\noindent
Выберем $C[0, \pi/2]$, и $x(t) = \cos t$, $y(t) = \sin t $. Заметим, что
\begin{equation*}
    \|x\|_{\infty} = \|y\|_{\infty} = 1,
    \hspace{5 mm}
    \|x+y\|_{\infty} = \sqrt{2}, \hspace{5 mm} \|x-y\|_\infty = 1,
    \hspace{5 mm} 
    2+1 \neq 2+2,
\end{equation*}
таким образом пространство не гильбертово.

