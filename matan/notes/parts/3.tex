\subsubsection*{Неравенства Гёльдера и Минковского}


\begin{to_def}
    \textit{Абсолютно интегрирумыми функциями} на измеримом $X \subseteq \mathbb{R}^n$ называют $f \colon X \mapsto \mathbb{R}$ с конечным интегралом $\int_X |f(x)| \d x$. \textit{Расстоянием}\footnote{
        В силу неравенства $|f(x) - g(x)| \leq |f(x)| + |g(x)|$ расстояние конечно.
    } между функциями $f$ и $g$ будем считать $\int_X |f(x)-g(x)| \d x$.
\end{to_def}

\begin{to_def}
    Обозначим через $L_1 (X)$ факторпространство  линейного пространства абсолютно интегрируемых функций по его линейному подпространству почти всюду равных нулю функций. То есть функции на $0$ расстоянии считаем равными. \textit{Нормой} будем считать
    \begin{equation*}
        \|f\|_1 = \int_X |f(x)| \d x.
    \end{equation*}
\end{to_def}

\begin{to_def}
    Для измеримого по Лебегу $X \subset \mathbb{R}^n$ и числа $p \geq 1$ \textit{факторпространство} измеримых по Лебегу функций на $X$ с конечной (полу)нормой
    \begin{equation*}
        \|f\|_p
        = 
        \left(
            \int_X |f|^p \d x
        \right)^{1/p},
    \end{equation*}
    по модулю функций равных нулю почти всюду,
    назовём $L_p (X)$.
\end{to_def}

\begin{to_thr}[Неравенство Гёльдера]
    Возьмём $p, \, q > 1$ такие, что $1/p + 1/q = 1$. Пусть $f \in L_p (X)$ и $g \in L_q(X)$. Тогда
    \begin{equation*}
        \int_X |fg| \d x \leq \|f\|_p \cdot \|g\|_q.
    \end{equation*}
\end{to_thr}

\begin{proof}[$\triangle$]
    Для доказательства достаточно проинтегрировать неравенство вида
    \begin{equation*}
        |fg| \leq \frac{|f|^p}{p} + \frac{|g|^q}{q}.
    \end{equation*}
    \red{Осталось получить само неравенство.}
\end{proof}

\begin{to_con}
    Для измеримых функций и чисел $p, \, q > 0$, таких что $1/p + 1/q = 1$, имеет место формула
    \begin{equation}
        \label{8_1}
        \|f\|_p = \sup \left\{
            \int_X fg \d x \ \bigg| \  \|g\|_q \leq 1
        \right\}.
    \end{equation}
\end{to_con}


\begin{proof}[$\triangle$]
По неравенству Гёльдера норма $f$ не менее супремума правой части \red{(?)}, более того равенство достигается при выборе
\begin{equation*}
    g(x) = \frac{\sign f(x) |f(x)|^{p-1}}{\|f\|_p^{p-1}}.
\end{equation*}
\end{proof}

\begin{to_def}
    Функция $f \colon V \mapsto \mathbb{R}$ на векторном пространство называется выпуклой, если для любых $x, \, y \in V$ и любого $t \in (0, 1)$ имеет место неравенство
    \begin{equation*}
        f ( (1-t) x + ty) \leq (1-t) f(x) + t f(y).
    \end{equation*}
    Функция называется \textit{строго выпуклой}, если неравенство строгое $\forall x \neq y$ и $t \in (0, 1)$. 
\end{to_def}

\begin{to_lem}
    Если в семействе функций $f_\alpha \colon V \mapsto \mathbb{R}$, $\alpha \in A$, все функции выпуклые, то
    \begin{equation*}
        f(x) = \sup \{f_\alpha (x) \mid \alpha \in A\}
    \end{equation*}
    тоже выпуклая\footnote{
        Если разрешить в определении выпуклости значение $+ \infty$.
    }.
\end{to_lem}


\begin{to_thr}[Неравенство Минковского]
    Для функций $f, \, g \in L_p$ при $p \geq 1$
    \begin{equation*}
        \|f + g\|_p \leq \|f\|_p + \|g\|)_p.
    \end{equation*}
\end{to_thr}



\subsubsection*{Полнота пространства интегрируемых функций}

Далее в разделе всегда предполагается суммирование по $k$ от $1$ до $\infty$. Глобально можно сказать, что \texttt{в нормированном пространстве вопрос полноты сводится в вопросу сходимости рядов}, у которых сходятся суммы норм. 

\begin{to_lem}
    Пусть у последовательности функций $(u_k)$ из $L_p (X)$ сумма
    $\sum \|u_k\|_p$
    оказалась конечной. Тогда $S(x) = \sum u_k (x)$ определена для почти всех $x$ и
    $\|S\|_p \leq \sum \|u_k\|_p.$
\end{to_lem}


\begin{to_lem}
    Пусть у последовательности функций $(u_k)$ из $L_p (x)$ сумма
    $\sum \|u_k\|_p$
    оказалась конечной. Тогда $S(x) = \sum u_k (x)$ определена для почти всех $x$ и
    $S = \sum u_k$
    в смысле сходимости в пространстве $L_p (X)$.
\end{to_lem}


\begin{to_thr}[]
    Пространство $L_p (X)$ полно.
\end{to_thr}



Вообще сходимость в $L_p (X)$ может не означать поточечной сходимости ни в одной точке.