\subsection{Банаховы пространства и их двойственные}





\subsubsection*{Т8}

Здесь, и далее $p(x) = \|x\|$, $q(x) = \|x\|'$. Нормы \textit{эквивалентны}, если
\begin{equation*}
    \exists m, M \ \colon  \ m p(x) \leq q(x) \leq M p(x) \  \ \forall x.
\end{equation*}
% ульянов бахвалов -- Rn, 3,99
% Кудрявцев, том 3
% базис Гамиля есть всегда!
Так вот, всегда есть $\{e_k\}_{k=1}^n$ базис Гамиля, такой что $x = \sum_{k=1}^n x_k e_k$, где естественно ввести норму вида
\begin{equation*}
    p(x) = \sum_{k=1}^n |x_k|.
\end{equation*}
Пусть $q(x)$ -- ещё одна норма на $X$, в качестве мажоранты выберем $M = \max\limits_{i=1, \ldots, n}  q(e_i)$.
Теперь можем оценить сумму сверху:
\begin{equation*}
    q(x) = q\left(
        \sum_{k=1}^n x_k e_k
    \right) \leq \sum_{k=1}^{n} |x_k| q(e_k) \leq M \cdot p(x).
\end{equation*}
И оценить снизу:
\begin{equation*}
    |q(x) - q(y)| \leq q(x-y) \leq M \cdot p (x-y),
\end{equation*}
вообще это значит, что $q$ -- липшецев функционал, -- непрерывный функционал на $X$ с нормой $p$, а тогда и $q(x)$ непрерывный функционал $X$ с нормой $p(x)$. 


\begin{to_lem}
    Шары в пространстве компактны тогда, и только тогда, когда $\dim X < + \infty$.
\end{to_lem}

Рассмотрим сферу $S = \{x \in X \mid p(x) = 1\}$ -- компакт. Но мы знаем, что непрерывный функционал на компакте достигает своего миниимума:
\begin{equation*}
    \min_{x \in S} q(x) = \min_{p(x)=1} q(x) = m > 0.
\end{equation*}
Тогда на сфере $S$ верно, что $q(x) \geq m$. Тогда в $X$ $q(x) \geq m \cdot p(x)$. Действительно,
\begin{equation*}
    q(tx) = |t|q(x), \ \  p(tx) = |t| \, p(x), \hspace{0.5cm} \Rightarrow \hspace{0.5cm}
    q(tx) = \frac{p(tx)}{p(x)} q(x) \geq m\, p(tx).
\end{equation*}
Собственно, $m p(x) \leq q(x) \leq M \cdot p(x)$,  $Q.\, E.\, D.$




\subsubsection*{Т9. Пространство \texorpdfstring{$c$}{с}}

Пространство состоит из некоторых бесконечномерных <<векторов>>  (последовательностей):
\begin{equation*}
    x = \big(x(1), x(2), \ldots, x(k), \ldots\big),
    \hspace{5 mm}
    \bigg|
        \lim_{k \to \infty} x(k)
    \bigg| < + \infty.
\end{equation*}
Норма определена, как
\begin{equation*}
    p(x) = \|x\|_c = \sup_{k \in \mathbb{N}} |x(k)| = \|x\|_{\infty}.
\end{equation*}
Докажем, что это пространство является банаховым, а именно полноту по $\|\cdot\|_{\infty}$ норме. 

Рассмотрим последовательность $x_n$, где 
\begin{equation*}
    x_n = (x_n (1), \ldots, x_n(k), \ldots).
\end{equation*}
Глобально хотим показать, что
\begin{equation*}
    \forall \varepsilon > 0 \ \ 
    \exists N(\varepsilon) \in \mathbb{N} \ \ 
    \forall n \geq N(\varepsilon) \ \ 
    \forall l \in \mathbb{N} \ \  \ 
    \|x_{n+l} -x_{n}\|_{\infty} = \sup_{k \in \mathbb{N}} |x_{n+l} (k) - x_n (k)| < \varepsilon.
\end{equation*}
Попробуем через это продраться: из сходимости следует, что
\begin{equation*}
    \forall k \in \mathbb{N} \ \ |x_{n+l} (k) - x_n (k) | < \varepsilon.
\end{equation*}
Здесь можем выделить $(x_n(k))_{n \in \mathbb{N}}$ -- числовая фундаментальная в $\mathbb{R}$. 
По критерию Коши:
\begin{equation*}
    \forall k \in \mathbb{N} \ \ 
    \lim_{n \to \infty} x_n (k) = y(k) \in \mathbb{R},
\end{equation*}
уставнавливается покомпонентая сходимость. 
Теперь рассмотрим
\begin{equation*}
    \sup_{k\in \mathbb{N}} |x_n (k) - y(k)| = \|x_n - y\|_{\infty} < \varepsilon,
\end{equation*}
что автоматически означает, что $\exists y$ такой, что
\begin{equation*}
    \lim_{n \to \infty} x_n  = y.
\end{equation*}

Следующий этап -- показать, что
\begin{equation*}
    \exists \lim_{k \to \infty} y(k) \in \mathbb{R},
\end{equation*}
то есть показать полноту пространства:
\begin{align*}
    |y(k+q)-y(k)| 
    &= |y(k+q) - x_n (k+q) + x_n (k+q) - x_n (k) + x_n (k) - x_n (k)|\\
    &\leq |y(k+q)-x_n (k+q)| + |y(k) - x_n (k)| + |x_n (k+q) - x_n (k)| \\
    & < \varepsilon/3 + \varepsilon/3 + \varepsilon/3 = \varepsilon.
\end{align*}
Таким образом мы доказали полноту пространства\footnote{
    \red{$c_0, c_{00}, l_\infty$ -- банаховы ли?
    $\encircled{!}$: $c_0$ (сходящиеся к $0$), $c_{00}$ (финитные), $l_\infty$ (ограниченные). 
    }  
} . 


\subsubsection*{Т10. Критерий Йордана-фон Неймана}

% \begin{to_def}
%     Если норма в банаховом пространстве $E$ порождается положительно определенным скалярным произведением 
%     \begin{equation*}
%         \|x\| = \sqrt{(x, x)},
%     \end{equation*}
%     то $E$ называется \textit{гильбертовым пространством}.
% \end{to_def}

Хочется понять, можно ли ввести на пространстве $C[a,b]$ скалярное произведение так, что норма пространства будет получаться из этого скадяного произведения.

\begin{to_thr}[критерий Йордана-фон Неймана]
    Норма $\|\circ\|_X$ порождается скалярным произведением тогда, и тоглько тогда, когда
    $\|\circ\|_X$ удовлетворяет правилу параллелограмма:
    \begin{equation*}
        \forall x, y \in X \ \ \ 
        \|x+y\|^2_X + \|x-y\|^2_X = 2 \|x\|_X^2 + 2 \|y\|_X^2.
    \end{equation*}
\end{to_thr}

\noindent
Выберем $C[0, \pi/2]$, и $x(t) = \cos t$, $y(t) = \sin t $. Заметим, что
\begin{equation*}
    \|x\|_{\infty} = \|y\|_{\infty} = 1,
    \hspace{5 mm}
    \|x+y\|_{\infty} = \sqrt{2}, \hspace{5 mm} \|x-y\|_\infty = 1,
    \hspace{5 mm} 
    2+1 \neq 2+2,
\end{equation*}
таким образом пространство не гильбертово.



\subsubsection*{Т11. Поиск функционала}

Далее будем обозначать за $\mathcal D(A)$ область определения оператора $A$, и $\mathcal R (A)$ -- область значений. Оператор действует $A \colon  X \mapsto Y$, где $X$ и $Y$ -- линейные нормированные пространства. 

\begin{to_def}
    Говорится, что линейный оператор $A \colon  X \mapsto Y$ \textit{непрерывен} а точка $x \in \mathcal D(A)$, если $\forall \, \{x_n\}_{n=1}^{\infty} \subset \mathcal D(A)$, сходящейся к $x$ в $X$, $A x_n \to A x$ в $Y$. Оператор \textit{глобально непрерывен}, если он непрерывен $\forall x \in \mathcal D(A)$. 
\end{to_def}

\begin{to_lem}
    Для того, чтобы линейный оператор $A$ был непрерывен на всей $\mathcal D(A)$, необходимо и достаточно, чтобы он был непрерывен в нуле.
\end{to_lem}

\begin{to_def}
    Линейный оператор $A \colon  X \mapsto Y$ называется \textit{ограниченным}, если $\exists C >0 \colon  \|A x\|_Y \leq C \cdot \|x\|_X$ $\forall x \in \mathcal D(A)$. Наименьшее из чисел $C$ называется \textit{нормой} оператора $A$ и обозначается $\|A\|$. 
\end{to_def}

\begin{to_lem}
    Для того, чтобы линейный оператор был ограниченным, необходимо и достаточно, чтобы он переводил всякое ограниченное в $X$ множество, в ограниченное в $Y$. 
\end{to_lem}

\begin{to_thr}[]
    Оператор $A$ непрерывен тогда, и только тогда, когда он ограничен.
\end{to_thr}

\begin{to_thr}[о норме линейного оператора]
    Верно, что 
    \begin{equation*}
        \|A\| = \sup_{\|x\|=1} \|Ax\| = \sup_{\|x\|\neq 0} \frac{\|Ax\|}{\|x\|}.
    \end{equation*}
\end{to_thr}


Найдём норму функционала
\begin{equation*}
    A \colon  f \mapsto \sum_{k=0}^N (-1)^k f\left(\frac{k}{N}\right),
\end{equation*}
на пространстве $C[0,1]$. 


Вообще нормированным пространством мы называем пару вида $(X, \|\circ\|_X)$. И пусть есть некоторый непрерывный ограниченный оператор из $X$ в $Y$. Если $Y = \mathbb{C}(\mathbb{R})$, 
\begin{equation*}
    A = F \colon  X \to \mathbb{C}(\mathbb{R}),
\end{equation*}
то  $A$ называют \textit{функционалом}. Выберем в качетсве $X = C[0, 1]$, а в качетсве $F \colon C[0, 1] \mapsto \mathbb{C}(\mathbb{R})$.
Функционал вида
\begin{equation*}
    F[f] = \sum_{k=0}^{n}(-1)^k f\left(\frac{k}{n}\right).
\end{equation*}
Что есть норма функционала? Норма функционала есть
\begin{align*}
    \|F\| 
    &= \sup_{\|f\|_{\infty} \leq 1} |F[f]|  
    = \sup_{\|f\|_{\infty} = 1} |F[f]| 
    = \inf \{
        L > 0 \mid |F[f]| \leq L\|f\|_\infty
    \}, 
    \hspace{5 mm} \forall f \in C[0, 1].
\end{align*}
\texttt{Глобально, это доказывается, например, в Константинове очень подробно.} 

\texttt{Всегда легко сверху ограничить.} Тривиальный шаг:
\begin{equation*}
    |F[f]| = \bigg|
        \sum_{k=0}^{n} (-1)^k f\left(
            \frac{k}{n}
        \right)
    \bigg| \leq \sum_{k=0}^{n} \bigg|
        f\left(\frac{k}{n}\right)
    \bigg| \leq \sum_{k=0}^{n} \sup_{x \in [0,1 ]} |f(x)| = (n+1) \cdot \|f\|_{\infty}.
\end{equation*}
Продолжаем, 
\begin{equation*}
    \frac{|F[f]|}{\|f\|_{\infty}} \leq n + 1,
    \hspace{0.5cm} \Rightarrow \hspace{0.5cm}
    \|F\| = \sup_{\|f\|_{\infty} = 1} |F[f]| \leq n+1.
\end{equation*}
Теперь выберем функцию $f_s(x) = f(k/n) = (-1)^k$. На ней мы действительно достигаем супремум, тогда
\begin{equation*}
    \|F\| = |F[f_s]| = n+1.
\end{equation*}
Таким образом нашли норму оператора. 

В более общем случае можем показать, что
\begin{equation*}
F[f] = \sum_{k=1}^{n}  c_k x(t_k),
\hspace{5 mm} 
    |F[f]| \leq \sum_{k=1}^{n} |c_k| \cdot \|f\|_{\infty},
    \hspace{0.5cm} \Rightarrow \hspace{0.5cm}
    \|F\| \leq \sum_{k=1}^{n} |c_k|.
\end{equation*}
Далее, определив схожим образом непрерывную функцию $\tilde{f}$, равную $\sign c_k$ в $t=t_k$ увидим, что $\|\tilde{f}\|=1$, 
\begin{equation*}
    \|F[\tilde{f}]\| \geq |F[\tilde{f}]| = \sum_{k=1}^{n}  |c_k|,
\end{equation*}
таким образом решили чуть более общую задачу. 







\subsubsection*{Т12}


Пусть функция $g$ непрерывна на $[a, b]$. Найдём норму линейного отображения $M_g \colon L_2[a, b] \mapsto L_2[a,b]$, где $A_g(f) = [f]$ -- мультипликативный оператор. Здесь $X=Y=L_2[a,b]$. 

По опредению, норма оператора $\|A_g\| = \sup_{\|f\|_X = 1} \|A_g [f]\|_Y$. Аналогично, ищем ограничение сверху:
\begin{equation*}
    \|A_g [f]\|_2^2 = \|gf\|_2^2 = \int_{[a, b]} |gf|^2 (x) \mu(\d x) \leq 
    \int_{[a, b]} \left\{
        \sup_{x\in[a,b]} |g(x)|
    \right\}^2 |f(x)|^2 \mu(\d x).
\end{equation*}
Вынесенный супремум позволит записать:
\begin{equation*}
    \|A_g[f]\|_2^2 \leq \|g\|_\infty^2 \|f\|_2^2,
    \hspace{0.5cm} \Rightarrow \hspace{0.5cm}
    \|A_g\| = \sup_{\|f\|_2=1} \|A_g[f]\|_2 \leq \|g\|_\infty.
\end{equation*}
Далее покажем, что норма не достигается, но сколь угодно близко приближается.


Есть функция 
\begin{equation*}
    \sup_{x\in[a,b]} |g(x)| = |g(c)|,
\end{equation*}
есть некоторая $f_\varepsilon \in L_2[a, b]$ вида
\begin{equation*}
    f(x) = \left\{\begin{aligned}
        &\alpha_\varepsilon, &x\in[c-\varepsilon, c+\varepsilon], \\
        &0, &x\notin[c-\varepsilon, c+\varepsilon],
    \end{aligned}\right.
    \hspace{0.5cm} \Rightarrow \hspace{0.5cm}
    \|f_\varepsilon\|_2^2 = \int_{[c-\varepsilon, c+\varepsilon]} \hspace{-10mm} \alpha_\varepsilon^2 \mu(d x) =
    \alpha_\varepsilon^2 \cdot 2 \varepsilon = 1,
    \hspace{5 mm} 
    \alpha_\varepsilon = \frac{1}{\sqrt{2\varepsilon}}.
\end{equation*}
В таком случае рассмотрим
\begin{equation*}
    \|A_g [f_\varepsilon]\|_2^2 = \|g f_\varepsilon\|_2^2 = \alpha_\varepsilon^2 \int_{[c-\varepsilon, c+\varepsilon]} \hspace{-10mm} |g(x)|^2 \mu(dx) = \alpha_\varepsilon^2 \cdot 2 \varepsilon |g(x_{c,\varepsilon})|^2 \underset{\varepsilon\to 0}{\to} \|g\|_\infty^2,
\end{equation*}
в силу непрерывности $g$, по теореме о среднем.


Можно пойти другим путем, по определению:
\begin{equation*}
    \forall \varepsilon \in (0, \|g\|_\infty), \hspace{5 mm} 
    \exists x_\varepsilon \subseteq [a, b] \ 
    g(x) \geq \|g\|_\infty - \varepsilon,
\end{equation*}
почти всюду на $X_\varepsilon$. Выберем $h(x)$ вида
\begin{equation*}
    h(x) = \sign g(x) \ \cf{X_\varepsilon} (x),
    \hspace{5 mm} 
    \|h_\varepsilon\|_1 = \|h_\varepsilon\|_2 = \mu(X_\varepsilon), 
\end{equation*}
тогда верно, что
\begin{equation*}
    \|A_g\| \geq \|A_g [h_\varepsilon]\|_1 \cdot \|h_\varepsilon\|_1 = \int_{[a, b]} |g(x)| \cf{X_\varepsilon} (x) \mu(\d x) \geq \left(
        \|g\|_{\infty} - \varepsilon
    \right) \chi \mu(X_\varepsilon),
    \hspace{0.5cm} \Rightarrow \hspace{0.5cm}
    \|A_g\| = \|g\|_\infty.
\end{equation*}
Аналогично в $L_2$:
\begin{equation*}
    \|A_g\|^2 \geq \\|g| \cf{X_\varepsilon}\|_2^2 \cdot \|h_\varepsilon\|_2^2 \geq \|g\|_\infty^2 \mu^2(X_\varepsilon),
\end{equation*}
что приводит такому же результату. 


\subsubsection*{Т13}


Сначала  найдём норму оператора $F$, откуда уже получим значение нормы для $J$, где
\begin{equation*}
    F[f] = \int_a^b g(t) f(t) \d t,
    \hspace{5 mm} 
    J[f] = \int_{a}^{b} K(x, y) f(y) \d y,
\end{equation*}
где $g\in C[a,b]$, а $F, \ J$ -- линейные функционалы на $C[a, b]$. 


\textbf{Первая часть}. Функционал $F$ ограничен в силу
\begin{equation*}
    |F[f]| \leq \int_{a}^{b} |g(t) |f(t)| \d t \leq \|f\|_\infty \cdot \int_{a}^{b} |g(t)| \d t.
\end{equation*}
Далее выберем произвольное $\varepsilon  > 0$. По \textit{теореме Кантора} найдётся такое разбиение отрезка $[a, b]$ точками $a=t_0 < t_1 < \ldots < t_n = b$, что колебание $\omega_i(g)$ функции $g$ на $i$-ом отрезке $\Delta_i = [t_{i-1}, t_i]$ удовлетворяет неравенствам 
\begin{equation*}
    \omega_i(g) < \varepsilon, \hspace{5 mm} i = 1,\, 2,\, \ldots,\, n.
\end{equation*}
Разобьём все $\Delta_i$ на две группы. В первую группу отнесем те отрезки, на которых $g$ сохраняет знак. Пусть это будут отрезки $\Delta_1',\ldots,\Delta_r'$. Вторую группу $\Delta_1'', \ldots, \Delta_s''$ образуют отрезки, на которых $g$ меняется знак. В каждом промежутке второго типа существует точка, в которой $g$ обращается в нуль. Ввиду установленных неравенств там $|g(t)|<\varepsilon$. 

На промежутках первого типа положим $\tilde{f}(t) = \sign g(t)$, в остальных точках $\tilde{f}(t)$ -- линейная непрерывная функция, удовлетворяющая неравенству $|\tilde{f}| \leq 1$. Тогда $\|\tilde{f}\|=1$, и 
\begin{align*}
    \|F\| &= \sup_{\|f\|=1} |F[f]| \geq |F[\tilde{f}]| = \bigg|
        \int_a^b g(t) \tilde{f}(t) \d t
    \bigg| = 
    \bigg|
        \sum_{k=1}^{r} \int_{\Delta'_k} |g(t)|\d t + \sum_{i=1}^{s} \int_{\Delta''_i} g(t) \tilde{f}(t) \d t
    \bigg| 
    \geq \\ &\geq
    \sum_{k=1}^{r}  \int_{\Delta'_k} |g(t)| \d t - 
    \sum_{i=1}^{s} \int_{\Delta_i''} = \int_a^b |g(t)| \d t - 
    2 \sum_{i=1}^{s}  \int_{\Delta''_i} |g(t)| \d t \geq \int_{a}^{b} |g(t)| \d t - 2 \varepsilon \cdot \mu[a, b],
\end{align*}
что ввиду произвольности $\varepsilon$ означает, что $\|F\|\geq \int_a^b |g(t)| \d t$, что вместе со знанием супремума позволяет утверждать: $\|f\|=\int_a^b |g(t)| \d t$.


\textbf{Вторая часть}. Переходим к поиску нормы $J$:
\begin{align*}
    \|J[f]\| = \sup_{t\in[a, b]} \bigg| 
        \int_{a}^{b} K(t, s) f(s) \d s
    \bigg| \leq \sup_{t\in[a, b]} \int_{a}^{b}  |K(t, s)| \cdot |f(s)| \d s \leq 
    \|f\| \cdot \sup_{t\in[a, b]} \int_{a}^{b} |K(t, s)| \d s,
\end{align*}
таким образом, по определению
\begin{equation*}
    \|J\| \leq \sup_{t\in[a, b]} \int_{a}^{b} |K(t, s)| \d s \overset{\mathrm{def}}{=}  M.
\end{equation*}
Так как ядро $K$ непрерывно, то непрерывен и интеграл $\int_a^b |K|\d s$, поэтому $\exists t_0 \in [a, b]$ такой, что $M = \int_{a}^{b} |K(t_0, s)|\d s$. 

Как было показано в первой части, $q(x) = \int_{a}^{b}  |K(t_0, s)| f(s) \d s$ -- линейный непрерывный функционал на $C[a, b]$ с нормой равной $M$. Таким образом, выбирая $\tilde{f}$ так, чтобы $\sign \tilde{f(s)} = \sign K(t_0, s)$ может утверждать, что супремум достигается, и 
\begin{equation*}
    \|J\| = M = \sup_{t\in[a, b]} \int_{a}^{b}  |K(t, s)| \d s.
\end{equation*}



\begin{to_thr}[Теоремма Бэра для открытых множеств]
    Счётное семейство открытых всюду плотных подмножеств банахова пространства имеет непустое пересечение.
\end{to_thr}

\begin{to_thr}[Теорема Бэра для замкнутых множеств]
    Если банахово пространство $E$ покрыто счётным семейством замкнутых множеств, то одно из них имеет непустую внутренность. 
\end{to_thr}

\subsubsection*{Т14}

Докажем, что алгебраический базис бесконечномерного банахова пространства не может быть счётным. 

Вводился алгебраический базис Гамиля $\{e_\alpha\}_{\alpha\in A}$, где $\forall x \in E$ представляется в виде
$x \ \sum_{k=1}^{n}  x_k e_{\alpha_k}$. Получается, что нужно показать, что в бесконечномерном банаховом пространстве такой базис не может быть счётным: докажем от противного.


Пусть $\{e_n\}_{n\in \mathbb{N}}$, тогда пространство описывется, как
\begin{equation*}
    E_n = \left\{
        \sum_{k=1}^{n} x_k e_k \mid x_1, \ldots, x_k \in \mathbb{R}
    \right\} = \langle E_1, \ldots, e_n \rangle,
    \hspace{0.5cm} \Rightarrow \hspace{0.5cm}
    E = \bigcup_{n\in \mathbb{N} } E_n.
\end{equation*}
Но по теореме Бэра для замкнутых множеств $E$ не может быть счётным объединением нигде не плотных множеств.

Точнее, это было бы возможно, только с случае непустой внутренности одного из пространств $E_n$, что невозможно. 


% Мы знаем, что $E_n$ -- замкнуто, $E$ -- полно. Тогда, по следствию из теоремы Бэра,
% \begin{equation*}
%     \exists n_0 \in \mathbb{N} \colon \ \ E \subset E_{n_0}, 
% \end{equation*}
% что приводит к противоречию, в силу бесконечности $E$. 


\subsubsection*{Т15}

Приведем пример плотного в $X = C[a,b]$ банахова пространства, со счётным базисом. 

 
По теореме Вейерштрассе система степеней $A$ полна в $C[a, b]$, что равносильно тому, что линейная оболочка системы степеней $A$ плотна на $C[a,b]$. Таким образом, $A$ со счётным базисом, является ответом на задачу.

\begin{to_def}
    Последовательность элементов $\{e_n\}_{n\in \mathbb{N}}$ называется \textit{базисом} в пространстве\footnote{
Если линейное нормированное пространство имеет не более, чем счётный базис, то оно сепарабельно. Однако существуют сепарабельные банаховы пространства без базиса.
    }  $X$, если $\forall x \in X$ существует единственный набор $\{x_i\}_{i \in \mathbb{N}_0}$ таких, что сумма вида (не конечная не при каком $n$)
    \begin{equation*}
        x = \sum_{k=1}^{\infty} x_k e_k = \lim_{n \to \infty} \sum_{k=1}^{n} x_k e_k,
        \hspace{5 mm} \Leftrightarrow \hspace{5 mm} 
        \exists ! \{x_i\}_{i \in \mathbb{N}_0} \ 
        \forall \varepsilon > 0 \ 
        \exists N_\varepsilon \in \mathbb{N}_0 \ 
        \forall n \geq N_\varepsilon \ 
        \|x - \sum_{k=0}^{n} x_k e_k\|_X = \|x-S_n\| < \varepsilon.
    \end{equation*}
\end{to_def}

\begin{to_thr}[Теорема Банаха-Штейнгауза для линейных функционалов]
    Пусть семейство линейный функционалов $Y \subset E'$ ограничено в любой точке банахова пространства $E$, то есть для любого $x \in E$ множество чисел $\{\lambda(x) \mid \lambda \in Y\}$ ограничено. Тогда $Y$ ограничено в смысле нормы в $E'$. 
\end{to_thr}


% см. Конспект Карасева сразу после Банаха-Щтейнгауза

\subsubsection*{Т16}


\begin{to_thr}[Расходимость ряда Фурье в точке]
    Существует непрерывная $2\pi$-периодическая функция, ряд Фурье которой расходится в точке 0.
\end{to_thr}

\begin{proof}[$\triangle$]
На пространстве $\dot{C}[-\pi, \pi]$ непрерывных $2\pi$ -периодических функций с нормой $\|\cdot\|_C$ определим линейный функционал 
\begin{equation*}
    \lambda_n (f) = \int_{-\pi}^{\pi} f(t) D_n(t) \d t,
\end{equation*}
это значение $n$-й частичной суммы ряда Фурье в точке $0,\, T_n(f, 0)$. Можно заметить по определению нормы, что его норма равна
\begin{equation*}
    \|\lambda_n\| = \int_{-\pi}^{\pi} |D_n(t)| \d t.
\end{equation*}
Оценим интеграл модуля ядра Дирихле стандартным способом:
\begin{align*}
    I &= \int_{-\pi}^{\pi} \frac{|\sin(n+1/2)x|}{2 \pi |\sin x/2|} \d x \geq 
    \int_{-\pi}^{\pi} \frac{|\sin(n+1/2)x|}{\pi |x|} \d x =  \int_{-\pi(1+1/2)}^{\pi(1+1/2)} \frac{|\sin u|}{\pi |u|} \d u 
    \geq \\ &\geq 
    \int_{-\pi(1+1/2)}^{\pi(1+1/2)} \frac{\sin^2 u}{\pi |u|} \d u = \int_{-\pi(1+1/2)}^{\pi(1+1/2)} \frac{1-\cos 2u}{2\pi |u|}\d u \to 
    \int_{-\infty}^{\infty} \frac{1 - \cos 2 u}{2\pi |u|} \d u = + \infty,
    \hspace{5 mm} 
    n \to \infty.
\end{align*}
Получается, то нормы функционалов $\lambda_n$ при $n \to \infty$ не являются ограниченными. Следовательно, по теореме Банаха-Штейгауза, примененной в обратную сторону, для некоторой функции $f \in \dot{C}[-\pi, \pi]$ значения $\lambda_n (f) = T_n(f, 0)$ не будут ограничены, и, следоватеьно, расходятся при $n \to \infty$. 


\end{proof}
\subsubsection*{Т17}

Для последовательностей 
\begin{equation*}
    x = (x(1), \ldots, x(k), \ldots),
\end{equation*}
рассмотрим пространство вида
\begin{equation*}
    l_p = \{x \mid \|x\|_p \in \mathbb{R} \},
    \hspace{5 mm} 
    \|x\|_p = \l(
        \sum_{k=1}^{\infty} |x(k)|^p
    \r)^{1/p}.
\end{equation*}
Возьмём пространство $l_p$ как множество, но добавим норму из пространства $l_q$, где $\infty > q > p$. Покажем, что в таком <<дырявом>> пространстве не выполняется теорема Бэра и принцип равномерной ограниченности. 

Рассмотрим шар $A_n$ вида
\begin{equation*}
    A_n = \{x \in l_p \mid \|x\|_p \leq n\},
    \hspace{5 mm}
    l_p = \bigcap_{n=1}^{\infty} A_n.
\end{equation*}
Докажем от противного, что $A_n$ нигде не плотно. 

Пусть существует такой $R > 0$ и $x_0 \in A_n \colon B_R (x_0) \subset \cl A_n = A_n$. 
\begin{equation*}
    \forall x \in l_p \colon  \ \ \ 
    \rho_q (x, x_0) < R,
    \hspace{0.5cm} \Rightarrow \hspace{0.5cm}
    x \in A_n \hspace{0.5cm} \Rightarrow \hspace{0.5cm}
    \|x\|_p \leq n.
\end{equation*}
Рассмотрим некоторую последовательность
\begin{equation*}
    z(k) = \frac{R}{2} \frac{1}{\sum_{\kappa=1}^{\infty} \frac{1}{\kappa^{q/p}}} \frac{1}{k^{1/p}}.
\end{equation*}
Для начала,
\begin{equation*}
    \left(
        \sum_{k=1}^{\infty} (z(k))^{q}
    \right)^{1/q} = \|z\|_q = \frac{R}{2} < + \infty.
\end{equation*}
Далее, видим гармонический ряд
\begin{equation*}
    \sum_{k=1}^{\infty} (z(k))^p = + \infty,
    \hspace{0.5cm} \Rightarrow \hspace{0.5cm}
    \exists N  \colon  \sum_{k=1}^{N} (z(k))^p > (2n)^p.
\end{equation*}
Теперь рассмотрим набор <<частниных последовательностей>>
\begin{equation*}
    y(k) = \left\{\begin{aligned}
        &z(k), \ &k \leq N,
        &0, &k > N.
    \end{aligned}\right.
\end{equation*}
Теперь рассмотрим последовательность $h(k) = (x_0 + y) (k)$, для которой верно, что
\begin{enumerate}
    \item $\rho_q (h, x_0) = \|y\|_q \leq R/2$, откуда следует $\|h\|_p \leq n$.
    \item $\|h\|_p \geq \|y\|_p - \|x_0\|_p > 2n -n=n$, а тогда $\|h\|_p >n$, таким образом пришли к противоречию. 
\end{enumerate}

Полное пространство нельзя представить, как объединение нигде не плотных множеств, получается $l_p$ не полно. Осталось доказать, что $A_n$ замкнуто.

Пусть $t$ -- точка прикосновения. Тогда $\forall \varepsilon > 0$ найдётся 
\begin{equation*}
    \forall\varepsilon > 0 \ \ \ 
    \exists x_\varepsilon \in A_n \colon 
    \rho_q (t, x_\varepsilon) < \varepsilon,
    \hspace{0.5cm} \encircled{\Rightarrow} \hspace{0.5cm}
    \sum_{k=1}^{N} |t(k) - x_\varepsilon(k)|^q < \varepsilon
    \hspace{0.5cm} \Rightarrow \hspace{0.5cm}
    |t(k) - x_\varepsilon (k)|< \varepsilon^{1/q},
\end{equation*}
получается это правда и для
\begin{equation*}
    \encircled{\Rightarrow} \hspace{5 mm}
    \forall N \in \mathbb{N} \ 
    \left(
        \sum_{k=1}^{N} |t(k)|^p
    \right)^{1/p} \leq t(k) - x_\varepsilon (k) + x_\varepsilon(k) 
    \leq
    \left(
        \sum_{k=1}^{N} |t(k) - x_\varepsilon (k)|^p
    \right)^{1/p}  + 
    \left(
        \sum_{k=1}^{N} |x_\varepsilon (k)|^p
    \right)^{1/p} \leq 
    \left(
        N \varepsilon^{p/q}
    \right)^{1/p} + n,
\end{equation*}
что стремится к $n$ при $\varepsilon \to 0$. Таким образом $\|t\|_p \leq n$. 


И, наконец, докажем, что не выполняеся принцип равномерной ограниченности. Рассмотрим функционалы
\begin{equation*}
    F_n [x] = \sum_{k=1}^{n} x(k).
\end{equation*}
Верно, что
\begin{equation*}
    \forall x \in l_1 \ \ 
    |F_n[x]| \leq \|x\|_1.
\end{equation*}
По норме $\|\circ\|_2$ верно, что эти функционалы можно переписать в виде скалярного произведения $(x, e_n)$, где $e_n = (1, \ldots, 1, 0,\ldots,0,\ldots)$:
\begin{equation*}
    F_n [x] = (x, e_n) = \sum_{k=1}^{\infty}  x_{(k)} (e_n)_{(k)} = \sum_{k=1}^{n} x(k),
\end{equation*}
что является проявлением одной из теорем Рисса. Положив $x=e_n$ видим, что норма достигается и $\|F_n\| = n \to \infty$ при $n \to \infty$. Таким образом мы показали, что на таком пространстве не работает принцип равномерной сходимости. 




\subsubsection*{Т18}

Докажем, что в бесконечномерном банаховом пространстве $E$ единичный шар не явяется компактным. 


\begin{to_lem}[Лемма Рисса или лемма о перпендикуляре]
    Если $X_0$ -- замкнутое линейное подпространство в нормированом пространстве $X$, $X_0 \neq X$, тогда
    \begin{equation*}
        \forall \varepsilon > 0, \ \exists x_\varepsilon \in X \colon \|x_\varepsilon\| = 1,
        \hspace{5 mm} 
        \|x_\varepsilon - y\| \geq 1 - \varepsilon \ \ \forall y \in X_0.
    \end{equation*}
\end{to_lem}

\begin{proof}[$\triangle$]
    Найдётся $z \in X \backslash X_0$, положим $\delta = \inf\{
        \|z - u\| \mid y \in X_0
    \} > 0$.
    Тогда выберем
    \begin{equation*}
        \varepsilon_0 > 0 \colon  \frac{\delta}{\delta+\varepsilon_0} > 1 - \varepsilon,
    \end{equation*}
    выберем $y_0 \in X_0$ такой, что $\|z - y_0\| < \delta + \varepsilon_0$.

    Далее, считая
    \begin{equation*}
        x_\varepsilon = \frac{z-y_0}{\|z-y_0\|}, \  \ \forall y \in X_0.
    \end{equation*}
    Теперь оценим
    \begin{equation*}
        \|x_\varepsilon - y\| = \frac{1}{\|z-y_0\|} \|z-y_0-\|z-y_0\|y\| \geq \frac{\delta}{\delta+\varepsilon_0} > 1 - \varepsilon.
    \end{equation*}
    Заметим, что
    \begin{equation*}
        v = y_0 + \|z-y_0\| y \in X_0,
        \hspace{0.5cm} \Rightarrow \hspace{0.5cm}
        \|z-v\| \geq \delta.
    \end{equation*}
\end{proof}



\begin{to_con}
    В $\forall X$  (бесконеномерном, нормированном пространстве) $\exists (x_n) \colon  \|x_n\| = 1$ и
    $\|x_n - x_k\| \geq 1$, $n \neq k$.
    Как следставие все шары $R > 0$ в $X$ некомпактны. 
\end{to_con}

\begin{proof}[$\triangle$]

Всякое бесконечное подмножество компакта имеет предельную точку. 
Последовательность $x_n$ строится по индукции с помошью леммы Рисса. 

\end{proof}




\begin{to_thr}[Теорема Хана-Банаха]
    Пусть $E$ -- банахово пространтво, $F \subset E$ -- его линейное подпространство. Тогда всякий ограниченный линейный функционал $\lambda \in \mathbb{F'}$ продолжается до линейногофункционала на всём $E$ без увеличения его нормы. 
\end{to_thr}

\begin{to_con}
    Для всякого банахова пространства $E$ и его ненулевого элемента $x \in E$ найдётся $\lambda \in E'$, такой что $\|\lambda\|=1$ и $\lambda[x] = \|x\|$.
\end{to_con}

\begin{to_con}
    Естественное отображение банахова пространства в двойственное к его двойственному (второе двойственное) 
    \begin{equation*}
        E \mapsto E'',
        \hspace{5 mm} 
        x \mapsto \left(\lambda \mapsto \lambda(x)\right)
    \end{equation*}
    является вложением, сохраняющим норму. 
\end{to_con}


\begin{to_thr}[Теорема Радона-Никодима в $\mathbb{R}^n$]
    Пусть неотрицательная конечная борелевская мера $\mu$ на $\mathbb{R}^n$ абсолютно непрерывна относительно меры Лебега. Тогда у меры $\nu$ есть плотность, то есть борелевская $f \geq 0$, такая что для всякого борелевского $X$ $\nu(X) = \int_X f(x) \d x$. 
\end{to_thr}



\subsubsection*{Т19}

Выведем из теоремы Хана-Банаха, что всякое конечномерное подпространство $V$ в банаховом постранстве $E$ имеет замкнутое дополнение $W \subseteq E$, такое что $E = V \oplus W$. 

\begin{to_thr}[]
    Для всякого ненулевого элемента $x$ нормированного пространства $X$ найдётся такой функционал $l$, что $\|l\|=1$ и $l[f]=\|f\|$. 
\end{to_thr}

\begin{proof}[$\triangle$]
На одномерном пространствеЮ порожденном $x$Ю положим $l_0(tx) = t \|x\|$. Тогда $l_0 (x) = \|x\|$ и $\|l_0\|$ =1. Остается продолжить $l$ на $x$ с сохранением нормы. 
\end{proof}

Из этой теоремы можно получить, что в случае бесконечномерного пространства $X$ для всякого $n$ найдутся такие векторы $x_1, \ldots, x_n \in X$ и функционалы $l_1, \ldots, l_n \in X^*$, что $l_i (x_j) = \delta_{ij}$. В частности поэтому, сопряженное пространство тоже бесконечномерно. 

\begin{to_con}
    Пусть $X_0$ -- конечномерное подпростанство нормированного пространства $X$. Тогда $X_0$ топологически дополняемо в $X$, т.е. существует такое замкнутое линейное подпространство $X_1$, что $X$ является прямой алгебраической суммой $X_0$ и $X_1$, а естественные алгебраические проекции $P_0$ и $P_1$ на $X_0$ и $X_1$ непрерыны. 
\end{to_con}

\begin{proof}[$\triangle$]
    \textit{Можно} найти базис $x_1, \ldots, x_n$ пространства $X_0$ и элементы $l_i \in X^*$ с $l_i (x_j) = \delta_{ij}$. Положим
    \begin{equation*}
        X_1 \overset{\mathrm{def}}{=}  \bigcap_{i=1}^{n} \Ker l_i,
        \hspace{5 mm} 
        P_0 [x] \overset{\mathrm{def}}{=} \sum_{i=1}^{n} l_i (x) x_i,
        \hspace{5 mm} 
        P_1[x] \overset{\mathrm{def}}{=} x - P_0 x.
    \end{equation*}
    Для всякого $j$ имеем $P_0 [x_j] - l_j (x_j) x_j = x_j$. В таком контексте становится понятно, что $P_0 |_{X_1} = 0$, и $X_0 \cap X_1 = \{0\}$, $X = X_) \oplus X_1$, ибо $x - P_0 x \in X_1$ ввиду равенств $l_j (x-P_0 x) = l_j(x) - l_j (x) l_j(x_j) = 0$. Непрерывность $P_0$ и $P_1$ понятна из опредления, более того сопадают с алгебраическими проектированиями на $X_0$ и $X_1$.
\end{proof}




\subsubsection*{Т20}


Приведем пример замкнутого в топологии нормы множества $X \subset E'$ (двойственное к некоторому банахову пространству), которое не замкнутое в его *-слабой топологии. 


Ответ -- \textit{сфера}, докажем это. Покажем, что для $X \subset E'$ $\cl X = X$ и $w. \cl X \neq X$. Что есть сфера? Сфера есть
\begin{equation*}
    S = \{f \in E' \mid \|f\| = 1\}, 
    \hspace{5 mm} \cl S = S, 
    \hspace{5 mm} w. \cl S = \bar{B},
    \hspace{5 mm} 
    \bar{B} = \{F \in E' \mid \|f\|\leq 1\}.
\end{equation*}
Введём дополнение $S_C \overset{\mathrm{def}}{=} E\backslash S$, и покажем, что оно открыто. 


Выберем $g \in S_c$ с $\|g\|<1$ и $\varepsilon = 1- \|g\|> 0$. Пусть $h \in B_\varepsilon(g)$, более того
\begin{equation*}
    \|h\| = \|g + h - g\| \leq \|g\| + \|h-g\| < 1,
    \hspace{0.5cm} \Rightarrow \hspace{0.5cm}
    B_\varepsilon (g) \subseteq S_c. 
\end{equation*}
Далее, пусть $g \in S_c$ и $\|g\|>1$, тогда $\varepsilon = \|g\|-1 > 0$. Выберем $h \in B_\varepsilon (g)$, тогда 
\begin{equation*}
    \|g\| =  \|h+g-h\| \leq \|h\| + \|g-h\|,
    \hspace{0.5cm} \Rightarrow \hspace{0.5cm}
    \|h\| \geq \|g\|-(\|g\|-1) = 1,
\end{equation*}
получается $\|h\| > 1$ и $B_\varepsilon (g) \subseteq S_c$. Таким образом $S_c$ открыто, $S$ замкнуто. 


Докажем теперь, что $w, \cl S = \bar{B}$. Во-первых $\forall g_0 \notin B$ верно, что
\begin{equation*}
    \|g_0\| > 1,
    \hspace{5 mm} 
    \exists x_0 \in E, \ 
    \exists \varepsilon_0 > 0 \ 
    \forall g \in U_{x_0, g_0, \varepsilon_0} \ \ \|g\|>1,
    \hspace{0.5cm} \Rightarrow \hspace{0.5cm}
    w.\cl S \subseteq B. 
\end{equation*}
В чатсности, покажем, что
\begin{equation*}
    \|g\| \geq |g[x_0]| = |g[x_0] - g_0[x_0] + g_0[x_0]| \geq 
    |g_0[x_0]| - |g[x_0] - g_0[x_0]|,
\end{equation*}
что уже можно сделать строго больше:
\begin{equation*}
    \|g\| > |g_0 [x_0]| - \varepsilon_0 = 1,
\end{equation*}
где $\varepsilon_0 = |g_0[x_0]|-1$. 


Пусть теперь $\forall$ фиксированного $g_0 \in \bar{B}$ с $\|g_0\|<1$. Тогда 
\begin{equation*}
    \exists U(g_0) \colon  g_0 \in \bigcap_{k=1}^N U_{x_k, g_k, \varepsilon_k} \subset U(g_0).
\end{equation*}
Утверждается, что существует ненулевой $g$ такой, что $\forall t \in \mathbb{R}$ с $g_0 + t g \in U(g_0)$. 

Осталось построить цилиндрическое множество по которому <<прогуляемся>> до  нужной нам области.  Пусть
\begin{equation*}
    \varphi(t) = \|g_0 + tg\| \in C(\mathbb{R}),
    \hspace{5 mm} 
    |\varphi(t_1) - \varphi(t_2)| \leq |t_1 - t_2|  \cdot \|g\|.
\end{equation*}
Понятно, что $\varphi(0) = \|g_0\| < 1$. Тогда $\varphi(t) \geq |t| \cdot \|g\|-\|g\|_0 \to \infty$ при $t \to \infty$. По теореме о промежуточных значениях непрерывной функции
\begin{equation*}
     \exists t_0 \in \mathbb{R} \colon  \varphi(t_0) = 1,
     \hspace{0.5cm} \Rightarrow \hspace{0.5cm}
     g_0 +t_0 g \in S.
 \end{equation*} 
 Получается, что взяв точку из шара, и взяв её слабую окрестность, мы находим непустое пересечение этой окрестности со сферой. Из этого следует, что $g_0 \in w. \cl S$, а тогда и $\bar{B} \subseteq w. \cl S$, которое содержится в замкнутом шаре. Вывод: $\bar{B} = w. \cl S$. 







\subsubsection*{Т21}

Докажем, что $*$-слабой топологии $E'$ компактность некоторого множества влечет его замкнутость. 

\begin{to_lem}
    Слабая топология хаусдорфова.
\end{to_lem}


Пусть $K$ -- компакт в ХТП $X$. Пуcть $x \in X \backslash K$. Для $\forall y \in K$ $\exists U_y,\, V_y$ (открытые) такие, что $U_y \cap V_y = \varnothing$, где $x \in U_y$ и $y \in V_y$. 

Рассмотрим систему $S = \left\{V_y \mid y \in K\right\}$
-- открытое покрытие компакта $K$. Также $S_0 = \{V_y \mid y \in F\}$, $F$ -- конечное подмножество $K$ (т.к. $K$ -- компакт). 

Рассмотрим множество $U = \cap_{y\in F} U_y$ -- открытая окрестность точки $x$. Утверждается, что $U \cap K = \varnothing$. Перебирая все точки $x \in K$ получаем доказательство исходного утверждения.  
\subsubsection*{Т22}


Хочется найти такое топологическое пространство, в котором есть компактные, но не замкнутые подмножества. 
В качестве такого  \sout{хаусдорфова} топологического пространства можем выбрать $X = \{a, b\}$, базой топологии $\tau = \{\varnothing, \{a\}, \{a, b\}\}$. 

Пример выглядит искуственным, но, на мой взгляд, большинство примеров нехаумдорфовых пространств выглядят очень искуственно. 