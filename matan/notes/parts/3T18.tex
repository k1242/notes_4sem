\subsubsection*{Т18}

Докажем, что в бесконечномерном банаховом пространстве $E$ единичный шар не явяется компактным. 


\begin{to_lem}[Лемма Рисса или лемма о перпендикуляре]
    Если $X_0$ -- замкнутое линейное подпространство в нормированом пространстве $X$, $X_0 \neq X$, тогда
    \begin{equation*}
        \forall \varepsilon > 0, \ \exists x_\varepsilon \in X \colon \|x_\varepsilon\| = 1,
        \hspace{5 mm} 
        \|x_\varepsilon - y\| \geq 1 - \varepsilon \ \ \forall y \in X_0.
    \end{equation*}
\end{to_lem}

\begin{proof}[$\triangle$]
    Найдётся $z \in X \backslash X_0$, положим $\delta = \inf\{
        \|z - u\| \mid y \in X_0
    \} > 0$.
    Тогда выберем
    \begin{equation*}
        \varepsilon_0 > 0 \colon  \frac{\delta}{\delta+\varepsilon_0} > 1 - \varepsilon,
    \end{equation*}
    выберем $y_0 \in X_0$ такой, что $\|z - y_0\| < \delta + \varepsilon_0$.

    Далее, считая
    \begin{equation*}
        x_\varepsilon = \frac{z-y_0}{\|z-y_0\|}, \  \ \forall y \in X_0.
    \end{equation*}
    Теперь оценим
    \begin{equation*}
        \|x_\varepsilon - y\| = \frac{1}{\|z-y_0\|} \|z-y_0-\|z-y_0\|y\| \geq \frac{\delta}{\delta+\varepsilon_0} > 1 - \varepsilon.
    \end{equation*}
    Заметим, что
    \begin{equation*}
        v = y_0 + \|z-y_0\| y \in X_0,
        \hspace{0.5cm} \Rightarrow \hspace{0.5cm}
        \|z-v\| \geq \delta.
    \end{equation*}
\end{proof}



\begin{to_con}
    В $\forall X$  (бесконеномерном, нормированном пространстве) $\exists (x_n) \colon  \|x_n\| = 1$ и
    $\|x_n - x_k\| \geq 1$, $n \neq k$.
    Как следставие все шары $R > 0$ в $X$ некомпактны. 
\end{to_con}

\begin{proof}[$\triangle$]

Всякое бесконечное подмножество компакта имеет предельную точку. 
Последовательность $x_n$ строится по индукции с помошью леммы Рисса. 

\end{proof}


