\subsection{Т3}

Докажем \textit{формулу Фруллани}
\begin{equation*}
    \int_0^{+\infty} \frac{f(ax)-f(bx)}{x} \d x = f(0) \ln \frac{b}{a}, \hspace{5 mm}
    a > 0, \ b > 0,
\end{equation*}
где $f$ -- непрерывная функция и $\int_A^{\infty} \frac{f(x)}{x} \d x$ сходится $\forall A > 0$. 

В силу условий теоремы
\begin{equation*}
    \int_{A}^{+\infty} \frac{f(ax)}{x} \d x = \int_{Aa}^{+\infty} \frac{f(t)}{t} \d t,
    \hspace{10 mm}
    \int_{A}^{+\infty} \frac{f(bx)}{x} \d x = \int_{Ab}^{+\infty} \frac{f(t)}{t} \d t,
    \hspace{0.5cm} \Rightarrow \hspace{0.5cm}
    \int_{A}^{+\infty}  \frac{f(ax)-f(bx)}{x} \d x = \int_{Aa}^{Ab} \frac{f(t)}{t} \d t.
\end{equation*}
По \textit{первой теореме о среднем}, получаем
\begin{equation*}
    \int_{A}^{+\infty} \frac{f(ax)-f(bx)}{x} \d x = f(\xi) \int_{Aa}^{Ab} \frac{\d t}{t} = f(\xi) \ln \frac{b}{a},
    \hspace{5 mm} 
    Aa \leq \xi \leq Ab.
\end{equation*}
Поскольку функция $f$ непрерывна, то $\lim_{A \to +0} f(\xi) = f(0)$, откуда находим
\begin{equation}
     \lim_{A \to +0} \int_{A}^{+\infty}  \frac{f(ax)-f(bx)}{x} \d x = \int_0^{+\infty} \frac{f(ax)-f(bx)}{x} \d x = f(0) \ln \frac{b}{a}.
 \end{equation} 
Стоит заметить, что если $\int_A^\infty f(x)/x \d x$ расходится, то
\begin{equation*}
    \exists \ \lim_{x \to + \infty} f(x) = f(+\infty),
    \hspace{5 mm}
    \exists \ \int_{A}^{+\infty} \frac{f(x)-f(+\infty)}{x} \d x \ \in \mathbb{R},
    \hspace{0.5cm} \Rightarrow \hspace{0.5cm}
    \int_0^{+\infty}  \frac{f(ax)-f(bx)}{x} \d x = 
    \left(
        f(0) - f(+\infty)
    \right) \ln \frac{b}{a}.
\end{equation*}