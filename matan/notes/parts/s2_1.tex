\subsection{Собственные интегралы с параметром}


\subsubsection{13.8(3)}


Стоит следующий вопрос
\begin{equation*}
    \int_0^1 \d x \int_0^1 \d \alpha f(x, \alpha) - \int_0^1 \d \alpha \int_0^1 \d x f(x, \alpha) \overset{?}{=} 0.
\end{equation*}
Выберем в качестве $f(x, \alpha)$ 
\begin{equation*}
    f(x, \alpha) = \left(
        \frac{x^5}{\alpha^4} - \frac{2 x^3}{\alpha^3}
    \right) e^{-x^2/\alpha}.
\end{equation*}
Для этого вычислим интеграл
\begin{equation*}
    g(x) = \int_0^1 \d \alpha f(x, \alpha) = \int_0^1 \d \alpha \left(
        \frac{x^5}{\alpha^4} - \frac{2 x^3}{\alpha^3}
    \right) e^{-x^2/\alpha}
\end{equation*}
Теперь вполне логично сделать замену переменных
\begin{equation*}
    d t = x^2 (\frac{-1}{\alpha^2}) \d \alpha 
\end{equation*}
и получить
\begin{equation*}
    g(x) = \int_{x^2}^{\infty} \d t \left(
        \frac{x^3}{\alpha^2} - \frac{2 x}{\alpha}
    \right) e^{-t}
\end{equation*}
который взяв по частям и получаем
\begin{equation*}
    \frac{x^3}{\alpha^2} - \frac{2 x}{\alpha} = \frac{1}{x} (t^2 - 2t)
    \hspace{0.5cm} \Rightarrow \hspace{0.5cm}
    g(x) = \frac{1}{x} \int_{x^2}^{\infty} (t^2 - 2t) e^{-t} \d t = 
    \frac{1}{x} \left\{-t^2 e^{-t}\right\} \bigg|_{x^2}^{\infty} = x^3 e^{-x^2}.
\end{equation*}
Идём дальше, возвращаясь к первоначальному интегрированию,
\begin{equation*}
    \int_0^1 g(x) \d x = \frac{1}{2} \int_0^1 x^2 e^{-x^2} \d (x^2) = 
    \frac{1}{2} \int_0^1 t e^{-t} \d t = - \frac{1}{2} (t+1) e^{-t} \big|_0^1 = \frac{1}{2}-\frac{1}{e}.
\end{equation*}
Теперь пойдём в другую сторону
\begin{equation*}
    h(\alpha) = \int_0^1 \d x f(x, \alpha) = \frac{1}{2\alpha} \int_0^{1/\alpha} (t^2 - 2t) e^{-t} \d t = \frac{1}{2\alpha} \left\{
        - t^2 e^{-t}
    \right\}\bigg|^{1/\alpha}_0 = - \frac{1}{2\alpha^3} e^{-1/\alpha}.
\end{equation*}
Остается посчитать интеграл $\alpha$ 
\begin{equation*}
    \int_0^1 h(\alpha) \d \alpha = - \frac{1}{2} \int_1^{\infty} t e^{-t} \d t = - \frac{1}{e}.
\end{equation*}


\subsubsection{13.14(3)}

Найти $\Phi'(\alpha)$, если 
\begin{equation*}
    \Phi(\alpha) = \int_{\sin \alpha}^{\cos \alpha} e^{\alpha\sqrt{1-x^2}} \d x.
\end{equation*}
Тут смотрим на К3, стр 325 (7), условия которой выполняются. Остается только посчитать
\begin{equation*}
    \Phi'(\alpha) = e^{\alpha |\sin \alpha|} (-\sin \alpha) - e^{\alpha |\cos \alpha|} \cos \alpha + \int_{\sin \alpha}^{\cos \alpha} e^{\alpha \sqrt{1-x^2}} \sqrt{1-x^2} \d x.
\end{equation*}


\subsubsection{13.17}

Есть интеграл вида
\begin{equation*}
    I(\alpha) = \int_0^b \frac{]d x}{x^2 + \alpha^2}.
\end{equation*}
Дифференцируя его по параметру $\alpha > 0$ вычислим интграл
\begin{equation*}
    J(\alpha) = \int_0^b \frac{\d x}{(x^2 + \alpha^2)^2}.
\end{equation*}
\textbf{Нужно} проверить, что условия теоремы выполняются, и вообще дифференцировать можем.
Тут всё хорошо, так что
\begin{equation*}
    \frac{\partial I(\alpha)}{\partial \alpha} = \int_0^\Lambda dx \ \frac{\partial }{\partial \alpha} \frac{1}{x^2+\alpha^2} = - 2 \alpha \int_0^\Lambda \frac{\d x}{(x^2 + \alpha^2)^2} = - 2 \alpha J(\alpha).
\end{equation*}
Таким образом приходим к
\begin{equation*}
    J(\alpha) = \frac{1}{2\alpha} \frac{\partial }{\partial \alpha} \left(
        \frac{1}{\alpha} \arctg \frac{\Lambda}{\alpha}
    \right) = \frac{1}{2\alpha^3} \left\{
        \arctg \frac{\Lambda}{\alpha} + \frac{\Lambda \alpha}{\Lambda^2 + \alpha^2}
    \right\}.
\end{equation*}


\subsubsection*{13.18 (1)}

Теперь применяя дифференцирование по параметру $\alpha$, вычислить
\begin{equation*}
    I(\alpha) = \int_0^{\pi/2}  \ln \left(
        \alpha^2 - \sin^2 \varphi
    \right) \d \varphi.
\end{equation*}
Утверждается, что \textbf{можно} дифференцировать по параметру. Ну и посчитаем тогда
\begin{equation*}
    \frac{d I(\alpha)}{d \alpha} = \int_0^{\pi/2} \d \varphi \frac{\partial }{\partial \alpha} \ln \left(\alpha^2 - \sin^2 \varphi\right) = 
    \int_0^{\pi/2} \frac{2\alpha \d \varphi}{\alpha^2 - \sin^2 \varphi} = \frac{\pi}{\sqrt{\alpha^2-1}}.
\end{equation*}
Таким образом находим, что
\begin{equation*}
    I(\alpha) = \pi \ln ( \alpha + \sqrt{\alpha^2-1}) + C.
\end{equation*}
С другой стороны
\begin{align*}
    I(\alpha) &= \int_0^{\pi/2} \{2 \ln \alpha + o(1)\} \d \varphi = \pi \ln \alpha + o(1)
    I(\alpha) &= \pi \ln \alpha + \pi \ln 2 + C + o(1).
\end{align*}
при больших $\alpha$. Получается, что
\begin{equation*}
    I(\alpha) = \pi \ln \left\{
        \frac{1}{2}\left(
            \alpha + \sqrt{\alpha^2-1}
        \right)
    \right\}
\end{equation*}


\subsection{Равномерная сходимость несобственных интегралов, зависящих от параметра}

Пусть $\alpha \in E$, подумаем о сходимости интгралов, зависящих от параметра
\begin{equation*}
    \int_a^{\infty} f(x \alpha) \d x.
\end{equation*}
По определению сходится (поточечно)
\begin{equation*}
    \forall \alpha \in E \ \ 
    \forall \varepsilon > 0 \ \
    \exists \delta[\varepsilon, \alpha] > a \ \
    \forall \xi > \delta[\varepsilon, \alpha] \ \
    \bigg|
        \int_{\xi}^{\infty} f(x, \alpha) \d x
    \bigg| < \varepsilon.
\end{equation*}
В случае же равномерной сходимости
\begin{equation*}
    \forall \varepsilon > 0 \ \
    \exists \delta(\varepsilon) > a \ \ 
    \forall \xi > \delta(\varepsilon) \ \ 
    \bigg|
        \int_{\xi}^{\infty} f(x, \alpha) \d x
    \bigg| < \varepsilon.
\end{equation*}

\subsubsection{14.1(1)}

По признаку Вейерштрассе $x^\alpha \geq x^{\alpha_0}$, если $x > 1$, $\alpha > \alpha_0 > 1$
\begin{equation*}
    \int_1^{\infty} \frac{dx}{x^\alpha} \leq \int_1^{\infty} \frac{\d x}{x^{\alpha_0}}
    \hspace{0.5cm} \Rightarrow \hspace{0.5cm}
    M(x) = \frac{1}{x^{\alpha_0}}.
\end{equation*}
что соответствует сходимости. Аналогично 14.1(2).
