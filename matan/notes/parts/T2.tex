\subsection{Т2}

\textbf{Интеграл Дирихле}. Вычислим \textit{интеграл Дирихле}
\begin{equation}
    I(\alpha) = \int_0^\infty \frac{\sin \alpha x}{x} \d x
\end{equation}
Для начала вычислим некоторый другой интеграл:
\begin{equation*}
    \Phi(\alpha, \beta) = \int_0^{\infty} e^{-\beta x} \frac{\sin (\alpha x)}{x} \d x,
    \hspace{10 mm}
    \Phi'_\alpha (\alpha, \beta) = 
    \int_0^\infty e^{-\beta x} \cos (\alpha x) \d x = \frac{\beta}{\alpha^2 + \beta^2}.
\end{equation*}
Действительно, считая $f'_\alpha (x, \alpha) = e^{-\beta x} \cos (\alpha x)$, заметим, что $f$ и $f'_\alpha$ непрерывны на $E$, $\int_0^{\infty} f(x, \alpha) \d x$ сходится $\forall \alpha \in \mathbb{R}$ по Дирихле:
\begin{equation*}
    \bigg|
        \int_0^{\infty} \sin (\alpha x) \d x
    \bigg| = \bigg|
        \frac{\cos (\alpha t) - 1}{\alpha}
    \bigg| \leq \frac{2}{|\alpha|}, \hspace{5 mm} \alpha \neq 0,
\end{equation*}
а функция $x^{-1} e^{-\beta x}$ убывает на промежутке $(0, +\infty)$, также верно, что $\int_0^{\infty} f'_\alpha (x, \alpha) \d x$ сходится равномерно по признаку Вейерштрассе, следовательно можем дифференцировать под знаком интеграла. 

Теперь, интегриря $\alpha$ на отрезке $[0, \alpha]$ находим
\begin{equation*}
    \Phi(\alpha, \beta) - \Phi(0, \beta) = \beta \int_{0}^{\alpha} \frac{d t}{t^2 + \beta^2} = 
    \arctg \frac{\alpha}{\beta} 
    % \overset{\mathrm{?}}{\to} I(\alpha)
    .
\end{equation*}
Понятно, что $I(\alpha) = - I(\alpha)$, так что далее считаем $\alpha > 0$. Имеем право рассмотреть $\beta \in [0, 1]$, точнее предел
\begin{equation*}
    \lim_{\beta \to +0} \int_0^{+\infty} e^{-\beta x} \frac{\sin(\alpha x)}{x} \d x = I(\alpha) =
    \lim_{\beta \to +0} \arctg \frac{\alpha}{\beta} = \frac{\pi}{2}.
\end{equation*}
Таким образом для произвольного $\alpha$ верно, что
\begin{equation}
    \int_0^{\infty} \frac{\sin(\alpha x)}{x} \d x = \frac{\pi}{2} \sign (\alpha).
\end{equation}

\textbf{Интеграл Лапласа}. Вычислим интегралы Лапласа
\begin{equation*}
    I(\alpha) = \int_0^\infty \frac{\cos \alpha x}{1 + x^2} \d x = \int_0^\infty f(x, \alpha) \d x, 
    \hspace{10 mm}
    K(\alpha) = \int_0^\infty \frac{x \sin \alpha x}{1 + x^2} \d x.
\end{equation*}
Без ограничения общности рассмотрим $\alpha > 0$. Проверим, что можем дифференцировать под знаком интеграла: $f(x, \alpha)$ непрерывна $\forall \alpha, \, x$, интеграл
\begin{equation*}
    \int_0^{+\infty} f'_\alpha \d x = - \int_0^{+\infty} \frac{x \sin \alpha x}{1 + x^2} \d x,
\end{equation*}
сходится равномерно по  $\alpha$ на $[a_0, +\infty)$ для $\forall \alpha_0 > 0$, получается верно, что
\begin{equation*}
    I'(\alpha) = - \int_0^{+\infty}  \frac{x \sin \alpha x}{1 + x^2} = 
    - K(\alpha).
\end{equation*}
Складывая с известным выражением интеграла Дирихле, находим
\begin{equation*}
    I'(\alpha) + \frac{\pi}{2} = \int_0^{+\infty}  \left(
        \frac{\sin \alpha x}{x} - \frac{x \sin \alpha x}{1+ x^2} 
    \right) \d x = 
    \int_0^{+\infty} \frac{\sin \alpha x}{x (1 + x^2)} \d x.
\end{equation*}
Аргумент интеграла непрерывен, как и его производная по $\alpha$, они Лебег-интегрируемы, поэтому, дифференцируя под знаком интеграла, находим
\begin{equation*}
    I''(\alpha) = \int_0^{+\infty}  \frac{\cos \alpha x}{1 + x^2} \d x.
\end{equation*}
Так мы приходим к дифференциальному уравнению на $I(\alpha)$:
\begin{equation*}
    I''(\alpha) - I(\alpha) = 0,
    \hspace{0.5cm} \Rightarrow \hspace{0.5cm}
    I(\alpha) = C_1 e^\alpha + C_2 e^{-\alpha}.
\end{equation*}
Рассматривая пределы $\alpha \to 0$ и $\alpha \to + \infty$, находим константы интегрирования $C_1 = 0$ и $C_2 = \pi/2$. В силу четности $I(\alpha)$ находим
\begin{equation*}
    I(\alpha) = \frac{\pi}{2} e^{-|\alpha|}, \hspace{5 mm} \alpha \in \mathbb{R}.
\end{equation*}
Бонусом находим $K(\alpha) = - I'_\alpha (\alpha)$:
\begin{equation*}
    K(\alpha) = \frac{\pi}{2} e^{-|\alpha|} \cdot \sign \alpha.
\end{equation*}

\textbf{Интегралы Френеля}. Вычислим \textit{интеграл Френеля}
\begin{equation*}
    I = \int_0^{+\infty} \sin^2 x^2 \d x.
\end{equation*}
Для нахождения нам понадобится \textit{интеграл Эйлера-Пуассона} и, возможно, \textit{интеграл Лапласа}:
\begin{equation*}
    \int_0^{+\infty} e^{-x^2} \d x = \frac{\sqrt{\pi}}{2},
    \hspace{10 mm}
    I(\alpha) = \int_0^{+\infty}  e^{-x^2} \cos 2 \alpha x \d x = \frac{\sqrt{\pi}}{2} e^{-\alpha^2}.
    .
\end{equation*}
Полагая $x^2 = t$ запишем интеграл $I$ в виде
\begin{equation*}
    I  = \frac{1}{2} \int_0^{+\infty}  \frac{\sin t}{\sqrt{ t}} \d t.
\end{equation*}
При $t > 0$ справедливо равенство
\begin{equation}
    \int_0^{+\infty} e^{-t u^2} \d u =
    \bigg/
        x = \sqrt{t} u
    \bigg/ = \frac{1}{2} \sqrt{\frac{\pi}{t}},
    \hspace{0.5cm} \Rightarrow \hspace{0.5cm}
    \frac{1}{\sqrt{t}} = \frac{2}{\pi} \int_0^{+\infty}  e^{-t u^2} \d u,
\end{equation}
Так приходим к двойному интегралу
\begin{equation*}
    I = \frac{1}{\sqrt{\pi}} \int_0^{+\infty} \sin t \d t \int_0^{+\infty}  e^{- t u^2} \d u.
\end{equation*}
Меняя порядок интегрирования, получаем
\begin{equation*}
    I = \frac{1}{\sqrt{\pi}} \int_0^{+\infty} \d u \int_0^{+\infty}  e^{-tu^2} \sin t \d t = 
    \frac{1}{\sqrt{\pi}} \int_0^{+\infty} \frac{\d u}{1 + u^4}.
\end{equation*}
Который легко вычисляется, если заметить, что
\begin{equation*}
    \int_0^{+\infty} \frac{x^2 \d x }{1 + x^4} = \int_0^{+\infty} \frac{(1/x^2) \d x}{1 + (1/x)^4} = \int_0^{+\infty} \frac{\d x}{1 + x^4}.
\end{equation*}
Поэтому 
\begin{equation*}
    2 \int_0^{+\infty}  \frac{\d x}{1 + x^4} = \int_0^{+\infty}  \frac{(1 + 1/x^2)\d x}{x^2 + 1/x^2} = \int_0^{+\infty}  \frac{\d (x - 1/x)}{ (x-1/x)^2 + 2} = \frac{1}{\sqrt{2}} \arctg\left(
        \frac{x-1/x}{\sqrt{2}}
    \right) \bigg|_{0}^{\infty} = \frac{\pi}{\sqrt{2}}.
\end{equation*}
Откуда уже и получаем
\begin{equation}
    I = 
    \int_0^{+\infty} \sin^2 x^2 \d x
    =
    \frac{1}{\sqrt{\pi}} \cdot \frac{\pi}{2 \sqrt{2}} = \frac{1}{2} \sqrt{\frac{\pi}{2}}.
\end{equation}

