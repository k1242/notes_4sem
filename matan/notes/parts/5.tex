\begin{to_def}
    Функция $f$ на промежутке $I$ имеет \textit{ограниченную вариацию}, если для любых $x_0 < x_1 < \ldots M x_N \in I$ (в любом количестве)
    \begin{equation*}
        |f(x_0) - f(x_1)| + 
        |f(x_1) - f(x_2)| + \ldots +
        |f(x_{N-1}) - f(x_N)| \leq M,
    \end{equation*}
    для некоторой константы $M$. Наименьшую константу $M$ в этом неравенстве назовём вариацией функции $f$ равную $\|f\|_B$, что задаёт \textit{полунормой}.
\end{to_def}


\begin{to_lem}
    Функцию ограниченной вариации на отрезке $[a, b]$ можно представить в виде суммы двух функций $f = u + d$, одна из которых возрастает, а другая убывает. При этом $\|f\|_B = \|u\|_B + \|d\|_B$ и если $f$ была непрерывной, то $u$ и $d$ тоже будут непрерывны.
\end{to_lem}


\red{Дополнить смыслом функций с ограниченной вариацией.}