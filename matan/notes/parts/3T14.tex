\subsubsection*{Т14}

Докажем, что алгебраический базис бесконечномерного банахова пространства не может быть счётным. 

Вводился алгебраический базис Гамиля $\{e_\alpha\}_{\alpha\in A}$, где $\forall x \in E$ представляется в виде
$x \ \sum_{k=1}^{n}  x_k e_{\alpha_k}$. Получается, что нужно показать, что в бесконечномерном банаховом пространстве такой базис не может быть счётным: докажем от противного.


Пусть $\{e_n\}_{n\in \mathbb{N}}$, тогда пространство описывется, как
\begin{equation*}
    E_n = \left\{
        \sum_{k=1}^{n} x_k e_k \mid x_1, \ldots, x_k \in \mathbb{R}
    \right\} = \langle E_1, \ldots, e_n \rangle,
    \hspace{0.5cm} \Rightarrow \hspace{0.5cm}
    E = \bigcup_{n\in \mathbb{N} } E_n.
\end{equation*}
Но по теореме Бэра для замкнутых множеств $E$ не может быть счётным объединением нигде не плотных множеств.

Точнее, это было бы возможно, только с случае непустой внутренности одного из пространств $E_n$, что невозможно. 


% Мы знаем, что $E_n$ -- замкнуто, $E$ -- полно. Тогда, по следствию из теоремы Бэра,
% \begin{equation*}
%     \exists n_0 \in \mathbb{N} \colon \ \ E \subset E_{n_0}, 
% \end{equation*}
% что приводит к противоречию, в силу бесконечности $E$. 

