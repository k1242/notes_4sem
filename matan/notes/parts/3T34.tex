\subsubsection*{Т34}

Докажем, что преобразование Фурье в $S'$ переводит распределение
\begin{equation*}
    \sum_{n=-\infty}^{\infty} \delta_{2\pi n} \hspace{5 mm} 
    \mapsto
    \hspace{5 mm} 
    \frac{1}{\sqrt{2\pi}} \sum_{n=-\infty}^{+\infty} \delta_n.
\end{equation*}


\begin{to_thr}[Формула Пуассона]
    Так называется следующее соотношение:
    \begin{equation*}
        \sqrt{2\pi} \sum_{n=-\infty}^{\infty}  \varphi(2 \pi n) = \sum_{n=-\infty}^{\infty} \hat{\varphi} (n).
    \end{equation*}
\end{to_thr}

\begin{proof}[$\triangle$]
Формула получается при $x= 0$ из равенства вида
\begin{equation*}
    \sqrt{2\pi} \sum_{m=-\infty}^{\infty} \varphi(x + 2 \pi n) = \sum_{n=-\infty}^{\infty}  \hat{\varphi} (n) e^{i n x},
\end{equation*}
которое мы и докажем. 

Поскольку $\varphi,\, \hat{\varphi} \in S$, ряды сходятся абсолютно и равномерно по $x$ на $\mathbb{R}$. Также чтоит заметить, что
\begin{equation*}
    f(x) = \sum_{n=-\infty}^{\infty}  \varphi(x + 2 \pi n)
\end{equation*}
бесконечно гладкая и $2\pi$-периодическая. Пусть $\{\hat{c}_k (f)\}$ -- её коэффициенты Фурье по ортонормированной  системе $\left\{\frac{1}{\sqrt{2\pi}} e^{ikx}, \ \ k \in \mathbb{Z}\right\}$. Тогда 
\begin{equation*}
    \hat{c}_k (f) = \frac{1}{\sqrt{2\pi}} \int_0^{2\pi} f(x) e^{ikx} \d x = 
    \sum_{n=-\infty}^{\infty} \frac{1}{\sqrt{2\pi}} \int_{2 \pi n}^{2p(n+1)} \varphi(x) e^{ikx} \d x = 
    \frac{1}{\sqrt{2\pi}} \int_{-\infty}^{+\infty} \varphi(x) e^{ikx} \d x \overset{\mathrm{def}}{=}  \hat{\varphi}(k). 
\end{equation*}
Но ряд фурье $f$ сходится к ней в любой точке $x \in \mathbb{R}$, значит в любой точке $x \in \mathbb{R}$ справедливо соотношение
\begin{equation*}
    \sum_{n=-\infty}^{\infty} \varphi(x+2\pi n) = f(x) = \sum_{n=-\infty}^{\infty} \hat{c}_n (f) \frac{e^{inx}}{\sqrt{2\pi}} = \frac{1}{\sqrt{2\pi}} \sum_{n=-\infty}^{\infty}  \hat{\varphi} (n) e^{ikx}, \QED
\end{equation*}

\end{proof}


Тогда в пределах задания можем переписать это в терминах обобщенных функций
\begin{equation*}
    \sqrt{2\pi} \sum_{n=-\infty}^{\infty}  \langle \delta(x-2\pi n) \,|\, \varphi \rangle  = \sum_{n=-\infty}^{\infty} 
    \langle \delta(x-n) \,|\, F[\varphi] \rangle  = \sum_{n=-\infty}^{\infty} \langle F[\delta(x-n)] \,|\, \varphi \rangle.
\end{equation*}
Тогда приходим к выражению вида
\begin{equation*}
    F\left[\frac{1}{\sqrt{2\pi}} \sum_{n=-\infty}^{\infty} \delta(x-n)\right] = \sum_{n=-\infty}^{\infty}  \delta(x-2 \pi n).
\end{equation*}
Вообще $\sum_{n=-\infty}^{\infty}  \delta(x-n) = G$ называют \textit{решеткой Дирака}. 


Утверждается, что $G_N$ сходится в $S'$ к $G$ $\forall \varphi$. В частности,
\begin{equation*}
    \lim_{N \to \infty} \langle G_N (x) \,|\, \varphi \rangle  = \lim_{N \to \infty} \sum_{n=-N}^{N} \varphi(n) = 
    \sum_{n=-\infty}^{\infty}  \varphi(n) = \bigg\langle \sum_{n=-\infty}^{\infty} \delta(x-n) \,\bigg|\, \varphi \bigg\rangle,
\end{equation*}
так что ряд дейтсвительно сходится и всё хорошо. 