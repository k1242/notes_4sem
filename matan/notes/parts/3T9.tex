
\subsubsection*{Т9. Пространство \texorpdfstring{$c$}{с}}

Пространство состоит из некоторых бесконечномерных <<векторов>>  (последовательностей):
\begin{equation*}
    x = \big(x(1), x(2), \ldots, x(k), \ldots\big),
    \hspace{5 mm}
    \bigg|
        \lim_{k \to \infty} x(k)
    \bigg| < + \infty.
\end{equation*}
Норма определена, как
\begin{equation*}
    p(x) = \|x\|_c = \sup_{k \in \mathbb{N}} |x(k)| = \|x\|_{\infty}.
\end{equation*}
Докажем, что это пространство является банаховым, а именно полноту по $\|\cdot\|_{\infty}$ норме. 

Рассмотрим последовательность $x_n$, где 
\begin{equation*}
    x_n = (x_n (1), \ldots, x_n(k), \ldots).
\end{equation*}
Глобально хотим показать, что
\begin{equation*}
    \forall \varepsilon > 0 \ \ 
    \exists N(\varepsilon) \in \mathbb{N} \ \ 
    \forall n \geq N(\varepsilon) \ \ 
    \forall l \in \mathbb{N} \ \  \ 
    \|x_{n+l} -x_{n}\|_{\infty} = \sup_{k \in \mathbb{N}} |x_{n+l} (k) - x_n (k)| < \varepsilon.
\end{equation*}
Попробуем через это продраться: из сходимости следует, что
\begin{equation*}
    \forall k \in \mathbb{N} \ \ |x_{n+l} (k) - x_n (k) | < \varepsilon.
\end{equation*}
Здесь можем выделить $(x_n(k))_{n \in \mathbb{N}}$ -- числовая фундаментальная в $\mathbb{R}$. 
По критерию Коши:
\begin{equation*}
    \forall k \in \mathbb{N} \ \ 
    \lim_{n \to \infty} x_n (k) = y(k) \in \mathbb{R},
\end{equation*}
уставнавливается покомпонентая сходимость. 
Теперь рассмотрим
\begin{equation*}
    \sup_{k\in \mathbb{N}} |x_n (k) - y(k)| = \|x_n - y\|_{\infty} < \varepsilon,
\end{equation*}
что автоматически означает, что $\exists y$ такой, что
\begin{equation*}
    \lim_{n \to \infty} x_n  = y.
\end{equation*}

Следующий этап -- показать, что
\begin{equation*}
    \exists \lim_{k \to \infty} y(k) \in \mathbb{R},
\end{equation*}
то есть показать полноту пространства:
\begin{align*}
    |y(k+q)-y(k)| 
    &= |y(k+q) - x_n (k+q) + x_n (k+q) - x_n (k) + x_n (k) - x_n (k)|\\
    &\leq |y(k+q)-x_n (k+q)| + |y(k) - x_n (k)| + |x_n (k+q) - x_n (k)| \\
    & < \varepsilon/3 + \varepsilon/3 + \varepsilon/3 = \varepsilon.
\end{align*}
Таким образом мы доказали полноту пространства\footnote{
    \red{$c_0, c_{00}, l_\infty$ -- банаховы ли?
    $\encircled{!}$: $c_0$ (сходящиеся к $0$), $c_{00}$ (финитные), $l_\infty$ (ограниченные). 
    }  
} . 

