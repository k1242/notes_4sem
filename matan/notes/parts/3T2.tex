\subsubsection*{Т2}


Приведем пример, когда последовательность функция $(f_n)$ сходится в пространстве $L_1[a, b]$, но для любого $x \in[a, b]$ последовательность чисел $f_n(x)$ расходится. 

Из сходимости в $L_1$ следует сходимость по мере, так что можем воспользоваться \textit{примером Рисса}. Пусть $f_n \underset{L_2}{\to}  f \equiv 0 $. Рассмотрим конструкцию вида
\begin{equation*}
    \varphi_{m,k} (x) = \xi\left[\frac{k}{2^m},\, \frac{k+1}{2^m}\right](x),
    \hspace{5 mm} 
    m \in \mathbb{N}_0,
\end{equation*}
где $k=0,1\ldots,2^{m}-1$. Утверждается, что $\forall n \in \mathbb{N} \ \exists ! m, k \ \colon  \ n = 2^m + k$. Таким нетривиальным образом мы (точнее Рисс) решили дробить ступеньку. Верно, что
\begin{equation*}
    \|f_n-0\|_1 = \int_{k/2^m}^{(k+1)/2^m} \varphi_{m ,k} (x) \d x = \frac{1}{2^m} \underset{n\to\infty}{\to} 0.
\end{equation*}
Однако, для $\forall x \in[0, 1]$ существует бесконечное число сленов последовательности равных $0$ и $1$. Таким образом поточечно последовательнсоть расходится.
