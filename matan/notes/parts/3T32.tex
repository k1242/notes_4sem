\subsubsection*{Т32}


Найдём преобразование Фурье в $S'$ некоторых функций.

\textbf{Синус}. Найдём преобразование Фурье вида
\begin{align*}
    \big\langle F^{-1}[\delta(x-x_0)] \,\big|\, \varphi \big\rangle &=
    \big\langle \delta(x-x_0) \,\big|\, F^{-1}[\varphi] \big\rangle  = 
    F^{-1} [\varphi] (x_0) = \int_{-\infty}^{+\infty} \frac{\d t}{\sqrt{2\pi}} \varphi(t) e^{i x_0 t}  = \\
    &= \int_{-\infty}^{+\infty} \d t  f(t) \varphi(t) = \bigg\langle \frac{e^{i x_0 t}}{\sqrt{2\pi}} \,\bigg|\,  \varphi\bigg\rangle ,
    \hspace{0.5cm} \Rightarrow \hspace{0.5cm}
    F^{-1}[\delta(x-x_0)] (t) = \frac{e^{i x_0 t}}{\sqrt{2\pi}}.
\end{align*}
Отсуюда следует, что
\begin{equation*}
    \big\langle F[e^{i x_0 t}] \,\big|\, \varphi \big\rangle = \big\langle F\left[\sqrt{2\pi} F^{-1} [\delta(x-x_0)]\right] \,\big|\, \varphi \big\rangle,
\end{equation*}
тогда можем перегрупировать, и найти
\begin{equation*}
    \big\langle F[e^{i x_0 t}]  \,\big|\, \varphi \big\rangle = \langle \sqrt{2\pi} \delta(x-x_0) \,|\, \varphi \rangle.
\end{equation*}
Нас, правда, интересует Фурье от синуса
\begin{align*}
    \langle F[\sin(x_0 t)] \,|\, \varphi \rangle &=
    \bigg\langle F\left[\frac{e^{i x_0 t}-e^{-i x_0 t}}{2 i}\right] \,\bigg|\, \varphi \bigg\rangle  = 
    \bigg\langle \sqrt{\frac{\pi}{2}} i \left(
        \delta(x+x_0) - \delta(x-x_0)
    \right) \,\bigg|\,  \varphi \bigg\rangle.
\end{align*}
Тогда $\mathcal D'$ справедливо равенство вида
\begin{equation*}
    F[\sin (x_0 t)] \dseq \sqrt{\frac{\pi}{2}} i \big( 
        \delta(x+x_0) - \delta(x-x_0)
    \big).
\end{equation*}


\textbf{Дельта-функция}. Пользуясь формулой $n$-й производной
\begin{align*}
    \big\langle F[\delta^{(n)}(x)] \,\big|\, \varphi \big\rangle  &= \big\langle \delta^{(n)} (x) \,\big|\, F[\varphi] \big\rangle = 
    (-1)^n F^{(n)} [\varphi] (0) = \frac{(-1)^n}{i^n} F[x^n \varphi] (0) = 
    \bigg\langle \frac{(-1)^n}{i^n} \delta(x) \,\bigg|\, F[x^n \varphi] \bigg\rangle 
    = \\ &= 
    i^n \big\langle F[\delta(x)] \,\big|\, x^n \varphi \big\rangle = \langle (ix)^n F[\delta(x)] \,|\, \varphi \rangle = \bigg\langle \frac{(ix)^n}{\sqrt{2\pi}} \,\bigg|\, \varphi \bigg\rangle,
\end{align*}
таким образом пришли к равенству вида
\begin{equation*}
    F[\delta^{(n)} (x)] \dseq \frac{(ix)^n}{\sqrt{2\pi}}.
\end{equation*}


\textbf{Фунция Хевисайда}. Для начала найдём преобразование Фурье функции $\theta(x) e^{-tx}$ при $t>0$
\begin{equation*}
    F[\theta(x) e^{-tx}] = \frac{1}{\sqrt{2\pi}} \int_{0}^{\infty} e^{-x(t+iy)}\d x = \frac{-i}{\sqrt{2\pi} (y-it)}.
\end{equation*}
Покажем теперь, что в $S'$
\begin{equation*}
    \lim_{t\to+0} \theta(x) e^{-tx} = \theta(x).
\end{equation*}
Действительно, для каждой функции $\varphi \in S$ и любого числа $A$ имеем
\begin{equation*}
    |
        \langle \theta(x) \,|\, \varphi(x) \rangle - 
        \langle \theta(x) e^{-tx} \,|\, \varphi(x) \rangle 
    | = \bigg| \int_{0}^{\infty} (1-e^{-tx}) \varphi(x) \d x \bigg| \leq 
    \bigg| \int_{0}^{A} (1-e^{-tx}) \varphi(x) \d x \bigg| + 
    \bigg| \int_{A}^{\infty} (1-e^{-tx}) \varphi(x) \d x \bigg|.
\end{equation*}
Теперь зафиксируем $\sigma \in S$ и какое-либо число $\varepsilon >0$. В силу абсолютной интегрируемости $\varphi$, существует $A > 0$ такео, что $\int_A^{+\infty} < \varepsilon/2$ , тогда
\begin{equation*}
    \bigg| \int_{A}^{\infty}  (1-e^{-tx}) \varphi(x) \d x \bigg| \leq \frac{\varepsilon}{2}.
\end{equation*}
Выберем теперь $t_0 > 0$ так, чтобы при $0 < t < t_0$ было справедливо неравенство
\begin{equation*}
    (1 - e^{-t A}) \int_{0}^{A}  |\varphi(x)| \d x < \frac{\varepsilon}{2},
    \hspace{0.5cm} \Rightarrow \hspace{0.5cm}
        |
        \langle \theta(x) \,|\, \varphi(x) \rangle - 
        \langle \theta(x) e^{-tx} \,|\, \varphi(x) \rangle 
    | < \varepsilon.
\end{equation*}
Таким образом утверждение про $\lim_{t\to+0} \theta(x) e^{-tx} = \theta(x)$ верно. 


В силу непрерывности преобразования Фурье
\begin{equation*}
    \lim_{t\to +0} F\left[
        \theta(x) e^{-tx}
    \right] = F[\theta(x)],
    \hspace{0.5cm} \Rightarrow \hspace{0.5cm}
    F[\theta(x)] = -\frac{1}{\sqrt{2\pi}} \lim_{t\to +0} \frac{i}{y-it},
\end{equation*}
причём  мы сразу утверждаем, что $S'$ предел существует, и, кстати, обозначается за $\frac{i}{y-i 0}$. Тогда
\begin{equation*}
    F[\theta(x)] = - \frac{1}{\sqrt{2\pi}} \frac{i}{y-i0}.
\end{equation*}