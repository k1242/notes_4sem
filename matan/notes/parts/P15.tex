\subsubsection*{15.1(1, 2, 3, 4)}

\textbf{1)} Найдём интеграл вида
\begin{equation*}
    \int_{0}^{+\infty}  \left(
        \cos^2 (ax) - \cos^2 (bx)
    \right) \frac{dx}{x} = \frac{1}{2} \int_{0}^{+\infty}  \left(
        \cos(2ax) - \cos(2 b x)
    \right) \frac{dx}{x} = \frac{1}{2} \ln \frac{b}{a}.
\end{equation*}
где воспользовались формулой Фрулани, выбрав $\cos (2ax) = f(ax)$. 

\textbf{2)} Теперь найдём
\begin{equation*}
    \int_{0}^{+\infty}  \l(
        e^{-ax^2} - e^{-bx^2}
    \r) \frac{dx}{x} = \ln \sqrt{\frac{b}{a}} = \frac{1}{2} \ln \frac{b}{a},
\end{equation*}

\textbf{3)} Интеграл вида
\begin{equation*}
    \int_{0}^{+\infty} \frac{e^{-ax^2}-e^{-bx^2}}{x} \d x = \bigg/
        \begin{aligned}
            x &= \sqrt{t}, \\
            dx &= dt/(2\sqrt{t}) 
        \end{aligned}
    \bigg/ = 
    \int_{0}^{+\infty}
    \frac{e^{-a t}-e^{-b t}}{2 t} \d t = \frac{1}{2} \ln \frac{b}{a}.
\end{equation*}

\textbf{4)} И, наконец, вычислим интеграл вида
\begin{equation*}
    \int_0^1 \frac{x^a - x^b}{\ln x} \d x = \bigg/
    \red{
        \ln \frac{1}{x} = t
    }
    \bigg/ = 
    \int_{\infty}^0 \frac{\d t}{t} e^{-t} (e^{-at} - e^{-bt}) = 
    \int_0^\infty \l(
        e^{-(b+1)t} - e^{-(a+1)t}
    \r) \frac{\d t}{t} = \ln \frac{a+1}{b+1}
\end{equation*}

% в 15.2(3) t = x^2


\subsubsection*{15.2(1)}

Найдём интеграл, вида
\begin{equation*}
    \int_0^{+\infty} \frac{1 - \cos \alpha x}{x^2} \d x = -\int_0^{+\infty} \sin^2 \left(
        \frac{\alpha x}{2}
    \right) \d \left(\frac{1}{x}\right) = - \frac{\sin^2 \left(\frac{\alpha}{2}x\right)}{x} \bigg|_0^{+\infty} + 
    \frac{\alpha}{2} \int_0^{+\infty} \frac{\sin \alpha x}{x} \d x = \frac{\pi |\alpha|}{2},
\end{equation*}
где модуль вполне правомерен в силу чётности $\cos (\alpha x)$.


\subsubsection*{15.3(2)}

Интеграл
\begin{equation*}
    \int_0^\infty \sin x \cos^2 x \frac{dx}{x} = \int_{0}^{+\infty} 
    \sin(x) \frac{1 + \cos (2x)}{2} \frac{dx}{x} = \frac{1}{2} \int_0^\infty \frac{\sin x}{x} \d x + 
    \frac{1}{2} \int_0^\infty \frac{\sin x \cos 2 x}{x} \d x,
\end{equation*}
где уже хочется подставить $\sin (3 x) - \sin (x)= 2 \sin(x) \cos(2x)$:
\begin{equation*}
    \frac{1}{2} \frac{\pi}{2} + \frac{1}{2} \frac{1}{2} \int_0^\infty \frac{\sin(3x)}{x} \d x -
    \frac{1}{4} \int_0^\infty \frac{\sin x}{x} \d x = \frac{\pi}{4}.
\end{equation*}
% почему можно дифференцировать по параметру, равномерно ил он сходится


\subsubsection*{15.4(3)}

Есть интеграл вида
\begin{equation*}
    I(\alpha) = \int_{0}^{+\infty} \frac{\sin^4 (\alpha x)}{x^2} \d x = 
    \int_0^{+\infty} f(x, \alpha) \d x.
    .
\end{equation*}
Заметим, что $f$, $f'_\alpha$  существуют  почти всюду по $\alpha$, $f'_\alpha = \frac{4 \sin ^3(\alpha  x) \cos (\alpha  x)}{x}$ мажорируется $10 x^2$ при  малых $x$ и \textit{не абсолютно} интегрируема при больших по признаку Дирихле, соответственно можем нтегрировать под знаком интеграла
\begin{equation*}
    I'_\alpha (\alpha) = \int_{0}^{+\infty} 4 \sin^3 (\alpha x) \cos(\alpha x) \frac{\d x}{x}
    =
    \int_{0}^{+\infty} \frac{dx}{x} \l(
        \frac{1}{4} \sin (2 x\alpha) - \frac{1}{8} \sin (4 x \alpha)
    \r) = \frac{\pi}{4} \sign \alpha,
\end{equation*}
что верно $\forall \alpha$. 

Возвращаясь к интегралу, находим, что
\begin{equation*}
    I(\alpha) = \frac{\pi}{4} |\alpha| + 0,
\end{equation*}
так как $I(0) = 0$.


\begin{to_thr}[интерирование по частям]
Воообще 
\begin{equation*}
    \int_{a}^{+\infty}  f(x, \alpha) g'_x (x, \alpha) \d x 
    =
    f(x, \alpha) g(x, \alpha) \bigg|_a^{\infty} - \int_{a}^{+\infty}  f'_x (x, \alpha) g(x, \alpha) \d x,
\end{equation*}
работает, когда $f, g \in C^1$ по $x$ и любые два из трёх написанных пределов существуют.    
\end{to_thr}




\subsubsection*{15.5(6)}

Вычислим интеграл вида
\begin{equation*}
    I(\alpha) = \int_{0}^{+\infty}  \sin^3 (x) \cos (\alpha x) \frac{dx}{x^3}.
\end{equation*}
Интегрируя по частям
\begin{equation*}
    \sin^3 x \cos (\alpha x) = \frac{3}{8} \left(
        \sin(\alpha+1)x - \sin (\alpha - 1)x
    \right) - \frac{1}{8} \l(
        \sin(\alpha+3)x - \sin(\alpha_3)x
    \r),
\end{equation*}
для $\alpha > 3$. В общем приходим к выражению
\begin{align*}
    I(\alpha) 
    &=
     \int_{0}^{+\infty}  \sin^3 (x) \cos (\alpha x) \frac{dx}{x^3}
    =
     \int_0^\infty \sin^3 x \cos(\alpha x) \d \l(
        \frac{-1}{2x^2}
    \r) 
    = 
    \\
    &=
    \l(
        -\frac{1}{2x^2}
    \r) \sin^3 x \cos \alpha x \bigg|_0^\infty + \frac{1}{2} \int_{0}^{+\infty} 
    \frac{1}{x^2} \d (\sin^3 x \cos \alpha x) = 
    \frac{1}{2} \int_{0}^{+\infty} \l(
        \sin^3 (x) \cos (\alpha x)
    \r)'_x  f\left(-\frac{1}{x}\right) 
    = 
    \\
    &= 
    - \frac{1}{2x} \l(
        \sin^3 x \cos(\alpha x)
    \r)'_x \bigg|_0^\infty + 
    \frac{1}{2} \int_{0}^{+\infty}  \frac{1}{x} \d \l(
        \sin^3 x \cos \alpha x
    \r)'_x
    = 
    \\ 
    &=
    -\frac{1}{2} \int_{0}^{+\infty} \frac{\d x}{x} \bigg(
        \frac{3}{8}\left[
            (\alpha+1)^2 \sin (\alpha+1)x - (\alpha-1)2 \sin (\alpha-1)x
        \right] - \\ 
        & 
        \phantom{=
    -\frac{1}{2} \int_{0}^{+\infty} \frac{\d x}{x} \bigg(\ }
         \frac{1}{8}\left[
            (\alpha+3)^2 \sin (\alpha+3) x - (\alpha-3)^2 \sin  (\alpha-3)x
        \right]
    \bigg) 
    = \\
    &=
    -\frac{\pi}{4} \left\{
        \frac{3}{8} (\alpha+1)^2 - \frac{3}{8} (\alpha-1)^2 - \frac{1}{8} (\alpha + 3)^2+ \frac{1}{8} (\alpha-3)^2
    \right\} = 0 
\end{align*}


\subsubsection*{15.6(3)}

C помощью дифференцирования по параметру вычислим
\begin{equation*}
    I(\alpha) = \int_0^{+\infty} \frac{e^{-\alpha x}-e^{-\beta x}}{x} \sin \lambda x \d x = \int_0^{+\infty} f(x, \alpha) \d x, 
    \hspace{5 mm}
    \alpha > 0, \ \beta > 0, \ \lambda \neq 0.
\end{equation*}
Для начала проверим, что можем дифференцировать по параметру $\lambda$. Действительно $f \in \mathcal L (X)$, $f'_\lambda = \left(
    e^{-\alpha x} - e^{-\beta x}
\right) \cos (\lambda x)$ существует, конечна и Лебег-интегрируема ($< e^{-\alpha x} - e^{-\beta x}$) $\forall \lambda$. Тогда, дифференцируя под знаком интеграла
\begin{equation*}
    I'_\lambda (\lambda) = \int_{0}^{+\infty}  \left(
        e^{-\alpha x} - e^{-\beta x}
    \right) \cos (\lambda x) \d x = \frac{\alpha}{\alpha^2 + \lambda^2} - \frac{\beta}{\beta^2 + \lambda^2}.
\end{equation*}
В таком случае $I(\lambda)$
\begin{equation*}
    I(\lambda) = \int I'_\lambda (\lambda) \d \lambda = \arctg\left(
        \frac{\lambda}{\alpha}
    \right) - \arctg\left(
        \frac{\lambda}{\beta} 
    \right) + C = \arctg\left(
        \frac{\lambda}{\alpha}
    \right) - \arctg\left(
        \frac{\lambda}{\beta} 
    \right),
\end{equation*}
где $C = 0$ так как $I(0) = 0$.


\subsubsection*{15.6(5)}
При выполнении всех условий о дифференцирование интеграла по параметру, для интеграла 
\begin{equation*}
    I(\alpha) = \int_0^1 \frac{\arctg (\alpha x)}{x \sqrt{1-x^2}} \d x,
\end{equation*}
может быть так посчитан. 

Действительно,
\begin{align*}
    \frac{d I(\alpha)}{d \alpha} 
    &=
    \int_0^1 \frac{1}{x \sqrt{1-x^2}} \frac{1}{1+(\alpha x)^2} 
    = 
    \int_0^1 dx \left[
        \sqrt{1-x^2} (1+ (\alpha x)^2)
    \right]^{-1}
    = \\ &=
    \bigg/
        x = \cos t, \ d x = - \sin t \d t
    \bigg/ = \frac{1}{\sqrt{1+\alpha^2}} \arctg \l(
        \frac{\tg t}{\sqrt{1+\alpha^2}}
    \r) \bigg|_0^{\pi/2} = \frac{\pi}{2 \sqrt{1+\alpha^2}}.
\end{align*}
Тогда $I(\alpha)$
\begin{equation*}
    I(\alpha) = \frac{\pi}{2} \int \frac{d \alpha}{\sqrt{1+\alpha^2}} + C 
    = 
    \frac{\pi}{2} \ln \bigg|
        \alpha + \sqrt{\alpha^2 + 1}
    \bigg| + C = 
    \frac{\pi}{2} \ln\left(
        \alpha + \sqrt{1 + \alpha^2}
    \right)
    ,
\end{equation*}
где $I(0) = 0$ так что $C = 0$.



\subsubsection*{15.13(5)}

Попробуем через интеграл Эйлера-Пуассона доказать, что
\begin{equation*}
    \int_0^{+\infty} e^{-(x^2 + \alpha^2/x^2)} \d x = \frac{\sqrt{\pi}}{2} e^{-2\alpha}, \hspace{5 mm} \alpha > 0.
\end{equation*}
Представим интеграл в виде
\begin{equation*}
    \int_0^1 \exp\left(
        -x^2 - \frac{\alpha^2}{x^2}
    \right) + \int_{1}^{+\infty} \l(
        -x^2 - \frac{\alpha^2}{x^2}
    \r) \d x,
\end{equation*}
далее, произведя замену $y = 1/x$ в первом интеграле получаем
\begin{equation*}
    I(\alpha) = \int_{1}^{+\infty} \exp \l(
        -\alpha^2 y^2 + \frac{1}{y^21}
    \r) \frac{\d y}{y^2} + \int_{1}^{+\infty} \exp\left(
        -y^2 - \frac{\alpha^2}{y^2}
    \right) \d y.
\end{equation*}
Так как подынтегральные функции $f_1$ и $f_2$ сходятся непрерывны при всех $\alpha$ и $1 \leq y < + \infty$, а соответствующие интегралы, по признаку Вейерштрасса, сходятся равномерно:
\begin{equation*}
    |f_1| \leq \frac{1}{y^2}, \hspace{5 mm} |f_2| \leq e^{-y^2},
\end{equation*}
и интегралы
\begin{equation*}
    \int_1^{+\infty} \frac{\d y}{y^2}, \hspace{5 mm} \int_1^{+\infty} e^{-y^2} \d y
\end{equation*}
сходятся, то функция $I$ непрерывна $\forall |\alpha| \in \mathbb{R}$. 

Пусть $|\alpha| \geq \varepsilon > 0$. Поскольку функции $\partial_\alpha f_1$ и $\partial_\alpha f_2$ непрерывны в области $|\alpha| \geq \varepsilon, \ 1 \leq y < + \infty$, а соответствующие интегралы от них, в силу мажорантного признака, сходятся равномерно, то функция $I'$ непрерывна при $\alpha \neq 0$. Следовательно
\begin{equation*}
    I'_\alpha (\alpha) = - 2 \alpha \int_{0}^{+\infty} \exp\left(
        -x^2 - \frac{\alpha^2}{x^2}
    \right) \frac{\d x}{x^2}.
\end{equation*}
Кроме того, положив в исходном интеграле $x = \alpha/y$, $y > 0$, можем написать
\begin{equation*}
    I(\alpha) = \alpha \int_{0}^{+\infty} \exp\left(
        -y^2 - \frac{\alpha^2}{y^2}
    \right) \frac{\d y}{y^2}.
\end{equation*}
Сравнивая последние два интеграла, получаем дифференциальное уравнение $I' + 2 I = 0$, решая которое, находим
\begin{equation*}
    I(\alpha) = C e^{-2 \alpha}.
\end{equation*}
В силу непрерывности $I(\alpha)$ находим, что $I(0) = \sqrt{\pi}/2$, откуда $C = \sqrt{\pi}/2$. Окончательно,
\begin{equation*}
    I(\alpha) = \frac{\sqrt{\pi}}{2} \exp\left(
        - 2 |\alpha|
    \right).
\end{equation*}



\begin{to_lem}
    Верно представление, вида
    \begin{equation*}
        \frac{1}{x^2 + 1} = \int_0^{+\infty} e^{-y (1 + x^2)} \d y.
    \end{equation*}
\end{to_lem}



\subsubsection*{15.15(1, 4)}

\noindent
\textbf{1)} Найдём интеграл, вида
\begin{equation*}
    \int_0^{+\infty} \frac{\sin^2 (\alpha x)}{1+ x^2} \d x = \frac{1}{2} \int_0^{\infty} \frac{1-\cos(2 \alpha x)}{1 + x^2} \d x 
    = 
    \frac{1}{2} \int_{0}^{+\infty} \frac{\d x}{1 + x^2} - \frac{1}{2} 
    \int_{0}^{+\infty}  \frac{\cos (2 \alpha x)}{1 + x^2} \d x = \frac{\pi}{4} \left(
        1 - e^{-2|\alpha|}
    \right).
\end{equation*}

\noindent
\textbf{4)} Теперь хочется взять интеграл
\begin{equation*}
    I(\alpha) = \int_0^\infty \frac{\sin^2 (\alpha x)}{x^2 (1+x^2)} \d x = \int_0^{+\infty} f(x, \alpha) \d x.
\end{equation*}
Заметим, что $f(x, \alpha)$ Лебег-интегрируема $\forall \alpha \in E$. Рассмотрим
\begin{equation*}
    f'_\alpha = \frac{\sin 2\alpha x}{x^2 (1 + x^2)}.
\end{equation*}
для которой верно, что
\begin{equation*}
    \bigg|
        \frac{2 \sin 2 \alpha x}{x (1 + x^2)}
    \bigg| \leq \bigg|
        \frac{1}{1 + x^2}
    \bigg|, \hspace{5 mm} x < 0.1/\alpha,
    \hspace{10 mm}
    \bigg|
        \frac{2 \sin 2 \alpha x}{x (1 + x^2)}
    \bigg| \leq \bigg|
        \frac{2}{x^3}
    \bigg|, \hspace{5 mm} x > 1,
\end{equation*}
соответственно, $f'_\alpha$ Лебег-интегрируема. Тогда верно, что
\begin{equation*}
    I'_\alpha (\alpha) = \int_0^{+\infty}  \frac{\sin 2 \alpha x}{x (1 + x^2)} \d x.
\end{equation*}
Дифференцируем дальше, по крайней мере хотим, для этого необходимо, чтобы $f'_\alpha$ и $f''_{\alpha, \alpha}$ были бы Лебег-интегрируемы и существуют $\forall \alpha$, что верно. Тогда
\begin{align*}
    \frac{d^2 I(\alpha)}{d \alpha^2} &= 2 \int_{0}^{+\infty} 
    \frac{\cos(2 \alpha x)}{1+x^2} \d x = 2 \frac{\pi}{2} e^{-2 |\alpha|}.
\end{align*}
Последний интеграл уже берется:
\begin{equation*}
    I''(\alpha) = \int_0^{+\infty}  \frac{2 \cos 2 \alpha x}{1 + x^2} \d x = 2 \frac{\pi}{2} e^{-2 \alpha}.
\end{equation*}
Отсюда находим
\begin{equation*}
    I'_\alpha (\alpha) = - \frac{\pi}{2} e^{- 2\alpha} + C_1 = - \frac{\pi}{2} e^{-2\alpha} + \frac{\pi}{2},
\end{equation*}
и, наконец, находим
\begin{equation*}
    I(\alpha) = \frac{\pi}{4} e^{-2\alpha} + \frac{\pi}{2} \alpha + C_2,
    \hspace{0.5cm} \Rightarrow \hspace{0.5cm}
    I(\alpha) = \frac{\pi}{4}\left(
        e^{-2\alpha} + 2 \alpha  -1
    \right), \hspace{5 mm} \alpha > 0.
\end{equation*}



