\subsection{Дифференцирование и интегрирование по параметру несобственных интегралов}

\subsubsection*{Т2}

Вычислим интеграл Дирихле
\begin{equation*}
    \int_0^\infty \frac{\sin \alpha x}{x} \d x
\end{equation*}
Пусть
\begin{equation*}
    \Phi(\alpha, \beta) = \int_0^{\infty} e^{-\beta x} \frac{\sin (\alpha x)}{x} \d x,
\end{equation*}
что называют smooth cut of. 
\begin{equation*}
    \Phi'_\alpha (\alpha, \beta) = 
    \int_0^\infty e^{-\beta x} \cos (\alpha x) \d x = \frac{\beta}{\alpha^2 + \beta^2},
\end{equation*}
по правилу Лейбница (см. теормин). Теперь, интегриря $\alpha$ на отрезке $[0, \alpha]$ находим
\begin{equation*}
    \Phi(\alpha, \beta) - \Phi(0, \beta) = \beta \int_{0}^{\alpha} \frac{d t}{t^2 + \beta^2} = 
    \arctg \frac{\alpha}{\beta} \overset{\mathrm{?}}{\to} I(\alpha).
\end{equation*}
Понятно, что $I(\alpha) = - I(\alpha)$, так что далее считаем $\alpha > 0$. Имеем правно рассмотреть $\beta \in [0, 1]$.

Тогда рассмотрим предел
\begin{equation*}
    \lim_{\beta \to +0} \int_0^{+\infty} e^{-\beta x} \frac{\sin(\alpha x)}{x} \d x = I(\alpha) =
    \lim_{\beta \to +0} \arctg \frac{\alpha}{\beta} = \frac{\pi}{2}.
\end{equation*}
Таким образом для произвольного $\alpha$ верно, что
\begin{equation}
    \int_0^{\infty} \frac{\sin(x)}{x} \d x = \frac{\pi}{2} \sign (\alpha).
\end{equation}


% далее интегралы Лапласа с дробно рациональной функцией -- пример 2.
% ДЗ -- пример 5 -> интегралы Френеля. К3, стр 352
% 


\subsubsection*{15.1(1)}

Теперь хочется посчитать
\begin{equation*}
    \int_{0}^{+\infty}  \left(
        \cos^ (ax) - \cos^2 (bx)
    \right) \frac{dx}{x} = \frac{1}{2} \int_{0}^{+\infty}  \left(
        \cos(2ax) - \cos(2 b x)
    \right) \frac{dx}{x} = \frac{1}{2} \ln \frac{b}{a}.
\end{equation*}
где уже можем выбрать $\cos (2ax) = f(ax)$. 

\subsubsection*{15.1(2)}

Теперь найдём
\begin{equation*}
    \int_{0}^{+\infty}  \l(
        e^{-ax^2} - e^{-bx^2}
    \r) \frac{dx}{x} = \ln \sqrt{\frac{b}{a}} = \frac{1}{2} \ln \frac{b}{a},
\end{equation*}

\subsubsection*{15.1(4)}

На этот раз наш интеграл
\begin{equation*}
    \int_0^1 \frac{x^a - x^b}{\ln x} \d x = \bigg/
    \red{
        \ln \frac{1}{x} = t
    }
    \bigg/ = 
    \int_{\infty}^0 \frac{\d t}{t} e^{-t} (e^{-at} - e^{-bt}) = 
    \int_0^\infty \l(
        e^{-(b+1)t} - e^{-(a+1)t}
    \r) \frac{\d t}{t} = \ln \frac{a+1}{b+1}
\end{equation*}

% в 15.2(3) t = x^2

\subsubsection*{15.3(2)}

Интеграл
\begin{equation*}
    \int_0^\infty \sin x \cos^ x \frac{dx}{x} = \int_{0}^{+\infty} 
    \frac{dx}{x}\sin(x) \frac{1}{2} \l(
        1 + \cos (2x)
    \r) = \frac{1}{2} \int_0^\infty \frac{\sin x}{x} \d x + 
    \frac{1}{2} \int_0^\infty \frac{\sin x \cos 2 x}{x} \d x,
\end{equation*}
где уже хочется подставить раскрытый $\sin (3 x)$:
\begin{equation*}
    \frac{1}{2} \frac{\pi}{2} + \frac{1}{2} \frac{1}{2} \int_0^\infty \frac{\sin(3x)}{x} \d x -
    \frac{1}{4} \int_0^\infty \frac{\sin x}{x} \d x = \frac{\pi}{4}.
\end{equation*}
% почему можно дифференцировать по параметру, равномерно ил он сходится


\subsubsection*{15.4(3)}

Есть интграл вида
\begin{equation*}
    I(\alpha) = \int_{0}^{+\infty} \frac{\sin^4 (\alpha x)}{x^2} \d x.
\end{equation*}
% хочется проверить условие дифференцируемости
...

Теперь можем посчитать
\begin{equation*}
    I'_\alpha (\alpha) = \int_{0}^{+\infty} 4 \sin^2 (\alpha x) \cos(\alpha x) \frac{\d x}{x}
    =
    \int_{0}^{+\infty} \frac{dx}{x} \l(
        \frac{1}{4} \sin (2 x\alpha) - \frac{1}{8} \sin (4 x \alpha)
    \r) = \frac{\pi}{4} \sign \alpha,
\end{equation*}
что верно $\forall \alpha$. 

Возвращаясь к интегралу, находим, что
\begin{equation*}
    I(\alpha) = \frac{\pi}{4} |\alpha| + 0,
\end{equation*}
так как $I(0) = 0$.


\subsubsection*{Интерирование по частям}

Воообще 
\begin{equation*}
    \int_{a}^{+\infty}  f(x, \alpha) g'_x (x, \alpha) \d x 
    =
    f(x, \alpha) g(x, \alpha) \bigg|_a^{\infty} - \int_{a}^{+\infty}  f'_x (x, \alpha) g(x, \alpha) \d x,
\end{equation*}
работает, когда $f, g \in C^1$ по $x$ и любые два из трёх написанных пределов существуют.



\subsubsection*{15.5(6)}

Интеграл 
\begin{equation*}
    I(\alpha) = \int_{0}^{+\infty}  \sin^3 (x) \cos (\alpha x) \frac{dx}{x^3}
\end{equation*}
хотелось бы вычислить.

Интегрируя по частям
\begin{equation*}
    \sin^3 x \cos (\alpha x) = \frac{3}{8} \left(
        \sin(\alpha+1)x - \sin (\alpha - 1)x
    \right) - \frac{1}{8} \l(
        \sin(\alpha+3)x - \sin(\alpha_3)x
    \r),
\end{equation*}
для $\alpha > 3$. В общем приходим к выражению
\begin{align*}
    I(\alpha) 
    &=
     \int_{0}^{+\infty}  \sin^3 (x) \cos (\alpha x) \frac{dx}{x^3}
    =
     \int_0^\infty \sin^3 x \cos(\alpha x) \d \l(
        \frac{-1}{2x^2}
    \r) 
    = 
    \\
    &=
    \l(
        -\frac{1}{2x^2}
    \r) \sin^3 x \cos \alpha x \bigg|_0^\infty + \frac{1}{2} \int_{0}^{+\infty} 
    \frac{1}{x^2} \d (\sin^3 x \cos \alpha x) = 
    \frac{1}{2} \int_{0}^{+\infty} \l(
        \sin^3 (x) \cos (\alpha x)
    \r)'_x  f\left(-\frac{1}{x}\right) 
    = 
    \\
    &= 
    - \frac{1}{2x} \l(
        \sin^3 x \cos(\alpha x)
    \r)'_x \bigg|_0^\infty + 
    \frac{1}{2} \int_{0}^{+\infty}  \frac{1}{x} \d \l(
        \sin^3 x \cos \alpha x
    \r)'_x
    = 
    \\ 
    &=
    -\frac{1}{2} \int_{0}^{+\infty} \frac{\d x}{x} \l(
        \frac{3}{8}\left[
            (\alpha+1)^2 \sin (\alpha+1)x - (\alpha-1)2 \sin (\alpha-1)x
        \right] - 
        \frac{1}{8}\left[
            (\alpha+3)^2 \sin (\alpha+3) x - (\alpha-3)^2 \sin  (\alpha-3)x
        \right]
    \r) 
    = \\
    &=
    -\frac{\pi}{4} \left\{
        \frac{3}{8} (\alpha+1)^2 - \frac{3}{8} (\alpha-1)^2 - \frac{1}{8} (\alpha + 3)^2+ \frac{1}{8} (\alpha-3)^2
    \right\} = 0 
\end{align*}


\subsubsection*{15.6(5)}
При выполнении всех условий о дифференцирование интеграла по параметру, для интеграла 
\begin{equation*}
    I(\alpha) = \int_0^1 \frac{\arctg (\alpha x)}{x \sqrt{1-x^2}} \d x,
\end{equation*}
может быть так посчитан. 

Действительно,
\begin{align*}
    \frac{d I(\alpha)}{d \alpha} 
    &=
    \int_0^1 \frac{1}{x \sqrt{1-x^2}} \frac{1}{1+(\alpha x)^2} 
    = 
    \int_0^1 dx \left[
        \sqrt{1-x^2} (1+ (\alpha x)^2)
    \right]^{-1}
    = \\ &=
    \bigg/
        x = \cos t, \ d x = - \sin t \d t
    \bigg/ = \frac{1}{\sqrt{1+\alpha^2}} \arctg \l(
        \frac{\tg t}{\sqrt{1+\alpha^2}}
    \r) \bigg|_0^{\pi/2} = \frac{\pi}{2 \sqrt{1+\alpha^2}}.
\end{align*}
Тогда $I(\alpha)$
\begin{equation*}
    I(\alpha) = \frac{\pi}{2} \int \frac{d \alpha}{\sqrt{1+\alpha^2}} + C -
    \frac{\pi}{2} \ln \bigg|
        \alpha + \sqrt{\alpha^2 + 1}
    \bigg| + C,
\end{equation*}
где $I(0) = 0$ так что $C = 0$.



\subsubsection*{15.15(4)}

Теперь хочется взять интеграл
\begin{equation*}
    I(\alpha) = \int_0^\infty \frac{\sin^2 (\alpha x)}{x^2 (1+x^2)} \d x
\end{equation*}
И снова по частям (при выполнении условий теоремы о дифференцировании по параметру \red{[!]}) справедливо утверждать
\begin{align*}
    \frac{d I(\alpha)}{d \alpha}  
    &= \int_0^\infty \frac{\sin (2 \alpha x)}{x (1+x^2)} \d x, \\
    \frac{d^2 I(\alpha)}{d \alpha^2} &= 2 \int_{0}^{+\infty} 
    \frac{\cos(2 \alpha x)}{1+x^2} \d x = 2 \frac{\pi}{2} e^{-2 |\alpha|}.
\end{align*}
