% банахово пространство и теорема Бэра

\begin{to_def}
    Векторное пространство $E$ \textit{нормировано}, если для всякого вектора $v \in E$ имеется неотрицательное исло $\|v\|$, удовлетворяющее свойствам:
    \begin{enumerate}
        \item $\|av\| = |a| \cdot \|v\|$ (однородность при умножении на константу);
        \item $\|v + w\| \leq \|v\| + \|w\|$ (неравенство треугольника);
        \item $\|v\|=0 \Leftrightarrow v = 0$ (невырожденность).
    \end{enumerate}
\end{to_def}

Например, шаром с центром $c$ и с радиусом $r$ в нормированном пространстве $E$ называется множество
\begin{equation*}
    B_c (r) = \{x \in E \mid
    \|x-c\|\leq r
    \}.
\end{equation*}
Заметим, что норма полностью определяется единичным шаром с центром в нуле $B_0 (1)$, а именно
\begin{equation*}
    \|x\| = \inf\{
        |1/t|\, \mid \, t x \in B_0 (1)
    \}.
\end{equation*}


\begin{to_thr}[Теорема Бэра для открытых множеств]
    Счётное семейство открытых всюду плотных подмножеств банахова пространства $E$ имеет непустое пересечение.
\end{to_thr}

\begin{to_con}[Теорема Бэра для замкнутых множеств]
    Если банахово пространство $E$ покрыто счётным семейством замкнутых множеств, то одно из них имеет непустую внутренность.
\end{to_con}

\begin{to_thr}[Неподвижные точки сжимающих отображений]
    Пусть $E$ -- банахово пространство. Пусть $X \subset E$ -- замкнутое подмножество и $f \colon  X \mapsto X$ является сжимающим, то есть
    \begin{equation*}
        \exists C < 1 \ \colon  \ \forall x, y \in X \ 
        \|f(x)-f(y)\| \leq C \|x-y\|.
    \end{equation*}
    Тогда $f$ имеет неподвижную точку $x \in X$, такую что $f(x) = x$.
\end{to_thr}


% 