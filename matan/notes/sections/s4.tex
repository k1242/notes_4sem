\begin{to_def}
    Обозначим \textit{частичную сумму} тригонометрического ряда Фурье для $2\pi$-периодической функции $f$ как
    \begin{equation*}
        T_n (f, x) = \sum_{k=-n}^n c_k (f) e^{ikx}.
    \end{equation*}
\end{to_def}

\begin{to_lem}
    Для $n$-й частичной суммы ряда Фурье $2\pi$-периодической функции имеет место формула в виде свёртки
    \begin{equation*}
        T_n (f, x) = \int_{-\pi}^{\pi} f(x+t) D_n (t) \d T,
    \end{equation*}
    с ядром Дирихле
    \begin{equation*}
        D_n (t) = \frac{1}{2\pi} \frac{\sin \big(\left(n+\frac{1}{2}\right) t\big)}{\sin\left(\frac{1}{2} t\right)}.
    \end{equation*}
\end{to_lem}


\begin{to_lem}[Равномерная ограниченность интегралов от ядра Дирихле]
    Существует такая константа $C$, что 
    \begin{equation*}
        \left|
        \int_a^b D_n (t) \d t
        \right| \leq C
    \end{equation*}
    для любых $a, b \in [-\pi, \pi], \ n \in \mathbb{N}$.
\end{to_lem}

\begin{to_thr}[Равномерный принцип локализации]
    Запищем для $\delta \in (0, \pi)$
    \begin{equation*}
        T_n (f, x) - f(x) = 
        \int_{-\pi}^{\pi} 
        \left(
            f(x+t) - f(x)
        \right) D_n (t) \d t =
        \int_{-\delta}^{\delta} \left(
            f(x+t) - f(x)
        \right) D_n (t) \d t + 
        \int_M 
        (f(x+t)-f(x)) D_n (t) \d t,
    \end{equation*}
    где $M = \left\{t \mid \delta \leq |t| \leq \pi\right\}$. Если $f \in L_1 [-\pi, \pi]$, то
    \begin{equation*}
        \int_M \left(
            f(x+t) - f(x)
        \right) D_n (t) \d t
        \ \to \ 0, \hspace{0.5 cm} n \to \infty.
    \end{equation*}
    Если $f$ ограничена на отрезке $[a, b]$, то это выражение стремится к нулю равномерно по $x \in [a, b]$.
\end{to_thr}


\begin{to_def}
    Функция $f$ называется гёльдеровой степени $\alpha > 0$, если для любых $x, \, y$ из области определения
    \begin{equation*}
        |f(x) -f(y)| \leq C |x-y|^{\alpha}
    \end{equation*}
    с некоторой константой $C$.
\end{to_def}


\begin{to_thr}[Признак Липшица сходимости ряда Фурье]
    Для абсолютно интегрируемой $2\pi$-периодической функции, которая является гёльдеровой с некоторыми $C$, $\alpha > 0$ на интервале $(A, B) \supset [a, b]$
    \begin{equation*}
        T_n (f, x) \to f(x)
    \end{equation*}
    равномерно $x \in [a, b]$ при $n \to \infty$.
\end{to_thr}

\begin{to_thr}[Признак Дирихле сходимости ряда Фурье]
    Для абсолютно интегрируемой $2\pi$-периодической функции, которая является непрерывной с ограниченной вариацией на интервале $(A, B) \supset [a, b]$
    \begin{equation*}
        T_n (f, x) \to f(x)
    \end{equation*}
    равномерно по $x \in [a, b]$ при $n \to \infty$.
\end{to_thr}

\red{Далее несколько лемм, сформулированных в виде задач, а именно признак Дирихле сходимости ряда Фурье в точке, признак Липшица сходимости ряда Фурье в точке, признак Дини сходимости ряда Фурье в точке. Ага, это 13 тема. А потом будут темы 14 - 17.}



\subsubsection*{Интегрирование ряда Фурье}

\begin{to_thr}[Почленное интегрирование ряда Фурье]
    Пусть $f \in L_1 [-\pi, \pi]$ соответствует не обязательно сходящийся ряд Фурье, записанный в действительном виде как
    \begin{equation*}
        a_0 + \sum_{n=1}^{\infty} (a_n \cos nx + B_n \sin nx).
    \end{equation*}
    Тогда ряд Фурье можно почленно интегрировать, то есть выполняется формула
    \begin{equation*}
        \int_a^b f(x) \d x =    
        \frac{1}{2} a_0 (b-a) + 
        \sum_{n=1}^{\infty} \left(
            \frac{a_n \sin n x}{n} -
            \frac{b_n \cos nx }{n}
        \right) \bigg|_a^b.
    \end{equation*}
\end{to_thr}



\begin{to_lem}
    \red{Разложим $\cos ax$ на отрезке $[-\pi, \pi]$ при $a \notin \mathbb{Z}$ в ряд Фурье.} Легко получить, что
    \begin{align*}
        &\ctg x
        &=
        & v. p. \sum_{k=-\infty}^{\infty} \frac{1}{x-\pi k} \\
        &\frac{1}{\sin x}
        &=
        & v. p. \sum_{k=-\infty}^{\infty} \frac{(-1)^k}{x-\pi k} \\
        &\sin x
        &=
        & x \prod_{k=1}^{\infty} \left(
            1 - \frac{x^2}{\pi^2 k^2}
        \right).
    \end{align*}
\end{to_lem}


\begin{to_lem}
    \textit{Формула дополнения} для бета-функции про $p \in (0, 1)$
    \begin{equation*}
        B(p, 1-p) = \int_0^1 t^{p-1} (1-t)^{-p} \d t = \frac{\pi}{\sin \pi p}.
    \end{equation*}
\end{to_lem}

\begin{to_lem}
    Для $0 < |x| < \pi$ верно, что
    \begin{equation*}
        \frac{1}{x} - \ctg x = 
        \sum_{n, k \geq 1} \frac{2 x^{2k-1}}{\pi^{2k} n^{2k}},
    \end{equation*}
    \red{
        откуда можно получить значения сумм $\sum_{n=1}^{\infty} \frac{1}{n^2}$ и $\sum_{n=1}^{\infty} \frac{1}{n^4}$.
    }
\end{to_lem}


\subsubsection*{Суммирование тригонометрических рядов по Фейеру}


\begin{to_def}
    Определим \textit{ядро Фейера} 
    \begin{equation*}
        \Phi_n (t) = 
        \frac{D_0 (t) + D_1 (t) + \ldots + D_n (t)}{n+1} = 
        \frac{1}{2\pi} \sum_{k=-n}^{n} 
        \frac{n+1 - |k|}{n+1} e^{ikx},
    \end{equation*}
    как усреднение ядер Дирихле. Соответствующая \textit{сумма Фейера} будет соответствовать усреднением первых $n+1$ частичных сумм ряда Фурье,
    \begin{equation*}
        S_n (f, x) = \int_{-\pi}^{\pi} 
        f(x+\xi) \Phi_n (\xi) \d \xi = 
        \frac{T_0 (f, x) + \ldots + T_n (f, x)}{n+1}.
    \end{equation*}
\end{to_def}

Записав
\begin{equation*}
    D_n (t) 
    =
    \frac{1}{2\pi}
     \frac{\sin \left( \left(n + \frac{1}{2}\right)t\right)}{\sin \left(\frac{1}{2} t\right)} 
     = 
     \frac{1}{4\pi}
    \frac{
    \cos nt - \cos \left((n+1)t\right)
    }{
    \sin^2 \left(\frac{1}{2}t\right)
    },
\end{equation*}
и суммируя, получаем
\begin{equation*}
    \Phi_n (t) = 
    \frac{1}{4 \pi}
    \frac{
        1 - \cos \left((n+1)t\right)
    }{
        (n+1) \sin^2 \left(\frac{1}{2}t\right)
    }
    =
    \frac{1}{2\pi}
    \frac{
        \sin^2 \left(
            \frac{n+1}{2}t
        \right)
    }{
        (n+1) \sin^2 \left(\frac{1}{2}t\right)
    }
\end{equation*}


\begin{to_thr}
    Для непрерывной $2\pi$-периодической $f$ 
    \begin{equation*}
        S_n (f, x) \rightrightarrows f(x),
    \end{equation*}
    то есть сходится равномерно.
\end{to_thr}
