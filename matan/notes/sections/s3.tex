\begin{to_thr}[Теорема Вейерштрасса для тригонометрических многочленов]
    \label{thr_4.77}
    Всякую непрерывную на $[-\pi, \pi]$ функцию $f$, для которой $f(-\pi)=f(\pi)$, можно сколь угодно близко равномерно приблизить тригонометрическими многочленами вида
    \begin{equation*}
        T(x) = a_0 + \sum_{k=1}^{n} (a_k \cos kx + b_k \sin kx).
    \end{equation*}
\end{to_thr}

\begin{to_thr}[Теорема Стоуна-Вейерштрасса]
    Пусть у нас зафиксирован компакт $K$ и дана алгебра непрерывных функций $\mathcal A$ на этом компакте, которая разделяет точки, то есть для любых $x \neq y \in K$ найдётся $f \in \mathcal A$, такая что $f(x) \neq f(y)$. Тогда Всякую непрерывную на $K$ функцию можно сколь угодно близко равномерно приблизить функциями из $\mathcal A$.
\end{to_thr}

\red{Вспомнить про $\|f\|_C$.} Равномерное приближение является приближением по норме $L_2$, так как на отрезке $[-\pi, \pi]$ имеется неравенство $\|f\|_2 \leq \sqrt{2\pi} \|f\|_C$. В случае $L_2$ нормы определим коэффициенты, которыми собираемся приближать.

\begin{to_thr}[Оптимальность коэффициентов Фурье]
    Для всякой $f \in L_2[-\pi, \pi]$ и данного числа $n$ лучшее по норме $L_2$ приближение $f$ тригонометрическим многочленом $\sum_{-n}^{+n} c_k e^{ikx}$ дают коэффициенты Фурье
    \begin{equation*}
        c_k = \frac{1}{2\pi} \int_{-\pi}^{\pi} f(x) e^{ikx} \d x.
    \end{equation*}
\end{to_thr}


\begin{to_lem}[неравенство Бесселя]
    Из доказательства предыдущей теоремы, можем получить, что
    \begin{equation*}
        \left\|f - \sum_{k=1}^N c_k \varphi_k \right\|_2^2 = 
        \|f\|_2^2 - \sum_{k=1}^{N} |c_k|^2 \|\varphi_k\|_2^2,
        \hspace{0.7 cm} \Rightarrow \hspace{0.7   cm}
        \|f\|_2^2 \geq  \sum_{k=1}^{\infty} |c_k|^2 \|\varphi_k\|^2_2,
        \hspace{0.5cm} \overset{\mathrm{trig}}{\Rightarrow}  \hspace{0.5cm}
        \|f\|_2^2 \geq 2\pi \sum_{k=-n}^n |c_k|^2.
    \end{equation*}
    \red{Точно ли до $n$?}
\end{to_lem}

\begin{to_lem}[Представление действительнозначной функции]
    Для действительнозначной функции представление в виде ряда Фурье перепишется в виде
    \begin{equation*}
        f = \sum_{k=0}^{n} (a_k \cos kx + b_k \sin kx),
        \hspace{1 cm}
        a_k = \frac{1}{\pi}\int_{-\pi}^{\pi} f(x) \cos kx \d x,
        \hspace{0.5 cm}
        b_k = \frac{1}{\pi} \int_{-\pi}^{\pi} f(x) \sin kx \d x,
    \end{equation*}
    для $k \geq 1$. Неравенство Бесселя тогда запишется так:
    \begin{equation*}
        \|f\|_2^2 \geq \frac{\pi}{2} |a_0|^2 + 
        \pi \sum_{k=1}^{\infty} (|a_k|^2 + |b_k|^2).
    \end{equation*}
\end{to_lem}

\begin{to_thr}[Сходимость ряда Фурье в среднеквадратичном]
    Для вской комплекснозначной $f \in L_2 [-\pi, \pi]$
    \begin{equation*}
        f = \sum_{k=-\infty}^{\infty} 
        c_k e^{ikx} = 
        \lim_{n \to \infty} \sum_{k=-n}^{n} c_k e^{ikx}
    \end{equation*}
    в смысле сходимости суммы в пространстве $L_2[-\pi, \pi]$, а также выполняется равенство Парсеваля
    \begin{equation*}
        \|f\|_2^2 = 2 \pi \sum_{k=-\infty}^{\infty} |c_k|^2.
    \end{equation*}
\end{to_thr}

\texttt{Пока мы не доказали, что в полученную формулу можно подставить хоть одно конкретное значение $x$. Тот факт, что ряд Фурье функции из
$L_2[-\pi, \pi]$ на самом деле сходится к этой функции почти всюду, был доказан Л. Карлесоном (1966), а до этого был известен как гипотеза Лузина.} 