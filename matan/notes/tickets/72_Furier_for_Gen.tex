\sbs{72}{Преобразование Фурье для обобщённый функций}
Тут нам уже потрубется пространство $\mathcal{S}(\mathbb{R})$. Так как
\begin{to_def}
	Преобразование Фурье непрерывно переводит $\mathcal{S}(\mathbb{R})$ в $\mathcal{S}(\mathbb{R})$.
\end{to_def}

\begin{to_def}
	$\mathcal{S}(\mathbb{R})$ -- \textbf{пространство бесконечно дифференцируемых функций} $f \colon \mathbb{R} \to \mathbb{C}$, у которрой конечны все полунормы $(k,n \geq 0)$
	\begin{equation*}
		\|f\|_{n,k} = \sup \{x^n f^{(k)}(x) \mid x \in \mathbb{R}\}.
	\end{equation*}
\end{to_def}

Соответсвенно можем определить преобразование Фурье:
\begin{equation*}
	\langle F[\lambda], \varphi\rangle = \langle \lambda, F[\varphi]\rangle.
\end{equation*}

Наверное тут хочется посмотреть на хороший пример: обозначим $F[\delta] = \hat{\delta}$, тогда
\begin{equation*}
	(\hat{\delta}, \varphi) = (\delta, \hat{\varphi}) = \hat{\varphi}(0) = \frac{1}{\sqrt{2 \pi}} \int_{-\infty}^{+\infty} \varphi(x) e^{- i x y} d x\big|_{y=0} = \frac{1}{\sqrt{2 \pi}} \int_{-\infty}^{+\infty} \varphi(x) d x = \left(\frac{1}{\sqrt{2 \pi}}, \varphi\right)
\end{equation*}
Ого, мы полчуили, что $F[\delta] = 1/\sqrt{2\pi}$, а для обратного тогда $F^{-1}[1] = \sqrt{2\pi}\delta$. И подобным же образом получается и прямое преобразование фурье для единицы, что ранее не в обобщенных нам было не доступно.
