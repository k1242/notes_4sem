\sbs{17}{Полные системы в пространстве \texorpdfstring{$L_2$}{L2}} 


Пусть $\{\varphi_i\}$ -- ортогональная система в $L_2$. Допустим $f = \sum_i c_i \varphi_i$, где коэффиценты $c_i$ могут быть найдены непосредственно:
\begin{equation*}
    c_i = \frac{(f,\, \varphi_i)}{\left(\varphi_i,\, \varphi_i\right)},
\end{equation*}
что упрощается в случае ортонормированной системы до $c_i = (f,\, \tilde{\varphi}_i)$. Числа $c_i$ и называются \textit{коэффицентами Фурье элемента} $f$ в ортогональной системе $\varphi_i$. 


В таких терминах можем определить и \textit{ряд Фурье} элемента $f$ по ортогональной системе $\{\varphi_k\}$:
\begin{equation*}
    f \sim \sum_{k=1}^{\infty} \frac{(f,\, \varphi_i)}{\left(\varphi_i,\, \varphi_i\right)} \varphi_k,
\end{equation*}
где если система $\varphi_k$ конечна, то ряд сводится к конечной сумме. 

Так например можно выделить ортогональную систему $\{1, \cos k x, \sin kx; \ k \in \mathbb{N}\}$. Или, например, многочлены Лежандра
\begin{equation*}
    P_n (x) = \frac{1}{2^n n!} \frac{d^n}{d z^n} \left(z^2-1\right)^n,
\end{equation*}
образующих ортогональную систему. 

 
\begin{to_def}
    Система $\{\varphi_\alpha; \alpha \in \mathcal A\}$ векторов нормированного пространства $X$ называется \textit{полной по отношению к множеству} $E \subseteq X$ (полной в $E$), если любой вектор $x \in E$ можно сколь угодно точно в смысле нормы пространства $X$ приблизить конечными линейными комбинациями векторов системы. Другими словами $E \subset \bar{L}\{\varphi_\alpha\}$ -- замыкание линейной оболочки векторов. 
\end{to_def}



\begin{to_thr}[условие полноты ортогональной системы]
    Пусть $X$ -- линейное пространство со скалярным произведением, а $\varphi_k$ -- конечная или счётная система ортогональных векторов в $X$. Тогда следующие условия эквивалентны: 
    \vspace{-2mm}
    \begin{enumerate*}
        \item система $\{\varphi_k\}$ полна по отношению к множеству $E \subseteq X$;
        \item для любого вектора в $f \in E \subset X$ имеет место разложение в ряд Фурье в смысле нормы;
        \item для любого вектора $f \in E \subset X$ имеет место равенство Парсеваля $\|f\|^2 = \sum_k |(f, \varphi_k)|^2/(\varphi_k, \varphi_k)$.
    \end{enumerate*}
\end{to_thr}

\begin{uproof}
Из (1) $\Rightarrow$ (2) в силу экстремального свойства коэффициентов Фурье. Из (2) в (3) по теореме Пифагора. Из (3) $\Rightarrow$ (1) т.к. ввиду леммы о перпендикуляре по теореме Пифагора ...
\red{по Зоричу можно дописать}.
\end{uproof}



\begin{to_def}
    Система элементов линейного нормированного простанства $X$ называется \textit{базисом пространства} $X$, если любая конечная её подсистема состоит из линейно независимых векторов и любой вектор $x \in X$ может быть представлен в виде $f = \sum_k \alpha_k x_k$, где $\alpha_k$ -- коэффициенты из поля констант пространства $X$, а сходимость понимается по норме пространства $X$. 
\end{to_def}






Для доказательства неравенства Бесселя достаточно требовать ортогональность системы. В случае же равенства Парсеваля необходима \textit{полнота} системы -- возможность приблизить любую функцию $L_2$ линейной комбинацикй функций рассматриваемой системы сколь угодно точно. 



