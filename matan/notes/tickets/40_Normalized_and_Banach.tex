\sbs{40}{Нормированные векторные и банаховы пространства}

\begin{to_def}
	Векторное $E$ -- \textbf{нормировано}, если $\forall v \in E$ имеется $\|v\|$ такое, что:
	\begin{itemize}
		\item Однородность: $\|\alpha v\| = |\alpha| \|v\|$
		\item Неравенство треугольника: $\|v + w\| \leq \|v\| + \|w\|$
		\item Невырожденность: $\|v\| = 0 \Leftrightarrow v = 0$
	\end{itemize}
\end{to_def}

Можно эквивалентно определить норму единичным шаром с центром в $B_0(1)$:
\begin{equation*}
	\|x\| = \inf \{|1/t| \mid t x \in B_0(1)\}
	\hspace{1 cm}
	\text{where }
	B_{c}(r) = \{x \in E \mid \|x-c\| \leq r\}.
\end{equation*}

\begin{to_def}
	Полное нормированное пространство называется \textbf{банаховым}.
\end{to_def}

Так например, $L_p$ и $C[a, b]$ (с нормой $\|\cdot\|_{\infty}$) -- банаховы, а $C[a, b]$ в $L_2$ не ялвяется полным. 



