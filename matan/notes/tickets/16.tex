\sbs{16}{Неравенство Бесселя и оптимальность коэффициентов Фурье}


\begin{to_thr}[Оптимальность коэффициентов Фурье]
    Для всякой $f \in L_2[-\pi, \pi]$ и данного числа $n$ лучшее по норме $L_2$ приближение $f$ тригонометрическим многочленом $\sum_{-n}^{+n} c_k e^{ikx}$ дают коэффициенты Фурье
    \begin{equation*}
        c_k = \frac{1}{2\pi} \int_{-\pi}^{\pi} f(x) e^{ikx} \d x.
    \end{equation*}
\end{to_thr}

\begin{uproof}
    Воспользуемся скалярным произведением в $L_2$, занумеруем $e^{ikx}$ в некотором порядке $\varphi_1$, $\varphi_2$, $\ldots$, где далее будет важна лишь орогональность этих функций относительно введенного скалярного произведения. Пусть мы приближаем $\varphi = \sum_{k=1}^N a_k \varphi_k$ и оптимизируем $a_k$, тогда
    \begin{equation*}
        \left\|
            f - \sum_{k=1}^{N} a_k \varphi_k
        \right\|_2^2 = \|f\|_2^2 - 
        \sum_{k=1}^{N} \bar{a}_k (f, \varphi_k) - \sum_{k=1}^{N} a_k (\varphi_k, f) + \sum_{k=1}^{N} |a_k|^2 \|\varphi_k\|_2^2.
    \end{equation*}
    Далее, по определению коэффициентов Фурье в виде $(f, \varphi_k) = c_k \|\varphi\|_2^2$ находим
    \begin{equation*}
        \left\|
            f - \sum_{k=1}^{N} a_k \varphi_k
        \right\|_2^2 = \|f\|_2^2 - \sum_{k=1}^{N} \left(
            \bar{a}_k c_k + a_k \bar{c}_k - |a_k|^2
        \right) \|\varphi_k\|_2^2 = 
        \|f\|_2^2 - \sum_{k=1}^{N} |c_k|^2 \|\varphi_k\|_2^2 + \sum_{k=1}^{N} |c_k - a_k|^2 \|\varphi_k\|_2^2,
    \end{equation*}
    откуда оптимально положить $a_k = c_k$. 
\end{uproof}


\begin{to_lem}[неравенство Бесселя]
    Из доказательства предыдущей теоремы, можем получить, что
    \begin{equation*}
        \left\|f - \sum_{k=1}^N c_k \varphi_k \right\|_2^2 = 
        \|f\|_2^2 - \sum_{k=1}^{N} |c_k|^2 \|\varphi_k\|_2^2,
        \hspace{0.7 cm} \Rightarrow \hspace{0.7   cm}
        \|f\|_2^2 \geq  \sum_{k=1}^{N} |c_k|^2 \|\varphi_k\|^2_2,
        \hspace{0.5cm} \overset{\mathrm{trig}}{\Rightarrow}  \hspace{0.5cm}
        \|f\|_2^2 \geq 2\pi \sum_{k=-n}^n |c_k|^2.
    \end{equation*}
    % \red{Точно ли до $n$?}
\end{to_lem}

\begin{to_lem}[Представление действительнозначной функции]
    Для действительнозначной функции представление в виде ряда Фурье перепишется в виде
    \begin{equation*}
        f = a_0 + \sum_{k=1}^{n} (a_k \cos kx + b_k \sin kx),
        \hspace{0.25 cm}
        a_0 = \frac{1}{2\pi} \int_{-\pi}^{+\pi} f(x) \d x,
        \hspace{2.5 mm} 
        a_k = \frac{1}{\pi}\int_{-\pi}^{\pi} f(x) \cos kx \d x,
        \hspace{0.25 cm}
        b_k = \frac{1}{\pi} \int_{-\pi}^{\pi} f(x) \sin kx \d x,
    \end{equation*}
    для $k \geq 1$. Неравенство Бесселя тогда запишется так:
    \begin{equation*}
        \|f\|_2^2 \geq \frac{\pi}{2} |a_0|^2 + 
        \pi \sum_{k=1}^{\infty} (|a_k|^2 + |b_k|^2).
    \end{equation*}
\end{to_lem}

