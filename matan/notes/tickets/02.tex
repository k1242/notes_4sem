\sbs{2}{Приближение \texorpdfstring{$2\pi$}{2pi}-периодических функций тригонометрическими многочленами}
% 115 страница (4.8)

\begin{to_thr}[теорема Вейерштрасса]
    Всякую непрерывную
    на $[-\pi, \pi]$ функцию $2\pi$-периодичную $f \colon \mathbb{R} \to \mathbb{R}$, для которой $f(-\pi)=f(\pi)$
       можно сколько угодно точно равномерно приблизить $$T(x) = a_0 + \sum\limits_{k = 1}^{n}(a_k \cos(k x) + b_k \sin(k x)). $$
\label{Vthr}
\end{to_thr}


\begin{uproof}
    Многочлен от тригонометрическего многочлена -- всё ещё многочлен. Рассмотрим некотрую непрерывную $g(\cos x)$, которую можем приблизить на компакте $P(\cos x)$. В частности, можем приблизить $2 \pi$-периодическую функцию 
    \begin{equation*}
        \psi_\delta(x) = \sum_{k \in \mathbb{Z}} \varphi_\delta (x-  2 \pi k),
    \end{equation*}
    так как она чётна и $2\pi$-периодична, а значит зависит от $\cos x$ непрерывно. Далее любую непрерывную $2\pi$-периодическую $f$ будем приближать суммами
    \begin{equation*}
        f_m (x) = \sum_{k=0}^{m} f( 2 \pi k /m) \psi_{2 \pi /m} (x-2\pi k /m),
    \end{equation*}
    аналогично раннее доказанной лемме. 
\end{uproof}