\sbs{50}{Теорема Арцела-Асколи}
\begin{to_def}
	Множество функций $X \subset C(K)$(над метрическим компактом) \textbf{равностепенно непрерывно} , если 
	\begin{equation*}
		\forall \varepsilon >0 \, \exists \delta > 0 \colon \forall f \in X\, \forall x, y \in K, \, \rho(x,y) < \delta \Leftrightarrow |f(x) - f(y)| < \varepsilon.
	\end{equation*}
	Если все функции ещё и $L$-липшецивы, то $|f(x) - f(y)| = L \rho(x,y)$.
\end{to_def}

\begin{to_def}
	\textbf{Модуль непрерывности} липшецивых функций:
	\begin{equation*}
		\omega_X(\delta) = \sup \left\{|f(x) - f(y)| \mid f \in X, \, \rho(x,y) < \delta \right\}.
	\end{equation*}
	И тогда,
	$X$ -- равностепенно непрерывно $\Longleftrightarrow$ $\omega_X(\delta) \to 0$ при $\delta \to +0$.
\end{to_def}


\begin{to_thr}[Арцела-Асколи]
	Множество $X \subset C(K)$ предкомпактно $\Longleftrightarrow$ $X$ равномерно ограниченно и равностепенно непрерывно.
\end{to_thr}