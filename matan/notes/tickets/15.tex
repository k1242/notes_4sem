\sbs{15}{Неравенство Коши-Буняковского}


% \begin{to_thr}[Теорема Вейерштрасса для тригонометрических многочленов]
%     \label{thr_4.77}
%     Всякую непрерывную на $[-\pi, \pi]$ функцию $f$, для которой $f(-\pi)=f(\pi)$, можно сколь угодно близко равномерно приблизить тригонометрическими многочленами вида
%     \begin{equation*}
%         T(x) = a_0 + \sum_{k=1}^{n} (a_k \cos kx + b_k \sin kx).
%     \end{equation*}
% \end{to_thr}

\textcolor{ugray}{
\begin{to_thr}[Неравенство Коши-Буняковского]
    Пусть функции $f,\, g \colon  X \mapsto \mathbb{R}$ измеримы по Лебегу, а также $|f|^2,\, |g|^2 \in L_1(X)$. Тогда
    \begin{equation*}
         \left(
            \int_X f(x) g(x) \d x
         \right)^2 \leq \left(
            \int_X |f(x)|^2 \d x
         \right) \cdot \left(
            \int_X |g(x)|^2 \d x
         \right).
     \end{equation*} 
\end{to_thr}
}

\begin{uproof}
    Домножая на константы добиваемся нормировки к $1$ интегралов $|f|^2$ и $|g|^2$. Тогда 
    \begin{equation*}
        |fg| \leq \frac{|f|^2}{2} + \frac{|g|^2}{2},
        \hspace{0.5cm} \Rightarrow \hspace{0.5cm}
        \int_X |fg| \d x \leq 1, \hspace{0.5cm} \Rightarrow \hspace{0.5cm}
        \bigg|
        \int_X fg \d x
        \bigg| \leq 1.
    \end{equation*}
\end{uproof}

По теореме \ref{thr8d17} любую $f \in L_2[-\pi, \pi]$ можно сколь угодно близко по норме приблизить бесконечно гладкой функцией с носителем строго в $(-\pi, \pi)$. Такая функция продолжается до бесконечно гладкой $2\pi$-периодической и по теореме \ref{Vthr} \textit{равномерно} приближается тригонометрическим многочленом $\sum_{k=-n}^{n} c_k e^{ikx}$. 

Равномерное приближение является приближением по норме $L_2$, так как на отрезке $[-\pi, \pi]$ имеется неравенство $\|f\|_2 \leq \sqrt{2\pi} \|f\|_C$. В случае $L_2$ нормы определим коэффициенты, которыми собираемся приближать.


% \red{Дописать про геометрическое представление коэффициентов Фурье.}