\sbs{23}{Признаки Липшица, Дирихле и Дини сходимости Фурье в точке}

\begin{to_thr}[признак Дини]
    Пусть $f$ -- $2\pi$-периодиечкая $\in L_1[-\pi, \pi]$. Если $x$ -- точка непрерывности или разрыва I рода и $\exists  \delta \in (0, \pi)$ такое, что $\int_0^\delta |f^*_x (t)|/t \d t$ сходится, то 
    ряд Фурье $f$ сходится в $x$ к $\frac{1}{2} \left(f(x+0)+ f(x-0)\right)$.
\end{to_thr}

Выше использовалась функция $f^*_x (t) = f(x-t) + f(x+t) - f(x-0) - f(x+0)$. В случае, если точка была регулярной, то Фурье к ней и сходится. Аналогично можно сформулировать это утвержение, как
\begin{equation*}
    \int_{-\delta}^{\delta} \bigg|\frac{f(x+t)-f(x)}{t} \bigg| \d t \text{ \ сходится}
    \hspace{0.5cm} \Rightarrow \hspace{0.5cm}  
    \text{ряд Фурье сходится к $f(x)$.}
\end{equation*}
Признаке Дирихле и Липшица в точке являются локальными аналогами признаков на отрезке. 





