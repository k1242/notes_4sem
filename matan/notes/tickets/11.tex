\sbs{11}{Абсолютная непрерывность произведения абсолютно непрерывных и обобщенное интегрирование по частям}


\begin{to_con}[Обобщенное интегрирование по частям]
    Если $f \in L_1 [a, b]$, а $g$ абсолютно непрерывна, то верна формула интегрирования по частям
    \begin{equation*}
        \int_a^b f g \d x = F(x) g(x) \bigg|_a^b
        - \int_a^b F(x) g'(x) \d x,
    \end{equation*}
    где $F(x) = \int_a^x f(t) \d t$.
\end{to_con}

\begin{uproof}
    Производная $g'$ существует почти всюду, функция $F$ абсолютно непрерывна по раннее доказанной теореме, тогда $Fg$ тоже абсолютно непрерывна:
    \begin{equation*}
        F(y) g(y) - F(x) g(x) = \left[
            F(y) - F(x)
        \right] g(y) + 
        \left[
            g(y) - g(x)
        \right] F(x).
    \end{equation*}
    Тогда $(Fg)' = fg + F g'$, к её приращению применима формула Ньютона-Лейбница, так и получаем интегрирование по частям. 
\end{uproof}

\textit{Дополнительно}. 
    Функция $f \colon [a, b] \mapsto \mathbb{R}$ абсолютно непрерывна тогда и только тогда, когда она может быть сколь угодно близко в $B$-норме приближена кусочно-линейными функциями.

% \red{А дальше про борелевские меры на отрезках и интеграл Лебега–Стилтьеса.}

