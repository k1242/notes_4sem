\sbs{5}{Полнота пространства \texorpdfstring{$L_p$}{Lp}}

\subsubsection*{Полнота пространства интегрируемых функций}

Далее в разделе всегда предполагается суммирование по $k$ от $1$ до $\infty$, если не сказано иного. Глобально можно сказать, что \texttt{в нормированном пространстве вопрос полноты сводится в вопросу сходимости рядов}, у которых сходятся суммы норм. 

\begin{to_def}
    Назовём последовательность $(f_n)$ \textit{фундаментальной}, если
    \begin{equation*}
        \forall \varepsilon > 0 \ \ 
        \exists N_\varepsilon \colon 
        \forall n, m \geq N_\varepsilon \ \ 
        \|f_n - f_m\|_p < \varepsilon.
    \end{equation*}
\end{to_def}

\begin{to_lem}
    Пусть у последовательности функций $(u_k)$ из $L_p (X)$ сумма
    $\Sigma = \sum \|u_k\|_p$
    оказалась конечной. Тогда $S(x) = \sum u_k (x)$ определена для почти всех $x$ и
    $\|S\|_p \leq \sum \|u_k\|_p.$
\end{to_lem}

\begin{uproof}
    Определим возрастающую последовательность
    \begin{equation*}
        \rho_N (x) = \left(
            \sum_{k=1}^{N} |u_k (x)|
        \right)^p,
        \hspace{0.5cm} \overset{(1)}{\Rightarrow} \hspace{0.5cm}
        \int_X \rho_N (x) \leq \left(
            \sum_{k=1}^{N} \|u_k (x)\|
        \right)^p  \leq \Sigma^p
        \hspace{0.5cm} \overset{(2)}{\Rightarrow} \hspace{0.5cm}
        \rho(x) = \lim_{N \to \infty} \rho_N (x).
    \end{equation*}
    Первое следствие получается по неравенству Минковского, второе по теореме о монотонной сходимости функции: $\rho(x)$ почти всюду конечна и имеет конечный интеграл, что означает почти всюду абсолютную сходимость ряда $\sum u_k (x)$. 

    Функция $\sigma_N (x) = \big| \sum u_k (x)\big|^p$ сходится к $|S(x)|^p$ почти всюду и $\sigma_N (x) \leq \rho(x)$. По теореме об ограниченной сходимости
    \begin{equation*}
        \left\|\sum u_k(x)\right\|_p^p \to \|S\|_p^p,
        \hspace{0.5cm} \Rightarrow \hspace{0.5cm}
        \|S\|_p \leq \sum \|u_k\|_p,
    \end{equation*}
    по предельному переходу в неравенстве Минковского. 
\end{uproof}

\begin{to_lem}
    Пусть у последовательности функций $(u_k)$ из $L_p (x)$ сумма
    $\Sigma = \sum \|u_k\|_p$
    оказалась конечной. Тогда $S(x) = \sum u_k (x)$ определена для почти всех $x$ и
    (что отличает эту лемму от предыдущей)
    $S = \sum u_k$
    в смысле сходимости в пространстве $L_p (X)$.
\end{to_lem}

\begin{uproof}
    По предыдущей лемме для остатка $r_N (x) = \sum_{k=N+1}^{\infty} u_k(x)$:
    \begin{equation*}
        \|r_N\|_p \leq \sum_{k=N+1}^{\infty}  \|u_k\|_p, \hspace{5 mm} 
        \text{при \ \ } N \to \infty,
    \end{equation*}
    что и означает  сходимость в терминах $L_p$. 
\end{uproof}

\begin{to_thr}[]
    Пространство $L_p (X)$ полно.
\end{to_thr}

\begin{uproof}
    Рассмотрим фундаментальную последовательность $(f_k)$ в $L_p (x)$ для подпоследовательности которой докажем сходимость. Выберем её так, чтобы $\|f_k - f_l\|_p \leq 2^{-k -1}$ при всех $l > k$. 

    Пусть тогда $u_1 = f_1$, $u_k = f_k - f_{k-1}$, получается хотим доказать сходимость суммы телескопического ряда $\sum u_k$, для которых $\|u_k\|_p \leq 2^{-k}$. По предыдущей лемме ряд почти всюду сходится к $S \in L_p (X)$, а $(f_k)$ сходятся к $S$ по норме $L_p (X)$.
\end{uproof}


Так и сводится в $L_p$ вопрос полноты к вопросу сходимости рядов, со сходящейся суммой норм.
Вообще сходимость в $L_p (X)$ может не означать поточечной сходимости ни в одной точке.