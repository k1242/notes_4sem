\sbs{5}{Полнота пространства \texorpdfstring{$L_p$}{Lp}}

\subsubsection*{Полнота пространства интегрируемых функций}

Далее в разделе всегда предполагается суммирование по $k$ от $1$ до $\infty$. Глобально можно сказать, что \texttt{в нормированном пространстве вопрос полноты сводится в вопросу сходимости рядов}, у которых сходятся суммы норм. 

\begin{to_def}
    Назовём последовательность $(f_n)$ \textit{фундаментальной}, если
    \begin{equation*}
        \forall \varepsilon > 0 \ 
        \exists N_\varepsilon \colon 
        \forall n, m \geq N_\varepsilon \
        \|f_n - f_m\|_p < \varepsilon.
    \end{equation*}
\end{to_def}

\begin{to_lem}
    Пусть у последовательности функций $(u_k)$ из $L_p (X)$ сумма
    $\sum \|u_k\|_p$
    оказалась конечной. Тогда $S(x) = \sum u_k (x)$ определена для почти всех $x$ и
    $\|S\|_p \leq \sum \|u_k\|_p.$
\end{to_lem}


\begin{to_lem}
    Пусть у последовательности функций $(u_k)$ из $L_p (x)$ сумма
    $\sum \|u_k\|_p$
    оказалась конечной. Тогда $S(x) = \sum u_k (x)$ определена для почти всех $x$ и
    $S = \sum u_k$
    в смысле сходимости в пространстве $L_p (X)$.
\end{to_lem}


\begin{to_thr}[]
    Пространство $L_p (X)$ полно.
\end{to_thr}



Вообще сходимость в $L_p (X)$ может не означать поточечной сходимости ни в одной точке.