\sbs{13}{Порядок убывания коэффициентов Фурье абсолютно непрерывных функций}


\begin{to_lem}
    Если производная $f^{(k-1)}$ абсолютно непрерывна и производные до $k$-й включительно\footnote{
        Для $k$-й достаточно существования почти всюду.
    }  находятся в $L_1 (\mathbb{R})$, то
    \begin{equation*}
        c_f (y) = \int_{-\infty}^{+\infty} f(x) e^{-ixy} \d x = o \left(\frac{1}{y^k}\right),
        \hspace{5 mm}
        t \to \infty.
    \end{equation*}
    \label{lem8d43}
\end{to_lem}


\begin{uproof}
    Всё как раньше, но слагаемые вижа $f^{(l)} (x) e^{-ixy} |_{-\infty}^{+\infty1}$ исчезают в силу конечности пределов $f^{(l)}$ на бесконечности. Так как $f^{(l+1)} \in L_1 (\mathbb{R})$, то $f^{(l)}$ имеет конечные пределы в $-\infty$ и $+\infty$, которые должны быть равны нулю, так как $f^{(l)}$ конечного интеграла.
    \red{Я бы и эту штуку посмотрел в другой книжке.}
\end{uproof}
