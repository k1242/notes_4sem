\sbs{7}{Функции ограниченной вариации}

\begin{to_def}
    Функция $f$ на промежутке $I$ имеет \textit{ограниченную вариацию}, если для любых $x_0 < x_1 < \ldots  x_N \in I$ (в любом количестве)
    \begin{equation*}
        |f(x_0) - f(x_1)| + 
        |f(x_1) - f(x_2)| + \ldots +
        |f(x_{N-1}) - f(x_N)| \leq M,
    \end{equation*}
    для некоторой константы $M$. Наименьшую константу $M$  назовём \textit{вариацией} функции $f$ равную $\|f\|_B$, что задаёт \textit{полунорму}, вида
    \begin{equation*}
        \|f\|_B = \sup\left\{
                            |f(x_0) - f(x_1)| + 
                        |f(x_1) - f(x_2)| + \ldots +
                        |f(x_{N-1}) - f(x_N)| \ 
                        \bigg| \ 
                        N \in \mathbb{N}, a \leq x_1 \leq \ldots \leq b\right\}
    \end{equation*}
\end{to_def}

Вообще это длина кривой в одномерном варианте, в частности кривая в $\mathbb{R}^n$ спрямляема только при конечной вариации каждой своей координаты. Важно что вариация функции аддитивна и выпукла, в смысле $\|f + g\|_B \leq \|f\|_B + \|g\|_B$.

\begin{to_lem}
    Функцию ограниченной вариации на отрезке $[a, b]$ можно представить в виде суммы двух функций $f = u + d$, одна из которых возрастает, а другая убывает. При этом $\|f\|_B = \|u\|_B + \|d\|_B$ и если $f$ была непрерывной, то $u, d$ тоже будут непрерывны.
\end{to_lem}


\begin{uproof}
    Определим \textit{вариацию вверх} $u(x)$ как $\sup$ сумм положительных приращений и \textit{вариацию вниз} $d(x)$ как $\inf$ сумм отрицательных приращений. Любой набор приращений даст $f(x)$ и его можно разбить на две части, одна из которых даст $u(x)$ а другая $d(x)$. Тогда
    \begin{equation*}
        f(x) = u(x) + d(x), \hspace{5 mm} 
        \left\| f|_{[a, x]} \right\| = u(x) - d(x),
    \end{equation*}
    при чём $u(x) \uparrow$ и $d(x) \downarrow$. Так как вариация монотонной функции -- модуль её приращения, то $\|f\|_B = \|u\|_B + \|d\|_B$. 

    Покажем теперь, что $f \in C[a, b] \ \Rightarrow \ u, d \in C[a,b]$. Функции $u$ и $-d$ не убывают, покажем, что нет скачков. Их сумма $u(x) - d(x)$ равна $\left\| f|_{[a, x]} \right\|$ так что осталось показать, что у вариации нет скачков, что сводится к утверждению о непрерывности зависимости длины спрямляемой кривой от параметра. 
\end{uproof}


Вспомним, что для монотонной $g$ и $f \in L_1$ верна следующая теорема о среднем:

\textcolor{ugray}{
\begin{to_thr}[Вторая теорема о среднем]
    Если $f$ интегрируема по Лебегу с конечным интегралом, а $g$ монотонна и ограниченна на $[a, b]$, то при некотором $\nu \in [a, b]$
    \begin{equation*}
        \int_a^b f(x) g(x) \d x = g(a+0) \int_a^\nu f(x) \d x + g(b-0) \int_\nu^b f(x) \d x.
    \end{equation*}
\end{to_thr}
}

Таким образом приходим к утверждению о том, что \texttt{функции ограниченной вариации допускают оценку интеграла своего произведения с другой функцией}. В силу предыдущей леммы для любой функции ограниченной вариации $g$ из второй теоремы о среднем 
\begin{equation*}
    \bigg| \int_a^b f(x) g(x) \d x \bigg| \leq
    \left(
        |g(a+0)| + \|g\|_B
    \right) \cdot \sup \left\{
        \bigg|
            \int_\nu^b f(x) \d x
        \bigg| \text{ при } \nu \in [a, b]
    \right\}.
\end{equation*}