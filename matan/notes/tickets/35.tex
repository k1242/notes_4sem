\sbs{35}{Явление Гиббса}


Для функции $f$ одной переменной с разрывом первого рода в точке $x$ остаётся возможность, что ряд фурье или интеграл Фурье будут сходиться к среднему значению между пределами
\begin{equation*}
    M_f (x) = \textstyle \frac{1}{2}\left(
        f(x+0) + f(x-0)
    \right).
\end{equation*}
Действительно, верно, что
\begin{equation*}
    T_n (f, x) - M_f (x) = 
    \int_{0}^{\pi} D_n (t) \left(f(x_0 + t) - f(x_0 - t)\right) \d t
    -
    \frac{1}{2} (2 M_f(x))  \int_{\pi}^{\pi} D_n (t) \d t
    =
    \sum_{\pm} \int_{0}^{\pi} \left(
        f(x\pm t) - f(x \pm 0)
    \right) D_n (t) \d t
\end{equation*}
для некоторого $\xi$ от $0$ до $\pi$.

Так, например, услвоие Гёльдера для разрывной функции в окрестности точки разрывая $x$ в виде
\begin{equation*}
    |f(x \pm \xi) - f(x \pm  0)| \leq C |\xi|^\alpha,
\end{equation*}
с положительными $C$ и $\alpha$, гарантирует сходимость ряда Фурье к $M_f (x)$ в точке $x$. Ещё проще с функциями ограниченной вариации, для которорых возможен разрыв не более, чем первого рода. 


Однако сходимость к разрывной функции не может быть равномерной. Рассмотрим представление функции $\sign x$ через Фурье:
\begin{equation*}
    \sign x = \frac{1}{\pi} \, v.p.\ \int_{-\infty}^{+\infty} \frac{\sin xy}{y} \d y = \lim_{h \to + \infty} \frac{2}{\pi} \int_{0}^{h}  \frac{\sin xy}{y}\d y
    .
\end{equation*}
Рассмотрим $S(\eta)$  вида
\begin{equation*}
    S(\eta) = \frac{2}{\pi} \int_{0}^{\eta} \frac{\sin t}{t} \d t,
\end{equation*}
максимум которой достигается в $\pi$, и $S(\pi) = 1 + G$, где $G = 0.18$. Тогда запишем
\begin{equation*}
    \sign x = \lim_{h \to + \infty} \frac{2}{\pi} \int_{0}^{h} \frac{\sin xy}{y}\d y = \lim_{h \to + \infty} S(hx),
\end{equation*}
откуда вижно, что частичный интеграл Фурье в точках $\pi/h$ принимает значение $1 + G$, то есть вылезает за область значений $\sign x$ на $G$ при всех $h$, что и называется \textit{явлением Гиббса}. 




