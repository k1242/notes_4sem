\sbs{46}{Непрерывне линейные отображения}
\begin{to_def}
	\textbf{Норма} линейного $A \colon E \to F$ между банаховыми --- $||A|| = \sup\{||A(x)|| \mid x \in E , ||x|| \leq 1\}$.
\end{to_def}
Можно сформулировать утверждения:
\begin{equation*}
	\forall x \in E, \, ||Ax|| \leq ||A|| \cdot ||x||
\end{equation*}
и для $f \colon E \to F$ и $g \colon F \to G$ верно:
\begin{equation*}
	||g \circ f|| \leq ||g|| \cdot ||f||.
\end{equation*}
Ядро отображения между банаховыми это просто $\ker A = \{x \in E \mid A x = 0\} $.