\sbs{14}{Порядок убывания коэффициентов Фурье функций ограниченной вариации}


\begin{to_thr}[]
    Если $f \in L-1 (\mathbb{R})$ имеет ограниченную вариацию на $\mathbb{R}$, то выражение
    \begin{equation*}
        c_f (y) = \int_{-\infty}^{+\infty} f(x) e^{-ixy} \d x = O(1/y),
        \hspace{5 mm} y \to +\infty.
    \end{equation*}
\end{to_thr}

\begin{uproof}
    % \red{Это доказательство также оставило в моём мировозрение белые пятна.}
    Получим оценку для интеграла по $[a, b]$. Можем представить $f = u + g$ в виде суммы монотонно возрастающей и убывающей. Тогда по второй теореме о среднем
    \begin{equation*}
        c_{[a, b]}[f](y) = \int_{a}^{b} f(x) e^{-ixy} \d x = u(a+0) 
        \int_{a}^{\nu} e^{-ixy} \d x + u(b-0) \int_{\nu}^{b} e^{-ixy} \d x + 
        g(a+0) \int_{a}^{\psi} e^{-ixy} \d x + g(b-0) \int_{\psi}^{b} e^{-ixy} \d x.
    \end{equation*}
    Функция ограниченной вариации имеет пределы на бесконечности, а из интегрируемости следует их равенство нулю. Тогда значения $u(a+0),\, u(b-a),\, g(a+0),\, g(b-0)$ оцениваются полной вариацией $\|f\|_B$, а интегралы оцениваются по модулю как $\frac{2}{|y|}$. 
\end{uproof}




\begin{to_con}
    Пусть функция $f \colon \mathbb{R} \mapsto \mathbb{R}$ имеет абсолютно непрерывную $(k-1)$-ую производную, производные до $k$-й включительно находятся в $L_1 (\mathbb{R})$, а $f^{(k)}$ (возможно, после изменения на множестве меры нуль) имеет ограниченную вариацию на $\mathbb{R}$, тогда
    \begin{equation*}
        c_f (y) = \int_{-\infty}^{+\infty} 
        f(x) e^{-ixy} \d x =
        O\left(\frac{1}{y^{k+1}}\right), 
        \hspace{1 cm}
        y \to \infty.
    \end{equation*}
\end{to_con}


\begin{uproof}
    Можно получить интегрированием по частям, аналогично лемме \ref{lem8d43}, только используя предыдущую теорему. 
\end{uproof}




\subsubsection*{Периодические функции}


\begin{to_def}
    Для $2\pi$-\textit{периодической функции}  $f(x+2\pi) \equiv f(x)$ \textit{коэффициенты Фурье}  запишутся, как
    \begin{equation*}
        c_n = \frac{1}{2\pi} \int_{-\pi}^{\pi} 
        f(x) e^{-inx} \d x = 
        \frac{
        (f, e^{inx})
        }{
        \|e^{inx}\|_2^2
        },
    \end{equation*}
    где последнее выражение понимается в смысле скалярного произведения и нормы в $L_2 [-\pi, \pi]$.
\end{to_def}

Для таких функций сохраняются утверждения, доказанные выше.

\begin{to_thr}[]
    Пусть функция $f$ имеет период $2 \pi$ и абсолютно непрерывную $(k-1)$-ую производную, причём $f^(k)$ (возможно, после изменения на множестве меры нуль) имеет ограниченную вариацию на $[-\pi, \pi]$, тогда
    \begin{equation*}
        c_n = \frac{1}{2\pi} \int_{-\pi}^{\pi} f(x) e^{inx} \d x =
        O\left(\frac{1}{n^{k+1}}\right),
        \hspace{1 cm}
        n \to \infty.
    \end{equation*}
\end{to_thr}


\begin{uproof}
Здесь слагаемые $f(x) e^{-ixy} |_{-\pi}^{+\pi}$ обращаются в нуль в силу $2\pi$-периодичности, поэтому можем воспользоваться теоремой об ограниченной вариации. 
\end{uproof}

\begin{to_lem}
    Если у $2\pi$-периодической функции ограниченной вариации есть ненулевое конечное число разрывов, и она кусочно абсолютно непрерывна, то оценка $O(1/n)$ для коэффициентов Фурье неулучшаема.
\end{to_lem}


\begin{to_thr}[]
    Пусть функция $f$ непрерывна и $2\pi$-периодическая, тогда для коэффициента Фурье имеется оценка
    \begin{equation*}
        c_n =
        \frac{1}{2\pi} \int_{-\pi}^{+\pi} f(x) e^{inx} \d x
        = O(\omega_f (\pi/n)),
    \end{equation*}
    где $\omega_f$ -- модуль непрерывности $f$.
\end{to_thr}


\begin{uproof}
Перейдём к переменной $x = x' + \pi/n$, тогда
\begin{equation*}
    c_n = - \frac{1}{2\pi} \int_{-\pi}^{+\pi} f(x' + \pi/n) e^{-in x'} \d x',
    \hspace{0.5cm} \Rightarrow \hspace{0.5cm}
    |c_n| = \frac{1}{2\pi} \bigg|
        \int_{-\pi}^{+\pi} \frac{1}{2} \left(f(x+\pi/n)-f(x)\right) e^{-inx} \d x
    \bigg| \leq \frac{1}{2} \omega_f (\pi/n). 
\end{equation*}
Так и получаем не очень точную, но полезную оценку. 
\end{uproof}