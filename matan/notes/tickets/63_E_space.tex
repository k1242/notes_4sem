\sbs{63}{Пространство \texorpdfstring{$\mathcal{E}$}{E} и топология в нём}
Будем рассматривать функции на действительной прямой, чтобы не отвелкаться на технические тонкости. И так
\begin{to_def}
	$\mathcal{E}(\mathbb{R}) = C^{\infty}(\mathbb{R})$. Топологию в таком пространстве зададим полунормами:
	\begin{equation*}
		||f||_{K,k} = \sup \{ |f^{(k)}| \mid x \in K\},
	\end{equation*}
	где $K \subset \mathbb{R}$ -- компакты и $k \in \mathcal{Z}_+$. Ну то есть имеем семейство открытых:
	\begin{equation*}
		U_{K,k,\varepsilon} (f_0) = \{f \mid ||f - f_0||_{K,k} < \varepsilon\}.
	\end{equation*}
\end{to_def}

\begin{to_thr}
	Пространство $\mathcal{E}$ -- полно.
\end{to_thr}

