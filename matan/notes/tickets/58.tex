\sbs{58}{Теорема Тихонова}


\begin{to_def}
    Множество $X$ называется \textit{топологическим пространством}, если на нём введена \textit{топология} $\mathcal O \subset 2^X$, элементы ккоторой называются \textit{открытыми множествами}, и выполняются свойства:
    \vspace{-3mm}
    \begin{enumerate}
        \item $\varnothing,\, X \in \mathcal O$;
        \item для всех семейств открытых множеств $\mathcal U \subset \mathcal O$ , объединение $\cup \mathcal U \in \mathcal O$;
        \item для всех конечных семейств $U_1, \ldots, U_k \in \mathcal O$, пересечение $U_1 \cap \ldots \cap U_k \in \mathcal O$. 
    \end{enumerate}
\end{to_def}


\begin{to_def}
    Пусть $A$ -- некоторое множество, а $\{X_\alpha\}_{\alpha \in A}$ -- семество топологических пространств, индексированное этим множеством. Базой топологии в декартовом произведении
    \begin{equation*}
        \prod_{\alpha \in A} X_\alpha
    \end{equation*}
    являются всевозможные произведения $\prod_{\alpha \in A} U_\alpha$ открытых $U_\alpha \subset X_\alpha$, у которых только для конечного числа $\alpha$ нарушается, что $U_\alpha = X_\alpha$. Вся топология состояит из всевозможных объединений множеств её базы. 
\end{to_def}



\begin{to_thr}[Теорема Тихонова]
    Пусть $A$ -- некоторое множество, а $\{X_\alpha\}_{\alpha \in A}$ -- семество компактных топологических пространств, индексированное этим множеством. Тогда будет компактным и декартово произведение
    \begin{equation*}
        \prod_{\alpha \in A} X_\alpha.
    \end{equation*}
\end{to_thr}

\begin{to_def}
    Пусть на пространстве $X$ топология состоит из всевозможных объединений множеств семейства $\mathcal B \subseteq 2^X$, покрывающего $X$. Тогда $\mathcal B$ называется \textit{базой} этой топологии. 
\end{to_def}


\begin{to_def}
    \textit{Предбазой} топологического пространства $X$ называется такой набор открытх множеств $\mathcal P \subseteq 2^X$, что всевозможные конечные пересечения $U_1 \cap \ldots \cap U_N$ элементов $U_1, \ldots, U_N \in \mathcal P$ составляют базу его топологии, иначе говоря, топология $X$ задаётся произвольными объединенями конечных пересечений множеств из $\mathcal P$. 
\end{to_def}


\begin{to_def}
    \textit{Окрестностью точки} $x \in X$ в топологическом пространстве называется любое открытое множество $U$ в этой топологии, содержащее $x$. 
\end{to_def}



Получается, что если у топологии есть база, то любая окрестность точки $x$ по определению содержит в себе одно из множеств базы, содержащее $x$. 
% Окрестностью точки $x \in X$ в топологическом пространстве 



