\sbs{8}{Абсолютно непрерывные функции, абсолютная непрерывность интеграла с переменных верхним пределом}

Для формулы Ньютона-Лейбница условие липшицевости можно ослабить до следующего:

\begin{to_def}
    Функция $F$ на промежутке $I$ \textit{абсолютно непрерывна}, если $\forall \varepsilon > 0 \ \exists \delta_\varepsilon > 0$, такое что
    $\forall \, x_1 \leq y_1 \leq x_2 \leq y_2 \leq \ldots \leq x_N \leq y_N \in I$ из неравенства
    \begin{equation*}
        |x_1 - y_1| + |x_2 - y_2| + \ldots + |x_N - y_N| \leq \delta
    \end{equation*}
    следует, что
    \begin{equation*}
        |F(x_1)-F(y_1)| + 
        |F(x_2)-F(y_2)| + 
        \ldots +
        |F(x_N)-F(y_N)| \leq \varepsilon.
    \end{equation*}
    Говоря неформально, сумма модулей приращений функции на системе непересекающихся отрезков должна \\ стремиться к нулю при суммарной длине системы, стремящейся к нулю.
\end{to_def}


\begin{to_thr}[]
    Для некоторой $f \in L_1 [a, b]$, всякая обобщенная первообразная $F$ 
    \begin{equation*}
        F(x) = \int_a^x f(t) \d t,
    \end{equation*}
     является абсолютно непрерывной и её производная почти всюду существует и совпадает с $f$. 
\end{to_thr}

\begin{uproof}
    В силу теоремы о дифференцируемости интеграла с переменным верхним пределом, производная $F$ почти всюду равна $f$. Осталось показать абсолютную непрерывность $F$. Как и раньше, приблизим $f$ ограниченной $g \leq M$, так что $\|f-g\|_1 < \varepsilon$. Тогда при наборе отрезков $S$ длины $< \delta$
     \begin{equation*}
         \int_S |g(x)| \d x \leq M \delta, \hspace{5 mm} 
         \int_S |f(x) - g(x)| < \varepsilon,
         \hspace{0.25cm} \Rightarrow \hspace{0.25cm}
        \int_S |f(x)| \d x \leq M \delta  + \varepsilon,
        \hspace{0.25cm} \Rightarrow \hspace{0.25cm}
        \int_S |f(x)| \d x \leq 2 \varepsilon,
     \end{equation*}
     что и означает сумма приращений $F$ на отрезках $S$ не более $2\varepsilon$ при $\mu[S] < \delta$. 
\end{uproof}



