\sbs{8}{Абсолютно непрерывные функции}

Для формулы Ньютона-Лейбница условие липшицевости можно ослабить до следующего:

\begin{to_def}
    Функция $F$ на промежутке $I$ \textit{абсолютно непрерывна}, если $\forall \varepsilon > 0 \ \exists \delta_\varepsilon > 0$, такое что
    $\forall \, x_1 \leq y_1 \leq x_2 \leq y_2 \leq \ldots \leq x_N \leq y_N \in I$ из неравенства
    \begin{equation*}
        |x_1 - y_1| + |x_2 - y_2| + \ldots + |x_N - y_N| \leq \delta
    \end{equation*}
    следует, что
    \begin{equation*}
        |F(x_1)-F(y_1)| + 
        |F(x_2)-F(y_2)| + 
        \ldots +
        |F(x_N)-F(y_N)| \leq \varepsilon.
    \end{equation*}
    Говоря неформально, сумма модулей приращений функции на системе непересекающихся отрезков должна \\ стремиться к нулю при суммарной длине системы, стремящейся к нулю.
\end{to_def}




\begin{to_lem}
    Абсолютно непрерывная на отрезке функция $f$ имеет на нём ограниченную вариацию. Также на отрезке существует разложение $f$ в сумму двух монотонных абсолютно непрерывных функций.
\end{to_lem}

\begin{uproof}
    Для данной абсолютно непрерывной $f \colon  [a, b] \mapsto \mathbb{R}$ рассмотрим $f = u + d$, также вспомним $u[x] + (-d[x]) = v(x) = \left\|f|_{[a, x]}\right\|_B$. Осталось показать абсолютную непрерывность $v(x)$.

    От противного: $\exists  \varepsilon > 0$ такое, что сумма приращений $v$ на некоторых отрезках не менее $\varepsilon$. По аддитивности вариации $v(y_i) - v(x_i) = \left\|f|_{[x_i, y_i]}\right\|_B,$ тогда
    \begin{equation*}
        \exists \ [x_{i1}, y_{i1}], \ldots, [x_{iN_i}, y_{iNN_i}] \subset [x_i, y_i], \ \colon  \ |f(x_i1) - f(y_{i1}) + \ldots + |f(x_{i N_i}) - f(y_{iN_i})|
        \geq \frac{v(y_i) - v(x_i)}{2}.
    \end{equation*}
    Суммируя такие неравенства по всем $i = 1, \ldots, N$ получаем, что сумма модулей приращений $f$ не менее $\varepsilon/2$, что противоречит абсолютной непрерывности $f$. 
\end{uproof}



