\sbs{25}{Теорема Фейера}



\begin{to_def}
    Определим \textit{ядро Фейера} 
    \begin{equation*}
        \Phi_n (t) = 
        \frac{D_0 (t) + D_1 (t) + \ldots + D_n (t)}{n+1} = 
        \frac{1}{2\pi} \sum_{k=-n}^{n} 
        \frac{n+1 - |k|}{n+1} e^{ikx},
    \end{equation*}
    как усреднение ядер Дирихле. Соответствующая \textit{сумма Фейера} будет соответствовать усреднением первых $n+1$ частичных сумм ряда Фурье,
    \begin{equation*}
        S_n (f, x) = \int_{-\pi}^{\pi} 
        f(x+\xi) \Phi_n (\xi) \d \xi = 
        \frac{T_0 (f, x) + \ldots + T_n (f, x)}{n+1}.
    \end{equation*}
\end{to_def}

Записав
\begin{equation*}
    D_n (t) 
    =
    \frac{1}{2\pi}
     \frac{\sin \left( \left(n + \frac{1}{2}\right)t\right)}{\sin \left(\frac{1}{2} t\right)} 
     = 
     \frac{1}{4\pi}
    \frac{
    \cos nt - \cos \left((n+1)t\right)
    }{
    \sin^2 \left(\frac{1}{2}t\right)
    },
\end{equation*}
и суммируя, получаем
\begin{equation*}
    \Phi_n (t) = 
    \frac{1}{4 \pi}
    \frac{
        1 - \cos \left((n+1)t\right)
    }{
        (n+1) \sin^2 \left(\frac{1}{2}t\right)
    }
    =
    \frac{1}{2\pi}
    \frac{
        \sin^2 \left(
            \frac{n+1}{2}t
        \right)
    }{
        (n+1) \sin^2 \left(\frac{1}{2}t\right)
    }
\end{equation*}


\begin{to_thr}
    Для непрерывной $2\pi$-периодической $f$ 
    \begin{equation*}
        S_n (f, x) \rightrightarrows f(x),
    \end{equation*}
    то есть сходится равномерно.
\end{to_thr}
