\sbs{9}{Представление в виде суммы монотонных абсолютно непрерывных}


\begin{to_lem}
    Абсолютно непрерывная на отрезке функция $f$ имеет на нём ограниченную вариацию. Также на отрезке существует разложение $f$ в сумму двух монотонных абсолютно непрерывных функций.
\end{to_lem}

\begin{uproof}
    Для данной абсолютно непрерывной $f \colon  [a, b] \mapsto \mathbb{R}$ рассмотрим $f = u + d$, также вспомним $u[x] + (-d[x]) = v(x) = \left\|f|_{[a, x]}\right\|_B$. Осталось показать абсолютную непрерывность $v(x)$.

    От противного: $\exists  \varepsilon > 0$ такое, что сумма приращений $v$ на некоторых отрезках не менее $\varepsilon$. По аддитивности вариации $v(y_i) - v(x_i) = \left\|f|_{[x_i, y_i]}\right\|_B,$ тогда
    \begin{equation*}
        \exists \ [x_{i1}, y_{i1}], \ldots, [x_{iN_i}, y_{iN_i}] \subset [x_i, y_i], \ \colon  \ |f(x_i1) - f(y_{i1})| + \ldots + |f(x_{i N_i}) - f(y_{iN_i})|
        \geq \frac{v(y_i) - v(x_i)}{2}.
    \end{equation*}
    Суммируя такие неравенства по всем $i = 1, \ldots, N$ получаем, что сумма модулей приращений $f$ не менее $\varepsilon/2$, что противоречит абсолютной непрерывности $f$. 
\end{uproof}



