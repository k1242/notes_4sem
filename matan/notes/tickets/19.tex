\sbs{19}{Интегральное представление частичных сумм ряда Фурье, ядро Дирихле}

\begin{to_def}
    Обозначим \textit{частичную сумму} тригонометрического ряда Фурье для $2\pi$-периодической функции $f$ как
    \begin{equation*}
        T_n (f, x) = \sum_{k=-n}^n c_k (f) e^{ikx}.
    \end{equation*}
\end{to_def}

\begin{to_lem}
    Для $n$-й частичной суммы ряда Фурье $2\pi$-периодической функции имеет место формула в виде свёртки
    \begin{equation*}
        T_n (f, x) = \int_{-\pi}^{\pi} f(x+t) D_n (t) \d T,
    \end{equation*}
    с ядром Дирихле
    \begin{equation*}
        D_n (t) = \frac{1}{2\pi} \frac{\sin \big(\left(n+\frac{1}{2}\right) t\big)}{\sin\left(\frac{1}{2} t\right)}.
    \end{equation*}
\end{to_lem}

\begin{uproof}
    По определению:
    \begin{equation*}
        T_n[f](x) = \sum_{k=-n}^{n} c_k e^{ikx}
        =
        \frac{1}{2\pi} \sum_{k=-n}^{n} \int_{-\pi}^{+\pi}
        f(\xi) e^{ikx-ik\xi} \d \xi
        = \bigg/
            \xi = x + t
        \bigg/ = \int_{-\pi}^{+\pi} f(x+t) \left(
            \frac{1}{2\pi} \sum_{k=-n}^{n} e^{-ikt}
        \right) \d t.
    \end{equation*}
    Теперь раскрываем геометрическую прогрессию:
    \begin{equation*}
        D_n (t) = \frac{1}{2\pi} \sum_{k=-n}^{n} e^{-itk} = 
        - \frac{e^{int}}{2\pi} \frac{e^{it}-e^{-2int}}{1 - e^{it}} = 
        \frac{e^{i(n+1/2)t}-e^{-i(n+1/2)t}}{2\pi \left(e^{it/2}-e^{-it/2}\right)} = 
        \frac{1}{2\pi} \frac{\sin(n+1/2)t}{\sin t/2}.
    \end{equation*}
\end{uproof}


\begin{to_lem}[Равномерная ограниченность интегралов от ядра Дирихле]
    Существует такая константа $C$, что 
    \begin{equation*}
        \left|
        \int_a^b D_n (t) \d t
        \right| \leq C
    \end{equation*}
    для любых $a, b \in [-\pi, \pi], \ n \in \mathbb{N}$.
\end{to_lem}


\begin{uproof}
    Заметим, что $t/\sin(t/2)$ -- монотонная и ограниченная на $[-\pi, \pi]$ функция, тогда вынесем её из под знака интеграла:
    \begin{equation*}
        \bigg| \int_{a}^{b} \frac{1}{2\pi} \frac{\sin \big(\left(n+\frac{1}{2}\right) t\big)}{\sin\left(\frac{1}{2} t\right)} \d t \bigg| \sim
        \bigg|
           \int_a^b \frac{\sin \big(\left(n+\frac{1}{2}\right) t\big)}{t} \d t 
        \bigg| \sim 
        \bigg|
            \int_{a}^{b}  \frac{\sin t}{t} \d t
        \bigg|.
    \end{equation*}
    А оставшееся выражение принимает значения $\in [-1, 1]$, так что имеет конечный интеграл на отрезке.  
\end{uproof}

Также можем оценить интеграл от ядра Дирихле:
\begin{equation*}
    D_n (t) = \frac{1}{2\pi} \sum_{k=-n}^{n} e^{-ikt},
    \hspace{0.5cm} \Rightarrow \hspace{0.5cm}
    \int_{-\pi}^{\pi} D_n (t) = 1,
    \hspace{0.5cm} \Rightarrow \hspace{0.5cm}
    T_n[f](x) - f(x) = \int_{-\pi}^{+\pi} \left(
        f(x+t) -f(x)
    \right) D_n(t) \d t,
\end{equation*}
что исследуется равномерным принципом локализации.