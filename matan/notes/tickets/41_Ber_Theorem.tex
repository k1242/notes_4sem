\sbs{41}{Теорема Бэра в банаховом пространстве}

\begin{to_thr}
	Счетное ${U_k}$ -- открытых всюду плотных подмножеств банахова $E$ имеет $\bigcap_k U_k \neq \varnothing$.
\end{to_thr}

\begin{to_con}
	Если банахово $E$ покрыто счетным $(Z_k)$ замкнутых множеств, то $\exists m \colon \text{int}Z_m \neq \varnothing$.
\end{to_con}

\begin{to_thr}[Неподвижные точки сжимающих отображений]
	Для банахово $E$ замкнутого $X \subset E$ отображение $f \colon X \to X$ -- \textbf{сжимающее}, то есть
	\begin{equation*}
	 	\exists C <1 \colon \forall x,y \in X \, \|f(x) - f(y)\| \leq C\|x-y\|.
	 \end{equation*} 
	 Имеет неподвижную точку $x \in X$ такую, что $f(x) = x$.
\end{to_thr}