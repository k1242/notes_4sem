\sbs{21}{Признак Липшица равномерной сходимости ряда Фурье}

\begin{to_def}
    Функция $f$ называется гёльдеровой степени $\alpha > 0$, если для любых $x, \, y$ из области определения
    \begin{equation*}
        |f(x) -f(y)| \leq C |x-y|^{\alpha}
    \end{equation*}
    с некоторой константой $C$.
\end{to_def}


\begin{to_thr}[Признак Липшица сходимости ряда Фурье]
    Для абсолютно интегрируемой $2\pi$-периодической функции, которая является гёльдеровой с некоторыми $C$, $\alpha > 0$ на интервале $(A, B) \supset [a, b]$
    \begin{equation*}
        T_n (f, x) \to f(x)
    \end{equation*}
    равномерно по $x \in [a, b]$ при $n \to \infty$.
\end{to_thr}

\begin{uproof}
    Вспомним локальное представление $T_n[f](x) - f(x)$, как
    \begin{equation*}
        \bigg|
            \int_{-\delta}^{\delta} \left(f(x+t)-f(x)\right) D_n (t) \d t
        \bigg|  \leq 
        C \int_{-\delta}^{\delta} \frac{|t|^\alpha}{2\pi}
        \frac{1}{|\sin t/2|} \d t \leq 
        \frac{C}{2} \int_{-\delta}^{\delta} |t|^{\alpha-1} \d t \leq \frac{C}{\alpha} |\delta|^\alpha,
    \end{equation*}
    где мы воспользовались мыслью, что $\pi |\sin t/2| \geq t$ на $[-\pi, \pi]$. По произвольности $\delta$ и равномерного принципа локализации следует, что $T_n[f](x) - f(x)$ может быть равномерно сделано сколь угодно маленьким при некотором $\delta > 0$ и $n$. 
\end{uproof}


