\sbs{27}{Интеграл Дирихле}


\begin{to_lem}[нормировка интеграла Дирихле]
    Выполняется
    \begin{equation*}
        v.p.\ \int_{-\infty}^{+\infty}  D_h(t) \d t = v.p. \ \int_{-\infty}^{+\infty}  \frac{\sin ht}{\pi t} = 1.
    \end{equation*}
\end{to_lem}

\begin{uproof}
Интеграл можем свести к условно сходящемуся интегралу
\begin{equation*}
    \int_{0}^{\infty}  \frac{\sin t}{t} \d t =\frac{\pi}{2}.
\end{equation*}
Рассмотрим интеграл с параметром $y \in [0, +\infty)$:
\begin{equation*}
    I(y) = \int_{0}^{\infty} e^{-yt} \frac{\sin t}{t} \d t,
\end{equation*}
который сходится абсолютно при $y > 0$. Разложим интеграл на два и проанализируем отдельно:
...
\end{uproof}



