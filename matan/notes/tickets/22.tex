\sbs{22}{Признак Дирихле равномерной сходимости тригонометрического ряда Фурье на отрезке}

Стоит заметить, что верна следующая цепочка вложений:
\begin{equation*}
    C^1[-\pi, \pi] \ \subseteq \ 
     L\text{-Lipshitz непрерывные}  \ \subseteq \ 
     AC[-\pi, \pi] \ \subseteq \ 
     BV[-\pi, \pi] \ \subseteq \ 
     \text{дифференцируемые почти всюду},
\end{equation*}
где $BV$ -- банахово несепарабельное пространство функций ограниченной вариации. 

Так, например $x \sin (1/x)$ -- непрерывная функция неограниченной вариации. Функция Кантора на $[0, 1]$ -- функция ограниченной вариации, но не абсолютно непрерывная. 




\begin{to_thr}[Признак Дирихле сходимости ряда Фурье]
    Для абсолютно интегрируемой $2\pi$-периодической функции, которая является непрерывной с ограниченной вариацией на интервале $(A, B) \supset [a, b]$
    \begin{equation*}
        T_n (f, x) \to f(x)
    \end{equation*}
    равномерно по $x \in [a, b]$ при $n \to \infty$.
\end{to_thr}


% \red{Далее несколько лемм, сформулированных в виде задач, а именно признак Дирихле сходимости ряда Фурье в точке, признак Липшица сходимости ряда Фурье в точке, признак Дини сходимости ряда Фурье в точке. Ага, это 13 тема. А потом будут темы 14 - 17.}



