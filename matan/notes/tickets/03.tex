\sbs{3}{* Алгебры непрерывных на компактах функций. Теорема Стоуна-Вейерштрасса}

\begin{to_def}
    Множество $\mathcal{A} \subseteq C(x)$ (-- непрерывные на компакте функции) называется \textit{алгеброй}, если она содержит константы ($\mathbb{R} \subseteq \mathcal{A}$) и топологически "замкнута" относительно операций $\cdot$ и 
    %$\bullet$
    $+$.
\end{to_def}

\begin{to_def}
    \textit{Алгебра разделяющая точки} --- $\forall a, b \in \mathbb{R},\, x=y \in X,\, \, \exists f \in A$ такая что $f(x)=a$, а $f(y) = b$.
\end{to_def}

\begin{to_thr}[теорема Стоуна-Вейерштрасса]
    Пусть у нас зафиксирован компакт $K$ и дана алгебра непрерывных функций $\mathcal A$ на этом компакте, которая разделяет точки. Тогда любую непрерывную на $K$ функцию можно сколь угодно близко равномерно приблизить функциями из $\mathcal A$. 
\end{to_thr}