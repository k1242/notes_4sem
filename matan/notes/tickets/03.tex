\sbs{3}{* Алгебры непрерывных на компактах функций. Теорема Стоуна-Вейерштрасса}

\begin{to_def}
    $\mathcal{A} \subseteq C(x)$(-- непрерывные на компакте функции) называется \textit{алгеброй}, если она содержит константы ($\mathbb{R} \subseteq \mathcal{A}$) и топологически "замкнута" относительно операций $\cdot$ и 
    %$\bullet$
    $+$.
\end{to_def}

\begin{to_def}
    \textit{Алгебра разделяющая точки} --- $\forall a, b \in \mathbb{R},\, x=y \in X,\, \, \exists f \in A$ такая что $f(x)=a$, а $f(y) = b$.
\end{to_def}

\begin{to_thr}[теорема Стоуна-Вейерштрасса]
    Если $X$-метрический компакт, а алгебра $\mathcal{A}\in C(x)$ разделяет точки, \textbf{то} $\forall f \in C(x)$ можно сколь угодно точно равномерно приблизить функциями из $\mathcal{A}$.
\end{to_thr}