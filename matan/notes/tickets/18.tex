\sbs{18}{Равенство Парсеваля для Фурье функций из \texorpdfstring{$L_2[-\pi, \pi]$}{L2[-pi, pi]}} 



\begin{to_thr}[Сходимость ряда Фурье в среднеквадратичном]
    Для вской комплекснозначной $f \in L_2 [-\pi, \pi]$
    \begin{equation*}
        f = \sum_{k=-\infty}^{\infty} 
        c_k e^{ikx} = 
        \lim_{n \to \infty} \sum_{k=-n}^{n} c_k e^{ikx},
        \hspace{10 mm} 
        c_k = \frac{1}{2\pi} \int_{-\pi}^{+\pi} f(x) e^{ikx} \d x
    \end{equation*}
    в смысле сходимости суммы в пространстве $L_2[-\pi, \pi]$, а также выполняется равенство Парсеваля
    \begin{equation*}
        \|f\|_2^2 = 2 \pi \sum_{k=-\infty}^{\infty} |c_k|^2.
    \end{equation*}
\end{to_thr}

\begin{uproof}
    Сначала функцию $f$ приближаем по $L_2$ норме тригонометрическим многочленом. Формула для квадрата точности приближения
    \begin{equation*}
         \left\|
            f - \sum_{k=1}^{N} c_k \varphi_k
         \right\|_2^2 = \|f\|_2^2 - 2\pi \sum_{k=1}^{N} |c_k|^2 < \varepsilon,
     \end{equation*} 
     откуда при $N \uparrow$ можем говорить про сходимость ряда Фурье по $L_2$ норме по определению. Также получаем в пределе в неравенстве Бесселя равенство Парсеваля. 
\end{uproof}

Стоит заметить что в последней теореме использовали <<симметричное>> сумирование -- \textit{суммирование в смысле главного значения}:
\begin{equation*}
    v.p. \ \sum_{k=-\infty}^{+\infty} c_k e^{ikx}  = \lim_{n \to \infty} \sum_{k=-n}^{n}  c_k e^{ikx}.
\end{equation*}

\texttt{Пока мы не доказали, что в полученную формулу можно подставить хоть одно конкретное значение $x$. Тот факт, что ряд Фурье функции из
$L_2[-\pi, \pi]$ на самом деле сходится к этой функции почти всюду, был доказан Л. Карлесоном (1966), а до этого был известен как гипотеза Лузина.} 

