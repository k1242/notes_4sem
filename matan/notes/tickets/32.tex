\sbs{32}{Пространство \texorpdfstring{$S$}{S} и его инвариантность}


\begin{to_def}
    Пространство $cal S (\mathbb{R})$ -- пространство бесконечно дифференцируемых функций $f\colon \mathbb{R} \mapsto \mathbb{C}$, у которых конечны все полунормы ($k,\, n \geq 0$) вида
    \begin{equation*}
        \|f\|_{n,k} = \sup\left\{
            |x^n f^{(k)} (x) \mid x \in \mathbb{R}
        \right\}.
    \end{equation*}
    Другими словами все их производны убывают на бесконечности быстрее любой степени. 
\end{to_def}

Так, например, $e^{-x^2/2}$ лежит в $\mathcal S (\mathbb{R})$. 



\begin{to_def}
    Последовательность $(f_m)$ функций из $\mathcal S (\mathbb{R})$ стремится к $f_0$, если для любых $n,\, k \geq 0$
    \begin{equation*}
        \|f_m - f_0\|_{n, k} \to 0, \hspace{5 mm}  m \to \infty.
    \end{equation*}
\end{to_def}


Также можем определить топологию на $\mathcal S(\mathbb{R})$, объявив предбазовыми открытми окрестностями функции $f_0 \in \mathcal S(\mathbb{R})$ множества
\begin{equation*}
    U_{n. k, \varepsilon} (f_0) = \left\{
        f \in \mathcal S(\mathbb{R}) \\medskip
        \|f-f_0\|_{n, k} < \varepsilon
    \right\},
\end{equation*}
объявив базовыми окрестностями $f_0$ любые конечные пересечения предбазовых окрестностей $f_0$ и объявив открытыми те множества, которые содержат каждый свой элемент вместе со своей базовой открытой окрестностью. 

\begin{to_thr}[]
    Преобразование Фурье непрерывно переводит $\mathcal S\left(\mathbb{R}\right)$ в $\mathcal S(\mathbb{R})$. 
\end{to_thr}

\begin{uproof}
    Заметим, что $\sup |F[f]|$ ограничен $L_1$-нормой $\|f\|_1$ с точностью до константы. Также заметим, что
    \begin{equation*}
        \|f\|_1 \leq \pi\left(
            \|f\|_{0, 0} + \|f\|_{2, 0}
        \right),
    \end{equation*}
    так как если $|f| \leq M$ и $|x^2 f| \leq N$, то всюуда $(1+x^2)|f| \leq M + N$ и интеграл от $|f|$ не более $M + N$  на интеграл от $\frac{1}{x^2+1}$, который равен $\pi$. 

    Тогда можем ограничить $\|F[f]\|_{0, 0}$ в терминах полунорм исходной функции $f$. Если же нас интересует упремум выражения вида
    \begin{equation*}
        y^n \frac{d^k}{d y^k}  F[f],
    \end{equation*}
    то с точностью до константы это выражение является преобразованием Фурье от
    \begin{equation*}
        \frac{d^n}{dx^n} \left(x^k f(x)\right).
    \end{equation*}
    Тогда, по формуле Лейбниа, имеет место выражение ... которое приведет нас к оценке вида
    \begin{equation*}
        \|F[f]\|_{n, k} \leq \sum_{k' \leq k + 2,\, n' \leq n} C_{k', n', k, n} \|f\|_{k', n'},
    \end{equation*}
    которая доказывает определенность Фуре как линейного отображения $\mathcal S (\mathbb{R}) \mapsto \mathcal S (\mathbb{R})$. Отсюда же получаем и непрерывность преобразования Фурье по Гейне. 
\end{uproof}

Также, по Коши, можно показать, что для любой базовой окрестности $U \ni F[f_0]$ найдётся базовая окрестность $V \ni f_0$, такая что $F(V) \subseteq U$. 

\begin{to_thr}[Формула суммирования Пуассона]
    Для функции $f \in \mathcal S (\mathbb{R})$ верна формула
    \begin{equation*}
        \sum_{n \in \mathbb{Z}} f(2 \pi n) = \frac{1}{\sqrt{2\pi}} \sum_{n \in \mathbb{Z}} \hat{f}(n).
    \end{equation*}
\end{to_thr}



