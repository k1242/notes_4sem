\sbs{69}{Дифференцирование обобщенных функций}
Общая идея выведения действия какой-либо операции на обобщенные функции -- проделать её с регулярной, а затем обобщить и на нерегулярные. Будем теперь писать действия обобщенной функции как: $\lambda_f(\varphi) = \langle \lambda_f, \varphi\rangle$.

\begin{to_def}
	Производная от обобщенной функции: $\langle \lambda', \varphi \rangle  = - \langle \lambda, \varphi'\rangle$.
\end{to_def}
Здесь это вынесено определением, но несложно показать интегрируя по частям функцию с компактным носителем:
\begin{equation*}
	\langle \lambda_f', \varphi \rangle = - \int_{-\infty}^{+\infty} f'(x)\varphi(x) d x = - \int_{-\infty}^{+\infty} f(x) \varphi'(x)d x
\end{equation*}
Таким образом получили линейный непрерывный функционал.

Можно сформулировать утверждения вытекающие из определения производной:
\begin{enumerate}
	\item $\forall f \in \mathcal{D}'$ имеет производные всех порядков;
	\item Если $(f_k)\to f$ для обобщенных. То и $f_k' \to f'$. И так далее;
\end{enumerate}
 
 \begin{to_lem}
 	Всякий сходящийся ряд из обобщенных функций можно дифференцировать почленно любое количество раз.
 \end{to_lem}

 И для примера рассмотрим тоже популярную функцию
 \begin{equation*}
 	f(x) = \left\{
 	\begin{aligned}
 		&1, \, x > 0\\
 		&0, \, x \leq 0
 	\end{aligned}\right.
 	\hspace{0.5 cm}
 	\leadsto
 	\hspace{0.5 cm}
 	\langle \lambda_f,\varphi\rangle = \int_{0}^{\infty} \varphi(x) d x
 \end{equation*}
 Это \textbf{Функция Хевисайда} она обладает хайповым свойством -- её производная это дельта-функция:
 \begin{equation*}
 	\langle \lambda_f', \varphi\rangle = - \langle \lambda_f, \varphi'\rangle = - \int_{0}^{\infty} \varphi'(x) d x = \varphi(0).
 \end{equation*}

Покажем ещё непрерывность в смысле топологии по конспекту Романа Николаевича:
\begin{equation*}
	\lambda' \in U_{\varphi,a,b} 
	\hspace{0.3 cm}
	\Leftrightarrow
	\hspace{0.3 cm}
	a <\lambda'(\varphi) < b 
	\hspace{0.3 cm}
	\Leftrightarrow
	\hspace{0.3 cm}
	a < - \lambda(\varphi') < b
	\hspace{0.3 cm}
	\Leftrightarrow
	\hspace{0.3 cm}
	\lambda \in U_{- \varphi', a, b}.
\end{equation*}
