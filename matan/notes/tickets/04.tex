\sbs{4}{Пространства \texorpdfstring{$L_p$}{Lp}. Неравенства Гёльдера и Минковского.}

\begin{to_def}
    \textit{Абсолютно интегрирумыми функциями} на измеримом $X \subseteq \mathbb{R}^n$ называют $f \colon X \mapsto \mathbb{R}$ с конечным интегралом $\int_X |f(x)| \d x$. \textit{Расстоянием}\footnote{
        В силу неравенства $|f(x) - g(x)| \leq |f(x)| + |g(x)|$ расстояние конечно.
    } между функциями $f$ и $g$ будем считать $\int_X |f(x)-g(x)| \d x$.
\end{to_def}

\begin{to_def}
    \textit{Нормой} в векторном пространстве $V$ над полем $\mathbb{F}$ называется функционал $p \colon  V \mapsto \mathbb{R}_+$, обладающий своствами: \vspace{-3mm}
    \begin{enumerate*}
        \item $p(x) = 0 \ \Rightarrow \ x = 0_V$ -- невырожденность нормы (в \textit{полунорме} это неверно);
        \item $\forall x,\, y \in V,\, \ p(x + y) \leq p(x) + p(y)$ -- неравенство треугольника;
        \item $\forall  \alpha \in \mathbb{F},\, \forall x \in V,\,\ p(\alpha x) = |\alpha| p(x)$. 
    \end{enumerate*}
\end{to_def}

\begin{to_def}
    Обозначим через $L_1 (X)$ факторпространство  линейного пространства абсолютно интегрируемых функций по его линейному подпространству почти всюду равных нулю функций. То есть функции на $0$ расстоянии считаем равными. \textit{Нормой} будем считать
    \begin{equation*}
        \|f\|_1 = \int_X |f(x)| \d x.
    \end{equation*}
\end{to_def}

\begin{to_def}
    Для измеримого по Лебегу $X \subset \mathbb{R}^n$ и числа $p \geq 1$ \textit{факторпространство} измеримых по Лебегу функций на $X$ с конечной (полу)нормой
    \begin{equation*}
        \|f\|_p
        = 
        \left(
            \int_X |f|^p \d x
        \right)^{1/p},
    \end{equation*}
    по модулю функций равных нулю почти всюду,
    назовём $L_p (X)$.
\end{to_def}



\texttt{Очень хорошим, симметричным, актуальным для описания квантовой механики оказывается $L_2$ простран-\\ство, на котором естественно вводить скалярное произведение, его порождающее.} 

 \begin{to_def}
     В комплексном случае норма $L_2$ порождена \textit{скалярным произведением}
     \begin{equation*}
         (f, g) = \int_{-\infty}^{+\infty} f(x) \overline{g(x)} \d x
         \hspace{0.5cm} \longrightarrow \hspace{0.5cm}
         \|f\|_2 = \sqrt{(f, f)}.
     \end{equation*}
 \end{to_def}


\begin{to_thr}[Неравенство Гёльдера]
    Возьмём $p, \, q > 1$ такие, что $1/p + 1/q = 1$. Пусть $f \in L_p (X)$ и $g \in L_q(X)$. Тогда
    \begin{equation*}
        \int_X |fg| \d x \leq \|f\|_p \cdot \|g\|_q.
    \end{equation*}
\end{to_thr}

\begin{uproof}
    Добьёмся (домножением на константу) ситуации с $\|f\|_p = \|g\|_q=1$. 
    Тогда достаточно проинтегрировать неравенство вида
    \begin{equation*}
        |fg| \leq \frac{|f|^p}{p} + \frac{|g|^q}{q}.
    \end{equation*}
    Неравенство же можем получить из выпуклости логарифма
    \begin{equation*}
        \ln(\alpha a + \beta b) \geq \alpha \ln a + \beta \ln b, \hspace{2.5 mm} \alpha + \beta = 1,
        \hspace{0.5cm} \Rightarrow  \bigg/ \left.\begin{aligned}
            \alpha = p^{-1} \\
            \beta = q^{-1}
        \end{aligned}\right. \bigg/ \Rightarrow\hspace{0.5cm}
        \ln\left(\frac{a}{p} + \frac{b}{q}\right) \geq \frac{\ln a}{p} + \frac{\ln b}{q} = \ln (a^{1/p} b^{1/q}).
    \end{equation*}
\end{uproof}

\begin{to_con}
    Для измеримых функций и чисел $p, \, q > 0$, таких что $1/p + 1/q = 1$, имеет место формула
    \begin{equation}
        \label{8_1}
        \|f\|_p = \sup \left\{
            \int_X fg \d x \ \bigg| \  \|g\|_q \leq 1
        \right\}.
    \end{equation}
\end{to_con}

\begin{proof}[$\triangle$]
По неравенству Гёльдера норма $f$ не менее супремума правой части, более того равенство достигается при выборе
\begin{equation*}
    g(x) = \frac{\sign f(x) |f(x)|^{p-1}}{\|f\|_p^{p-1}}.
\end{equation*}
\end{proof}

\begin{to_def}
    Функция $f \colon V \mapsto \mathbb{R}$ на векторном пространство называется \textit{выпуклой}, если для любых $x, \, y \in V$ и любого $t \in (0, 1)$ имеет место неравенство
    \begin{equation*}
        f ( (1-t) x + ty) \leq (1-t) f(x) + t f(y).
    \end{equation*}
    Функция называется \textit{строго выпуклой}, если неравенство строгое $\forall x \neq y$ и $t \in (0, 1)$. 
\end{to_def}

\begin{to_lem}
    Если в семействе функций $f_\alpha \colon V \mapsto \mathbb{R}$, $\alpha \in A$, все функции выпуклые, то
    \begin{equation*}
        f(x) = \sup \{f_\alpha (x) \mid \alpha \in A\}
    \end{equation*}
    тоже выпуклая\footnote{
        Если разрешить в определении выпуклости значение $+ \infty$.
    }.
\end{to_lem}

\begin{uproof}
    Выпуклость функции нескольких переменных означает выпуклость всех её ограничений на прямые, а значит достаточно доказать это для функции одной переменной, \texttt{что допускает графическое доказательство}. 
\end{uproof}

\begin{to_thr}[Неравенство Минковского]
    Для функций $f, \, g \in L_p$ при $p \geq 1$
    \begin{equation*}
        \|f + g\|_p \leq \|f\|_p + \|g\|_p.
    \end{equation*}
\end{to_thr}

\begin{uproof}
    В силу предыдущих двух утверждений норма $\|\cdot\|_p$ -- выпуклая функция на $L_p$, тогда, в частности
    \begin{equation*}
        \left\|\frac{f+g}{2}\right\|_p \leq \frac{1}{2} \left(
            \|f\|_p + \|g\|_p
        \right),
        \hspace{0.5cm} \Rightarrow \hspace{0.5cm}  
        \left\|\frac{f}{2} + \frac{g}{2}\right\|_p \leq \left\|\frac{f}{2}\right\|_p + \left\|\frac{g}{2}\right\|_p,
    \end{equation*}
    где последнее верно по 1-однородности нормы. 
\end{uproof}

