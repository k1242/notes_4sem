\sbs{1}{Приближение функций кусочно-линейными и  многочленами}
% 112 страница (4.7)

\textit{Носитель} функции -- дополнение к объдинению всех открытых множеств, на которых функция равна нулю, иначе -- замыкание множества точек, в которых функция не равна нулю. Получается носитель функции всегда замкнут и для функций на $\mathbb{R}^n$ компактнось носителя означает ограниченность. 

\begin{to_lem}
    Для непрерывной с компактным носителем $f(x) \colon \mathbb{R} \to \mathbb{R}$ и $t_n \to 0$ при $n \to \infty$, последовательность $f_n(x) = f(x + t_n) \rr f$.
\end{to_lem}

\begin{uproof}
Непрерывня функция с компактным носителем равномерно непрерывна, то есть
\begin{equation*}
    \forall \varepsilon > 0 \,  \exists \delta > 0 \, \forall x \in \mathbb{R},\, \forall t,\, \ 
    \left(
        |t| < \delta \ \Rightarrow \ |f(x-t) - f(x)| < \varepsilon,
    \right)
\end{equation*}
что можно интепретировать как равномерной сходимости $f(x-t_n) \rr f(x)$. 
\end{uproof}

% --------------------------------

\begin{to_thr}[]
    Для $|x| < 1$ и $\alpha \in \mathbb{R}$ $$(1+x)^\alpha = \sum_{n=0}^{\infty} \begin{pmatrix}
        a \\ n
    \end{pmatrix} x^n$$ с радиуом сходимости не менее $1$. 
\end{to_thr}

\begin{to_lem}
    $f(x) = \sqrt{x}$ можно равномерно приблизить многочленами на любом отрезке $[0,a]$.
\end{to_lem}

\begin{uproof}
    Заменой переменной $x = a -y$ сведем вопрос к приближению функции
    \begin{equation*}
        g(y) = \sqrt{a + \delta} \sqrt{1 - \frac{y}{a+\delta}},
    \end{equation*}
    который раскладывается по предыдущей лемме в степенной ряд при $|y| \leq a + \delta$, причём при $|y| \leq a$ ряд сходится равномерно, тогда $g(y)$ приближается многочленом на $[0, a]$, соответственно и $\sqrt{x}$ тоже. 
\end{uproof}


% --------------------------------

\begin{to_lem}
    $f(x) = |x|$ можно равномерно приблизить многочленами на любом отрезке $[-a,a]$.
\end{to_lem}

% \begin{uproof}
% Идея 
% \end{uproof}

\begin{uproof}
    На отрезке $[0, a^2]$ приближаем $\sqrt{t}$ многочленом $|\sqrt{t} - P(t)| < \varepsilon$. Подставим $x = \sqrt{t}$, тогда на $x \in [0, a]$ верно $|x - P(x^2)| < \varepsilon$, что можно продолжить на $[-a, a]$, продолжая $x$ чётным образом как $|x|$: $\big| |x| - P(x^2) \bigg| < \varepsilon$. 
\end{uproof}

% --------------------------------

\begin{to_thr}
    Всякую непрерывную кусочно-линейную на отрезке $[a, b]$ функцию можно сколь угодно близко равномерно приблизить многочленом.
\end{to_thr}

\begin{uproof}
Если функция со скачком производной на $\Delta$, то $f(x) - \Delta/2 |x-x_i|$ будет уже без скачка, тогда кусочно-линейная представится в виде
\begin{equation*}
    f(x) = \sum_i c_i |x - x_i | + a x + b,
\end{equation*}
где каждое слагаемой уже приближаемо. 
\end{uproof}

Этого достаточно, чтобы приближать кусочно-линейные многочленами. Осталось понять, как приближать непрерывные на отрезке функции  кусочно-линейными. Определим
\begin{equation*}
    \varphi_\delta (x) = \left\{\begin{aligned}
        &0, & x < - \delta, \\
        & 1 - |x|/\delta, &|x| \leq \delta \\ 
        &0, & x > \delta.
    \end{aligned}\right.
\end{equation*}
Такая функция кусочно линейная, непрерывная, и её носитель -- $[-\delta, \delta]$.

% --------------------------------

\begin{to_lem}
    Для непрерывной $f \colon [0,1] \to \mathbb{R}$: $\sum\limits_{k = 0}^{m} f(k/m) \varphi_{1/m} (x - k/m) \rr f$.
\end{to_lem}

\begin{uproof}
Воспользуемся разбиением единицы
\begin{equation*}
    \sum_{k=0}^{m} \varphi_{1/m} (x - k/m) = 1.
\end{equation*}
Умножая это на $f(x)$ и вычитая $f_m(x)$, получаем
\begin{equation*}
    f(x) - f_m (x) = \sum_{k=0}^{m}  \left(
        f(x)- f(k/m)
    \right) \varphi_{1/m} (x-k/m).
\end{equation*}
При фиксированном $x$ в правой части слагаемые ненулевые только при $|x-k/m|<1/m$. Тогда правую часть оценим через модуль непрерывности 
\begin{equation*}
    \bigg| \sum_{k=0}^{m}  \left(
        f(x)- f(k/m)
    \right) \varphi_{1/m} (x-k/m) \bigg| \leq 
    \sum_{k=0}^{m}  \omega_f (1/m) \varphi_{1/m} (x-k/m) = \omega_f (1/m),
\end{equation*}
который стремится к нулю при $m \to \infty$ по непрерывности $f$. Напомним, что
\begin{equation*}
    \omega_f (\delta) = \sup \left\{
        \rho(f(x) - f(y)) \mid x, y \in M,\, \ \rho(x, y) < \delta.
    \right\}
\end{equation*}


\end{uproof}

% --------------------------------

\begin{to_thr}
    Всякую $f \colon [a_1, b_1] \times [a_n, b_n] \to \mathbb{R}$ можно сколь угодно близко
равномерно приблизить многочленом.
\end{to_thr}


\begin{uproof}
    Сначала масштабируем параллелепипед в единичный куб. Потом равномерно приближаем непрерывную функцию комбинацией произведений кусочно-линейных функций отдельных переменных:
\begin{equation*}
    f \colon  [0, 1]^n \mapsto \mathbb{R}, \hspace{5 mm} 
    f_m (x) = \sum_{k_1, \ldots, k_n} f\left(
        \frac{k_1}{m}, \ldots, \frac{k_n}{m}
    \right) \varphi_{1/m}\left(x_1 -\frac{k_1}{m}\right) \ldots \varphi_{1/m} \left(x_n - \frac{k_n}{m}\right).
\end{equation*}
     Потом их приближаем многочленами. 
\end{uproof}sw