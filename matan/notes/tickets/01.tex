\sbs{1}{Приближение функций кусочно-линейными и кусочно-линейных многочленами}
% 112 страница (4.7)

\begin{to_lem}
    Для непрерывной с компактным носителем $f(x) \colon \mathbb{R} \to \mathbb{R}$ и $t_n \to 0$ при $n \to \infty$, последовательность $f_n(x) = f(x + t_n) \rr f$.
\end{to_lem}

\begin{to_lem}
    $f(x) = \sqrt{x}$ можно равномерно приблизить многочленами на любом отрезке $[0,a]$.
\end{to_lem}

\begin{to_lem}
    $f(x) = |x|$ можно равномерно приблизить многочленами на любом отрезке $[-a,a]$.
\end{to_lem}

\begin{to_thr}
    Всякую непрерывную кусочно-линейную на отрезке $[a, b]$ функцию можно сколь угодно близко равномерно приблизить многочленом.
\end{to_thr}

\begin{to_lem}
    Для непрерывной $f \colon [0,1] \to \mathbb{R}$: $\sum\limits_{k = 0}^{m} f(k/m) \varphi_{1/m} (x - k/m) \rr f$.
\end{to_lem}

\begin{to_thr}
    Всякую $f \colon [a_1, b_1] \times [a_n, b_n] \to \mathbb{R}$ можно сколь угодно близко
равномерно приблизить многочленом.
\end{to_thr}
