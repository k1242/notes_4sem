% сенсоры
% вычисление и хранение
% коммуникации

% 
% 
% -андрей турлапов
% кирилл лахманский



\section{Лазерное охлаждение} 

\subsection{Холодные атомы}

Цель охлаждения -- фазовая плотность:
\begin{equation*}
    N = \int \frac{d^3 x d^3 p}{(2\pi \hbar)^3} \frac{N}{V} 
    \left(
        \frac{2\pi \hbar^2}{mT}
    \right)^{3/2} e^{-p^2/2mT}.
\end{equation*}
На $\Delta p \cdot \Delta x$ приходится $2 \pi \hbar$. Работаем в приближении классического Максвелловского газа. 

Можем получить оценку для характерной фазовой плотности:
\begin{equation*}
    \rho \sim n     \left(
        \frac{2\pi \hbar^2}{mT}
    \right)^{3/2}  = \frac{N}{V} \lambda^3_{\text{dB}}.
\end{equation*}
% snpp:     \text{db}



Начинаем охлаждать: $N(v) = v^2 e^{-mv^2/T}$ с максимум в $\sqrt{T/m}$. Охладить -- сужение распределения по скоростям. 



\phantom{42}

\noindent
\textbf{Домашнее задание:} \hrulefill
\vspace{-2mm}
\begin{enumerate*}
    \item Оцените фазовую плотность азота в атмосфере при нормальных условиях.
    \item Оцените фазовую плотность атомов лития при температоре 100 мкК и межчастичном расстоянии 1 мкм.
\end{enumerate*}
\hrule
\phantom{42}

Рассматриваем 2 типа охлаждения: резонансный свет, испарение. Так в 1981 году охладили до $1.5$ К. 

Так вот, охлаждать -- замедлить движение. 
% Разрешим в модели двухуровнего атома в состояниях $|2\rangle$  и $|1\rangle$. 
Если атом летит навстречу, то атом ловит фотон -- из-за доплеровского эффекта. Частота лазера немного меньше резонансной.


Освещаем со всех трёх сторон -- получаем трёхмерное охлаждение. Но начнём смотреть на цифры в одномерном случае:
\begin{equation*}
    m a = \hbar k \frac{\Gamma}{2},
\end{equation*}
где $1/\Gamma = 27$ нс, -- обратное время жизни возбужденного состояния.

\phantom{42}

\noindent
\textbf{Домашнее задание:} \hrulefill
\vspace{-2mm}
\begin{enumerate*}
    \item Атом лития замедляется со скорости $500$ м/c, рассчитать расстояние, на котором скорость падает до $0$. 
    Рассчитать ускорение, сравнить с $g$. Переход $2 s\to 2p$.
\end{enumerate*}
\hrule

\phantom{42}


Посмотрим на облачко атомов. Хочется подумать про магнито-оптическую ловушку (1987 г.). Усложним ситуацию до $m_j \in {-1, 0, +1}$. Атом может находиться на $2s$ и $2p$. В магнитном поле уровни раздвигаются: гап на $\mu_B B$. 

Сделаем так, чтобы свет имел циркулярные поляризации. 
% добавьте номера слайдов
Подробное описание см. на слайде \textit{магнито-оптическая ловушка}.




Какой предел охлаждения? Предел охлаждения из-за фотонной отдачи:
\begin{equation*}
    T_{\text{min}} = \frac{p_{\text{photon}}^2}{2m},
    \hspace{5 mm} 
    p_{\text{photon}} = \frac{2\pi \hbar}{\lambda}.
\end{equation*}
Так находим $T_{\text{min}} = 4$ мкК для Li. 

Есть также предел Летохова-Миногина-Павлова (1977):
\begin{equation*}
    T_{\text{min}} = \hbar \Gamma / 2 = 150 \text{ мкК для Li,}
\end{equation*}
что приводит к фазовой плотности порядка $0.01$.





\subsection{Ещё холоднее}


Оптическая дипольня ловушка -- частота лазера много меньше резонансной, также возьмём $P = 100$ Вт. В постоянном внешнем поле возникает дипольный момент $\vc{d} = |e| \vc{l}$, более того
\begin{equation*}
    V = - \vc{d} \cdot \vc{E} = - \frac{\alpha}{2} \vc{E}^2,
\end{equation*}
что очень похоже на оптический пинцет. Так можем удержать.

Охлаждать же будем методом испарения. Сложности при охлаждении испарением: частота межатомных испарений и резонансное рассеяние. Частота $\nu = n \sigma v$.  

\newpage

\begin{hw2}
\item Оцените частоту классически столкновений атома лития с другими атомами при межчастичное расстояние 1 мкм, температура 100 мкК, -- они ооочень редко сталкиваются. 
\item ...
\end{hw2}