\subsection{Обработка квантовой информации на холодных ионах в ловушках}

\begin{to_def}
    \textit{Кубит} -- произвольная квантовая двухуровневая система.
\end{to_def}

Можно ввести \textit{сферу Блоха} -- описание состояния кубита, с $\alpha = \cos \theta/2$ и $\beta = e^{i\varphi} \sin (\theta/2)$.


Есть несколько моделей квантовых вычисленийЮ например, на гейтах (вентилях) -- унитарные квантовые операторы. Любую однокубитную операцию можно представить, как вращение на сфере Блоха. 

Так например можно реализовать гейты CNOT, SWAP, и т.д. Любой квантовый алгоритм представляется в виде (см. слайд 4.11). В конце всегда измеренеия. 

\begin{to_thr}[]
    Набор квантовых операций должен состоять из произвольной однокубитной операции, и двухкубитного запутывающего гейта является универсальным.
\end{to_thr}

\begin{to_con}
    Набор операций, состоящий из произвольных вращений вокруг осей сферы Блоха, и двухкубитного запутывающего гейта является универсальным.
\end{to_con}


% все ли кубиты эффективно используются?
%  пример квантового параллелизма
% 


% \subsection{Однокубитные вентили}
% Двухуровневая система, взаимодействующая с (почти) резонансным лазером оп





% вентиль цирака-цоллера
% вентиль молмера-соренсена
% 1-10
% 30-100
% 



% 300 мс 00 


% quantum dynamics of single trapped ions
% 







