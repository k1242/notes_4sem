Пространство элементарных исходов назовём дискретным, если множество $\Omega$ конечно или счётно: $\Omega = \{\omega_1, .., \omega_n, \ldots\}$. 

\begin{to_def}
    Сопоставим каждому элементарному исходу $\omega_i$ число $p_i \in [0, 1]$ так, чтобы $\sum p_i = 1$. Вероятностью события $A$ называют число
    \begin{equation*}
        \P(A) = \sum_{\omega_i \in A} p_i,
    \end{equation*}
    где с случае $A = \varnothing$ считаем $\P(A) = 0$. 
\end{to_def}


\begin{to_def}[Классическое определение вероятности]
    Говорят, что эксперимент описывается \textit{классической вероятностной моделью}, если пространство его элементарных исходов состоит из конечного числа равновозможных исходов. Для любого события верно, что
    \begin{equation}
        \P (A) = \frac{\card A}{\card \Omega}.
    \end{equation}
    Эту формулу называют \textit{классическим определением вероятности}.
\end{to_def}

Тут стоит вспомнить три схемы из модели с урнами: 
схема выбора с возвращением и с учётом порядка ($n^k$), 
выбора без возвращения и с учётом порядка ($A_n^k$), а также выбора
без возвращения и без учёта порядка ($C_n^k$), 
описываются классической вероятностной моделью. 
А вот схема выбора с возвращением и без учёта порядкауже не описывается классической вероятностью. 


\subsubsection*{Пример с гипергеометрическим распределением}
 Из урны, в которой $K$ белых и $N - K$ чёрных шаров, наудачу и без возвращения вынимают $n$ шаров, где $n \leq N$.
Термин «наудачу» означает, что появление любого набора из $n$ шаров равновозможно. Найти вероятность того, что будет выбрано $k$ белых и $n - k$ чёрных шаров.

Результат -- набор из $n$ шаров. Общее число $\card \Omega = C_N^n$. Пусть $A_k$ -- событие, состоящее в том, что в наборе окажется $k$ белых и $n-k$ черных. Есть ровно $C_K^k$ способов выбрать $k$ белых шаров из $K$, и 
$C_{N-K}^{n-k}$ способов выбрать $n-k$ черных шаров из $N-K$. Тогда $\card A_k = C_K^k C_{N_K}^{n-k}$,
\begin{equation*}
    \P (A_k) = \frac{\card A_k}{\card \Omega} = \frac{C_K^k C_{N_K}^{n-k}}{C_N^n}.
\end{equation*}
Этот набор вероятностей называется \textit{гипергеометрическим распределением} вероятностей. 



