
\subsection*{Фазовая плотность газа}

Оценим фазовую плотность
\begin{equation*}
    \rho \sim n \left(\frac{2 \pi \hbar^2}{m kT}\right)^{3/2}
\end{equation*}
азота в атмосфере при нормальных условиях и для холодного лития.

\textbf{Для азота} соответствующие величины
\begin{equation*}
    T \approx 273 \text{ K}, \hspace{5 mm}  
    P \approx 10^5 \text{ Па}, \hspace{5 mm} 
    m \approx \frac{28 \cdot 10^{-3}}{N_\text{A}} \text{ кг}, \hspace{5 mm} 
    n \approx 0.78 \frac{P}{k T} \approx 2.1 \cdot 10^{25} \text{ м}^{-3}.
\end{equation*}
Коэффициент 0.78 -- количество объёмное количество азота в атмосфере. Итого находим
\begin{equation*}
    \rho\left(\text{N}_2\right) \sim 1.6 \de{-7}
\end{equation*}

\textbf{Для лития} соответствующие величины
\begin{equation*}
    T \approx 100 \de{-6} \text{ K}, \hspace{5 mm}  
    m \approx \frac{7 \cdot 10^{-3}}{N_\text{A}} \text{ кг}, \hspace{5 mm} 
    n \approx 0.2 \cdot \frac{1}{a^{3}} \approx 2.0 \de{17} \text{ м}^{-3}.
\end{equation*}
Здесь коэффицент $0.2$ в выражение для концентрации -- результат расчёта межчастичного расстояния для идеального газа, с вероятностью нахождения на радиусе $r$:
\begin{equation*}
    P(r) = \frac{3}{r_{s}} \left(\frac{r}{r_{s}}\right)^2 \exp\left(-\left[\frac{r}{r_{s}}\right]^3\right),
    \hspace{5 mm} 
    r_{s} = \left(\frac{3}{4\pi n}\right)^{1/3},
    \hspace{5 mm} 
    \langle P(r)\rangle = 0.893 \cdot  r_s,
    \hspace{0.5cm} \Rightarrow \hspace{0.5cm}
    n \approx 0.17 \cdot \frac{1}{a^3},
\end{equation*}
что совпадает с численным моделированием. Итого, фазовая плотность для лития,
\begin{equation*}
    \rho\left(\text{Li}\right) \sim 1.7 \de{-4} \sim 1.1 \de{3} \rho\left(\text{N}_2\right).
\end{equation*}



\subsection*{Замедление лития}

Теперь оценим, ускорение с которым замедляется литий в пучке:
\begin{equation*}
    m w = \hbar k \frac{\Gamma}{2} = \frac{\pi \hbar \Gamma}{\lambda}, \hspace{5 mm}  \Gamma = \frac{1}{27} \de{-9} \text{ c}^{-1},
    \hspace{5 mm} 
    \lambda = 670 \text{ нм}.
\end{equation*}
Тогда
\begin{equation*}
    w \approx 1.6 \de{6} \text{ м/с}^2 \approx 1.6 \de{5} g,
\end{equation*}
а длина тормозного пути (с начальной скорости $500$ м/c) составит  $s = v_0^2/2g \approx 79$ мм, что не так уж и много.



\subsection*{Частота столкновений}


Оценка частоты столкновений (в одном кубометре):
\begin{equation*}
    f = \frac{1}{\sqrt{2}} n^2 \sqrt{\frac{8 k T}{\pi m}} \sigma,
\end{equation*}
и здесь вопрос только к $\sigma$ для холодного лития. 


\textbf{Для холодного лития}.
Судя по \href{https://arxiv.org/pdf/physics/0107075.pdf}{этой} эффективное сечение для температуры порялка 50 мкК и столкновений Li-Cs порядка
\begin{equation*}
    \sigma_{\text{CsLi}} \sim 5 \de{-16} \text{ м}^2.
\end{equation*}
% кстати, там же указана частота для такого процесса. 
% \begin{equation*}
    % f_0 \sim 20 \text{ c}^{-1}.
% \end{equation*}
Так как нас интересует оценка порядков, то наверное не очень плохая затея считать цезий сильно меньше лития, и тогда $\sigma_{\text{LiLi}} \approx 2\sigma_{\text{CsLi}} \approx \sim 1 \de{-15} \text{ м}^2$. Вообще есть зависимость $\sigma(T)$ вида $\sigma = \sigma_0 (1 + S/T)$ (формула Сазерленда), но этим тоже пренебрежем. Тогда
\begin{equation*}
    f_1 = 1.1 \de{19} \text{ c}^{-1},
\end{equation*}
что много (?). Если взять просто размер (?) атомов лития $r = 1.8 \de{-12}$ м, то $\sigma \approx 2.1 \de{-19}$ м$^2$, и $f_2 \sim 2.3 \de{15} \text{ c}^{-1}.$ 

\textbf{Для азота в нормальных условиях}.
Здесь проще воспользоваться знанием длины свободного пробега для молекул азота в нормальных условиях
\begin{equation*}
    \lambda \approx 0.6 \de{-7} \text{ м},
\end{equation*}
откуда
\begin{equation*}
    \sigma = \frac{1}{\sqrt{2} \lambda n} \approx 5.7 \de{-19} \text{ м}^{2}.
\end{equation*}
Тогда частота столкновений
\begin{equation*}
    f \approx 1.3 \de{35} \text{ с}^{-1},
\end{equation*}
что много больше частоты столкновений для атомов лития.
