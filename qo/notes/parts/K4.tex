\section{КвантМех в холодных ионах}

Квантовомеханические системы описываются квантовомеханическим уравнением Шредингера вида
\begin{equation*}
    i \hbar \partial_t \Psi = \hat{H} \Psi,
\end{equation*}
где для $N$ частиц $\Psi \equiv \Psi(\vc{r}_1, \ldots, \vc{r}_n, t)$. 
Гамильтониан же -- набор частных производных вида
\begin{equation*}
    \hat{H} = - \sum \frac{\hbar}{2m} \frac{\partial^2 }{\partial x_i^2} + V(x_1, \ldots, x_n),
\end{equation*}
а потенциальная энергия --
\begin{equation*}
    V = \sum \frac{e^2}{|r_i - r_j|} + \sum V(r_i).
\end{equation*}
Важно понимать, что есть набор стационарных решений вида
\begin{equation*}
    \Psi = \exp\left(
        - \frac{i E_n t}{\hbar}
    \right) \Psi_n (\vc{r}_1, \ldots, \vc{r}_N).
\end{equation*}
Переобозначим исходный гамильтониан за $H_0$, а добавку будем называть $\hat{V}$ вида
\begin{equation*}
\red{
    \hat{V} = - e \vc{E} \cdot (\vc{r}_1+ \ldots+ \vc{r}_N) =  \vc{E} \cdot \vc{d}.
}
\end{equation*}
Так вот, $H = \hat{H}_0 + \hat{V}$, допустим мы знаем, что
\begin{equation*}
    \hat{H}_0 \Psi_n = E_n \Psi_n,
\end{equation*}
тогда
\begin{equation*}
    \Psi = \sum c_n (t) e^{-i E_n t} \Psi_n.
\end{equation*}
Подставляя это в уравнение Шредингера, получим
\begin{equation*}
    i \hbar\, \partial_t c_n = \sum_{m} e^{i(E_n - E_m)t} V_{nm} (t) c_m.
\end{equation*}

Светим лазером сейчас на частоте близкой к резонансной, поэтому предлагается оставить только резонансную пару $c_0, \, c_1$.  Если ограничиться всего двумя членами, то
\begin{equation*}
    E(t) = E_0 \cos \left(
        (E_1-E_0) t
    \right),
    \hspace{5 mm} 
    i \partial_t \begin{pmatrix}
        c_0 \\ c_1
    \end{pmatrix} = 
    \begin{pmatrix}
        0 & (Ed)^* \\
        Ed & 0
    \end{pmatrix}
    \begin{pmatrix}
        c_0 \\ c_1
    \end{pmatrix},
\end{equation*}
которое уже решается. В результате находим осциллирующую резонансную пару
\begin{equation*}
        c_0 (t) = c_0 (0) \cos\left(
            \frac{|E_0 d|}{\hbar} \, t
        \right) + i e^{i \varphi} c_1 (0) \sin \left(
            \frac{|E_0 d|}{\hbar} \, t
        \right).
\end{equation*}
Есть матрицы Паули
\begin{equation*}
    \sigma_x = \begin{pmatrix} 0 & 1 \\ 1 & 0 \\ \end{pmatrix};
    \hspace{5 mm} 
    \sigma_y = \begin{pmatrix} 0 & -i\\ i & 0 \\ \end{pmatrix};
    \hspace{5 mm} 
    \sigma_z = \begin{pmatrix} 1 & 0 \\ 0 & -1 \\ \end{pmatrix},
\end{equation*}
 также вводится $\sigma_{+-} = \frac{1}{2} \left(
    \sigma_x \pm i \sigma_y
 \right)$, тогда гамильтониан перепишется в виде
 \begin{equation*}
     i \hbar \partial_t c = \left[
        (Ed)^* \sigma_- + (Ed) \sigma_+
     \right].
 \end{equation*}


\subsection{Начало движения ядра}

Теперь ядро $(\vc{R})$ не неподвижно, но решать будем в предположение о факторизации системы
\begin{equation*}
    \Psi = \Psi(\vc{R}, \vc{r}_1, \ldots, \vc{r}_N) = \tilde{\Psi} (\tilde{R}) \cdot \Psi(\vc{r}_1, \ldots, \vc{r}_N).
\end{equation*}
Внешний потенциал ловушки практически гармонический
\begin{equation*}
    \hat{H} = \frac{\hat{p}}{2M} + \frac{m \omega^2 R^2}{2}.
\end{equation*}
Его можно с помощью повыщающих и понижающих операторов
\begin{equation*}
    a, a^+ = \sqrt{
    \frac{\hbar}{2 M \omega}
    } x \pm i \sqrt{\frac{\hbar \omega}{2m}} \hat{p}.
\end{equation*}
Тогда гамильтониан перепишется в виде
\begin{equation*}
    \hat{H} = \hbar \omega \left(
        a^+ a + \frac{1}{2}
    \right).
\end{equation*}
Очень важно, что $[a, a^+ ] = 1$. Например, можно решить
\begin{equation*}
    a |0\rangle = 0,
    \hspace{0.5cm} \Rightarrow \hspace{0.5cm}
    H |0\rangle = \frac{\hbar \omega}{2} |0\rangle,
\end{equation*}
где $|0\rangle \equiv \Psi_0 = \Psi(\vc{R})$. Найдём $\Psi_n = (a^+)^n |0\rangle.$
Утверждается, что
\begin{equation*}
    H \Psi_n = \hbar \omega  \left(n + \frac{1}{2}\right) \Psi_n.
\end{equation*}
Это можно доказать. Так, например,
\begin{equation*}
    [a^+ a, a] = a^+ a \, a - a a^+ a = - a,
    \hspace{5 mm} 
    [a^+ a, \, a^+] = a^+.
\end{equation*}
% написать функцию: send message
% написать функцию: make note
% def mknt(): s = input(); with open("D:\\Kami\\git_folder\\notes_4sem\\qo\\notes\\parts\\K0.tex", "w") as f: f.write(s);
Теперь можем записать, что
\begin{equation*}
    \vc{E}(t) = \vc{E}_0 \cos(\Omega t - \vc{k} \cdot \vc{R}),
    \hspace{5 mm} 
    \vc{R} = \sqrt{\frac{1}{2 m \hbar \omega}} (a + a^+).
\end{equation*}
В предположении о малости $\vc{k} \cdot \vc{R}$, можем найти
\begin{equation*}
    \vc{E} \approx \frac{\vc{E}_0}{2} 
        e^{-i \Omega t} \left(
            1 + i k R
        \right),
\end{equation*}
что запишем как соответствующий член в гамильтониан Лэмбэ-Дикке.
\begin{equation*}
    \hat{H} = H_0 + \frac{E_0}{2} \left(
        e^{-\Omega t} \left(
            1 + ik \sqrt{\frac{1}{2 m \hbar \omega}} (a + a^+)(\sigma_+ + \sigma_-)
        \right) + h.c.
    \right),
\end{equation*}
% ☺1111}▐Ї☺☼BcTc♠♠♥§♀|}{╝╗╜║╣╕╖↕╢╡│2☺☺
Выбрав свет в резонанс на один колебательный квант -- мы запутываем ион с одной степенью свободы,
\begin{equation*}
    \Omega = \frac{\varepsilon_1 - \varepsilon_0}{\hbar} \pm \omega_0.
\end{equation*}


















